% This is samplepaper.tex, a sample chapter demonstrating the
% LLNCS macro package for Springer Computer Science proceedings;
% Version 2.21 of 2022/01/12
%
\documentclass[runningheads]{llncs}
%
\usepackage[T1]{fontenc}
% T1 fonts will be used to generate the final print and online PDFs,
% so please use T1 fonts in your manuscript whenever possible.
% Other font encondings may result in incorrect characters.
%
\usepackage{graphicx}
% Used for displaying a sample figure. If possible, figure files should
% be included in EPS format.
%
% If you use the hyperref package, please uncomment the following two lines
% to display URLs in blue roman font according to Springer's eBook style:
%\usepackage{color}
%\renewcommand\UrlFont{\color{blue}\rmfamily}
%\urlstyle{rm}
\usepackage[utf8]{inputenc}
\usepackage{graphicx}
\usepackage{stmaryrd}
\usepackage{amsmath}
\usepackage{amssymb}
\usepackage{multicol}
\usepackage{xparse}
\usepackage{hyperref}
\usepackage[capitalise,noabbrev]{cleveref}
\usepackage{listings}
\usepackage{mathpartir}
\usepackage{xcolor}
\usepackage{mathtools}
\usepackage{wasysym}
\usepackage{agda}
\usepackage{src/agdadimmed}
% \usepackage{src/agdatheme}
\usepackage{newunicodechar}
\usepackage{mdframed}
\usepackage{wrapfig}
\usepackage{boxedminipage}
\usepackage{dsfont}
\usepackage{etoolbox}
\usepackage{scalerel}
\usepackage{graphicx}
\usepackage{mathbbol}
\usepackage{lineno}
\usepackage{setspace} % for GS course
\usepackage{ifthen}
\usepackage{tikz-cd} 
\usepackage{fontspec} 
%
\newtoggle{draft}
\toggletrue{draft} % enables notes

\iftoggle{draft}{
\linenumbers
\doublespacing
}

\input{src/macros/todo.tex}

%%
%% Agda typesetting commands shorthands, for
%% manual typesetting of inline code
%%

\newcommand{\af}{\AgdaFunction}
\newcommand{\un}{\AgdaUnderscore}
\newcommand{\ad}{\AgdaDatatype}
\newcommand{\ab}{\AgdaBound}
\newcommand{\ac}{\AgdaInductiveConstructor}
\newcommand{\aF}{\AgdaField}
\newcommand{\as}{\AgdaSymbol}
\newcommand{\aS}{\AgdaString}
\newcommand{\ak}{\AgdaKeyword}
\newcommand{\ap}{\AgdaPrimitiveType}
\newcommand{\an}{\AgdaNumber}
\newcommand{\aC}{\AgdaComment}
\newcommand{\am}{\AgdaModule}


%%
%% Unicode for typesetting Agda code
%%%%%%%%%%%%%%%%%%%
%% AGDA UNICODE %%%
%%%%%%%%%%%%%%%%%%%

% hacks

\newunicodechar{◂}{} % for repeating defs from stdlib for example

%%
%% Symbols
%%

\newcommand\superequiv{\mathrel{\rlap{\raisebox{\fontdimen22\textfont2}{$=$}}\raisebox{-0.5\fontdimen22\textfont2}{$ = $}}}

\newunicodechar{×}{\ensuremath{\mathnormal\times}}
\newunicodechar{→}{\ensuremath{\mathnormal\to}}
\newunicodechar{←}{\ensuremath{\mathnormal\leftarrow}}
\newunicodechar{⟦}{\ensuremath{\mathnormal\llbracket}}
\newunicodechar{⟧}{\ensuremath{\mathnormal\rrbracket}}
\newunicodechar{ℕ}{\ensuremath{\mathbb{N}}}
\newunicodechar{⊕}{\ensuremath{\mathnormal\oplus}}
\newunicodechar{⊗}{\ensuremath{\mathnormal\otimes}}
\newunicodechar{ẋ}{\ensuremath{\dot{\times}}}
\newunicodechar{∔}{\ensuremath{\dot{+}}}
\newunicodechar{⋎}{\ensuremath{\mathnormal\curlyvee}}
\newunicodechar{⊞}{\ensuremath{\mathnormal\boxplus}}
\newunicodechar{⇒}{\ensuremath{\mathnormal\Rightarrow}}
\newunicodechar{⇛}{\ensuremath{\mathnormal\Rrightarrow}}
\newunicodechar{⟨}{\ensuremath{\mathnormal\langle}}
\newunicodechar{⟩}{\ensuremath{\mathnormal\rangle}}
\newunicodechar{∪}{\ensuremath{\mathnormal\cup}}
\newunicodechar{⦃}{\ensuremath{\mathnormal\{\mskip-4.5mu\mid}}
\newunicodechar{⦄}{\ensuremath{\mathnormal\mid\mskip-4.5mu\}}}
\newunicodechar{⊎}{\ensuremath{\mathnormal\uplus}}
% \newunicodechar{∗}{\ensuremath{\mathnormal\ast}}
\newunicodechar{✴}{\ensuremath{\mathnormal\ast}}
\newunicodechar{↦}{\ensuremath{\mathnormal\mapsto}}
\newunicodechar{≡}{\ensuremath{\mathnormal\equiv}}
\newunicodechar{∀}{\ensuremath{\mathnormal\forall}}
\newunicodechar{∙}{\ensuremath{\mathnormal\bullet}}
\newunicodechar{≣}{\ensuremath{\mathnormal\superequiv}}
\newunicodechar{▿}{\ensuremath{\mathnormal\triangledown}}
\newunicodechar{▸}{\raisebox{0.25ex}{\scaleobj{0.75}{\ensuremath{\mathnormal\blacktriangleright}}}}
\newunicodechar{∼}{\ensuremath{\mathnormal\sim}}
\newunicodechar{≤}{\ensuremath{\mathnormal\leq}}
\newunicodechar{↔}{\ensuremath{\mathnormal\leftrightarrow}}
\newunicodechar{⇔}{\ensuremath{\mathnormal\Leftrightarrow}}
\newunicodechar{⊂}{\ensuremath{\mathnormal\subset}}
\newunicodechar{∘}{\ensuremath{\mathnormal\circ}}
\newunicodechar{∃}{\ensuremath{\mathnormal\exists}}
\newunicodechar{↓}{\ensuremath{\mathnormal\downarrow}}
\newunicodechar{↑}{\ensuremath{\mathnormal\uparrow}}
\newunicodechar{⊔}{\ensuremath{\mathnormal\sqcup}}
\newunicodechar{∐}{\ensuremath{\mathnormal\amalg}}
\newunicodechar{⊢}{\ensuremath{\mathnormal\vdash}}
\newunicodechar{◇}{\ensuremath{\mathnormal\diamond}}
\newunicodechar{◆}{}
\newunicodechar{⊙}{\ensuremath{\mathnormal\odot}}
\newunicodechar{⊤}{\ensuremath{\mathnormal\top}}
\newunicodechar{⊥}{\ensuremath{\mathnormal\bot}}
\newunicodechar{∣}{\ensuremath{\mathnormal\mid}}
% \newunicodechar{‵}{\ensuremath{^\backprime}}
\newunicodechar{′}{\ensuremath{^\prime}}
\newunicodechar{″}{\ensuremath{^{\prime\prime}}}
\newunicodechar{‴}{\ensuremath{^{\prime\prime\prime}}}
\newunicodechar{⅋}{}
\newunicodechar{∅}{\ensuremath{\emptyset}}
\newunicodechar{≺}{\ensuremath{\mathnormal{\prec}}}
\newunicodechar{≼}{\ensuremath{\mathnormal{\preceq}}}
\newunicodechar{∩}{\ensuremath{\mathnormal{\cap}}}
\newunicodechar{⟪}{\ensuremath{\langle\kern-.2em\langle}}
\newunicodechar{⟫}{\ensuremath{\rangle\kern-.2em\rangle}}
\newunicodechar{⊆}{\ensuremath{\mathnormal{\subseteq}}}
\newunicodechar{⊇}{\ensuremath{\mathnormal{\supseteq}}}
\newunicodechar{▻}{\ensuremath{\mathnormal\vartriangleright}}
\newunicodechar{∷}{\ensuremath{::}}
\newunicodechar{►}{\ensuremath{\mathnormal{\blacktriangleright}}}
\newunicodechar{▹}{\ensuremath{\mathnormal{\scaleobj{0.75}{\triangleright}}}}
\newunicodechar{□}{\ensuremath{\mathnormal{\square}}}
\newunicodechar{⋯}{\ensuremath{\mathnormal{\cdots}}}
\newunicodechar{▣}{\ensuremath{\mathnormal{\ldots}}}
\newunicodechar{⋮}{\ensuremath{\mathnormal{\quad\quad\vdots}}}
\newunicodechar{∈}{\ensuremath{\mathnormal{\in}}}
\newunicodechar{⊑}{\ensuremath{\mathbin{\sqsubseteq}}}
\newunicodechar{𝓑}{\ensuremath{\mathnormal{\gg\!\!=}}}
\newunicodechar{𝓢}{\ensuremath{\mathnormal{\gg}}}
\newunicodechar{𝓒}{\ensuremath{\mathbf{\mathscr{C}}}}
\newunicodechar{𝓓}{\ensuremath{\mathbf{\mathscr{D}}}}
\newunicodechar{ℒ}{\ensuremath{\mathscr{L}}}
\newunicodechar{♯}{\ensuremath{\sharp}}
\newunicodechar{∎}{\ensuremath{\qed}}
\newunicodechar{↣}{\ensuremath{\rightarrowtail}}
\newunicodechar{⟶}{\ensuremath{\longrightarrow}}
\newunicodechar{–}{--}
\newunicodechar{◃}{\ensuremath{\mathnormal{\scaleobj{0.75}{\triangleleft}}}}
\newunicodechar{≟}{\ensuremath{\stackrel{?}{=}}}
\newunicodechar{⌊}{\ensuremath{\lfloor}}
\newunicodechar{⌋}{\ensuremath{\rfloor}}

%%
%% Greek
%%

\newunicodechar{φ}{\ensuremath{\phi}}
\newunicodechar{Φ}{\ensuremath{\Phi}}
\newunicodechar{ψ}{\ensuremath{\psi}}
\newunicodechar{μ}{\ensuremath{\mu}}
\newunicodechar{α}{\ensuremath{\alpha}}
\newunicodechar{β}{\ensuremath{\beta}}
\newunicodechar{σ}{\ensuremath{\sigma}}
\newunicodechar{ξ}{\ensuremath{\xi}}
\newunicodechar{Ξ}{\ensuremath{\Xi}}
\newunicodechar{λ}{\ensuremath{\lambda}}
\newunicodechar{ε}{\ensuremath{\epsilon}}
\newunicodechar{γ}{\ensuremath{\gamma}}
\newunicodechar{Σ}{\ensuremath{\Sigma}}
\newunicodechar{Δ}{\ensuremath{\Delta}}
\newunicodechar{Π}{\ensuremath{\Pi}}
\newunicodechar{Γ}{\ensuremath{\Gamma}}
\newunicodechar{η}{\ensuremath{\eta}}
\newunicodechar{ζ}{\ensuremath{\zeta}}
\newunicodechar{ν}{\ensuremath{\nu}}
\newunicodechar{δ}{\ensuremath{\delta}}


%%
%% subscript/superscript
%%

\newunicodechar{₀}{\ensuremath{_{0}}}
\newunicodechar{₁}{\ensuremath{_{1}}}
\newunicodechar{₂}{\ensuremath{_{2}}}
\newunicodechar{₃}{\ensuremath{_{3}}}
\newunicodechar{₄}{\ensuremath{_{4}}}
\newunicodechar{₅}{\ensuremath{_{5}}}
\newunicodechar{₆}{\ensuremath{_{6}}}
\newunicodechar{₇}{\ensuremath{_{7}}}
\newunicodechar{₈}{\ensuremath{_{8}}}
\newunicodechar{₉}{\ensuremath{_{9}}}
\newunicodechar{¹}{\ensuremath{^{1}}}
\newunicodechar{⁻}{$^{-}$}
\newunicodechar{ᴬ}{$^{A}$}
\newunicodechar{ᴮ}{$^{B}$}
\newunicodechar{ᴱ}{$^{E}$}
\newunicodechar{ᴴ}{$^{H}$}
\newunicodechar{ˣ}{$^{×}$}
\newunicodechar{ᵈ}{$^{d}$}
\newunicodechar{ᵘ}{$^{u}$}
\newunicodechar{ᶠ}{$^{F}$}
\newunicodechar{ⁱ}{$^{i}$}
\newunicodechar{ᵒ}{$^{o}$}
\newunicodechar{ˢ}{$^{s}$}
\newunicodechar{ˡ}{$^{l}$}
\newunicodechar{ʳ}{$^r$}
\newunicodechar{ᴰ}{$^D$}
\newunicodechar{ᵇ}{$^{b}$}
\newunicodechar{ᵐ}{$_m$}
\newunicodechar{ⁿ}{$_n$}
\newunicodechar{ₚ}{$_p$}
\newunicodechar{ₒ}{$_o$}
\newunicodechar{ₖ}{$_k$}
\newunicodechar{ₙ}{$_n$}
\newunicodechar{ₘ}{$_m$}
\newunicodechar{ᵛ}{$_v$}
\newunicodechar{ᵍ}{$_f$}
\newunicodechar{ᵢ}{$_i$}
\newunicodechar{ⱼ}{$_j$}
\newunicodechar{ₗ}{$_l$}
\newunicodechar{ᵣ}{$_r$}
\newunicodechar{ₛ}{$_s$}
\newunicodechar{ₜ}{$_t$}
\newunicodechar{ᶜ}{$^c$}
\newunicodechar{ₐ}{$_a$}
\newunicodechar{∶}{$:$}
\newunicodechar{̅}{$^{\textit{d}}$}

\newunicodechar{𝑡}{\textit{t}}
\newunicodechar{ℎ}{\textit{h}}
\newunicodechar{𝑟}{\textit{r}}
\newunicodechar{𝑜}{\textit{o}}
\newunicodechar{𝑤}{\textit{w}}
\newunicodechar{𝑐}{\textit{c}}
\newunicodechar{𝑎}{\textit{a}}
\newunicodechar{𝑡}{\textit{t}}

\newunicodechar{𝓑}{\ensuremath{\mathnormal{\gg\!\!=}}}

\newunicodechar{𝟘}{\ensuremath{\mathbb{0}}}
\newunicodechar{𝟙}{\ensuremath{\mathbb{1}}}
\newunicodechar{⅀}{\ensuremath{\mathbb{\Sigma}}}
\newunicodechar{ℿ}{\ensuremath{\mathbb{\Pi}}}
\newunicodechar{ℕ}{\ensuremath{\mathbb{N}}}


\pagestyle{plain} % for page numbers
%
\begin{document}
%
\title{Parsing Data-Dependent Grammars using Derivatives}
%
%\titlerunning{Abbreviated paper title}
% If the paper title is too long for the running head, you can set
% an abbreviated paper title here
%
\author{First Author\inst{1}\orcidID{0000-1111-2222-3333} \and
Second Author\inst{2,3}\orcidID{1111-2222-3333-4444} \and
Third Author\inst{3}\orcidID{2222--3333-4444-5555}}
%
\authorrunning{F. Author et al.}
% First names are abbreviated in the running head.
% If there are more than two authors, 'et al.' is used.
%
\institute{Princeton University, Princeton NJ 08544, USA \and
Springer Heidelberg, Tiergartenstr. 17, 69121 Heidelberg, Germany
\email{lncs@springer.com}\\
\url{http://www.springer.com/gp/computer-science/lncs} \and
ABC Institute, Rupert-Karls-University Heidelberg, Heidelberg, Germany\\
\email{\{abc,lncs\}@uni-heidelberg.de}}

\maketitle              % typeset the header of the contribution

\begin{abstract}
Parsing is the process of recovering structure from strings, an essential part of implementing programming languages.
Ambiguity, being able to parse a string in multiple different ways, has been a long-standing problem in parsing.
Recently, data-dependent grammars have emerged as a general, declarative, and efficient solution to this problem.
However, their current formalization in terms of automatons makes it difficult to reason about their behavior.

In this paper, we specify data-dependent grammars in type theory as predicates over strings.
Based on this specification, we derive a parser and formally prove properties about our specification which were previously only assumed.
In the future, our specification can guide the implementation of provably correct efficient parsers for data-dependent grammars.

\keywords{First keyword  \and Second keyword \and Another keyword.}
\end{abstract}

%
% the environments 'definition', 'lemma', 'proposition', 'corollary',
% 'remark', and 'example' are defined in the LLNCS documentclass as well.
%
\begin{code}[hide]%
\>[0]\AgdaSymbol{\{-\#}\AgdaSpace{}%
\AgdaKeyword{OPTIONS}\AgdaSpace{}%
\AgdaPragma{--cubical}\AgdaSpace{}%
\AgdaPragma{--guarded}\AgdaSpace{}%
\AgdaSymbol{\#-\}}\<%
\\
\>[0]\AgdaKeyword{module}\AgdaSpace{}%
\AgdaModule{1-introduction}\AgdaSpace{}%
\AgdaKeyword{where}\<%
\\
\>[0]\<%
\end{code}

\section{Introduction}

Parsing is the conversion of flat, human-readable text into a tree structure
that is easier for computers to manipulate.  As one of the central
pillars of compiler tooling since the 1960s, today almost every automated
transformation of computer programs requires a form of parsing.
Though it is such a mature research subject, it is still actively studied, for example the question of how to resolve ambiguities in context-free grammars \cite{one-parser-to-rule-them-all}. 

Recent work by Elliot uses interactive theorem provers to state simple specifications of languages and that proofs of desirable properties of these language specifications transfer easily to their parsers \cite{conal-languages}. Unfortunately, this work only considers regular languages which are not powerful enough to describe practical programming languages.

In this paper, we formalize context-free languages and show how to parse them, extending Elliot’s type theoretic approach to language specfication.  One of the main challenges is that the recursive nature of context-free languages does not map directly onto automated theorem provers as they do not support general recursion. We use a fuel-based approach to solve this problem.

We make the following concrete contributions:
\begin{itemize}
\item We extend Elliot's type theoretic formalization of regular languages to context-free languages.
\end{itemize}

For this paper we have chosen Agda as our type theory and interactive theorem prover. We believe our definitions should transfer easily to other theories and tools. This paper itself is a literate Agda file; all highlighted Agda code has been accepted by Agda's type checker, giving us a high confidence of correctness. Unfortunately, we are still working out the proof of three postulates in \cref{sec:cfg-parsing}. These are the only postulates that we have yet to prove.
\begin{code}[hide]%
\>[0]\AgdaComment{--\ Cannot\ be\ safe\ because\ we\ postulate\ funext}\<%
\\
\>[0]\AgdaComment{--\ \{-\#\ OPTIONS\ --safe\ \#-\}}\<%
\\
\>[0]\AgdaKeyword{module}\AgdaSpace{}%
\AgdaModule{2-overview}\AgdaSpace{}%
\AgdaKeyword{where}\<%
\\
%
\\[\AgdaEmptyExtraSkip]%
\>[0]\AgdaKeyword{open}\AgdaSpace{}%
\AgdaKeyword{import}\AgdaSpace{}%
\AgdaModule{Agda.Primitive}\AgdaSpace{}%
\AgdaKeyword{renaming}\AgdaSpace{}%
\AgdaSymbol{(}\AgdaPrimitive{Set}\AgdaSpace{}%
\AgdaSymbol{to}\AgdaSpace{}%
\AgdaPrimitive{Type}\AgdaSpace{}%
\AgdaSymbol{;}\AgdaSpace{}%
\AgdaPrimitive{Setω}\AgdaSpace{}%
\AgdaSymbol{to}\AgdaSpace{}%
\AgdaPrimitive{Typeω}\AgdaSymbol{)}\<%
\\
%
\\[\AgdaEmptyExtraSkip]%
\>[0]\AgdaKeyword{import}\AgdaSpace{}%
\AgdaModule{Function.Properties.Equivalence}\AgdaSpace{}%
\AgdaSymbol{as}\AgdaSpace{}%
\AgdaModule{⇔}\<%
\\
\>[0]\AgdaKeyword{import}\AgdaSpace{}%
\AgdaModule{Data.Bool}\AgdaSpace{}%
\AgdaSymbol{as}\AgdaSpace{}%
\AgdaModule{Bool}\<%
\\
\>[0]\AgdaKeyword{open}\AgdaSpace{}%
\AgdaKeyword{import}\AgdaSpace{}%
\AgdaModule{Data.Bool}\AgdaSpace{}%
\AgdaKeyword{using}\AgdaSpace{}%
\AgdaSymbol{(}\AgdaDatatype{Bool}\AgdaSpace{}%
\AgdaSymbol{;}\AgdaSpace{}%
\AgdaInductiveConstructor{true}\AgdaSpace{}%
\AgdaSymbol{;}\AgdaSpace{}%
\AgdaInductiveConstructor{false}\AgdaSymbol{)}\<%
\\
\>[0]\AgdaKeyword{open}\AgdaSpace{}%
\AgdaKeyword{import}\AgdaSpace{}%
\AgdaModule{Data.Char}\AgdaSpace{}%
\AgdaKeyword{using}\AgdaSpace{}%
\AgdaSymbol{(}\AgdaPostulate{Char}\AgdaSpace{}%
\AgdaSymbol{;}\AgdaSpace{}%
\AgdaOperator{\AgdaFunction{\AgdaUnderscore{}≟\AgdaUnderscore{}}}\AgdaSymbol{)}\<%
\\
\>[0]\AgdaKeyword{open}\AgdaSpace{}%
\AgdaKeyword{import}\AgdaSpace{}%
\AgdaModule{Data.List}\AgdaSpace{}%
\AgdaSymbol{as}\AgdaSpace{}%
\AgdaModule{List}\AgdaSpace{}%
\AgdaKeyword{hiding}\AgdaSpace{}%
\AgdaSymbol{(}\AgdaFunction{foldl}\AgdaSymbol{)}\<%
\\
\>[0]\AgdaKeyword{open}\AgdaSpace{}%
\AgdaKeyword{import}\AgdaSpace{}%
\AgdaModule{Data.Empty}\<%
\\
\>[0]\AgdaKeyword{open}\AgdaSpace{}%
\AgdaKeyword{import}\AgdaSpace{}%
\AgdaModule{Data.Product}\<%
\\
\>[0]\AgdaKeyword{open}\AgdaSpace{}%
\AgdaKeyword{import}\AgdaSpace{}%
\AgdaModule{Data.Sum}\AgdaSpace{}%
\AgdaSymbol{as}\AgdaSpace{}%
\AgdaModule{Sum}\<%
\\
\>[0]\AgdaKeyword{open}\AgdaSpace{}%
\AgdaKeyword{import}\AgdaSpace{}%
\AgdaModule{Data.Unit}\AgdaSpace{}%
\AgdaKeyword{hiding}\AgdaSpace{}%
\AgdaSymbol{(}\AgdaOperator{\AgdaFunction{\AgdaUnderscore{}≟\AgdaUnderscore{}}}\AgdaSymbol{)}\<%
\\
\>[0]\AgdaKeyword{open}\AgdaSpace{}%
\AgdaKeyword{import}\AgdaSpace{}%
\AgdaModule{Relation.Nullary.Decidable}\AgdaSpace{}%
\AgdaSymbol{as}\AgdaSpace{}%
\AgdaModule{Dec}\AgdaSpace{}%
\AgdaKeyword{hiding}\AgdaSpace{}%
\AgdaSymbol{(}\AgdaFunction{from-yes}\AgdaSpace{}%
\AgdaSymbol{;}\AgdaSpace{}%
\AgdaFunction{from-no}\AgdaSymbol{)}\<%
\\
\>[0]\AgdaKeyword{open}\AgdaSpace{}%
\AgdaKeyword{import}\AgdaSpace{}%
\AgdaModule{Relation.Nullary.Reflects}\AgdaSpace{}%
\AgdaKeyword{using}\AgdaSpace{}%
\AgdaSymbol{(}\AgdaInductiveConstructor{ofʸ}\AgdaSpace{}%
\AgdaSymbol{;}\AgdaSpace{}%
\AgdaInductiveConstructor{ofⁿ}\AgdaSymbol{)}\<%
\\
\>[0]\AgdaKeyword{open}\AgdaSpace{}%
\AgdaKeyword{import}\AgdaSpace{}%
\AgdaModule{Level}\AgdaSpace{}%
\AgdaKeyword{hiding}\AgdaSpace{}%
\AgdaSymbol{(}\AgdaPrimitive{zero}\AgdaSpace{}%
\AgdaSymbol{;}\AgdaSpace{}%
\AgdaPrimitive{suc}\AgdaSymbol{)}\<%
\\
\>[0]\AgdaKeyword{open}\AgdaSpace{}%
\AgdaKeyword{import}\AgdaSpace{}%
\AgdaModule{Relation.Binary.PropositionalEquality}\<%
\\
\>[0]\AgdaKeyword{open}\AgdaSpace{}%
\AgdaKeyword{import}\AgdaSpace{}%
\AgdaModule{Function}\<%
\\
\>[0]\AgdaKeyword{open}\AgdaSpace{}%
\AgdaKeyword{import}\AgdaSpace{}%
\AgdaModule{Data.Fin}\AgdaSpace{}%
\AgdaKeyword{hiding}\AgdaSpace{}%
\AgdaSymbol{(}\AgdaOperator{\AgdaFunction{\AgdaUnderscore{}≟\AgdaUnderscore{}}}\AgdaSymbol{)}\<%
\\
\>[0]\AgdaKeyword{open}\AgdaSpace{}%
\AgdaKeyword{import}\AgdaSpace{}%
\AgdaModule{Data.Nat}\AgdaSpace{}%
\AgdaKeyword{hiding}\AgdaSpace{}%
\AgdaSymbol{(}\AgdaOperator{\AgdaFunction{\AgdaUnderscore{}≟\AgdaUnderscore{}}}\AgdaSymbol{)}\<%
\\
\>[0]\AgdaKeyword{open}\AgdaSpace{}%
\AgdaKeyword{import}\AgdaSpace{}%
\AgdaModule{Relation.Nullary.Negation}\<%
\\
\>[0]\AgdaKeyword{import}\AgdaSpace{}%
\AgdaModule{Data.String}\AgdaSpace{}%
\AgdaSymbol{as}\AgdaSpace{}%
\AgdaModule{String}\<%
\\
\>[0]\AgdaKeyword{open}\AgdaSpace{}%
\AgdaKeyword{import}\AgdaSpace{}%
\AgdaModule{Agda.Builtin.FromString}\<%
\\
%
\\[\AgdaEmptyExtraSkip]%
\>[0]\AgdaFunction{transport}\AgdaSpace{}%
\AgdaSymbol{:}\AgdaSpace{}%
\AgdaSymbol{∀\{}\AgdaBound{A}\AgdaSpace{}%
\AgdaBound{B}\AgdaSpace{}%
\AgdaSymbol{:}\AgdaSpace{}%
\AgdaPrimitive{Type}\AgdaSymbol{\}}\AgdaSpace{}%
\AgdaSymbol{→}\AgdaSpace{}%
\AgdaBound{A}\AgdaSpace{}%
\AgdaOperator{\AgdaDatatype{≡}}\AgdaSpace{}%
\AgdaBound{B}\AgdaSpace{}%
\AgdaSymbol{→}\AgdaSpace{}%
\AgdaBound{A}\AgdaSpace{}%
\AgdaSymbol{→}\AgdaSpace{}%
\AgdaBound{B}\<%
\\
\>[0]\AgdaFunction{transport}\AgdaSpace{}%
\AgdaInductiveConstructor{refl}\AgdaSpace{}%
\AgdaBound{x}\AgdaSpace{}%
\AgdaSymbol{=}\AgdaSpace{}%
\AgdaBound{x}\<%
\\
%
\\[\AgdaEmptyExtraSkip]%
\>[0]\AgdaFunction{≡→⇔}\AgdaSpace{}%
\AgdaSymbol{:}\AgdaSpace{}%
\AgdaSymbol{∀}\AgdaSpace{}%
\AgdaSymbol{\{}\AgdaBound{A}\AgdaSpace{}%
\AgdaBound{B}\AgdaSpace{}%
\AgdaSymbol{:}\AgdaSpace{}%
\AgdaPrimitive{Type}\AgdaSymbol{\}}\AgdaSpace{}%
\AgdaSymbol{→}\AgdaSpace{}%
\AgdaBound{A}\AgdaSpace{}%
\AgdaOperator{\AgdaDatatype{≡}}\AgdaSpace{}%
\AgdaBound{B}\AgdaSpace{}%
\AgdaSymbol{→}\AgdaSpace{}%
\AgdaBound{A}\AgdaSpace{}%
\AgdaOperator{\AgdaFunction{⇔}}\AgdaSpace{}%
\AgdaBound{B}\<%
\\
\>[0]\AgdaFunction{≡→⇔}\AgdaSpace{}%
\AgdaInductiveConstructor{refl}\AgdaSpace{}%
\AgdaSymbol{=}\AgdaSpace{}%
\AgdaFunction{⇔.refl}\<%
\\
%
\\[\AgdaEmptyExtraSkip]%
\>[0]\AgdaFunction{lift⊎₂}\AgdaSpace{}%
\AgdaSymbol{:}\AgdaSpace{}%
\AgdaSymbol{∀\{}\AgdaBound{A}\AgdaSpace{}%
\AgdaBound{B}\AgdaSpace{}%
\AgdaBound{C}\AgdaSpace{}%
\AgdaBound{D}\AgdaSpace{}%
\AgdaSymbol{:}\AgdaSpace{}%
\AgdaPrimitive{Type}\AgdaSymbol{\}}\AgdaSpace{}%
\AgdaSymbol{→}\AgdaSpace{}%
\AgdaSymbol{(}\AgdaBound{A}\AgdaSpace{}%
\AgdaSymbol{→}\AgdaSpace{}%
\AgdaBound{B}\AgdaSpace{}%
\AgdaSymbol{→}\AgdaSpace{}%
\AgdaBound{C}\AgdaSymbol{)}\AgdaSpace{}%
\AgdaSymbol{→}\AgdaSpace{}%
\AgdaBound{A}\AgdaSpace{}%
\AgdaOperator{\AgdaDatatype{⊎}}\AgdaSpace{}%
\AgdaBound{D}\AgdaSpace{}%
\AgdaSymbol{→}\AgdaSpace{}%
\AgdaBound{B}\AgdaSpace{}%
\AgdaOperator{\AgdaDatatype{⊎}}\AgdaSpace{}%
\AgdaBound{D}\AgdaSpace{}%
\AgdaSymbol{→}\AgdaSpace{}%
\AgdaBound{C}\AgdaSpace{}%
\AgdaOperator{\AgdaDatatype{⊎}}\AgdaSpace{}%
\AgdaBound{D}\<%
\\
\>[0]\AgdaFunction{lift⊎₂}\AgdaSpace{}%
\AgdaBound{f}\AgdaSpace{}%
\AgdaSymbol{(}\AgdaInductiveConstructor{inj₁}\AgdaSpace{}%
\AgdaBound{x}\AgdaSymbol{)}\AgdaSpace{}%
\AgdaSymbol{(}\AgdaInductiveConstructor{inj₁}\AgdaSpace{}%
\AgdaBound{y}\AgdaSymbol{)}\AgdaSpace{}%
\AgdaSymbol{=}\AgdaSpace{}%
\AgdaInductiveConstructor{inj₁}\AgdaSpace{}%
\AgdaSymbol{(}\AgdaBound{f}\AgdaSpace{}%
\AgdaBound{x}\AgdaSpace{}%
\AgdaBound{y}\AgdaSymbol{)}\<%
\\
\>[0]\AgdaFunction{lift⊎₂}\AgdaSpace{}%
\AgdaSymbol{\AgdaUnderscore{}}\AgdaSpace{}%
\AgdaSymbol{(}\AgdaInductiveConstructor{inj₁}\AgdaSpace{}%
\AgdaSymbol{\AgdaUnderscore{})}\AgdaSpace{}%
\AgdaSymbol{(}\AgdaInductiveConstructor{inj₂}\AgdaSpace{}%
\AgdaBound{y}\AgdaSymbol{)}\AgdaSpace{}%
\AgdaSymbol{=}\AgdaSpace{}%
\AgdaInductiveConstructor{inj₂}\AgdaSpace{}%
\AgdaBound{y}\<%
\\
\>[0]\AgdaFunction{lift⊎₂}\AgdaSpace{}%
\AgdaSymbol{\AgdaUnderscore{}}\AgdaSpace{}%
\AgdaSymbol{(}\AgdaInductiveConstructor{inj₂}\AgdaSpace{}%
\AgdaBound{x}\AgdaSymbol{)}\AgdaSpace{}%
\AgdaSymbol{\AgdaUnderscore{}}\AgdaSpace{}%
\AgdaSymbol{=}\AgdaSpace{}%
\AgdaInductiveConstructor{inj₂}\AgdaSpace{}%
\AgdaBound{x}\<%
\\
%
\\[\AgdaEmptyExtraSkip]%
\>[0]\AgdaFunction{String}\AgdaSpace{}%
\AgdaSymbol{:}\AgdaSpace{}%
\AgdaPrimitive{Type}\<%
\\
\>[0]\AgdaFunction{String}\AgdaSpace{}%
\AgdaSymbol{=}\AgdaSpace{}%
\AgdaDatatype{List}\AgdaSpace{}%
\AgdaPostulate{Char}\<%
\\
\>[0]\AgdaKeyword{instance}\<%
\\
\>[0][@{}l@{\AgdaIndent{0}}]%
\>[2]\AgdaFunction{string}\AgdaSpace{}%
\AgdaSymbol{:}\AgdaSpace{}%
\AgdaRecord{IsString}\AgdaSpace{}%
\AgdaFunction{String}\<%
\\
%
\>[2]\AgdaField{IsString.Constraint}\AgdaSpace{}%
\AgdaFunction{string}\AgdaSpace{}%
\AgdaSymbol{\AgdaUnderscore{}}\AgdaSpace{}%
\AgdaSymbol{=}\AgdaSpace{}%
\AgdaRecord{⊤}\<%
\\
%
\>[2]\AgdaField{IsString.fromString}\AgdaSpace{}%
\AgdaFunction{string}\AgdaSpace{}%
\AgdaBound{xs}\AgdaSpace{}%
\AgdaSymbol{=}\AgdaSpace{}%
\AgdaPrimitive{String.toList}\AgdaSpace{}%
\AgdaBound{xs}\<%
\\
%
\\[\AgdaEmptyExtraSkip]%
\>[0]\AgdaFunction{foldl}\AgdaSpace{}%
\AgdaSymbol{:}\AgdaSpace{}%
\AgdaSymbol{∀}\AgdaSpace{}%
\AgdaSymbol{\{}\AgdaBound{ℓ₁}\AgdaSpace{}%
\AgdaBound{ℓ₂}\AgdaSymbol{\}}\AgdaSpace{}%
\AgdaSymbol{\{}\AgdaBound{A}\AgdaSpace{}%
\AgdaSymbol{:}\AgdaSpace{}%
\AgdaPrimitive{Type}\AgdaSpace{}%
\AgdaBound{ℓ₁}\AgdaSymbol{\}}\AgdaSpace{}%
\AgdaSymbol{\{}\AgdaBound{B}\AgdaSpace{}%
\AgdaSymbol{:}\AgdaSpace{}%
\AgdaPrimitive{Type}\AgdaSpace{}%
\AgdaBound{ℓ₂}\AgdaSymbol{\}}\AgdaSpace{}%
\AgdaSymbol{→}\AgdaSpace{}%
\AgdaSymbol{(}\AgdaBound{A}\AgdaSpace{}%
\AgdaSymbol{→}\AgdaSpace{}%
\AgdaBound{B}\AgdaSpace{}%
\AgdaSymbol{→}\AgdaSpace{}%
\AgdaBound{B}\AgdaSymbol{)}\AgdaSpace{}%
\AgdaSymbol{→}\AgdaSpace{}%
\AgdaBound{B}\AgdaSpace{}%
\AgdaSymbol{→}\AgdaSpace{}%
\AgdaDatatype{List}\AgdaSpace{}%
\AgdaBound{A}\AgdaSpace{}%
\AgdaSymbol{→}\AgdaSpace{}%
\AgdaBound{B}\<%
\\
\>[0]\AgdaFunction{foldl}\AgdaSpace{}%
\AgdaBound{k}\AgdaSpace{}%
\AgdaBound{z}\AgdaSpace{}%
\AgdaInductiveConstructor{[]}\AgdaSpace{}%
\AgdaSymbol{=}\AgdaSpace{}%
\AgdaBound{z}\<%
\\
\>[0]\AgdaFunction{foldl}\AgdaSpace{}%
\AgdaBound{k}\AgdaSpace{}%
\AgdaBound{z}\AgdaSpace{}%
\AgdaSymbol{(}\AgdaBound{c}\AgdaSpace{}%
\AgdaOperator{\AgdaInductiveConstructor{∷}}\AgdaSpace{}%
\AgdaBound{w}\AgdaSymbol{)}\AgdaSpace{}%
\AgdaSymbol{=}\AgdaSpace{}%
\AgdaFunction{foldl}\AgdaSpace{}%
\AgdaBound{k}\AgdaSpace{}%
\AgdaSymbol{(}\AgdaBound{k}\AgdaSpace{}%
\AgdaBound{c}\AgdaSpace{}%
\AgdaBound{z}\AgdaSymbol{)}\AgdaSpace{}%
\AgdaBound{w}\<%
\\
%
\\[\AgdaEmptyExtraSkip]%
\>[0]\AgdaKeyword{variable}\<%
\\
\>[0][@{}l@{\AgdaIndent{0}}]%
\>[4]\AgdaGeneralizable{ℓ}\AgdaSpace{}%
\AgdaGeneralizable{ℓ′}\AgdaSpace{}%
\AgdaSymbol{:}\AgdaSpace{}%
\AgdaPostulate{Level}\<%
\\
%
\>[4]\AgdaGeneralizable{A}\AgdaSpace{}%
\AgdaSymbol{:}\AgdaSpace{}%
\AgdaPrimitive{Type}\AgdaSpace{}%
\AgdaGeneralizable{ℓ}\<%
\\
%
\>[4]\AgdaGeneralizable{c}\AgdaSpace{}%
\AgdaGeneralizable{c'}\AgdaSpace{}%
\AgdaSymbol{:}\AgdaSpace{}%
\AgdaPostulate{Char}\<%
\\
%
\>[4]\AgdaGeneralizable{w}\AgdaSpace{}%
\AgdaSymbol{:}\AgdaSpace{}%
\AgdaFunction{String}\<%
\end{code}

\section{Languages and Parsers}

In this section, we introduce background information, namely how we define languages, basic language combinators, and parsers. Our exposition follows Elliot~\cite{conal-languages}. In \cref{sec:context-free}, we extend these concepts to context free languages.

\subsection{Languages}

We define languages as being functions from strings to types.\footnote{We use \af{Type} as a synonym for Agda's \af{Set} to avoid confusion.}
\begin{code}[hide]%
\>[0]\AgdaFunction{Lang}\AgdaSpace{}%
\AgdaSymbol{:}\AgdaSpace{}%
\AgdaPrimitive{Type₁}\<%
\end{code}
\begin{code}%
\>[0]\AgdaFunction{Lang}\AgdaSpace{}%
\AgdaSymbol{=}\AgdaSpace{}%
\AgdaFunction{String}\AgdaSpace{}%
\AgdaSymbol{→}\AgdaSpace{}%
\AgdaPrimitive{Type}\<%
\end{code}
The result type can be thought of as the type of proofs that the string is in the language.
\begin{remark}
Note that a language may admit multiple different proofs for the same string. That is an important difference between the type theoretic approach and the more common set theoretic approach, which models languages as sets of strings.
\end{remark}
This is a broad definition of what a language is; it includes languages that are outside the class of context-free languages. 
\begin{example}\label{ex:non-context-free}
The language $a^n b^n c^n$ can be specified as follows:
\begin{code}[hide]%
\>[0]\AgdaFunction{repeat}\AgdaSpace{}%
\AgdaSymbol{:}\AgdaSpace{}%
\AgdaDatatype{ℕ}\AgdaSpace{}%
\AgdaSymbol{→}\AgdaSpace{}%
\AgdaPostulate{Char}\AgdaSpace{}%
\AgdaSymbol{→}\AgdaSpace{}%
\AgdaFunction{String}\<%
\\
\>[0]\AgdaFunction{repeat}\AgdaSpace{}%
\AgdaInductiveConstructor{zero}\AgdaSpace{}%
\AgdaSymbol{\AgdaUnderscore{}}\AgdaSpace{}%
\AgdaSymbol{=}\AgdaSpace{}%
\AgdaInductiveConstructor{[]}\<%
\\
\>[0]\AgdaFunction{repeat}\AgdaSpace{}%
\AgdaSymbol{(}\AgdaInductiveConstructor{suc}\AgdaSpace{}%
\AgdaBound{n}\AgdaSymbol{)}\AgdaSpace{}%
\AgdaBound{c}\AgdaSpace{}%
\AgdaSymbol{=}\AgdaSpace{}%
\AgdaBound{c}\AgdaSpace{}%
\AgdaOperator{\AgdaInductiveConstructor{∷}}\AgdaSpace{}%
\AgdaFunction{repeat}\AgdaSpace{}%
\AgdaBound{n}\AgdaSpace{}%
\AgdaBound{c}\<%
\end{code}
\begin{code}%
\>[0]\AgdaFunction{abc}\AgdaSpace{}%
\AgdaSymbol{:}\AgdaSpace{}%
\AgdaFunction{Lang}\<%
\\
\>[0]\AgdaFunction{abc}\AgdaSpace{}%
\AgdaBound{w}\AgdaSpace{}%
\AgdaSymbol{=}\AgdaSpace{}%
\AgdaFunction{Σ[}\AgdaSpace{}%
\AgdaBound{n}\AgdaSpace{}%
\AgdaFunction{∈}\AgdaSpace{}%
\AgdaDatatype{ℕ}\AgdaSpace{}%
\AgdaFunction{]}\AgdaSpace{}%
\AgdaBound{w}\AgdaSpace{}%
\AgdaOperator{\AgdaDatatype{≡}}\AgdaSpace{}%
\AgdaFunction{repeat}\AgdaSpace{}%
\AgdaBound{n}\AgdaSpace{}%
\AgdaString{'a'}\AgdaSpace{}%
\AgdaOperator{\AgdaFunction{++}}\AgdaSpace{}%
\AgdaFunction{repeat}\AgdaSpace{}%
\AgdaBound{n}\AgdaSpace{}%
\AgdaString{'b'}\AgdaSpace{}%
\AgdaOperator{\AgdaFunction{++}}\AgdaSpace{}%
\AgdaFunction{repeat}\AgdaSpace{}%
\AgdaBound{n}\AgdaSpace{}%
\AgdaString{'c'}\<%
\end{code}
We can show that the string $aabbcc$ is in this language by choosing $n$ to be $2$, from which the required equality follows by reflexivity after normalization:
\begin{code}%
\>[0]\AgdaFunction{aabbcc}\AgdaSpace{}%
\AgdaSymbol{:}\AgdaSpace{}%
\AgdaFunction{abc}\AgdaSpace{}%
\AgdaString{"aabbcc"}\<%
\\
\>[0]\AgdaFunction{aabbcc}\AgdaSpace{}%
\AgdaSymbol{=}\AgdaSpace{}%
\AgdaNumber{2}\AgdaSpace{}%
\AgdaOperator{\AgdaInductiveConstructor{,}}\AgdaSpace{}%
\AgdaInductiveConstructor{refl}\<%
\end{code}
\end{example}
\cref{ex:non-context-free} shows that it is possible to specify languages and prove that certain strings are in those languages, but for practical applications we do not want to be burdened with writing such proofs ourselves. The compiler should be able to decide whether or not your program is valid by itself.

\begin{itemize}
\item Agda is too powerful: it can specify undecidable languages \jr{do I need to give an example?}
\item So, we need to define a simpler language which still supports all the features we need.
\end{itemize}

\subsection{Basic Language Combinators}

Let's start with a simple example: POSIX file system permissions. These are usually summarized using the characters `r', `w', and `x' if the permissions are granted, or `-' in place of the corresponding character if the permission is denied. For example the string ``r-x'' indicates that read and execute permissions are granted, but the write permission is denied. The full language can be expressed using the following BNF grammar:\jr{cite: BNF}

\begin{grammar}
<permissions>  ::= <read> <write> <execute>

<read>         ::= `-' | `r'

<write>        ::= `-' | `w'

<execute>      ::= `-' | `x'
\end{grammar}

\begin{code}[hide]%
\>[0]\AgdaKeyword{variable}\AgdaSpace{}%
\AgdaGeneralizable{ℒ}\AgdaSpace{}%
\AgdaGeneralizable{ℒ₁}\AgdaSpace{}%
\AgdaGeneralizable{ℒ₂}\AgdaSpace{}%
\AgdaSymbol{:}\AgdaSpace{}%
\AgdaFunction{Lang}\<%
\\
%
\\[\AgdaEmptyExtraSkip]%
\>[0]\AgdaKeyword{module}\AgdaSpace{}%
\AgdaModule{◇}\AgdaSpace{}%
\AgdaKeyword{where}\<%
\end{code}

\begin{figure}
\begin{minipage}{.63\textwidth}
\begin{code}%
\>[0][@{}l@{\AgdaIndent{1}}]%
\>[4]\AgdaOperator{\AgdaFunction{`\AgdaUnderscore{}}}\AgdaSpace{}%
\AgdaSymbol{:}\AgdaSpace{}%
\AgdaPostulate{Char}\AgdaSpace{}%
\AgdaSymbol{→}\AgdaSpace{}%
\AgdaFunction{Lang}\<%
\\
%
\>[4]\AgdaSymbol{(}\AgdaOperator{\AgdaFunction{`}}\AgdaSpace{}%
\AgdaBound{c}\AgdaSymbol{)}\AgdaSpace{}%
\AgdaBound{w}\AgdaSpace{}%
\AgdaSymbol{=}\AgdaSpace{}%
\AgdaBound{w}\AgdaSpace{}%
\AgdaOperator{\AgdaDatatype{≡}}\AgdaSpace{}%
\AgdaBound{c}\AgdaSpace{}%
\AgdaOperator{\AgdaInductiveConstructor{∷}}\AgdaSpace{}%
\AgdaInductiveConstructor{[]}\<%
\end{code}
\begin{code}%
%
\>[4]\AgdaOperator{\AgdaFunction{\AgdaUnderscore{}∪\AgdaUnderscore{}}}\AgdaSpace{}%
\AgdaSymbol{:}\AgdaSpace{}%
\AgdaFunction{Lang}\AgdaSpace{}%
\AgdaSymbol{→}\AgdaSpace{}%
\AgdaFunction{Lang}\AgdaSpace{}%
\AgdaSymbol{→}\AgdaSpace{}%
\AgdaFunction{Lang}\<%
\\
%
\>[4]\AgdaSymbol{(}\AgdaBound{P}\AgdaSpace{}%
\AgdaOperator{\AgdaFunction{∪}}\AgdaSpace{}%
\AgdaBound{Q}\AgdaSymbol{)}\AgdaSpace{}%
\AgdaBound{w}\AgdaSpace{}%
\AgdaSymbol{=}\AgdaSpace{}%
\AgdaBound{P}\AgdaSpace{}%
\AgdaBound{w}\AgdaSpace{}%
\AgdaOperator{\AgdaDatatype{⊎}}\AgdaSpace{}%
\AgdaBound{Q}\AgdaSpace{}%
\AgdaBound{w}\<%
\end{code}
\begin{code}%
%
\>[4]\AgdaOperator{\AgdaFunction{\AgdaUnderscore{}∗\AgdaUnderscore{}}}\AgdaSpace{}%
\AgdaSymbol{:}\AgdaSpace{}%
\AgdaFunction{Lang}\AgdaSpace{}%
\AgdaSymbol{→}\AgdaSpace{}%
\AgdaFunction{Lang}\AgdaSpace{}%
\AgdaSymbol{→}\AgdaSpace{}%
\AgdaFunction{Lang}\<%
\\
%
\>[4]\AgdaSymbol{(}\AgdaBound{P}\AgdaSpace{}%
\AgdaOperator{\AgdaFunction{∗}}\AgdaSpace{}%
\AgdaBound{Q}\AgdaSymbol{)}\AgdaSpace{}%
\AgdaBound{w}\AgdaSpace{}%
\AgdaSymbol{=}\AgdaSpace{}%
\AgdaFunction{∃[}\AgdaSpace{}%
\AgdaBound{u}\AgdaSpace{}%
\AgdaFunction{]}\AgdaSpace{}%
\AgdaFunction{∃[}\AgdaSpace{}%
\AgdaBound{v}\AgdaSpace{}%
\AgdaFunction{]}\AgdaSpace{}%
\AgdaBound{w}\AgdaSpace{}%
\AgdaOperator{\AgdaDatatype{≡}}\AgdaSpace{}%
\AgdaBound{u}\AgdaSpace{}%
\AgdaOperator{\AgdaFunction{++}}\AgdaSpace{}%
\AgdaBound{v}\AgdaSpace{}%
\AgdaOperator{\AgdaFunction{×}}\AgdaSpace{}%
\AgdaBound{P}\AgdaSpace{}%
\AgdaBound{u}\AgdaSpace{}%
\AgdaOperator{\AgdaFunction{×}}\AgdaSpace{}%
\AgdaBound{Q}\AgdaSpace{}%
\AgdaBound{v}\<%
\end{code}
\end{minipage}
\begin{minipage}{.36\textwidth}
\begin{code}%
%
\>[4]\AgdaFunction{∅}\AgdaSpace{}%
\AgdaSymbol{:}\AgdaSpace{}%
\AgdaFunction{Lang}\<%
\\
%
\>[4]\AgdaFunction{∅}\AgdaSpace{}%
\AgdaSymbol{\AgdaUnderscore{}}\AgdaSpace{}%
\AgdaSymbol{=}\AgdaSpace{}%
\AgdaFunction{⊥}\<%
\end{code}
\begin{code}%
%
\>[4]\AgdaFunction{ε}\AgdaSpace{}%
\AgdaSymbol{:}\AgdaSpace{}%
\AgdaFunction{Lang}\<%
\\
%
\>[4]\AgdaFunction{ε}\AgdaSpace{}%
\AgdaBound{w}\AgdaSpace{}%
\AgdaSymbol{=}\AgdaSpace{}%
\AgdaBound{w}\AgdaSpace{}%
\AgdaOperator{\AgdaDatatype{≡}}\AgdaSpace{}%
\AgdaInductiveConstructor{[]}\<%
\end{code}
\begin{code}%
%
\>[4]\AgdaOperator{\AgdaFunction{\AgdaUnderscore{}·\AgdaUnderscore{}}}\AgdaSpace{}%
\AgdaSymbol{:}\AgdaSpace{}%
\AgdaPrimitive{Type}\AgdaSpace{}%
\AgdaSymbol{→}\AgdaSpace{}%
\AgdaFunction{Lang}\AgdaSpace{}%
\AgdaSymbol{→}\AgdaSpace{}%
\AgdaFunction{Lang}\<%
\\
%
\>[4]\AgdaSymbol{(}\AgdaBound{A}\AgdaSpace{}%
\AgdaOperator{\AgdaFunction{·}}\AgdaSpace{}%
\AgdaBound{P}\AgdaSymbol{)}\AgdaSpace{}%
\AgdaBound{w}\AgdaSpace{}%
\AgdaSymbol{=}\AgdaSpace{}%
\AgdaBound{A}\AgdaSpace{}%
\AgdaOperator{\AgdaFunction{×}}\AgdaSpace{}%
\AgdaBound{P}\AgdaSpace{}%
\AgdaBound{w}\<%
\end{code}
\end{minipage}
\begin{code}[hide]%
%
\>[4]\AgdaKeyword{infix}\AgdaSpace{}%
\AgdaNumber{22}\AgdaSpace{}%
\AgdaOperator{\AgdaFunction{`\AgdaUnderscore{}}}\<%
\\
%
\>[4]\AgdaKeyword{infixr}\AgdaSpace{}%
\AgdaNumber{21}\AgdaSpace{}%
\AgdaOperator{\AgdaFunction{\AgdaUnderscore{}∗\AgdaUnderscore{}}}\<%
\\
%
\>[4]\AgdaKeyword{infix}\AgdaSpace{}%
\AgdaNumber{21}\AgdaSpace{}%
\AgdaOperator{\AgdaFunction{\AgdaUnderscore{}·\AgdaUnderscore{}}}\<%
\\
%
\>[4]\AgdaKeyword{infixr}\AgdaSpace{}%
\AgdaNumber{20}\AgdaSpace{}%
\AgdaOperator{\AgdaFunction{\AgdaUnderscore{}∪\AgdaUnderscore{}}}\<%
\end{code}
\caption{Basic language combinators.}\label{fig:combinators}
\end{figure}

This grammar uses three important features: sequencing, choice, and matching character literals. We can define these features are combinators in Agda as shown in \cref{fig:combinators} and use them to write our permissions grammar as follows:

\begin{code}[hide]%
%
\>[4]\AgdaFunction{permissions}\AgdaSpace{}%
\AgdaFunction{read}\AgdaSpace{}%
\AgdaFunction{write}\AgdaSpace{}%
\AgdaFunction{execute}\AgdaSpace{}%
\AgdaSymbol{:}\AgdaSpace{}%
\AgdaFunction{Lang}\<%
\end{code}
\begin{code}%
%
\>[4]\AgdaFunction{permissions}%
\>[17]\AgdaSymbol{=}\AgdaSpace{}%
\AgdaFunction{read}\AgdaSpace{}%
\AgdaOperator{\AgdaFunction{∗}}\AgdaSpace{}%
\AgdaFunction{write}\AgdaSpace{}%
\AgdaOperator{\AgdaFunction{∗}}\AgdaSpace{}%
\AgdaFunction{execute}\<%
\\
%
\>[4]\AgdaFunction{read}%
\>[17]\AgdaSymbol{=}\AgdaSpace{}%
\AgdaOperator{\AgdaFunction{`}}\AgdaSpace{}%
\AgdaString{'-'}\AgdaSpace{}%
\AgdaOperator{\AgdaFunction{∪}}\AgdaSpace{}%
\AgdaOperator{\AgdaFunction{`}}\AgdaSpace{}%
\AgdaString{'r'}\<%
\\
%
\>[4]\AgdaFunction{write}%
\>[17]\AgdaSymbol{=}\AgdaSpace{}%
\AgdaOperator{\AgdaFunction{`}}\AgdaSpace{}%
\AgdaString{'-'}\AgdaSpace{}%
\AgdaOperator{\AgdaFunction{∪}}\AgdaSpace{}%
\AgdaOperator{\AgdaFunction{`}}\AgdaSpace{}%
\AgdaString{'w'}\<%
\\
%
\>[4]\AgdaFunction{execute}%
\>[17]\AgdaSymbol{=}\AgdaSpace{}%
\AgdaOperator{\AgdaFunction{`}}\AgdaSpace{}%
\AgdaString{'-'}\AgdaSpace{}%
\AgdaOperator{\AgdaFunction{∪}}\AgdaSpace{}%
\AgdaOperator{\AgdaFunction{`}}\AgdaSpace{}%
\AgdaString{'x'}\<%
\end{code}

\subsection{Parsers}

We want to write a program which can prove for us that a given string is in the language. What should this program return for strings that are not in the language? We want to make sure our program does find a proof if it exists, so if it does not exist then we want a proof that the string is not in the language. We can capture this using a type called \af{Dec} from the Agda standard library. It can be defined as follows:

\begin{code}%
%
\>[4]\AgdaKeyword{data}\AgdaSpace{}%
\AgdaDatatype{◂Dec}\AgdaSpace{}%
\AgdaSymbol{(}\AgdaBound{A}\AgdaSpace{}%
\AgdaSymbol{:}\AgdaSpace{}%
\AgdaPrimitive{Type}\AgdaSymbol{)}\AgdaSpace{}%
\AgdaSymbol{:}\AgdaSpace{}%
\AgdaPrimitive{Type}\AgdaSpace{}%
\AgdaKeyword{where}\<%
\\
\>[4][@{}l@{\AgdaIndent{0}}]%
\>[8]\AgdaInductiveConstructor{◂yes}\AgdaSpace{}%
\AgdaSymbol{:}\AgdaSpace{}%
\AgdaBound{A}\AgdaSpace{}%
\AgdaSymbol{→}\AgdaSpace{}%
\AgdaDatatype{◂Dec}\AgdaSpace{}%
\AgdaBound{A}\<%
\\
%
\>[8]\AgdaInductiveConstructor{◂no}\AgdaSpace{}%
\AgdaSymbol{:}\AgdaSpace{}%
\AgdaOperator{\AgdaFunction{¬}}\AgdaSpace{}%
\AgdaBound{A}\AgdaSpace{}%
\AgdaSymbol{→}\AgdaSpace{}%
\AgdaDatatype{◂Dec}\AgdaSpace{}%
\AgdaBound{A}\<%
\end{code}

A parser for a language, then, is a program which can tell us whether any given string is in the language or not.

\begin{code}%
%
\>[4]\AgdaFunction{Parser}\AgdaSpace{}%
\AgdaSymbol{:}\AgdaSpace{}%
\AgdaFunction{Lang}\AgdaSpace{}%
\AgdaSymbol{→}\AgdaSpace{}%
\AgdaPrimitive{Set}\<%
\\
%
\>[4]\AgdaFunction{Parser}\AgdaSpace{}%
\AgdaBound{P}\AgdaSpace{}%
\AgdaSymbol{=}\AgdaSpace{}%
\AgdaSymbol{(}\AgdaBound{w}\AgdaSpace{}%
\AgdaSymbol{:}\AgdaSpace{}%
\AgdaFunction{String}\AgdaSymbol{)}\AgdaSpace{}%
\AgdaSymbol{→}\AgdaSpace{}%
\AgdaRecord{Dec}\AgdaSpace{}%
\AgdaSymbol{(}\AgdaBound{P}\AgdaSpace{}%
\AgdaBound{w}\AgdaSymbol{)}\<%
\end{code}

\begin{remark}
Readers familiar with Haskell might see similarity between this type and the type \verb|String -> Maybe a|, which is one way to implement parser combinators (although usually the return type is \verb|Maybe (a, String)| giving parsers the freedom to consume only a prefix of the input string and return the rest). The differences are that the result of our $\af{Parser}$ type depends on the language specification and input string, and that a failure carries with it a proof that the string cannot be part of the language. This allows us to separate the specification of our language from the implementation while ensuring correctness.
\end{remark}

\begin{remark}
Note that the \af{Dec} type only requires our parsers to produce a single result; it does not have to exhaustively list all possible ways to parse the input string. In Haskell, one might write \verb|String -> [(a, String)]|\jr{cite: monadic parser combinators}, which allows a parser to return multiple results but still does not enforce exhaustiveness. Instead, we could use:\jr{This should be explained in more detail}
%
\begin{itemize}
\item completely unique account of enumeration.
\item bijection with $\af{Fin}~\ab{n}$ for some $\ab{n}$ or $\af{Nat}$.
\end{itemize}
%
In this paper, however, we use $\af{Dec}$ to keep the presentation simple.
\end{remark}

To construct a parser for our permissions language, we start by defining parsers for each of the language combinators. Let us start by considering the character combinator. If the given string is empty or has more than one character, it can never be in a language formed by one character. If the string does consist of only one character, then it is in the language if that character is the same as from the language specification. In Agda, we can write such a parser for characters as follows:

\begin{code}%
%
\>[4]\AgdaOperator{\AgdaFunction{◂`-parse\AgdaUnderscore{}}}\AgdaSpace{}%
\AgdaSymbol{:}\AgdaSpace{}%
\AgdaSymbol{(}\AgdaBound{x}\AgdaSpace{}%
\AgdaSymbol{:}\AgdaSpace{}%
\AgdaPostulate{Char}\AgdaSymbol{)}\AgdaSpace{}%
\AgdaSymbol{→}\AgdaSpace{}%
\AgdaFunction{Parser}\AgdaSpace{}%
\AgdaSymbol{(}\AgdaOperator{\AgdaFunction{`}}\AgdaSpace{}%
\AgdaBound{x}\AgdaSymbol{)}\<%
\\
%
\>[4]\AgdaSymbol{(}\AgdaOperator{\AgdaFunction{◂`-parse}}\AgdaSpace{}%
\AgdaSymbol{\AgdaUnderscore{})}\AgdaSpace{}%
\AgdaInductiveConstructor{[]}\AgdaSpace{}%
\AgdaSymbol{=}\AgdaSpace{}%
\AgdaInductiveConstructor{no}\AgdaSpace{}%
\AgdaSymbol{λ}\AgdaSpace{}%
\AgdaSymbol{()}\<%
\\
%
\>[4]\AgdaSymbol{(}\AgdaOperator{\AgdaFunction{◂`-parse}}\AgdaSpace{}%
\AgdaBound{x}\AgdaSymbol{)}\AgdaSpace{}%
\AgdaSymbol{(}\AgdaBound{c}\AgdaSpace{}%
\AgdaOperator{\AgdaInductiveConstructor{∷}}\AgdaSpace{}%
\AgdaInductiveConstructor{[]}\AgdaSymbol{)}\AgdaSpace{}%
\AgdaSymbol{=}\AgdaSpace{}%
\AgdaFunction{Dec.map}\AgdaSpace{}%
\AgdaSymbol{(}\AgdaFunction{mk⇔}\AgdaSpace{}%
\AgdaSymbol{(λ}\AgdaSpace{}%
\AgdaSymbol{\{}\AgdaSpace{}%
\AgdaInductiveConstructor{refl}\AgdaSpace{}%
\AgdaSymbol{→}\AgdaSpace{}%
\AgdaInductiveConstructor{refl}\AgdaSpace{}%
\AgdaSymbol{\})}\AgdaSpace{}%
\AgdaSymbol{(λ}\AgdaSpace{}%
\AgdaSymbol{\{}\AgdaSpace{}%
\AgdaInductiveConstructor{refl}\AgdaSpace{}%
\AgdaSymbol{→}\AgdaSpace{}%
\AgdaInductiveConstructor{refl}\AgdaSpace{}%
\AgdaSymbol{\}))}\AgdaSpace{}%
\AgdaSymbol{(}\AgdaBound{c}\AgdaSpace{}%
\AgdaOperator{\AgdaFunction{≟}}\AgdaSpace{}%
\AgdaBound{x}\AgdaSymbol{)}\<%
\\
%
\>[4]\AgdaSymbol{(}\AgdaOperator{\AgdaFunction{◂`-parse}}\AgdaSpace{}%
\AgdaSymbol{\AgdaUnderscore{})}\AgdaSpace{}%
\AgdaSymbol{(\AgdaUnderscore{}}\AgdaSpace{}%
\AgdaOperator{\AgdaInductiveConstructor{∷}}\AgdaSpace{}%
\AgdaSymbol{\AgdaUnderscore{}}\AgdaSpace{}%
\AgdaOperator{\AgdaInductiveConstructor{∷}}\AgdaSpace{}%
\AgdaSymbol{\AgdaUnderscore{})}\AgdaSpace{}%
\AgdaSymbol{=}\AgdaSpace{}%
\AgdaInductiveConstructor{no}\AgdaSpace{}%
\AgdaSymbol{λ}\AgdaSpace{}%
\AgdaSymbol{()}\<%
\end{code}

This is a correct implementation of a parser for languages that consist of a single character, but the implementation is hard to read and does not give much insight. Instead, we can factor this parser into two cases: the empty string case and the case where the string has at least one character. We call the former nullability and use the greek character $ν$ to signify it, and we call the latter derivative and use the greek character $δ$ to signify it. \Cref{fig:null-delta} shows how these cases can be defined and how they relate to the basic combinators. These properties motivate the introduction of three new basic combinators: guards $\af{_·_}$, the language consisting of only the empty string $\af{ε}$, and the empty language $\af{∅}$.\jr{This does not motivate the split into $ν$ and $δ$ well enough. Also, the new combinators can be motivated more clearly.}

\begin{code}[hide]%
%
\>[4]\AgdaKeyword{variable}\AgdaSpace{}%
\AgdaGeneralizable{L}\AgdaSpace{}%
\AgdaGeneralizable{P}\AgdaSpace{}%
\AgdaGeneralizable{Q}\AgdaSpace{}%
\AgdaSymbol{:}\AgdaSpace{}%
\AgdaFunction{Lang}\<%
\end{code}

\begin{code}[hide]%
%
\>[4]\AgdaFunction{⊥-dec}\AgdaSpace{}%
\AgdaSymbol{:}\AgdaSpace{}%
\AgdaRecord{Dec}\AgdaSpace{}%
\AgdaFunction{⊥}\<%
\\
%
\>[4]\AgdaFunction{⊥-dec}\AgdaSpace{}%
\AgdaSymbol{=}\AgdaSpace{}%
\AgdaInductiveConstructor{no}\AgdaSpace{}%
\AgdaSymbol{λ}\AgdaSpace{}%
\AgdaSymbol{()}\<%
\\
%
\\[\AgdaEmptyExtraSkip]%
%
\>[4]\AgdaFunction{⊤-dec}\AgdaSpace{}%
\AgdaSymbol{:}\AgdaSpace{}%
\AgdaRecord{Dec}\AgdaSpace{}%
\AgdaRecord{⊤}\<%
\\
%
\>[4]\AgdaFunction{⊤-dec}\AgdaSpace{}%
\AgdaSymbol{=}\AgdaSpace{}%
\AgdaInductiveConstructor{yes}\AgdaSpace{}%
\AgdaInductiveConstructor{tt}\<%
\end{code}

\begin{figure}
\begin{minipage}{0.45\textwidth}
\begin{code}[hide]%
%
\>[4]\AgdaFunction{ν}\AgdaSpace{}%
\AgdaSymbol{:}\AgdaSpace{}%
\AgdaSymbol{\{}\AgdaBound{ℓ}\AgdaSpace{}%
\AgdaSymbol{:}\AgdaSpace{}%
\AgdaPostulate{Level}\AgdaSymbol{\}}\AgdaSpace{}%
\AgdaSymbol{\{}\AgdaBound{P}\AgdaSpace{}%
\AgdaSymbol{:}\AgdaSpace{}%
\AgdaFunction{String}\AgdaSpace{}%
\AgdaSymbol{→}\AgdaSpace{}%
\AgdaPrimitive{Type}\AgdaSpace{}%
\AgdaBound{ℓ}\AgdaSymbol{\}}\AgdaSpace{}%
\AgdaSymbol{→}\AgdaSpace{}%
\AgdaSymbol{((}\AgdaBound{w}\AgdaSpace{}%
\AgdaSymbol{:}\AgdaSpace{}%
\AgdaFunction{String}\AgdaSymbol{)}\AgdaSpace{}%
\AgdaSymbol{→}\AgdaSpace{}%
\AgdaBound{P}\AgdaSpace{}%
\AgdaBound{w}\AgdaSymbol{)}\AgdaSpace{}%
\AgdaSymbol{→}\AgdaSpace{}%
\AgdaBound{P}\AgdaSpace{}%
\AgdaInductiveConstructor{[]}\<%
\\
%
\>[4]\AgdaOperator{\AgdaFunction{\AgdaUnderscore{}◂⇔\AgdaUnderscore{}}}\AgdaSpace{}%
\AgdaSymbol{:}\AgdaSpace{}%
\AgdaPrimitive{Set}\AgdaSpace{}%
\AgdaSymbol{→}\AgdaSpace{}%
\AgdaPrimitive{Set}\AgdaSpace{}%
\AgdaSymbol{→}\AgdaSpace{}%
\AgdaPrimitive{Set}\<%
\end{code}
\begin{code}%
%
\>[4]\AgdaFunction{ν}\AgdaSpace{}%
\AgdaBound{P}\AgdaSpace{}%
\AgdaSymbol{=}\AgdaSpace{}%
\AgdaBound{P}\AgdaSpace{}%
\AgdaInductiveConstructor{[]}\<%
\end{code}
\begin{code}%
%
\>[4]\AgdaBound{A}\AgdaSpace{}%
\AgdaOperator{\AgdaFunction{◂⇔}}\AgdaSpace{}%
\AgdaBound{B}\AgdaSpace{}%
\AgdaSymbol{=}\AgdaSpace{}%
\AgdaSymbol{(}\AgdaBound{A}\AgdaSpace{}%
\AgdaSymbol{→}\AgdaSpace{}%
\AgdaBound{B}\AgdaSymbol{)}\AgdaSpace{}%
\AgdaOperator{\AgdaFunction{×}}\AgdaSpace{}%
\AgdaSymbol{(}\AgdaBound{B}\AgdaSpace{}%
\AgdaSymbol{→}\AgdaSpace{}%
\AgdaBound{A}\AgdaSymbol{)}\<%
\end{code}
\begin{code}%
%
\>[4]\AgdaFunction{ν∅}%
\>[8]\AgdaSymbol{:}\AgdaSpace{}%
\AgdaFunction{⊥}%
\>[23]\AgdaOperator{\AgdaFunction{⇔}}\AgdaSpace{}%
\AgdaFunction{ν}\AgdaSpace{}%
\AgdaFunction{∅}\<%
\\
%
\>[4]\AgdaFunction{νε}%
\>[8]\AgdaSymbol{:}\AgdaSpace{}%
\AgdaRecord{⊤}%
\>[23]\AgdaOperator{\AgdaFunction{⇔}}\AgdaSpace{}%
\AgdaFunction{ν}\AgdaSpace{}%
\AgdaFunction{ε}\<%
\\
%
\>[4]\AgdaFunction{ν·}%
\>[8]\AgdaSymbol{:}\AgdaSpace{}%
\AgdaSymbol{(}\AgdaGeneralizable{A}\AgdaSpace{}%
\AgdaOperator{\AgdaFunction{×}}\AgdaSpace{}%
\AgdaFunction{ν}\AgdaSpace{}%
\AgdaGeneralizable{P}\AgdaSymbol{)}%
\>[23]\AgdaOperator{\AgdaFunction{⇔}}\AgdaSpace{}%
\AgdaFunction{ν}\AgdaSpace{}%
\AgdaSymbol{(}\AgdaGeneralizable{A}\AgdaSpace{}%
\AgdaOperator{\AgdaFunction{·}}\AgdaSpace{}%
\AgdaGeneralizable{P}\AgdaSymbol{)}\<%
\\
%
\>[4]\AgdaFunction{ν`}%
\>[8]\AgdaSymbol{:}\AgdaSpace{}%
\AgdaFunction{⊥}%
\>[23]\AgdaOperator{\AgdaFunction{⇔}}\AgdaSpace{}%
\AgdaFunction{ν}\AgdaSpace{}%
\AgdaSymbol{(}\AgdaOperator{\AgdaFunction{`}}\AgdaSpace{}%
\AgdaGeneralizable{c'}\AgdaSymbol{)}\<%
\\
%
\>[4]\AgdaFunction{ν∪}%
\>[8]\AgdaSymbol{:}\AgdaSpace{}%
\AgdaSymbol{(}\AgdaFunction{ν}\AgdaSpace{}%
\AgdaGeneralizable{P}\AgdaSpace{}%
\AgdaOperator{\AgdaDatatype{⊎}}\AgdaSpace{}%
\AgdaFunction{ν}\AgdaSpace{}%
\AgdaGeneralizable{Q}\AgdaSymbol{)}%
\>[23]\AgdaOperator{\AgdaFunction{⇔}}\AgdaSpace{}%
\AgdaFunction{ν}\AgdaSpace{}%
\AgdaSymbol{(}\AgdaGeneralizable{P}\AgdaSpace{}%
\AgdaOperator{\AgdaFunction{∪}}\AgdaSpace{}%
\AgdaGeneralizable{Q}\AgdaSymbol{)}\<%
\\
%
\>[4]\AgdaFunction{ν∗}%
\>[8]\AgdaSymbol{:}\AgdaSpace{}%
\AgdaSymbol{(}\AgdaFunction{ν}\AgdaSpace{}%
\AgdaGeneralizable{P}\AgdaSpace{}%
\AgdaOperator{\AgdaFunction{×}}\AgdaSpace{}%
\AgdaFunction{ν}\AgdaSpace{}%
\AgdaGeneralizable{Q}\AgdaSymbol{)}%
\>[23]\AgdaOperator{\AgdaFunction{⇔}}\AgdaSpace{}%
\AgdaFunction{ν}\AgdaSpace{}%
\AgdaSymbol{(}\AgdaGeneralizable{P}\AgdaSpace{}%
\AgdaOperator{\AgdaFunction{∗}}\AgdaSpace{}%
\AgdaGeneralizable{Q}\AgdaSymbol{)}\<%
\\
\>[0]\<%
\end{code}
\end{minipage}
\begin{minipage}{0.54\textwidth}
\begin{code}[hide]%
\>[0][@{}l@{\AgdaIndent{1}}]%
\>[4]\AgdaFunction{δ}\AgdaSpace{}%
\AgdaSymbol{:}\AgdaSpace{}%
\AgdaSymbol{\{}\AgdaBound{ℓ}\AgdaSpace{}%
\AgdaSymbol{:}\AgdaSpace{}%
\AgdaPostulate{Level}\AgdaSymbol{\}}\AgdaSpace{}%
\AgdaSymbol{\{}\AgdaBound{P}\AgdaSpace{}%
\AgdaSymbol{:}\AgdaSpace{}%
\AgdaFunction{String}\AgdaSpace{}%
\AgdaSymbol{→}\AgdaSpace{}%
\AgdaPrimitive{Type}\AgdaSpace{}%
\AgdaBound{ℓ}\AgdaSymbol{\}}\AgdaSpace{}%
\AgdaSymbol{(}\AgdaBound{c}\AgdaSpace{}%
\AgdaSymbol{:}\AgdaSpace{}%
\AgdaPostulate{Char}\AgdaSymbol{)}\AgdaSpace{}%
\AgdaSymbol{→}\AgdaSpace{}%
\AgdaSymbol{((}\AgdaBound{w}\AgdaSpace{}%
\AgdaSymbol{:}\AgdaSpace{}%
\AgdaFunction{String}\AgdaSymbol{)}\AgdaSpace{}%
\AgdaSymbol{→}\AgdaSpace{}%
\AgdaBound{P}\AgdaSpace{}%
\AgdaBound{w}\AgdaSymbol{)}\AgdaSpace{}%
\AgdaSymbol{→}\AgdaSpace{}%
\AgdaSymbol{((}\AgdaBound{w}\AgdaSpace{}%
\AgdaSymbol{:}\AgdaSpace{}%
\AgdaFunction{String}\AgdaSymbol{)}\AgdaSpace{}%
\AgdaSymbol{→}\AgdaSpace{}%
\AgdaBound{P}\AgdaSpace{}%
\AgdaSymbol{(}\AgdaBound{c}\AgdaSpace{}%
\AgdaOperator{\AgdaInductiveConstructor{∷}}\AgdaSpace{}%
\AgdaBound{w}\AgdaSymbol{))}\<%
\\
%
\>[4]\AgdaOperator{\AgdaFunction{\AgdaUnderscore{}⟺\AgdaUnderscore{}}}\AgdaSpace{}%
\AgdaSymbol{:}\AgdaSpace{}%
\AgdaFunction{Lang}\AgdaSpace{}%
\AgdaSymbol{→}\AgdaSpace{}%
\AgdaFunction{Lang}\AgdaSpace{}%
\AgdaSymbol{→}\AgdaSpace{}%
\AgdaPrimitive{Set}\<%
\end{code}
\begin{code}%
%
\>[4]\AgdaSymbol{(}\AgdaFunction{δ}\AgdaSpace{}%
\AgdaBound{c}\AgdaSpace{}%
\AgdaBound{P}\AgdaSymbol{)}\AgdaSpace{}%
\AgdaBound{w}\AgdaSpace{}%
\AgdaSymbol{=}\AgdaSpace{}%
\AgdaBound{P}\AgdaSpace{}%
\AgdaSymbol{(}\AgdaBound{c}\AgdaSpace{}%
\AgdaOperator{\AgdaInductiveConstructor{∷}}\AgdaSpace{}%
\AgdaBound{w}\AgdaSymbol{)}\<%
\end{code}
\begin{code}%
%
\>[4]\AgdaBound{P}\AgdaSpace{}%
\AgdaOperator{\AgdaFunction{⟺}}\AgdaSpace{}%
\AgdaBound{Q}\AgdaSpace{}%
\AgdaSymbol{=}\AgdaSpace{}%
\AgdaSymbol{∀}\AgdaSpace{}%
\AgdaSymbol{\{}\AgdaBound{w}\AgdaSymbol{\}}\AgdaSpace{}%
\AgdaSymbol{→}\AgdaSpace{}%
\AgdaBound{P}\AgdaSpace{}%
\AgdaBound{w}\AgdaSpace{}%
\AgdaOperator{\AgdaFunction{⇔}}\AgdaSpace{}%
\AgdaBound{Q}\AgdaSpace{}%
\AgdaBound{w}\<%
\end{code}
\begin{code}%
%
\>[4]\AgdaFunction{δ∅}%
\>[8]\AgdaSymbol{:}\AgdaSpace{}%
\AgdaFunction{∅}%
\>[27]\AgdaOperator{\AgdaFunction{⟺}}\AgdaSpace{}%
\AgdaFunction{δ}\AgdaSpace{}%
\AgdaGeneralizable{c}\AgdaSpace{}%
\AgdaFunction{∅}\<%
\\
%
\>[4]\AgdaFunction{δε}%
\>[8]\AgdaSymbol{:}\AgdaSpace{}%
\AgdaFunction{∅}%
\>[27]\AgdaOperator{\AgdaFunction{⟺}}\AgdaSpace{}%
\AgdaFunction{δ}\AgdaSpace{}%
\AgdaGeneralizable{c}\AgdaSpace{}%
\AgdaFunction{ε}\<%
\\
%
\>[4]\AgdaFunction{δ·}%
\>[8]\AgdaSymbol{:}\AgdaSpace{}%
\AgdaSymbol{(}\AgdaGeneralizable{A}\AgdaSpace{}%
\AgdaOperator{\AgdaFunction{·}}\AgdaSpace{}%
\AgdaFunction{δ}\AgdaSpace{}%
\AgdaGeneralizable{c}\AgdaSpace{}%
\AgdaGeneralizable{P}\AgdaSymbol{)}%
\>[27]\AgdaOperator{\AgdaFunction{⟺}}\AgdaSpace{}%
\AgdaFunction{δ}\AgdaSpace{}%
\AgdaGeneralizable{c}\AgdaSpace{}%
\AgdaSymbol{(}\AgdaGeneralizable{A}\AgdaSpace{}%
\AgdaOperator{\AgdaFunction{·}}\AgdaSpace{}%
\AgdaGeneralizable{P}\AgdaSymbol{)}\<%
\\
%
\>[4]\AgdaFunction{δ`}%
\>[8]\AgdaSymbol{:}\AgdaSpace{}%
\AgdaSymbol{((}\AgdaGeneralizable{c}\AgdaSpace{}%
\AgdaOperator{\AgdaDatatype{≡}}\AgdaSpace{}%
\AgdaGeneralizable{c'}\AgdaSymbol{)}\AgdaSpace{}%
\AgdaOperator{\AgdaFunction{·}}\AgdaSpace{}%
\AgdaFunction{ε}\AgdaSymbol{)}%
\>[27]\AgdaOperator{\AgdaFunction{⟺}}\AgdaSpace{}%
\AgdaFunction{δ}\AgdaSpace{}%
\AgdaGeneralizable{c}\AgdaSpace{}%
\AgdaSymbol{(}\AgdaOperator{\AgdaFunction{`}}\AgdaSpace{}%
\AgdaGeneralizable{c'}\AgdaSymbol{)}\<%
\\
%
\>[4]\AgdaFunction{δ∪}%
\>[8]\AgdaSymbol{:}\AgdaSpace{}%
\AgdaSymbol{(}\AgdaFunction{δ}\AgdaSpace{}%
\AgdaGeneralizable{c}\AgdaSpace{}%
\AgdaGeneralizable{P}\AgdaSpace{}%
\AgdaOperator{\AgdaFunction{∪}}\AgdaSpace{}%
\AgdaFunction{δ}\AgdaSpace{}%
\AgdaGeneralizable{c}\AgdaSpace{}%
\AgdaGeneralizable{Q}\AgdaSymbol{)}%
\>[27]\AgdaOperator{\AgdaFunction{⟺}}\AgdaSpace{}%
\AgdaFunction{δ}\AgdaSpace{}%
\AgdaGeneralizable{c}\AgdaSpace{}%
\AgdaSymbol{(}\AgdaGeneralizable{P}\AgdaSpace{}%
\AgdaOperator{\AgdaFunction{∪}}\AgdaSpace{}%
\AgdaGeneralizable{Q}\AgdaSymbol{)}\<%
\\
%
\>[4]\AgdaFunction{δ∗}%
\>[8]\AgdaSymbol{:}\AgdaSpace{}%
\AgdaSymbol{(}\AgdaFunction{ν}\AgdaSpace{}%
\AgdaGeneralizable{P}\AgdaSpace{}%
\AgdaOperator{\AgdaFunction{·}}\AgdaSpace{}%
\AgdaFunction{δ}\AgdaSpace{}%
\AgdaGeneralizable{c}\AgdaSpace{}%
\AgdaGeneralizable{Q}\AgdaSpace{}%
\AgdaOperator{\AgdaFunction{∪}}\AgdaSpace{}%
\AgdaFunction{δ}%
\>[689I]\AgdaGeneralizable{c}\AgdaSpace{}%
\AgdaGeneralizable{P}\AgdaSpace{}%
\AgdaOperator{\AgdaFunction{∗}}\AgdaSpace{}%
\AgdaGeneralizable{Q}\AgdaSymbol{)}\<%
\\
\>[.][@{}l@{}]\<[689I]%
\>[27]\AgdaOperator{\AgdaFunction{⟺}}\AgdaSpace{}%
\AgdaFunction{δ}\AgdaSpace{}%
\AgdaGeneralizable{c}\AgdaSpace{}%
\AgdaSymbol{(}\AgdaGeneralizable{P}\AgdaSpace{}%
\AgdaOperator{\AgdaFunction{∗}}\AgdaSpace{}%
\AgdaGeneralizable{Q}\AgdaSymbol{)}\<%
\end{code}
\end{minipage}
\caption{Nullability, derivatives, and how they relate to the basic combinators.}\label{fig:null-delta}
\end{figure}

\begin{code}[hide]%
%
\>[4]\AgdaFunction{ν∅}\AgdaSpace{}%
\AgdaSymbol{=}\AgdaSpace{}%
\AgdaFunction{⇔.refl}\<%
\\
%
\>[4]\AgdaFunction{δ∅}\AgdaSpace{}%
\AgdaSymbol{=}\AgdaSpace{}%
\AgdaFunction{⇔.refl}\<%
\end{code}

\begin{code}[hide]%
%
\>[4]\AgdaFunction{∅-parse}\AgdaSpace{}%
\AgdaSymbol{:}\AgdaSpace{}%
\AgdaFunction{Parser}\AgdaSpace{}%
\AgdaFunction{∅}\<%
\\
%
\>[4]\AgdaFunction{∅-parse}\AgdaSpace{}%
\AgdaInductiveConstructor{[]}%
\>[21]\AgdaSymbol{=}\AgdaSpace{}%
\AgdaFunction{Dec.map}\AgdaSpace{}%
\AgdaFunction{ν∅}\AgdaSpace{}%
\AgdaFunction{⊥-dec}\<%
\\
%
\>[4]\AgdaFunction{∅-parse}\AgdaSpace{}%
\AgdaSymbol{(}\AgdaBound{c}\AgdaSpace{}%
\AgdaOperator{\AgdaInductiveConstructor{∷}}\AgdaSpace{}%
\AgdaBound{w}\AgdaSymbol{)}%
\>[21]\AgdaSymbol{=}\AgdaSpace{}%
\AgdaFunction{Dec.map}\AgdaSpace{}%
\AgdaSymbol{(}\AgdaFunction{δ∅}\AgdaSpace{}%
\AgdaSymbol{\{}\AgdaBound{c}\AgdaSymbol{\}}\AgdaSpace{}%
\AgdaSymbol{\{}\AgdaBound{w}\AgdaSymbol{\})}\AgdaSpace{}%
\AgdaSymbol{(}\AgdaFunction{∅-parse}\AgdaSpace{}%
\AgdaBound{w}\AgdaSymbol{)}\<%
\\
%
\\[\AgdaEmptyExtraSkip]%
%
\>[4]\AgdaFunction{νε}\AgdaSpace{}%
\AgdaSymbol{=}\AgdaSpace{}%
\AgdaFunction{mk⇔}\AgdaSpace{}%
\AgdaSymbol{(λ}\AgdaSpace{}%
\AgdaSymbol{\{}\AgdaSpace{}%
\AgdaInductiveConstructor{tt}\AgdaSpace{}%
\AgdaSymbol{→}\AgdaSpace{}%
\AgdaInductiveConstructor{refl}\AgdaSpace{}%
\AgdaSymbol{\})}\AgdaSpace{}%
\AgdaSymbol{(λ}\AgdaSpace{}%
\AgdaSymbol{\{}\AgdaSpace{}%
\AgdaInductiveConstructor{refl}\AgdaSpace{}%
\AgdaSymbol{→}\AgdaSpace{}%
\AgdaInductiveConstructor{tt}\AgdaSpace{}%
\AgdaSymbol{\})}\<%
\\
%
\>[4]\AgdaFunction{δε}\AgdaSpace{}%
\AgdaSymbol{=}\AgdaSpace{}%
\AgdaFunction{mk⇔}\AgdaSpace{}%
\AgdaSymbol{(λ}\AgdaSpace{}%
\AgdaSymbol{())}\AgdaSpace{}%
\AgdaSymbol{(λ}\AgdaSpace{}%
\AgdaSymbol{())}\<%
\\
%
\\[\AgdaEmptyExtraSkip]%
%
\>[4]\AgdaFunction{ε-parse}\AgdaSpace{}%
\AgdaSymbol{:}\AgdaSpace{}%
\AgdaFunction{Parser}\AgdaSpace{}%
\AgdaFunction{ε}\<%
\\
%
\>[4]\AgdaFunction{ε-parse}\AgdaSpace{}%
\AgdaInductiveConstructor{[]}%
\>[21]\AgdaSymbol{=}\AgdaSpace{}%
\AgdaFunction{Dec.map}\AgdaSpace{}%
\AgdaFunction{νε}\AgdaSpace{}%
\AgdaFunction{⊤-dec}\<%
\\
%
\>[4]\AgdaFunction{ε-parse}\AgdaSpace{}%
\AgdaSymbol{(\AgdaUnderscore{}}\AgdaSpace{}%
\AgdaOperator{\AgdaInductiveConstructor{∷}}\AgdaSpace{}%
\AgdaBound{w}\AgdaSymbol{)}%
\>[21]\AgdaSymbol{=}\AgdaSpace{}%
\AgdaFunction{Dec.map}\AgdaSpace{}%
\AgdaFunction{δε}\AgdaSpace{}%
\AgdaSymbol{(}\AgdaFunction{∅-parse}\AgdaSpace{}%
\AgdaBound{w}\AgdaSymbol{)}\<%
\\
%
\\[\AgdaEmptyExtraSkip]%
%
\>[4]\AgdaFunction{ν·}\AgdaSpace{}%
\AgdaSymbol{=}\AgdaSpace{}%
\AgdaFunction{⇔.refl}\<%
\\
%
\>[4]\AgdaFunction{δ·}\AgdaSpace{}%
\AgdaSymbol{=}\AgdaSpace{}%
\AgdaFunction{⇔.refl}\<%
\\
%
\\[\AgdaEmptyExtraSkip]%
%
\>[4]\AgdaOperator{\AgdaFunction{\AgdaUnderscore{}·-parse\AgdaUnderscore{}}}\AgdaSpace{}%
\AgdaSymbol{:}\AgdaSpace{}%
\AgdaSymbol{(}\AgdaBound{x}\AgdaSpace{}%
\AgdaSymbol{:}\AgdaSpace{}%
\AgdaRecord{Dec}\AgdaSpace{}%
\AgdaGeneralizable{A}\AgdaSymbol{)}\AgdaSpace{}%
\AgdaSymbol{→}\AgdaSpace{}%
\AgdaFunction{Parser}\AgdaSpace{}%
\AgdaGeneralizable{P}\AgdaSpace{}%
\AgdaSymbol{→}\AgdaSpace{}%
\AgdaFunction{Parser}\AgdaSpace{}%
\AgdaSymbol{(}\AgdaGeneralizable{A}\AgdaSpace{}%
\AgdaOperator{\AgdaFunction{·}}\AgdaSpace{}%
\AgdaGeneralizable{P}\AgdaSymbol{)}\<%
\\
%
\>[4]\AgdaOperator{\AgdaFunction{\AgdaUnderscore{}·-parse\AgdaUnderscore{}}}\AgdaSpace{}%
\AgdaSymbol{\{}\AgdaArgument{P}\AgdaSpace{}%
\AgdaSymbol{=}\AgdaSpace{}%
\AgdaBound{P}\AgdaSymbol{\}}\AgdaSpace{}%
\AgdaBound{x}\AgdaSpace{}%
\AgdaBound{φ}\AgdaSpace{}%
\AgdaInductiveConstructor{[]}%
\>[35]\AgdaSymbol{=}\AgdaSpace{}%
\AgdaFunction{Dec.map}\AgdaSpace{}%
\AgdaSymbol{(}\AgdaFunction{ν·}\AgdaSpace{}%
\AgdaSymbol{\{}\AgdaArgument{P}\AgdaSpace{}%
\AgdaSymbol{=}\AgdaSpace{}%
\AgdaBound{P}\AgdaSymbol{\})}\AgdaSpace{}%
\AgdaSymbol{(}\AgdaBound{x}\AgdaSpace{}%
\AgdaOperator{\AgdaFunction{×-dec}}\AgdaSpace{}%
\AgdaSymbol{(}\AgdaFunction{ν}\AgdaSpace{}%
\AgdaBound{φ}\AgdaSymbol{))}\<%
\\
%
\>[4]\AgdaOperator{\AgdaFunction{\AgdaUnderscore{}·-parse\AgdaUnderscore{}}}\AgdaSpace{}%
\AgdaSymbol{\{}\AgdaArgument{P}\AgdaSpace{}%
\AgdaSymbol{=}\AgdaSpace{}%
\AgdaBound{P}\AgdaSymbol{\}}\AgdaSpace{}%
\AgdaBound{x}\AgdaSpace{}%
\AgdaBound{φ}\AgdaSpace{}%
\AgdaSymbol{(}\AgdaBound{c}\AgdaSpace{}%
\AgdaOperator{\AgdaInductiveConstructor{∷}}\AgdaSpace{}%
\AgdaBound{w}\AgdaSymbol{)}%
\>[35]\AgdaSymbol{=}\AgdaSpace{}%
\AgdaFunction{Dec.map}\AgdaSpace{}%
\AgdaSymbol{(}\AgdaFunction{δ·}\AgdaSpace{}%
\AgdaSymbol{\{}\AgdaArgument{P}\AgdaSpace{}%
\AgdaSymbol{=}\AgdaSpace{}%
\AgdaBound{P}\AgdaSymbol{\})}\AgdaSpace{}%
\AgdaSymbol{((}\AgdaBound{x}\AgdaSpace{}%
\AgdaOperator{\AgdaFunction{·-parse}}\AgdaSpace{}%
\AgdaSymbol{(}\AgdaFunction{δ}\AgdaSpace{}%
\AgdaBound{c}\AgdaSpace{}%
\AgdaBound{φ}\AgdaSymbol{))}\AgdaSpace{}%
\AgdaBound{w}\AgdaSymbol{)}\<%
\\
%
\\[\AgdaEmptyExtraSkip]%
%
\>[4]\AgdaFunction{ν`}\AgdaSpace{}%
\AgdaSymbol{=}\AgdaSpace{}%
\AgdaFunction{mk⇔}\AgdaSpace{}%
\AgdaSymbol{(λ}\AgdaSpace{}%
\AgdaSymbol{())}\AgdaSpace{}%
\AgdaSymbol{(λ}\AgdaSpace{}%
\AgdaSymbol{())}\<%
\\
%
\>[4]\AgdaFunction{δ`}\AgdaSpace{}%
\AgdaSymbol{=}\AgdaSpace{}%
\AgdaFunction{mk⇔}\AgdaSpace{}%
\AgdaSymbol{(λ}\AgdaSpace{}%
\AgdaSymbol{\{}\AgdaSpace{}%
\AgdaSymbol{(}\AgdaInductiveConstructor{refl}\AgdaSpace{}%
\AgdaOperator{\AgdaInductiveConstructor{,}}\AgdaSpace{}%
\AgdaInductiveConstructor{refl}\AgdaSymbol{)}\AgdaSpace{}%
\AgdaSymbol{→}\AgdaSpace{}%
\AgdaInductiveConstructor{refl}\AgdaSpace{}%
\AgdaSymbol{\})}\AgdaSpace{}%
\AgdaSymbol{(λ}\AgdaSpace{}%
\AgdaSymbol{\{}\AgdaSpace{}%
\AgdaInductiveConstructor{refl}\AgdaSpace{}%
\AgdaSymbol{→}\AgdaSpace{}%
\AgdaInductiveConstructor{refl}\AgdaSpace{}%
\AgdaOperator{\AgdaInductiveConstructor{,}}\AgdaSpace{}%
\AgdaInductiveConstructor{refl}\AgdaSpace{}%
\AgdaSymbol{\})}\<%
\end{code}

Now the implementation of parsers for languages consisting of a single character follows completely from the decomposition into nullability and derivatives.

\begin{code}%
%
\>[4]\AgdaOperator{\AgdaFunction{`-parse\AgdaUnderscore{}}}\AgdaSpace{}%
\AgdaSymbol{:}\AgdaSpace{}%
\AgdaSymbol{(}\AgdaBound{c'}\AgdaSpace{}%
\AgdaSymbol{:}\AgdaSpace{}%
\AgdaPostulate{Char}\AgdaSymbol{)}\AgdaSpace{}%
\AgdaSymbol{→}\AgdaSpace{}%
\AgdaFunction{Parser}\AgdaSpace{}%
\AgdaSymbol{(}\AgdaOperator{\AgdaFunction{`}}\AgdaSpace{}%
\AgdaBound{c'}\AgdaSymbol{)}\<%
\\
%
\>[4]\AgdaSymbol{(}\AgdaOperator{\AgdaFunction{`-parse}}\AgdaSpace{}%
\AgdaSymbol{\AgdaUnderscore{})}\AgdaSpace{}%
\AgdaInductiveConstructor{[]}%
\>[26]\AgdaSymbol{=}\AgdaSpace{}%
\AgdaFunction{Dec.map}\AgdaSpace{}%
\AgdaFunction{ν`}\AgdaSpace{}%
\AgdaFunction{⊥-dec}\<%
\\
%
\>[4]\AgdaSymbol{(}\AgdaOperator{\AgdaFunction{`-parse}}\AgdaSpace{}%
\AgdaBound{c'}\AgdaSymbol{)}\AgdaSpace{}%
\AgdaSymbol{(}\AgdaBound{c}\AgdaSpace{}%
\AgdaOperator{\AgdaInductiveConstructor{∷}}\AgdaSpace{}%
\AgdaBound{w}\AgdaSymbol{)}%
\>[26]\AgdaSymbol{=}\AgdaSpace{}%
\AgdaFunction{Dec.map}\AgdaSpace{}%
\AgdaFunction{δ`}\AgdaSpace{}%
\AgdaSymbol{(((}\AgdaBound{c}\AgdaSpace{}%
\AgdaOperator{\AgdaFunction{≟}}\AgdaSpace{}%
\AgdaBound{c'}\AgdaSymbol{)}\AgdaSpace{}%
\AgdaOperator{\AgdaFunction{·-parse}}\AgdaSpace{}%
\AgdaFunction{ε-parse}\AgdaSymbol{)}\AgdaSpace{}%
\AgdaBound{w}\AgdaSymbol{)}\<%
\end{code}

The implementation of $\af{·-parse}$, $\af{ε-parse}$, and $\af{∅-parse}$ are straightforward and can be found in our source code artifact.\jr{todo: reference this nicely}

\begin{code}[hide]%
%
\>[4]\AgdaFunction{ν∪}\AgdaSpace{}%
\AgdaSymbol{=}\AgdaSpace{}%
\AgdaFunction{⇔.refl}\<%
\\
%
\>[4]\AgdaFunction{δ∪}\AgdaSpace{}%
\AgdaSymbol{=}\AgdaSpace{}%
\AgdaFunction{⇔.refl}\<%
\end{code}

\begin{code}[hide]%
%
\>[4]\AgdaComment{--\ To\ make\ this\ work\ properly\ we\ have\ to\ do\ some\ ugly\ implicit\ parameter\ manipulation}\<%
\\
%
\>[4]\AgdaOperator{\AgdaFunction{\AgdaUnderscore{}∪-parse\AgdaUnderscore{}}}\AgdaSpace{}%
\AgdaSymbol{:}\AgdaSpace{}%
\AgdaFunction{Parser}\AgdaSpace{}%
\AgdaGeneralizable{P}\AgdaSpace{}%
\AgdaSymbol{→}\AgdaSpace{}%
\AgdaFunction{Parser}\AgdaSpace{}%
\AgdaGeneralizable{Q}\AgdaSpace{}%
\AgdaSymbol{→}\AgdaSpace{}%
\AgdaFunction{Parser}\AgdaSpace{}%
\AgdaSymbol{(}\AgdaGeneralizable{P}\AgdaSpace{}%
\AgdaOperator{\AgdaFunction{∪}}\AgdaSpace{}%
\AgdaGeneralizable{Q}\AgdaSymbol{)}\<%
\\
%
\>[4]\AgdaOperator{\AgdaFunction{\AgdaUnderscore{}∪-parse\AgdaUnderscore{}}}\AgdaSpace{}%
\AgdaSymbol{\{}\AgdaBound{P}\AgdaSymbol{\}}\AgdaSpace{}%
\AgdaSymbol{\{}\AgdaBound{Q}\AgdaSymbol{\}}\AgdaSpace{}%
\AgdaBound{φ}\AgdaSpace{}%
\AgdaBound{ψ}\AgdaSpace{}%
\AgdaInductiveConstructor{[]}%
\>[35]\AgdaSymbol{=}\AgdaSpace{}%
\AgdaFunction{Dec.map}\AgdaSpace{}%
\AgdaSymbol{(}\AgdaFunction{ν∪}\AgdaSpace{}%
\AgdaSymbol{\{}\AgdaBound{P}\AgdaSymbol{\}}\AgdaSpace{}%
\AgdaSymbol{\{}\AgdaBound{Q}\AgdaSymbol{\})}\AgdaSpace{}%
\AgdaSymbol{(}\AgdaFunction{ν}\AgdaSpace{}%
\AgdaBound{φ}\AgdaSpace{}%
\AgdaOperator{\AgdaFunction{⊎-dec}}\AgdaSpace{}%
\AgdaFunction{ν}\AgdaSpace{}%
\AgdaBound{ψ}\AgdaSymbol{)}\<%
\\
%
\>[4]\AgdaOperator{\AgdaFunction{\AgdaUnderscore{}∪-parse\AgdaUnderscore{}}}\AgdaSpace{}%
\AgdaSymbol{\{}\AgdaBound{P}\AgdaSymbol{\}}\AgdaSpace{}%
\AgdaSymbol{\{}\AgdaBound{Q}\AgdaSymbol{\}}\AgdaSpace{}%
\AgdaBound{φ}\AgdaSpace{}%
\AgdaBound{ψ}\AgdaSpace{}%
\AgdaSymbol{(}\AgdaBound{c}\AgdaSpace{}%
\AgdaOperator{\AgdaInductiveConstructor{∷}}\AgdaSpace{}%
\AgdaBound{w}\AgdaSymbol{)}%
\>[35]\AgdaSymbol{=}\AgdaSpace{}%
\AgdaFunction{Dec.map}\AgdaSpace{}%
\AgdaSymbol{(}\AgdaFunction{δ∪}\AgdaSpace{}%
\AgdaSymbol{\{}\AgdaBound{c}\AgdaSymbol{\}}\AgdaSpace{}%
\AgdaSymbol{\{}\AgdaBound{P}\AgdaSymbol{\}}\AgdaSpace{}%
\AgdaSymbol{\{}\AgdaBound{Q}\AgdaSymbol{\})}\AgdaSpace{}%
\AgdaSymbol{((}\AgdaFunction{δ}\AgdaSpace{}%
\AgdaBound{c}\AgdaSpace{}%
\AgdaBound{φ}\AgdaSpace{}%
\AgdaOperator{\AgdaFunction{∪-parse}}\AgdaSpace{}%
\AgdaFunction{δ}\AgdaSpace{}%
\AgdaBound{c}\AgdaSpace{}%
\AgdaBound{ψ}\AgdaSymbol{)}\AgdaSpace{}%
\AgdaBound{w}\AgdaSymbol{)}\<%
\\
%
\>[4]\AgdaKeyword{module}\AgdaSpace{}%
\AgdaModule{\AgdaUnderscore{}}\AgdaSpace{}%
\AgdaSymbol{\{}\AgdaBound{P}\AgdaSpace{}%
\AgdaBound{Q}\AgdaSpace{}%
\AgdaSymbol{:}\AgdaSpace{}%
\AgdaFunction{Lang}\AgdaSymbol{\}}\AgdaSpace{}%
\AgdaKeyword{where}\<%
\\
\>[4][@{}l@{\AgdaIndent{0}}]%
\>[8]\AgdaFunction{◂ν∪}%
\>[13]\AgdaSymbol{:}\AgdaSpace{}%
\AgdaSymbol{(}\AgdaFunction{ν}\AgdaSpace{}%
\AgdaBound{P}\AgdaSpace{}%
\AgdaOperator{\AgdaDatatype{⊎}}\AgdaSpace{}%
\AgdaFunction{ν}\AgdaSpace{}%
\AgdaBound{Q}\AgdaSymbol{)}%
\>[28]\AgdaOperator{\AgdaFunction{⇔}}\AgdaSpace{}%
\AgdaFunction{ν}\AgdaSpace{}%
\AgdaSymbol{(}\AgdaBound{P}\AgdaSpace{}%
\AgdaOperator{\AgdaFunction{∪}}\AgdaSpace{}%
\AgdaBound{Q}\AgdaSymbol{)}\<%
\\
%
\>[8]\AgdaFunction{◂ν∪}\AgdaSpace{}%
\AgdaSymbol{=}\AgdaSpace{}%
\AgdaFunction{ν∪}\AgdaSpace{}%
\AgdaSymbol{\{}\AgdaBound{P}\AgdaSymbol{\}}\AgdaSpace{}%
\AgdaSymbol{\{}\AgdaBound{Q}\AgdaSymbol{\}}\<%
\\
%
\>[8]\AgdaFunction{◂δ∪}%
\>[13]\AgdaSymbol{:}\AgdaSpace{}%
\AgdaSymbol{(}\AgdaFunction{δ}\AgdaSpace{}%
\AgdaGeneralizable{c}\AgdaSpace{}%
\AgdaBound{P}\AgdaSpace{}%
\AgdaOperator{\AgdaFunction{∪}}\AgdaSpace{}%
\AgdaFunction{δ}\AgdaSpace{}%
\AgdaGeneralizable{c}\AgdaSpace{}%
\AgdaBound{Q}\AgdaSymbol{)}%
\>[32]\AgdaOperator{\AgdaFunction{⟺}}\AgdaSpace{}%
\AgdaFunction{δ}\AgdaSpace{}%
\AgdaGeneralizable{c}\AgdaSpace{}%
\AgdaSymbol{(}\AgdaBound{P}\AgdaSpace{}%
\AgdaOperator{\AgdaFunction{∪}}\AgdaSpace{}%
\AgdaBound{Q}\AgdaSymbol{)}\<%
\\
%
\>[8]\AgdaFunction{◂δ∪}\AgdaSpace{}%
\AgdaSymbol{=}\AgdaSpace{}%
\AgdaFunction{δ∪}\AgdaSpace{}%
\AgdaSymbol{\{\AgdaUnderscore{}\}}\AgdaSpace{}%
\AgdaSymbol{\{}\AgdaBound{P}\AgdaSymbol{\}}\AgdaSpace{}%
\AgdaSymbol{\{}\AgdaBound{Q}\AgdaSymbol{\}}\<%
\end{code}
\begin{code}%
%
\>[8]\AgdaOperator{\AgdaFunction{\AgdaUnderscore{}◂∪-parse\AgdaUnderscore{}}}\AgdaSpace{}%
\AgdaSymbol{:}\AgdaSpace{}%
\AgdaFunction{Parser}\AgdaSpace{}%
\AgdaBound{P}\AgdaSpace{}%
\AgdaSymbol{→}\AgdaSpace{}%
\AgdaFunction{Parser}\AgdaSpace{}%
\AgdaBound{Q}\AgdaSpace{}%
\AgdaSymbol{→}\AgdaSpace{}%
\AgdaFunction{Parser}\AgdaSpace{}%
\AgdaSymbol{(}\AgdaBound{P}\AgdaSpace{}%
\AgdaOperator{\AgdaFunction{∪}}\AgdaSpace{}%
\AgdaBound{Q}\AgdaSymbol{)}\<%
\\
%
\>[8]\AgdaSymbol{(}\AgdaBound{φ}\AgdaSpace{}%
\AgdaOperator{\AgdaFunction{◂∪-parse}}\AgdaSpace{}%
\AgdaBound{ψ}\AgdaSymbol{)}\AgdaSpace{}%
\AgdaInductiveConstructor{[]}%
\>[32]\AgdaSymbol{=}\AgdaSpace{}%
\AgdaFunction{Dec.map}\AgdaSpace{}%
\AgdaFunction{◂ν∪}\AgdaSpace{}%
\AgdaSymbol{(}\AgdaFunction{ν}\AgdaSpace{}%
\AgdaBound{φ}\AgdaSpace{}%
\AgdaOperator{\AgdaFunction{⊎-dec}}\AgdaSpace{}%
\AgdaFunction{ν}\AgdaSpace{}%
\AgdaBound{ψ}\AgdaSymbol{)}\<%
\\
%
\>[8]\AgdaSymbol{(}\AgdaBound{φ}\AgdaSpace{}%
\AgdaOperator{\AgdaFunction{◂∪-parse}}\AgdaSpace{}%
\AgdaBound{ψ}\AgdaSymbol{)}\AgdaSpace{}%
\AgdaSymbol{(}\AgdaBound{c}\AgdaSpace{}%
\AgdaOperator{\AgdaInductiveConstructor{∷}}\AgdaSpace{}%
\AgdaBound{w}\AgdaSymbol{)}%
\>[32]\AgdaSymbol{=}\AgdaSpace{}%
\AgdaFunction{Dec.map}\AgdaSpace{}%
\AgdaFunction{◂δ∪}\AgdaSpace{}%
\AgdaSymbol{((}\AgdaFunction{δ}\AgdaSpace{}%
\AgdaBound{c}\AgdaSpace{}%
\AgdaBound{φ}\AgdaSpace{}%
\AgdaOperator{\AgdaFunction{∪-parse}}\AgdaSpace{}%
\AgdaFunction{δ}\AgdaSpace{}%
\AgdaBound{c}\AgdaSpace{}%
\AgdaBound{ψ}\AgdaSymbol{)}\AgdaSpace{}%
\AgdaBound{w}\AgdaSymbol{)}\<%
\end{code}

\begin{code}[hide]%
%
\>[4]\AgdaFunction{ν∗}\AgdaSpace{}%
\AgdaSymbol{=}\AgdaSpace{}%
\AgdaFunction{mk⇔}\AgdaSpace{}%
\AgdaSymbol{(λ}\AgdaSpace{}%
\AgdaBound{x}\AgdaSpace{}%
\AgdaSymbol{→}\AgdaSpace{}%
\AgdaInductiveConstructor{[]}\AgdaSpace{}%
\AgdaOperator{\AgdaInductiveConstructor{,}}\AgdaSpace{}%
\AgdaInductiveConstructor{[]}\AgdaSpace{}%
\AgdaOperator{\AgdaInductiveConstructor{,}}\AgdaSpace{}%
\AgdaInductiveConstructor{refl}\AgdaSpace{}%
\AgdaOperator{\AgdaInductiveConstructor{,}}\AgdaSpace{}%
\AgdaBound{x}\AgdaSymbol{)}\AgdaSpace{}%
\AgdaSymbol{(λ}\AgdaSpace{}%
\AgdaSymbol{\{}\AgdaSpace{}%
\AgdaSymbol{(}\AgdaInductiveConstructor{[]}\AgdaSpace{}%
\AgdaOperator{\AgdaInductiveConstructor{,}}\AgdaSpace{}%
\AgdaInductiveConstructor{[]}\AgdaSpace{}%
\AgdaOperator{\AgdaInductiveConstructor{,}}\AgdaSpace{}%
\AgdaInductiveConstructor{refl}\AgdaSpace{}%
\AgdaOperator{\AgdaInductiveConstructor{,}}\AgdaSpace{}%
\AgdaBound{x}\AgdaSymbol{)}\AgdaSpace{}%
\AgdaSymbol{→}\AgdaSpace{}%
\AgdaBound{x}\AgdaSpace{}%
\AgdaSymbol{\})}\<%
\\
%
\>[4]\AgdaFunction{δ∗}%
\>[997I]\AgdaSymbol{=}\AgdaSpace{}%
\AgdaFunction{mk⇔}\<%
\\
\>[997I][@{}l@{\AgdaIndent{0}}]%
\>[8]\AgdaSymbol{(λ}\AgdaSpace{}%
\AgdaKeyword{where}\<%
\\
\>[8][@{}l@{\AgdaIndent{0}}]%
\>[10]\AgdaSymbol{(}\AgdaInductiveConstructor{inj₁}\AgdaSpace{}%
\AgdaBound{x}\AgdaSymbol{)}\AgdaSpace{}%
\AgdaSymbol{→}\AgdaSpace{}%
\AgdaInductiveConstructor{[]}\AgdaSpace{}%
\AgdaOperator{\AgdaInductiveConstructor{,}}\AgdaSpace{}%
\AgdaSymbol{\AgdaUnderscore{}}\AgdaSpace{}%
\AgdaOperator{\AgdaInductiveConstructor{,}}\AgdaSpace{}%
\AgdaInductiveConstructor{refl}\AgdaSpace{}%
\AgdaOperator{\AgdaInductiveConstructor{,}}\AgdaSpace{}%
\AgdaBound{x}\<%
\\
%
\>[10]\AgdaSymbol{(}\AgdaInductiveConstructor{inj₂}\AgdaSpace{}%
\AgdaSymbol{(}\AgdaBound{u}\AgdaSpace{}%
\AgdaOperator{\AgdaInductiveConstructor{,}}\AgdaSpace{}%
\AgdaBound{v}\AgdaSpace{}%
\AgdaOperator{\AgdaInductiveConstructor{,}}\AgdaSpace{}%
\AgdaInductiveConstructor{refl}\AgdaSpace{}%
\AgdaOperator{\AgdaInductiveConstructor{,}}\AgdaSpace{}%
\AgdaBound{x}\AgdaSymbol{))}\AgdaSpace{}%
\AgdaSymbol{→}\AgdaSpace{}%
\AgdaSymbol{\AgdaUnderscore{}}\AgdaSpace{}%
\AgdaOperator{\AgdaInductiveConstructor{∷}}\AgdaSpace{}%
\AgdaBound{u}\AgdaSpace{}%
\AgdaOperator{\AgdaInductiveConstructor{,}}\AgdaSpace{}%
\AgdaBound{v}\AgdaSpace{}%
\AgdaOperator{\AgdaInductiveConstructor{,}}\AgdaSpace{}%
\AgdaInductiveConstructor{refl}\AgdaSpace{}%
\AgdaOperator{\AgdaInductiveConstructor{,}}\AgdaSpace{}%
\AgdaBound{x}\AgdaSymbol{)}\<%
\\
%
\>[8]\AgdaSymbol{(λ}\AgdaSpace{}%
\AgdaKeyword{where}\<%
\\
\>[8][@{}l@{\AgdaIndent{0}}]%
\>[10]\AgdaSymbol{(}\AgdaInductiveConstructor{[]}\AgdaSpace{}%
\AgdaOperator{\AgdaInductiveConstructor{,}}\AgdaSpace{}%
\AgdaSymbol{\AgdaUnderscore{}}\AgdaSpace{}%
\AgdaOperator{\AgdaInductiveConstructor{,}}\AgdaSpace{}%
\AgdaInductiveConstructor{refl}\AgdaSpace{}%
\AgdaOperator{\AgdaInductiveConstructor{,}}\AgdaSpace{}%
\AgdaBound{x}\AgdaSymbol{)}\AgdaSpace{}%
\AgdaSymbol{→}\AgdaSpace{}%
\AgdaInductiveConstructor{inj₁}\AgdaSpace{}%
\AgdaBound{x}\<%
\\
%
\>[10]\AgdaSymbol{(\AgdaUnderscore{}}\AgdaSpace{}%
\AgdaOperator{\AgdaInductiveConstructor{∷}}\AgdaSpace{}%
\AgdaBound{u}\AgdaSpace{}%
\AgdaOperator{\AgdaInductiveConstructor{,}}\AgdaSpace{}%
\AgdaBound{v}\AgdaSpace{}%
\AgdaOperator{\AgdaInductiveConstructor{,}}\AgdaSpace{}%
\AgdaInductiveConstructor{refl}\AgdaSpace{}%
\AgdaOperator{\AgdaInductiveConstructor{,}}\AgdaSpace{}%
\AgdaBound{x}\AgdaSymbol{)}\AgdaSpace{}%
\AgdaSymbol{→}\AgdaSpace{}%
\AgdaInductiveConstructor{inj₂}\AgdaSpace{}%
\AgdaSymbol{(}\AgdaBound{u}\AgdaSpace{}%
\AgdaOperator{\AgdaInductiveConstructor{,}}\AgdaSpace{}%
\AgdaBound{v}\AgdaSpace{}%
\AgdaOperator{\AgdaInductiveConstructor{,}}\AgdaSpace{}%
\AgdaInductiveConstructor{refl}\AgdaSpace{}%
\AgdaOperator{\AgdaInductiveConstructor{,}}\AgdaSpace{}%
\AgdaBound{x}\AgdaSymbol{))}\<%
\end{code}

\begin{code}[hide]%
%
\>[4]\AgdaFunction{map′-id-id}\AgdaSpace{}%
\AgdaSymbol{:}\AgdaSpace{}%
\AgdaSymbol{\{}\AgdaBound{x}\AgdaSpace{}%
\AgdaSymbol{:}\AgdaSpace{}%
\AgdaRecord{Dec}\AgdaSpace{}%
\AgdaGeneralizable{A}\AgdaSymbol{\}}\AgdaSpace{}%
\AgdaSymbol{→}\AgdaSpace{}%
\AgdaFunction{Dec.map′}\AgdaSpace{}%
\AgdaFunction{id}\AgdaSpace{}%
\AgdaFunction{id}\AgdaSpace{}%
\AgdaBound{x}\AgdaSpace{}%
\AgdaOperator{\AgdaDatatype{≡}}\AgdaSpace{}%
\AgdaBound{x}\<%
\\
%
\>[4]\AgdaFunction{map′-id-id}\AgdaSpace{}%
\AgdaSymbol{\{}\AgdaArgument{x}\AgdaSpace{}%
\AgdaSymbol{=}\AgdaSpace{}%
\AgdaInductiveConstructor{yes}\AgdaSpace{}%
\AgdaSymbol{\AgdaUnderscore{}\}}\AgdaSpace{}%
\AgdaSymbol{=}\AgdaSpace{}%
\AgdaInductiveConstructor{refl}\<%
\\
%
\>[4]\AgdaFunction{map′-id-id}\AgdaSpace{}%
\AgdaSymbol{\{}\AgdaArgument{x}\AgdaSpace{}%
\AgdaSymbol{=}\AgdaSpace{}%
\AgdaInductiveConstructor{no}\AgdaSpace{}%
\AgdaSymbol{\AgdaUnderscore{}\}}\AgdaSpace{}%
\AgdaSymbol{=}\AgdaSpace{}%
\AgdaInductiveConstructor{refl}\<%
\\
%
\>[4]\AgdaFunction{∪-parse-rewrite}\AgdaSpace{}%
\AgdaSymbol{:}\AgdaSpace{}%
\AgdaSymbol{∀\{}\AgdaBound{φ}\AgdaSpace{}%
\AgdaSymbol{:}\AgdaSpace{}%
\AgdaFunction{Parser}\AgdaSpace{}%
\AgdaGeneralizable{P}\AgdaSymbol{\}}\AgdaSpace{}%
\AgdaSymbol{\{}\AgdaBound{ψ}\AgdaSpace{}%
\AgdaSymbol{:}\AgdaSpace{}%
\AgdaFunction{Parser}\AgdaSpace{}%
\AgdaGeneralizable{Q}\AgdaSymbol{\}}\AgdaSpace{}%
\AgdaSymbol{→}\AgdaSpace{}%
\AgdaSymbol{(}\AgdaBound{φ}\AgdaSpace{}%
\AgdaOperator{\AgdaFunction{∪-parse}}\AgdaSpace{}%
\AgdaBound{ψ}\AgdaSymbol{)}\AgdaSpace{}%
\AgdaGeneralizable{w}\AgdaSpace{}%
\AgdaOperator{\AgdaDatatype{≡}}\AgdaSpace{}%
\AgdaSymbol{(}\AgdaBound{φ}\AgdaSpace{}%
\AgdaGeneralizable{w}\AgdaSpace{}%
\AgdaOperator{\AgdaFunction{⊎-dec}}\AgdaSpace{}%
\AgdaBound{ψ}\AgdaSpace{}%
\AgdaGeneralizable{w}\AgdaSymbol{)}\<%
\\
%
\>[4]\AgdaFunction{∪-parse-rewrite}\AgdaSpace{}%
\AgdaSymbol{\{}\AgdaArgument{w}\AgdaSpace{}%
\AgdaSymbol{=}\AgdaSpace{}%
\AgdaInductiveConstructor{[]}\AgdaSymbol{\}}\AgdaSpace{}%
\AgdaSymbol{=}\AgdaSpace{}%
\AgdaFunction{map′-id-id}\<%
\\
%
\>[4]\AgdaFunction{∪-parse-rewrite}\AgdaSpace{}%
\AgdaSymbol{\{}\AgdaArgument{w}\AgdaSpace{}%
\AgdaSymbol{=}\AgdaSpace{}%
\AgdaBound{x}\AgdaSpace{}%
\AgdaOperator{\AgdaInductiveConstructor{∷}}\AgdaSpace{}%
\AgdaBound{w}\AgdaSymbol{\}}\AgdaSpace{}%
\AgdaSymbol{=}\AgdaSpace{}%
\AgdaFunction{trans}\AgdaSpace{}%
\AgdaFunction{map′-id-id}\AgdaSpace{}%
\AgdaSymbol{(}\AgdaFunction{∪-parse-rewrite}\AgdaSpace{}%
\AgdaSymbol{\{}\AgdaArgument{w}\AgdaSpace{}%
\AgdaSymbol{=}\AgdaSpace{}%
\AgdaBound{w}\AgdaSymbol{\})}\<%
\end{code}

\begin{code}[hide]%
%
\>[4]\AgdaKeyword{infix}\AgdaSpace{}%
\AgdaNumber{22}\AgdaSpace{}%
\AgdaOperator{\AgdaFunction{`-parse\AgdaUnderscore{}}}\<%
\\
%
\>[4]\AgdaKeyword{infixr}\AgdaSpace{}%
\AgdaNumber{21}\AgdaSpace{}%
\AgdaOperator{\AgdaFunction{\AgdaUnderscore{}∗-parse\AgdaUnderscore{}}}\<%
\\
%
\>[4]\AgdaKeyword{infix}\AgdaSpace{}%
\AgdaNumber{21}\AgdaSpace{}%
\AgdaOperator{\AgdaFunction{\AgdaUnderscore{}·-parse\AgdaUnderscore{}}}\<%
\\
%
\>[4]\AgdaKeyword{infixr}\AgdaSpace{}%
\AgdaNumber{20}\AgdaSpace{}%
\AgdaOperator{\AgdaFunction{\AgdaUnderscore{}∪-parse\AgdaUnderscore{}}}\<%
\\
%
\>[4]\AgdaComment{--\ To\ convince\ Agda\ this\ terminates,\ we\ need\ to\ use\ the\ identity:}\<%
\\
%
\>[4]\AgdaComment{--\ (φ\ ∪-parse\ ψ)\ w\ ≡\ φ\ w\ ⊎-dec\ ψ\ w}\<%
\\
%
\>[4]\AgdaComment{--\ That\ is\ easy\ to\ prove\ (see\ ∪-parse-rewrite\ above),\ but\ convincing\ Agda\ to\ use\ this\ is\ a\ bit\ difficult}\<%
\\
%
\>[4]\AgdaOperator{\AgdaFunction{\AgdaUnderscore{}∗-parse\AgdaUnderscore{}}}\AgdaSpace{}%
\AgdaSymbol{:}\AgdaSpace{}%
\AgdaFunction{Parser}\AgdaSpace{}%
\AgdaGeneralizable{P}\AgdaSpace{}%
\AgdaSymbol{→}\AgdaSpace{}%
\AgdaFunction{Parser}\AgdaSpace{}%
\AgdaGeneralizable{Q}\AgdaSpace{}%
\AgdaSymbol{→}\AgdaSpace{}%
\AgdaFunction{Parser}\AgdaSpace{}%
\AgdaSymbol{(}\AgdaGeneralizable{P}\AgdaSpace{}%
\AgdaOperator{\AgdaFunction{∗}}\AgdaSpace{}%
\AgdaGeneralizable{Q}\AgdaSymbol{)}\<%
\\
%
\>[4]\AgdaSymbol{(}\AgdaBound{φ}\AgdaSpace{}%
\AgdaOperator{\AgdaFunction{∗-parse}}\AgdaSpace{}%
\AgdaBound{ψ}\AgdaSymbol{)}\AgdaSpace{}%
\AgdaInductiveConstructor{[]}%
\>[27]\AgdaSymbol{=}\AgdaSpace{}%
\AgdaFunction{Dec.map}\AgdaSpace{}%
\AgdaFunction{ν∗}\AgdaSpace{}%
\AgdaSymbol{(}\AgdaFunction{ν}\AgdaSpace{}%
\AgdaBound{φ}\AgdaSpace{}%
\AgdaOperator{\AgdaFunction{×-dec}}\AgdaSpace{}%
\AgdaFunction{ν}\AgdaSpace{}%
\AgdaBound{ψ}\AgdaSymbol{)}\<%
\\
%
\>[4]\AgdaSymbol{(}\AgdaBound{φ}\AgdaSpace{}%
\AgdaOperator{\AgdaFunction{∗-parse}}\AgdaSpace{}%
\AgdaBound{ψ}\AgdaSymbol{)}\AgdaSpace{}%
\AgdaSymbol{(}\AgdaBound{c}\AgdaSpace{}%
\AgdaOperator{\AgdaInductiveConstructor{∷}}\AgdaSpace{}%
\AgdaBound{w}\AgdaSymbol{)}%
\>[27]\AgdaSymbol{=}\AgdaSpace{}%
\AgdaFunction{Dec.map}\AgdaSpace{}%
\AgdaFunction{δ∗}\AgdaSpace{}%
\AgdaSymbol{((}\AgdaFunction{ν}\AgdaSpace{}%
\AgdaBound{φ}\AgdaSpace{}%
\AgdaOperator{\AgdaFunction{·-parse}}\AgdaSpace{}%
\AgdaFunction{δ}\AgdaSpace{}%
\AgdaBound{c}\AgdaSpace{}%
\AgdaBound{ψ}\AgdaSymbol{)}\AgdaSpace{}%
\AgdaBound{w}\AgdaSpace{}%
\AgdaOperator{\AgdaFunction{⊎-dec}}\AgdaSpace{}%
\AgdaSymbol{(}\AgdaFunction{δ}\AgdaSpace{}%
\AgdaBound{c}\AgdaSpace{}%
\AgdaBound{φ}\AgdaSpace{}%
\AgdaOperator{\AgdaFunction{∗-parse}}\AgdaSpace{}%
\AgdaBound{ψ}\AgdaSymbol{)}\AgdaSpace{}%
\AgdaBound{w}\AgdaSymbol{)}\<%
\\
%
\>[4]\AgdaComment{--\ So\ we\ just\ use\ the\ TERMINATING\ pragma\ for\ the\ code\ that\ we\ show\ in\ the\ paper.}\<%
\\
%
\>[4]\AgdaSymbol{\{-\#}\AgdaSpace{}%
\AgdaKeyword{TERMINATING}\AgdaSpace{}%
\AgdaSymbol{\#-\}}\<%
\end{code}
\begin{code}%
%
\>[4]\AgdaOperator{\AgdaFunction{\AgdaUnderscore{}◂∗-parse\AgdaUnderscore{}}}\AgdaSpace{}%
\AgdaSymbol{:}\AgdaSpace{}%
\AgdaFunction{Parser}\AgdaSpace{}%
\AgdaGeneralizable{P}\AgdaSpace{}%
\AgdaSymbol{→}\AgdaSpace{}%
\AgdaFunction{Parser}\AgdaSpace{}%
\AgdaGeneralizable{Q}\AgdaSpace{}%
\AgdaSymbol{→}\AgdaSpace{}%
\AgdaFunction{Parser}\AgdaSpace{}%
\AgdaSymbol{(}\AgdaGeneralizable{P}\AgdaSpace{}%
\AgdaOperator{\AgdaFunction{∗}}\AgdaSpace{}%
\AgdaGeneralizable{Q}\AgdaSymbol{)}\<%
\\
%
\>[4]\AgdaSymbol{(}\AgdaBound{φ}\AgdaSpace{}%
\AgdaOperator{\AgdaFunction{◂∗-parse}}\AgdaSpace{}%
\AgdaBound{ψ}\AgdaSymbol{)}\AgdaSpace{}%
\AgdaInductiveConstructor{[]}%
\>[28]\AgdaSymbol{=}\AgdaSpace{}%
\AgdaFunction{Dec.map}\AgdaSpace{}%
\AgdaFunction{ν∗}\AgdaSpace{}%
\AgdaSymbol{(}\AgdaFunction{ν}\AgdaSpace{}%
\AgdaBound{φ}\AgdaSpace{}%
\AgdaOperator{\AgdaFunction{×-dec}}\AgdaSpace{}%
\AgdaFunction{ν}\AgdaSpace{}%
\AgdaBound{ψ}\AgdaSymbol{)}\<%
\\
%
\>[4]\AgdaSymbol{(}\AgdaBound{φ}\AgdaSpace{}%
\AgdaOperator{\AgdaFunction{◂∗-parse}}\AgdaSpace{}%
\AgdaBound{ψ}\AgdaSymbol{)}\AgdaSpace{}%
\AgdaSymbol{(}\AgdaBound{c}\AgdaSpace{}%
\AgdaOperator{\AgdaInductiveConstructor{∷}}\AgdaSpace{}%
\AgdaBound{w}\AgdaSymbol{)}%
\>[28]\AgdaSymbol{=}\AgdaSpace{}%
\AgdaFunction{Dec.map}\AgdaSpace{}%
\AgdaFunction{δ∗}\AgdaSpace{}%
\AgdaSymbol{((}\AgdaFunction{ν}\AgdaSpace{}%
\AgdaBound{φ}\AgdaSpace{}%
\AgdaOperator{\AgdaFunction{·-parse}}\AgdaSpace{}%
\AgdaFunction{δ}\AgdaSpace{}%
\AgdaBound{c}\AgdaSpace{}%
\AgdaBound{ψ}\AgdaSpace{}%
\AgdaOperator{\AgdaFunction{∪-parse}}\AgdaSpace{}%
\AgdaFunction{δ}\AgdaSpace{}%
\AgdaBound{c}\AgdaSpace{}%
\AgdaBound{φ}\AgdaSpace{}%
\AgdaOperator{\AgdaFunction{∗-parse}}\AgdaSpace{}%
\AgdaBound{ψ}\AgdaSymbol{)}\AgdaSpace{}%
\AgdaBound{w}\AgdaSymbol{)}\<%
\end{code}

Using these combinators we can define a parser for the permissions language by
simply mapping each of the language combinators onto their respective parser
combinators.

\begin{code}[hide]%
%
\>[4]\AgdaFunction{permissions-parse}\AgdaSpace{}%
\AgdaSymbol{:}\AgdaSpace{}%
\AgdaFunction{Parser}\AgdaSpace{}%
\AgdaFunction{permissions}\<%
\\
%
\>[4]\AgdaFunction{read-parse}\AgdaSpace{}%
\AgdaSymbol{:}\AgdaSpace{}%
\AgdaFunction{Parser}\AgdaSpace{}%
\AgdaFunction{read}\<%
\\
%
\>[4]\AgdaFunction{write-parse}\AgdaSpace{}%
\AgdaSymbol{:}\AgdaSpace{}%
\AgdaFunction{Parser}\AgdaSpace{}%
\AgdaFunction{write}\<%
\\
%
\>[4]\AgdaFunction{execute-parse}\AgdaSpace{}%
\AgdaSymbol{:}\AgdaSpace{}%
\AgdaFunction{Parser}\AgdaSpace{}%
\AgdaFunction{execute}\<%
\end{code}
\begin{code}%
%
\>[4]\AgdaFunction{permissions-parse}%
\>[23]\AgdaSymbol{=}\AgdaSpace{}%
\AgdaFunction{read-parse}\AgdaSpace{}%
\AgdaOperator{\AgdaFunction{∗-parse}}\AgdaSpace{}%
\AgdaSymbol{(}\AgdaFunction{write-parse}\AgdaSpace{}%
\AgdaOperator{\AgdaFunction{∗-parse}}\AgdaSpace{}%
\AgdaFunction{execute-parse}\AgdaSymbol{)}\<%
\\
%
\>[4]\AgdaFunction{read-parse}%
\>[23]\AgdaSymbol{=}\AgdaSpace{}%
\AgdaSymbol{(}\AgdaOperator{\AgdaFunction{`-parse}}\AgdaSpace{}%
\AgdaString{'-'}\AgdaSymbol{)}\AgdaSpace{}%
\AgdaOperator{\AgdaFunction{∪-parse}}\AgdaSpace{}%
\AgdaSymbol{(}\AgdaOperator{\AgdaFunction{`-parse}}\AgdaSpace{}%
\AgdaString{'r'}\AgdaSymbol{)}\<%
\\
%
\>[4]\AgdaFunction{write-parse}%
\>[23]\AgdaSymbol{=}\AgdaSpace{}%
\AgdaSymbol{(}\AgdaOperator{\AgdaFunction{`-parse}}\AgdaSpace{}%
\AgdaString{'-'}\AgdaSymbol{)}\AgdaSpace{}%
\AgdaOperator{\AgdaFunction{∪-parse}}\AgdaSpace{}%
\AgdaSymbol{(}\AgdaOperator{\AgdaFunction{`-parse}}\AgdaSpace{}%
\AgdaString{'w'}\AgdaSymbol{)}\<%
\\
%
\>[4]\AgdaFunction{execute-parse}%
\>[23]\AgdaSymbol{=}\AgdaSpace{}%
\AgdaSymbol{(}\AgdaOperator{\AgdaFunction{`-parse}}\AgdaSpace{}%
\AgdaString{'-'}\AgdaSymbol{)}\AgdaSpace{}%
\AgdaOperator{\AgdaFunction{∪-parse}}\AgdaSpace{}%
\AgdaSymbol{(}\AgdaOperator{\AgdaFunction{`-parse}}\AgdaSpace{}%
\AgdaString{'x'}\AgdaSymbol{)}\<%
\end{code}

\subsection{Infinite Languages}

This permissions language is very simple. In particular, it is finite. In practice, many languages are inifinite, for which the basic combinators will not suffice. For example, file paths can be arbitrarily long on most systems.
Elliot~\cite{conal-languages} defines a Kleene star\jr{does this need citation?} combinator which enables him to specify regular languages such as file paths.

However, we want to go one step further, speficying and parsing context-free languages. Most practical programming languages are at least context-free, if not more complicated. One essential feature of many languages is the ability to recognize balanced brackets. A minimal example language with balanced brackets is the following:
%
\begin{grammar}
<brackets> ::= ε | `[' <brackets> `]' | <brackets> <brackets>
\end{grammar}
%
This is the language of all strings which consist of balanced square brackets. 
Many practical programming languages include some form of balanced brackets. Furthermore, this language is well known to be context-free and not regular. Thus, we need more powerful combinators.

We could try to naively transcribe the brackets grammar using our basic combinators, but Agda will justifiably complain that it is not terminating (here I've added a NON_TERMINATING pragma to make Agda to accept it any way).
%
\begin{code}%
%
\>[4]\AgdaSymbol{\{-\#}\AgdaSpace{}%
\AgdaKeyword{NON\AgdaUnderscore{}TERMINATING}\AgdaSpace{}%
\AgdaSymbol{\#-\}}\<%
\\
%
\>[4]\AgdaFunction{brackets}\AgdaSpace{}%
\AgdaSymbol{=}\AgdaSpace{}%
\AgdaFunction{ε}\AgdaSpace{}%
\AgdaOperator{\AgdaFunction{∪}}\AgdaSpace{}%
\AgdaOperator{\AgdaFunction{`}}\AgdaSpace{}%
\AgdaString{'['}\AgdaSpace{}%
\AgdaOperator{\AgdaFunction{∗}}\AgdaSpace{}%
\AgdaFunction{brackets}\AgdaSpace{}%
\AgdaOperator{\AgdaFunction{∗}}\AgdaSpace{}%
\AgdaOperator{\AgdaFunction{`}}\AgdaSpace{}%
\AgdaString{']'}\AgdaSpace{}%
\AgdaOperator{\AgdaFunction{∪}}\AgdaSpace{}%
\AgdaFunction{brackets}\AgdaSpace{}%
\AgdaOperator{\AgdaFunction{∗}}\AgdaSpace{}%
\AgdaFunction{brackets}\<%
\end{code}
%
We need to find a different way to encode this recursive relation.

\begin{code}%
%
\>[4]\AgdaKeyword{postulate}\AgdaSpace{}%
\AgdaPostulate{μ}\AgdaSpace{}%
\AgdaSymbol{:}\AgdaSpace{}%
\AgdaSymbol{(}\AgdaFunction{Lang}\AgdaSpace{}%
\AgdaSymbol{→}\AgdaSpace{}%
\AgdaFunction{Lang}\AgdaSymbol{)}\AgdaSpace{}%
\AgdaSymbol{→}\AgdaSpace{}%
\AgdaFunction{Lang}\<%
\\
%
\>[4]\AgdaFunction{bracketsμ}\AgdaSpace{}%
\AgdaSymbol{=}\AgdaSpace{}%
\AgdaPostulate{μ}\AgdaSpace{}%
\AgdaSymbol{(λ}\AgdaSpace{}%
\AgdaBound{P}\AgdaSpace{}%
\AgdaSymbol{→}\AgdaSpace{}%
\AgdaFunction{ε}\AgdaSpace{}%
\AgdaOperator{\AgdaFunction{∪}}\AgdaSpace{}%
\AgdaOperator{\AgdaFunction{`}}\AgdaSpace{}%
\AgdaString{'['}\AgdaSpace{}%
\AgdaOperator{\AgdaFunction{∗}}\AgdaSpace{}%
\AgdaBound{P}\AgdaSpace{}%
\AgdaOperator{\AgdaFunction{∗}}\AgdaSpace{}%
\AgdaOperator{\AgdaFunction{`}}\AgdaSpace{}%
\AgdaString{']'}\AgdaSpace{}%
\AgdaOperator{\AgdaFunction{∪}}\AgdaSpace{}%
\AgdaBound{P}\AgdaSpace{}%
\AgdaOperator{\AgdaFunction{∗}}\AgdaSpace{}%
\AgdaBound{P}\AgdaSymbol{)}\<%
\end{code}

\begin{itemize}
\item μ cannot be implemented just like that
\item we need to restrict the Lang → Lang function that we take a fixed point over
\item polynomial functors would work, but our grammars are slightly different.
\item Luckily, our basic combinators with variables added also works
\item We can make this obvious to agda by defining a data type of descriptions a la gentle art of levitation.
\end{itemize}

\endinput

% For starters, we define some structure on this definition of language in
% \cref{fig:combinators}. First, Languages form a semiring, with union
% $\af{\un{}∪\un{}}$, concatenation $\af{\un{}∗\un{}}$, the empty language
% $\af{∅}$ which is the unit of union, and the language which only includes the
% empty string $\af{ε}$ which is the unit of concatenation. Furthermore the
% $\af{`\un}$ combinator defines a language which contains exactly the string
% consisting of a single given character. Finally, the scalar multiplication
% $\af{\un{}·\un{}}$ combinator injects an Agda type into a language. The purpose
% of this combinator will become clearer in later sections\jr{mention specific sections}.

% \subsection{Decidability}

% From our type theoretic perspective, parsing a string is the same thing as producing an element of the result type of a language for that given input string, or showing that no such element can exist. In Agda, we encode this using the following \af{Dec} data type which is parameterized by a type \ab{A} and contains a constructor \ac{yes} for when you can produce an element of \ab{A} or \ac{no} if you can show that no such element exists.
% \begin{code}
% data Dec (A : Type) : Type where
    % yes : A → Dec A
    % no : (A → ⊥) → Dec A
% \end{code}
% Sometimes we want to change the parameter type of a \af{Dec}. For that we need to provide conversion functions between the old and the new type in both ways.
% \begin{code}
% map? : (A ↔ B) → Dec A → Dec B
% map? f (yes x) = yes (to f x)
% map? f (no ¬A) = no λ x → ¬A (from f x)
% \end{code}
% \begin{code}[hide]
% ⌊_⌋ : Dec A → Type
% ⌊_⌋ {A} _ = A

% _×?_ : Dec A → Dec B → Dec (A × B)
% yes x ×? yes y = yes (x , y)
% yes _ ×? no ¬y = no λ (_ , y) → ¬y y
% no ¬x ×? _ = no λ (x , _) → ¬x x

% _⊎?_ : Dec A → Dec B → Dec (A ⊎ B)
% yes x ⊎? y = yes (inl x)
% no x ⊎? yes y = yes (inr y)
% no ¬x ⊎? no ¬y = no λ where
    % (inl x) → ¬x x
    % (inr y) → ¬y y

% _≟_ : (c : Char) → (c′ : Char) → Dec (c ≡ c′)
% `a ≟ `a = yes refl
% `a ≟ `b = no λ ()
% `a ≟ `c = no λ ()
% `a ≟ `0 = no λ ()
% `a ≟ `1 = no λ ()
% `b ≟ `a = no λ ()
% `b ≟ `b = yes refl
% `b ≟ `c = no λ ()
% `b ≟ `0 = no λ ()
% `b ≟ `1 = no λ ()
% `c ≟ `a = no λ ()
% `c ≟ `b = no λ ()
% `c ≟ `c = yes refl
% `c ≟ `0 = no λ ()
% `c ≟ `1 = no λ ()
% `0 ≟ `a = no λ ()
% `0 ≟ `b = no λ ()
% `0 ≟ `c = no λ ()
% `0 ≟ `0 = yes refl
% `0 ≟ `1 = no λ ()
% `1 ≟ `a = no λ ()
% `1 ≟ `b = no λ ()
% `1 ≟ `c = no λ ()
% `1 ≟ `0 = no λ ()
% `1 ≟ `1 = yes refl
% `a ≟ `[ = no λ ()
% `a ≟ `] = no λ ()
% `b ≟ `[ = no λ ()
% `b ≟ `] = no λ ()
% `c ≟ `[ = no λ ()
% `c ≟ `] = no λ ()
% `0 ≟ `[ = no λ ()
% `0 ≟ `] = no λ ()
% `1 ≟ `[ = no λ ()
% `1 ≟ `] = no λ ()
% `[ ≟ `a = no λ ()
% `[ ≟ `b = no λ ()
% `[ ≟ `c = no λ ()
% `[ ≟ `0 = no λ ()
% `[ ≟ `1 = no λ ()
% `[ ≟ `[ = yes refl
% `[ ≟ `] = no λ ()
% `] ≟ `a = no λ ()
% `] ≟ `b = no λ ()
% `] ≟ `c = no λ ()
% `] ≟ `0 = no λ ()
% `] ≟ `1 = no λ ()
% `] ≟ `[ = no λ ()
% `] ≟ `] = yes refl

% \end{code}

\subsection{Grammars}\label{sec:gram-and-parsing}

We have seen in \cref{ex:non-context-free} that our definition of language is very general, comprising even context-sensitive languages. Parsing such languages automatically poses a significant challenge. Hence, we side-step this problem by restricting the scope of our parsers to a smaller well-defined subset of languages. In this subsection, we consider a subset of regular languages without Kleene star (i.e., closure under concatenation). In \cref{sec:context-free}, we extend this class of languages to include fixed points which subsume the Kleene star.

\begin{code}[hide]%
\>[0]\AgdaKeyword{module}\AgdaSpace{}%
\AgdaModule{◆}\AgdaSpace{}%
\AgdaKeyword{where}\<%
\end{code}
\begin{code}%
\>[0][@{}l@{\AgdaIndent{1}}]%
\>[4]\AgdaKeyword{data}\AgdaSpace{}%
\AgdaDatatype{Exp}\AgdaSpace{}%
\AgdaSymbol{:}\AgdaSpace{}%
\AgdaPrimitive{Type₁}\AgdaSpace{}%
\AgdaKeyword{where}\<%
\\
\>[4][@{}l@{\AgdaIndent{0}}]%
\>[8]\AgdaInductiveConstructor{∅}\AgdaSpace{}%
\AgdaSymbol{:}\AgdaSpace{}%
\AgdaDatatype{Exp}\<%
\\
%
\>[8]\AgdaInductiveConstructor{ε}\AgdaSpace{}%
\AgdaSymbol{:}\AgdaSpace{}%
\AgdaDatatype{Exp}\<%
\\
%
\>[8]\AgdaOperator{\AgdaInductiveConstructor{`\AgdaUnderscore{}}}\AgdaSpace{}%
\AgdaSymbol{:}\AgdaSpace{}%
\AgdaSymbol{(}\AgdaBound{c}\AgdaSpace{}%
\AgdaSymbol{:}\AgdaSpace{}%
\AgdaPostulate{Char}\AgdaSymbol{)}\AgdaSpace{}%
\AgdaSymbol{→}\AgdaSpace{}%
\AgdaDatatype{Exp}\<%
\\
%
\>[8]\AgdaOperator{\AgdaInductiveConstructor{\AgdaUnderscore{}·\AgdaUnderscore{}}}\AgdaSpace{}%
\AgdaSymbol{:}\AgdaSpace{}%
\AgdaSymbol{\{}\AgdaBound{A}\AgdaSpace{}%
\AgdaSymbol{:}\AgdaSpace{}%
\AgdaPrimitive{Type}\AgdaSymbol{\}}\AgdaSpace{}%
\AgdaSymbol{→}\AgdaSpace{}%
\AgdaRecord{Dec}\AgdaSpace{}%
\AgdaBound{A}\AgdaSpace{}%
\AgdaSymbol{→}\AgdaSpace{}%
\AgdaDatatype{Exp}\AgdaSpace{}%
\AgdaSymbol{→}\AgdaSpace{}%
\AgdaDatatype{Exp}\<%
\\
%
\>[8]\AgdaOperator{\AgdaInductiveConstructor{\AgdaUnderscore{}∪\AgdaUnderscore{}}}\AgdaSpace{}%
\AgdaSymbol{:}\AgdaSpace{}%
\AgdaDatatype{Exp}\AgdaSpace{}%
\AgdaSymbol{→}\AgdaSpace{}%
\AgdaDatatype{Exp}\AgdaSpace{}%
\AgdaSymbol{→}\AgdaSpace{}%
\AgdaDatatype{Exp}\<%
\\
%
\>[8]\AgdaOperator{\AgdaInductiveConstructor{\AgdaUnderscore{}∗\AgdaUnderscore{}}}\AgdaSpace{}%
\AgdaSymbol{:}\AgdaSpace{}%
\AgdaDatatype{Exp}\AgdaSpace{}%
\AgdaSymbol{→}\AgdaSpace{}%
\AgdaDatatype{Exp}\AgdaSpace{}%
\AgdaSymbol{→}\AgdaSpace{}%
\AgdaDatatype{Exp}\<%
\end{code}

This syntax maps directly onto the semantics we defined in \cref{fig:combinators}.

\begin{code}[hide]%
%
\>[4]\AgdaFunction{typeOfDec}\AgdaSpace{}%
\AgdaSymbol{:}\AgdaSpace{}%
\AgdaSymbol{\{}\AgdaBound{A}\AgdaSpace{}%
\AgdaSymbol{:}\AgdaSpace{}%
\AgdaPrimitive{Type}\AgdaSymbol{\}}\AgdaSpace{}%
\AgdaSymbol{→}\AgdaSpace{}%
\AgdaRecord{Dec}\AgdaSpace{}%
\AgdaBound{A}\AgdaSpace{}%
\AgdaSymbol{→}\AgdaSpace{}%
\AgdaPrimitive{Type}\<%
\\
%
\>[4]\AgdaFunction{typeOfDec}\AgdaSpace{}%
\AgdaSymbol{\{}\AgdaBound{A}\AgdaSymbol{\}}\AgdaSpace{}%
\AgdaSymbol{\AgdaUnderscore{}}\AgdaSpace{}%
\AgdaSymbol{=}\AgdaSpace{}%
\AgdaBound{A}\<%
\end{code}
\begin{code}%
%
\>[4]\AgdaOperator{\AgdaFunction{⟦\AgdaUnderscore{}⟧}}\AgdaSpace{}%
\AgdaSymbol{:}\AgdaSpace{}%
\AgdaDatatype{Exp}\AgdaSpace{}%
\AgdaSymbol{→}\AgdaSpace{}%
\AgdaFunction{Lang}\<%
\\
%
\>[4]\AgdaOperator{\AgdaFunction{⟦}}\AgdaSpace{}%
\AgdaInductiveConstructor{∅}\AgdaSpace{}%
\AgdaOperator{\AgdaFunction{⟧}}\AgdaSpace{}%
\AgdaSymbol{=}\AgdaSpace{}%
\AgdaFunction{◇.∅}\<%
\\
%
\>[4]\AgdaOperator{\AgdaFunction{⟦}}\AgdaSpace{}%
\AgdaInductiveConstructor{ε}\AgdaSpace{}%
\AgdaOperator{\AgdaFunction{⟧}}\AgdaSpace{}%
\AgdaSymbol{=}\AgdaSpace{}%
\AgdaFunction{◇.ε}\<%
\\
%
\>[4]\AgdaOperator{\AgdaFunction{⟦}}\AgdaSpace{}%
\AgdaOperator{\AgdaInductiveConstructor{`}}\AgdaSpace{}%
\AgdaBound{c}\AgdaSpace{}%
\AgdaOperator{\AgdaFunction{⟧}}\AgdaSpace{}%
\AgdaSymbol{=}\AgdaSpace{}%
\AgdaOperator{\AgdaFunction{◇.`}}\AgdaSpace{}%
\AgdaBound{c}\<%
\\
%
\>[4]\AgdaOperator{\AgdaFunction{⟦}}\AgdaSpace{}%
\AgdaBound{x}\AgdaSpace{}%
\AgdaOperator{\AgdaInductiveConstructor{·}}\AgdaSpace{}%
\AgdaBound{e}\AgdaSpace{}%
\AgdaOperator{\AgdaFunction{⟧}}\AgdaSpace{}%
\AgdaSymbol{=}\AgdaSpace{}%
\AgdaFunction{typeOfDec}\AgdaSpace{}%
\AgdaBound{x}\AgdaSpace{}%
\AgdaOperator{\AgdaFunction{◇.·}}\AgdaSpace{}%
\AgdaOperator{\AgdaFunction{⟦}}\AgdaSpace{}%
\AgdaBound{e}\AgdaSpace{}%
\AgdaOperator{\AgdaFunction{⟧}}\<%
\\
%
\>[4]\AgdaOperator{\AgdaFunction{⟦}}\AgdaSpace{}%
\AgdaBound{e}\AgdaSpace{}%
\AgdaOperator{\AgdaInductiveConstructor{∪}}\AgdaSpace{}%
\AgdaBound{e₁}\AgdaSpace{}%
\AgdaOperator{\AgdaFunction{⟧}}\AgdaSpace{}%
\AgdaSymbol{=}\AgdaSpace{}%
\AgdaOperator{\AgdaFunction{⟦}}\AgdaSpace{}%
\AgdaBound{e}\AgdaSpace{}%
\AgdaOperator{\AgdaFunction{⟧}}\AgdaSpace{}%
\AgdaOperator{\AgdaFunction{◇.∪}}\AgdaSpace{}%
\AgdaOperator{\AgdaFunction{⟦}}\AgdaSpace{}%
\AgdaBound{e₁}\AgdaSpace{}%
\AgdaOperator{\AgdaFunction{⟧}}\<%
\\
%
\>[4]\AgdaOperator{\AgdaFunction{⟦}}\AgdaSpace{}%
\AgdaBound{e}\AgdaSpace{}%
\AgdaOperator{\AgdaInductiveConstructor{∗}}\AgdaSpace{}%
\AgdaBound{e₁}\AgdaSpace{}%
\AgdaOperator{\AgdaFunction{⟧}}\AgdaSpace{}%
\AgdaSymbol{=}\AgdaSpace{}%
\AgdaOperator{\AgdaFunction{⟦}}\AgdaSpace{}%
\AgdaBound{e}\AgdaSpace{}%
\AgdaOperator{\AgdaFunction{⟧}}\AgdaSpace{}%
\AgdaOperator{\AgdaFunction{◇.∗}}\AgdaSpace{}%
\AgdaOperator{\AgdaFunction{⟦}}\AgdaSpace{}%
\AgdaBound{e₁}\AgdaSpace{}%
\AgdaOperator{\AgdaFunction{⟧}}\<%
\end{code}

\subsection{Parsing}

To facilitate proving the inclusion of strings in a language, we start by decomposing the problem. A string is either empty or a character followed by the tail of the string. We can decompose the problem of string inclusion along the same dimensions. First, we define nullability $ν$ as the inclusion of the empty string in a language as follows:
\begin{code}%
%
\>[4]\AgdaFunction{◇ν}\AgdaSpace{}%
\AgdaSymbol{:}\AgdaSpace{}%
\AgdaFunction{Lang}\AgdaSpace{}%
\AgdaSymbol{→}\AgdaSpace{}%
\AgdaPrimitive{Type}\<%
\\
%
\>[4]\AgdaFunction{◇ν}\AgdaSpace{}%
\AgdaBound{ℒ}\AgdaSpace{}%
\AgdaSymbol{=}\AgdaSpace{}%
\AgdaBound{ℒ}\AgdaSpace{}%
\AgdaInductiveConstructor{[]}\<%
\end{code}
Second, we define the derivative $δ$ of a language $ℒ$ with respect to the character $c$ to be all the suffixes of the words in $ℒ$ which start with the $c$.
\begin{code}%
%
\>[4]\AgdaFunction{◇δ}\AgdaSpace{}%
\AgdaSymbol{:}\AgdaSpace{}%
\AgdaPostulate{Char}\AgdaSpace{}%
\AgdaSymbol{→}\AgdaSpace{}%
\AgdaFunction{Lang}\AgdaSpace{}%
\AgdaSymbol{→}\AgdaSpace{}%
\AgdaFunction{Lang}\<%
\\
%
\>[4]\AgdaFunction{◇δ}\AgdaSpace{}%
\AgdaBound{c}\AgdaSpace{}%
\AgdaBound{ℒ}\AgdaSpace{}%
\AgdaSymbol{=}\AgdaSpace{}%
\AgdaSymbol{λ}\AgdaSpace{}%
\AgdaBound{w}\AgdaSpace{}%
\AgdaSymbol{→}\AgdaSpace{}%
\AgdaBound{ℒ}\AgdaSpace{}%
\AgdaSymbol{(}\AgdaBound{c}\AgdaSpace{}%
\AgdaOperator{\AgdaInductiveConstructor{∷}}\AgdaSpace{}%
\AgdaBound{w}\AgdaSymbol{)}\<%
\end{code}
The relevance of these definitions is shown by \cref{thm:nullability-after-derivatives}.
\begin{theorem}\label{thm:nullability-after-derivatives}
Nullability after repeated derivatives fully captures what a language is. Formally, we state this as follows:
\begin{code}[hide]%
%
\>[4]\AgdaFunction{ν∘foldlδℒ≡ℒ}\AgdaSpace{}%
\AgdaSymbol{:}\<%
\end{code}
\begin{code}%
\>[4][@{}l@{\AgdaIndent{1}}]%
\>[8]\AgdaFunction{◇ν}\AgdaSpace{}%
\AgdaOperator{\AgdaFunction{∘}}\AgdaSpace{}%
\AgdaFunction{foldl}\AgdaSpace{}%
\AgdaFunction{◇δ}\AgdaSpace{}%
\AgdaGeneralizable{ℒ}\AgdaSpace{}%
\AgdaOperator{\AgdaDatatype{≡}}\AgdaSpace{}%
\AgdaGeneralizable{ℒ}\<%
\end{code}
\begin{code}[hide]%
%
\>[4]\AgdaFunction{ν∘foldlδℒ≡ℒ′}\AgdaSpace{}%
\AgdaSymbol{:}\AgdaSpace{}%
\AgdaSymbol{(}\AgdaBound{ℒ}\AgdaSpace{}%
\AgdaSymbol{:}\AgdaSpace{}%
\AgdaFunction{Lang}\AgdaSymbol{)}\AgdaSpace{}%
\AgdaSymbol{(}\AgdaBound{w}\AgdaSpace{}%
\AgdaSymbol{:}\AgdaSpace{}%
\AgdaFunction{String}\AgdaSymbol{)}\AgdaSpace{}%
\AgdaSymbol{→}\AgdaSpace{}%
\AgdaFunction{◇ν}\AgdaSpace{}%
\AgdaSymbol{(}\AgdaFunction{foldl}\AgdaSpace{}%
\AgdaFunction{◇δ}\AgdaSpace{}%
\AgdaBound{ℒ}\AgdaSpace{}%
\AgdaBound{w}\AgdaSymbol{)}\AgdaSpace{}%
\AgdaOperator{\AgdaDatatype{≡}}\AgdaSpace{}%
\AgdaBound{ℒ}\AgdaSpace{}%
\AgdaBound{w}\<%
\\
%
\>[4]\AgdaFunction{ν∘foldlδℒ≡ℒ′}\AgdaSpace{}%
\AgdaBound{ℒ}\AgdaSpace{}%
\AgdaInductiveConstructor{[]}\AgdaSpace{}%
\AgdaSymbol{=}\AgdaSpace{}%
\AgdaInductiveConstructor{refl}\<%
\\
%
\>[4]\AgdaFunction{ν∘foldlδℒ≡ℒ′}\AgdaSpace{}%
\AgdaBound{ℒ}\AgdaSpace{}%
\AgdaSymbol{(}\AgdaBound{c}\AgdaSpace{}%
\AgdaOperator{\AgdaInductiveConstructor{∷}}\AgdaSpace{}%
\AgdaBound{w}\AgdaSymbol{)}\AgdaSpace{}%
\AgdaSymbol{=}\AgdaSpace{}%
\AgdaFunction{ν∘foldlδℒ≡ℒ′}\AgdaSpace{}%
\AgdaSymbol{(}\AgdaFunction{◇δ}\AgdaSpace{}%
\AgdaBound{c}\AgdaSpace{}%
\AgdaBound{ℒ}\AgdaSymbol{)}\AgdaSpace{}%
\AgdaBound{w}\<%
\\
%
\\[\AgdaEmptyExtraSkip]%
%
\>[4]\AgdaKeyword{postulate}\AgdaSpace{}%
\AgdaPostulate{funext}\AgdaSpace{}%
\AgdaSymbol{:}\AgdaSpace{}%
\AgdaSymbol{\{}\AgdaBound{ℓ₁}\AgdaSpace{}%
\AgdaBound{ℓ₂}\AgdaSpace{}%
\AgdaSymbol{:}\AgdaSpace{}%
\AgdaPostulate{Level}\AgdaSymbol{\}}\AgdaSpace{}%
\AgdaSymbol{\{}\AgdaBound{A}\AgdaSpace{}%
\AgdaSymbol{:}\AgdaSpace{}%
\AgdaPrimitive{Set}\AgdaSpace{}%
\AgdaBound{ℓ₁}\AgdaSymbol{\}}\AgdaSpace{}%
\AgdaSymbol{\{}\AgdaBound{B}\AgdaSpace{}%
\AgdaSymbol{:}\AgdaSpace{}%
\AgdaPrimitive{Set}\AgdaSpace{}%
\AgdaBound{ℓ₂}\AgdaSymbol{\}}\AgdaSpace{}%
\AgdaSymbol{\{}\AgdaBound{P}\AgdaSpace{}%
\AgdaBound{Q}\AgdaSpace{}%
\AgdaSymbol{:}\AgdaSpace{}%
\AgdaBound{A}\AgdaSpace{}%
\AgdaSymbol{→}\AgdaSpace{}%
\AgdaBound{B}\AgdaSymbol{\}}\AgdaSpace{}%
\AgdaSymbol{→}\AgdaSpace{}%
\AgdaSymbol{((}\AgdaBound{x}\AgdaSpace{}%
\AgdaSymbol{:}\AgdaSpace{}%
\AgdaBound{A}\AgdaSymbol{)}\AgdaSpace{}%
\AgdaSymbol{→}\AgdaSpace{}%
\AgdaBound{P}\AgdaSpace{}%
\AgdaBound{x}\AgdaSpace{}%
\AgdaOperator{\AgdaDatatype{≡}}\AgdaSpace{}%
\AgdaBound{Q}\AgdaSpace{}%
\AgdaBound{x}\AgdaSymbol{)}\AgdaSpace{}%
\AgdaSymbol{→}\AgdaSpace{}%
\AgdaBound{P}\AgdaSpace{}%
\AgdaOperator{\AgdaDatatype{≡}}\AgdaSpace{}%
\AgdaBound{Q}\<%
\\
%
\\[\AgdaEmptyExtraSkip]%
%
\>[4]\AgdaFunction{ν∘foldlδℒ≡ℒ}\AgdaSpace{}%
\AgdaSymbol{\{}\AgdaArgument{ℒ}\AgdaSpace{}%
\AgdaSymbol{=}\AgdaSpace{}%
\AgdaBound{ℒ}\AgdaSymbol{\}}\AgdaSpace{}%
\AgdaSymbol{=}\AgdaSpace{}%
\AgdaPostulate{funext}\AgdaSpace{}%
\AgdaSymbol{(}\AgdaFunction{ν∘foldlδℒ≡ℒ′}\AgdaSpace{}%
\AgdaBound{ℒ}\AgdaSymbol{)}\<%
\end{code}
\end{theorem}

\begin{code}%
%
\>[4]\AgdaFunction{ν}\AgdaSpace{}%
\AgdaSymbol{:}\AgdaSpace{}%
\AgdaSymbol{(}\AgdaBound{e}\AgdaSpace{}%
\AgdaSymbol{:}\AgdaSpace{}%
\AgdaDatatype{Exp}\AgdaSymbol{)}\AgdaSpace{}%
\AgdaSymbol{→}\AgdaSpace{}%
\AgdaRecord{Dec}\AgdaSpace{}%
\AgdaSymbol{(}\AgdaFunction{◇ν}\AgdaSpace{}%
\AgdaOperator{\AgdaFunction{⟦}}\AgdaSpace{}%
\AgdaBound{e}\AgdaSpace{}%
\AgdaOperator{\AgdaFunction{⟧}}\AgdaSymbol{)}\<%
\\
%
\>[4]\AgdaFunction{δ}\AgdaSpace{}%
\AgdaSymbol{:}\AgdaSpace{}%
\AgdaPostulate{Char}\AgdaSpace{}%
\AgdaSymbol{→}\AgdaSpace{}%
\AgdaDatatype{Exp}\AgdaSpace{}%
\AgdaSymbol{→}\AgdaSpace{}%
\AgdaDatatype{Exp}\<%
\\
%
\>[4]\AgdaFunction{δ-sound}\AgdaSpace{}%
\AgdaSymbol{:}\AgdaSpace{}%
\AgdaSymbol{∀}\AgdaSpace{}%
\AgdaBound{e}\AgdaSpace{}%
\AgdaSymbol{→}\AgdaSpace{}%
\AgdaOperator{\AgdaFunction{⟦}}\AgdaSpace{}%
\AgdaFunction{δ}\AgdaSpace{}%
\AgdaGeneralizable{c}\AgdaSpace{}%
\AgdaBound{e}\AgdaSpace{}%
\AgdaOperator{\AgdaFunction{⟧}}\AgdaSpace{}%
\AgdaGeneralizable{w}\AgdaSpace{}%
\AgdaSymbol{→}\AgdaSpace{}%
\AgdaFunction{◇δ}\AgdaSpace{}%
\AgdaGeneralizable{c}\AgdaSpace{}%
\AgdaOperator{\AgdaFunction{⟦}}\AgdaSpace{}%
\AgdaBound{e}\AgdaSpace{}%
\AgdaOperator{\AgdaFunction{⟧}}\AgdaSpace{}%
\AgdaGeneralizable{w}\<%
\\
%
\>[4]\AgdaFunction{δ-complete}\AgdaSpace{}%
\AgdaSymbol{:}\AgdaSpace{}%
\AgdaSymbol{∀}\AgdaSpace{}%
\AgdaBound{e}\AgdaSpace{}%
\AgdaSymbol{→}\AgdaSpace{}%
\AgdaFunction{◇δ}\AgdaSpace{}%
\AgdaGeneralizable{c}\AgdaSpace{}%
\AgdaOperator{\AgdaFunction{⟦}}\AgdaSpace{}%
\AgdaBound{e}\AgdaSpace{}%
\AgdaOperator{\AgdaFunction{⟧}}\AgdaSpace{}%
\AgdaGeneralizable{w}\AgdaSpace{}%
\AgdaSymbol{→}\AgdaSpace{}%
\AgdaOperator{\AgdaFunction{⟦}}\AgdaSpace{}%
\AgdaFunction{δ}\AgdaSpace{}%
\AgdaGeneralizable{c}\AgdaSpace{}%
\AgdaBound{e}\AgdaSpace{}%
\AgdaOperator{\AgdaFunction{⟧}}\AgdaSpace{}%
\AgdaGeneralizable{w}\<%
\end{code}

\begin{code}[hide]%
%
\>[4]\AgdaFunction{map'}\AgdaSpace{}%
\AgdaSymbol{:}\AgdaSpace{}%
\AgdaSymbol{∀\{}\AgdaBound{A}\AgdaSpace{}%
\AgdaBound{B}\AgdaSymbol{\}}\AgdaSpace{}%
\AgdaSymbol{→}\AgdaSpace{}%
\AgdaSymbol{(}\AgdaBound{A}\AgdaSpace{}%
\AgdaSymbol{→}\AgdaSpace{}%
\AgdaBound{B}\AgdaSymbol{)}\AgdaSpace{}%
\AgdaSymbol{→}\AgdaSpace{}%
\AgdaSymbol{(}\AgdaBound{B}\AgdaSpace{}%
\AgdaSymbol{→}\AgdaSpace{}%
\AgdaBound{A}\AgdaSymbol{)}\AgdaSpace{}%
\AgdaSymbol{→}\AgdaSpace{}%
\AgdaRecord{Dec}\AgdaSpace{}%
\AgdaBound{A}\AgdaSpace{}%
\AgdaSymbol{→}\AgdaSpace{}%
\AgdaRecord{Dec}\AgdaSpace{}%
\AgdaBound{B}\<%
\\
%
\>[4]\AgdaFunction{map'}\AgdaSpace{}%
\AgdaSymbol{=}\AgdaSpace{}%
\AgdaFunction{map′}\<%
\end{code}
\begin{code}%
%
\>[4]\AgdaFunction{parse}\AgdaSpace{}%
\AgdaSymbol{:}\AgdaSpace{}%
\AgdaSymbol{(}\AgdaBound{e}\AgdaSpace{}%
\AgdaSymbol{:}\AgdaSpace{}%
\AgdaDatatype{Exp}\AgdaSymbol{)}\AgdaSpace{}%
\AgdaSymbol{(}\AgdaBound{w}\AgdaSpace{}%
\AgdaSymbol{:}\AgdaSpace{}%
\AgdaFunction{String}\AgdaSymbol{)}\AgdaSpace{}%
\AgdaSymbol{→}\AgdaSpace{}%
\AgdaRecord{Dec}\AgdaSpace{}%
\AgdaSymbol{(}\AgdaOperator{\AgdaFunction{⟦}}\AgdaSpace{}%
\AgdaBound{e}\AgdaSpace{}%
\AgdaOperator{\AgdaFunction{⟧}}\AgdaSpace{}%
\AgdaBound{w}\AgdaSymbol{)}\<%
\\
%
\>[4]\AgdaFunction{parse}\AgdaSpace{}%
\AgdaBound{e}\AgdaSpace{}%
\AgdaInductiveConstructor{[]}\AgdaSpace{}%
\AgdaSymbol{=}\AgdaSpace{}%
\AgdaFunction{ν}\AgdaSpace{}%
\AgdaBound{e}\<%
\\
%
\>[4]\AgdaFunction{parse}\AgdaSpace{}%
\AgdaBound{e}\AgdaSpace{}%
\AgdaSymbol{(}\AgdaBound{c}\AgdaSpace{}%
\AgdaOperator{\AgdaInductiveConstructor{∷}}\AgdaSpace{}%
\AgdaBound{w}\AgdaSymbol{)}\AgdaSpace{}%
\AgdaSymbol{=}\AgdaSpace{}%
\AgdaFunction{map'}\AgdaSpace{}%
\AgdaSymbol{(}\AgdaFunction{δ-sound}\AgdaSpace{}%
\AgdaBound{e}\AgdaSymbol{)}\AgdaSpace{}%
\AgdaSymbol{(}\AgdaFunction{δ-complete}\AgdaSpace{}%
\AgdaBound{e}\AgdaSymbol{)}\AgdaSpace{}%
\AgdaSymbol{(}\AgdaFunction{parse}\AgdaSpace{}%
\AgdaSymbol{(}\AgdaFunction{δ}\AgdaSpace{}%
\AgdaBound{c}\AgdaSpace{}%
\AgdaBound{e}\AgdaSymbol{)}\AgdaSpace{}%
\AgdaBound{w}\AgdaSymbol{)}\<%
\end{code}

\subsection{Nullability}

\begin{lemma}
Two languages, $\ab{ℒ₁}$ and $\ab{ℒ₂}$, are nullable if and only if their concatenation, $\ab{ℒ₁}~\af{◇.∗}~\ab{ℒ₂}$, is nullable. 
\begin{code}%
%
\>[4]\AgdaFunction{ν∗}\AgdaSpace{}%
\AgdaSymbol{:}\AgdaSpace{}%
\AgdaSymbol{(}\AgdaFunction{◇ν}\AgdaSpace{}%
\AgdaGeneralizable{ℒ₁}\AgdaSpace{}%
\AgdaOperator{\AgdaFunction{×}}\AgdaSpace{}%
\AgdaFunction{◇ν}\AgdaSpace{}%
\AgdaGeneralizable{ℒ₂}\AgdaSymbol{)}\AgdaSpace{}%
\AgdaOperator{\AgdaFunction{⇔}}\AgdaSpace{}%
\AgdaFunction{◇ν}\AgdaSpace{}%
\AgdaSymbol{(}\AgdaGeneralizable{ℒ₁}\AgdaSpace{}%
\AgdaOperator{\AgdaFunction{◇.∗}}\AgdaSpace{}%
\AgdaGeneralizable{ℒ₂}\AgdaSymbol{)}\<%
\end{code}
\begin{code}[hide]%
%
\>[4]\AgdaFunction{ν∗}\AgdaSpace{}%
\AgdaSymbol{=}\AgdaSpace{}%
\AgdaFunction{mk⇔}\AgdaSpace{}%
\AgdaSymbol{(λ}\AgdaSpace{}%
\AgdaBound{x}\AgdaSpace{}%
\AgdaSymbol{→}\AgdaSpace{}%
\AgdaInductiveConstructor{[]}\AgdaSpace{}%
\AgdaOperator{\AgdaInductiveConstructor{,}}\AgdaSpace{}%
\AgdaInductiveConstructor{[]}\AgdaSpace{}%
\AgdaOperator{\AgdaInductiveConstructor{,}}\AgdaSpace{}%
\AgdaInductiveConstructor{refl}\AgdaSpace{}%
\AgdaOperator{\AgdaInductiveConstructor{,}}\AgdaSpace{}%
\AgdaBound{x}\AgdaSymbol{)}\AgdaSpace{}%
\AgdaSymbol{λ}\AgdaSpace{}%
\AgdaSymbol{\{}\AgdaSpace{}%
\AgdaSymbol{(}\AgdaInductiveConstructor{[]}\AgdaSpace{}%
\AgdaOperator{\AgdaInductiveConstructor{,}}\AgdaSpace{}%
\AgdaInductiveConstructor{[]}\AgdaSpace{}%
\AgdaOperator{\AgdaInductiveConstructor{,}}\AgdaSpace{}%
\AgdaInductiveConstructor{refl}\AgdaSpace{}%
\AgdaOperator{\AgdaInductiveConstructor{,}}\AgdaSpace{}%
\AgdaBound{x}\AgdaSymbol{)}\AgdaSpace{}%
\AgdaSymbol{→}\AgdaSpace{}%
\AgdaBound{x}\AgdaSpace{}%
\AgdaSymbol{\}}\<%
\end{code}
\end{lemma}

\begin{code}%
%
\>[4]\AgdaFunction{ν}\AgdaSpace{}%
\AgdaInductiveConstructor{∅}\AgdaSpace{}%
\AgdaSymbol{=}\AgdaSpace{}%
\AgdaInductiveConstructor{no}\AgdaSpace{}%
\AgdaSymbol{λ}\AgdaSpace{}%
\AgdaSymbol{()}\<%
\\
%
\>[4]\AgdaFunction{ν}\AgdaSpace{}%
\AgdaInductiveConstructor{ε}\AgdaSpace{}%
\AgdaSymbol{=}\AgdaSpace{}%
\AgdaInductiveConstructor{yes}\AgdaSpace{}%
\AgdaInductiveConstructor{refl}\<%
\\
%
\>[4]\AgdaFunction{ν}\AgdaSpace{}%
\AgdaSymbol{(}\AgdaOperator{\AgdaInductiveConstructor{`}}\AgdaSpace{}%
\AgdaBound{c}\AgdaSymbol{)}\AgdaSpace{}%
\AgdaSymbol{=}\AgdaSpace{}%
\AgdaInductiveConstructor{no}\AgdaSpace{}%
\AgdaSymbol{λ}\AgdaSpace{}%
\AgdaSymbol{()}\<%
\\
%
\>[4]\AgdaFunction{ν}\AgdaSpace{}%
\AgdaSymbol{(}\AgdaBound{x}\AgdaSpace{}%
\AgdaOperator{\AgdaInductiveConstructor{·}}\AgdaSpace{}%
\AgdaBound{e}\AgdaSymbol{)}\AgdaSpace{}%
\AgdaSymbol{=}\AgdaSpace{}%
\AgdaBound{x}\AgdaSpace{}%
\AgdaOperator{\AgdaFunction{×-dec}}\AgdaSpace{}%
\AgdaFunction{ν}\AgdaSpace{}%
\AgdaBound{e}\<%
\\
%
\>[4]\AgdaFunction{ν}\AgdaSpace{}%
\AgdaSymbol{(}\AgdaBound{e}\AgdaSpace{}%
\AgdaOperator{\AgdaInductiveConstructor{∪}}\AgdaSpace{}%
\AgdaBound{e₁}\AgdaSymbol{)}\AgdaSpace{}%
\AgdaSymbol{=}\AgdaSpace{}%
\AgdaFunction{ν}\AgdaSpace{}%
\AgdaBound{e}\AgdaSpace{}%
\AgdaOperator{\AgdaFunction{⊎-dec}}\AgdaSpace{}%
\AgdaFunction{ν}\AgdaSpace{}%
\AgdaBound{e₁}\<%
\\
%
\>[4]\AgdaFunction{ν}\AgdaSpace{}%
\AgdaSymbol{(}\AgdaBound{e}\AgdaSpace{}%
\AgdaOperator{\AgdaInductiveConstructor{∗}}\AgdaSpace{}%
\AgdaBound{e₁}\AgdaSymbol{)}\AgdaSpace{}%
\AgdaSymbol{=}\AgdaSpace{}%
\AgdaFunction{Dec.map}\AgdaSpace{}%
\AgdaFunction{ν∗}\AgdaSpace{}%
\AgdaSymbol{(}\AgdaFunction{ν}\AgdaSpace{}%
\AgdaBound{e}\AgdaSpace{}%
\AgdaOperator{\AgdaFunction{×-dec}}\AgdaSpace{}%
\AgdaFunction{ν}\AgdaSpace{}%
\AgdaBound{e₁}\AgdaSymbol{)}\<%
\end{code}

\subsection{Derivation}

\begin{code}%
%
\>[4]\AgdaFunction{δ}\AgdaSpace{}%
\AgdaBound{c}\AgdaSpace{}%
\AgdaInductiveConstructor{∅}\AgdaSpace{}%
\AgdaSymbol{=}\AgdaSpace{}%
\AgdaInductiveConstructor{∅}\<%
\\
%
\>[4]\AgdaFunction{δ}\AgdaSpace{}%
\AgdaBound{c}\AgdaSpace{}%
\AgdaInductiveConstructor{ε}\AgdaSpace{}%
\AgdaSymbol{=}\AgdaSpace{}%
\AgdaInductiveConstructor{∅}\<%
\\
%
\>[4]\AgdaFunction{δ}\AgdaSpace{}%
\AgdaBound{c}\AgdaSpace{}%
\AgdaSymbol{(}\AgdaOperator{\AgdaInductiveConstructor{`}}\AgdaSpace{}%
\AgdaBound{c₁}\AgdaSymbol{)}\AgdaSpace{}%
\AgdaSymbol{=}\AgdaSpace{}%
\AgdaSymbol{(}\AgdaBound{c}\AgdaSpace{}%
\AgdaOperator{\AgdaFunction{≟}}\AgdaSpace{}%
\AgdaBound{c₁}\AgdaSymbol{)}\AgdaSpace{}%
\AgdaOperator{\AgdaInductiveConstructor{·}}\AgdaSpace{}%
\AgdaInductiveConstructor{ε}\AgdaSpace{}%
\AgdaComment{--\ a\ bit\ interesting}\<%
\\
%
\>[4]\AgdaFunction{δ}\AgdaSpace{}%
\AgdaBound{c}\AgdaSpace{}%
\AgdaSymbol{(}\AgdaBound{x}\AgdaSpace{}%
\AgdaOperator{\AgdaInductiveConstructor{·}}\AgdaSpace{}%
\AgdaBound{e}\AgdaSymbol{)}\AgdaSpace{}%
\AgdaSymbol{=}\AgdaSpace{}%
\AgdaBound{x}\AgdaSpace{}%
\AgdaOperator{\AgdaInductiveConstructor{·}}\AgdaSpace{}%
\AgdaFunction{δ}\AgdaSpace{}%
\AgdaBound{c}\AgdaSpace{}%
\AgdaBound{e}\<%
\\
%
\>[4]\AgdaFunction{δ}\AgdaSpace{}%
\AgdaBound{c}\AgdaSpace{}%
\AgdaSymbol{(}\AgdaBound{e}\AgdaSpace{}%
\AgdaOperator{\AgdaInductiveConstructor{∪}}\AgdaSpace{}%
\AgdaBound{e₁}\AgdaSymbol{)}\AgdaSpace{}%
\AgdaSymbol{=}\AgdaSpace{}%
\AgdaFunction{δ}\AgdaSpace{}%
\AgdaBound{c}\AgdaSpace{}%
\AgdaBound{e}\AgdaSpace{}%
\AgdaOperator{\AgdaInductiveConstructor{∪}}\AgdaSpace{}%
\AgdaFunction{δ}\AgdaSpace{}%
\AgdaBound{c}\AgdaSpace{}%
\AgdaBound{e₁}\<%
\\
%
\>[4]\AgdaFunction{δ}\AgdaSpace{}%
\AgdaBound{c}\AgdaSpace{}%
\AgdaSymbol{(}\AgdaBound{e}\AgdaSpace{}%
\AgdaOperator{\AgdaInductiveConstructor{∗}}\AgdaSpace{}%
\AgdaBound{e₁}\AgdaSymbol{)}\AgdaSpace{}%
\AgdaSymbol{=}\AgdaSpace{}%
\AgdaSymbol{(}\AgdaFunction{δ}\AgdaSpace{}%
\AgdaBound{c}\AgdaSpace{}%
\AgdaBound{e}\AgdaSpace{}%
\AgdaOperator{\AgdaInductiveConstructor{∗}}\AgdaSpace{}%
\AgdaBound{e₁}\AgdaSymbol{)}\AgdaSpace{}%
\AgdaOperator{\AgdaInductiveConstructor{∪}}\AgdaSpace{}%
\AgdaSymbol{(}\AgdaFunction{ν}\AgdaSpace{}%
\AgdaBound{e}\AgdaSpace{}%
\AgdaOperator{\AgdaInductiveConstructor{·}}\AgdaSpace{}%
\AgdaFunction{δ}\AgdaSpace{}%
\AgdaBound{c}\AgdaSpace{}%
\AgdaBound{e₁}\AgdaSymbol{)}\AgdaSpace{}%
\AgdaComment{--\ interesting}\<%
\end{code}

The proofs are very straightforward:

\begin{code}%
%
\>[4]\AgdaFunction{δ-sound}\AgdaSpace{}%
\AgdaSymbol{(}\AgdaOperator{\AgdaInductiveConstructor{`}}\AgdaSpace{}%
\AgdaBound{c}\AgdaSymbol{)}\AgdaSpace{}%
\AgdaSymbol{(}\AgdaInductiveConstructor{refl}\AgdaSpace{}%
\AgdaOperator{\AgdaInductiveConstructor{,}}\AgdaSpace{}%
\AgdaInductiveConstructor{refl}\AgdaSymbol{)}\AgdaSpace{}%
\AgdaSymbol{=}\AgdaSpace{}%
\AgdaInductiveConstructor{refl}\<%
\\
%
\>[4]\AgdaFunction{δ-sound}\AgdaSpace{}%
\AgdaSymbol{(}\AgdaBound{x₁}\AgdaSpace{}%
\AgdaOperator{\AgdaInductiveConstructor{·}}\AgdaSpace{}%
\AgdaBound{e}\AgdaSymbol{)}\AgdaSpace{}%
\AgdaSymbol{(}\AgdaBound{x}\AgdaSpace{}%
\AgdaOperator{\AgdaInductiveConstructor{,}}\AgdaSpace{}%
\AgdaBound{y}\AgdaSymbol{)}\AgdaSpace{}%
\AgdaSymbol{=}\AgdaSpace{}%
\AgdaBound{x}\AgdaSpace{}%
\AgdaOperator{\AgdaInductiveConstructor{,}}\AgdaSpace{}%
\AgdaFunction{δ-sound}\AgdaSpace{}%
\AgdaBound{e}\AgdaSpace{}%
\AgdaBound{y}\<%
\\
%
\>[4]\AgdaFunction{δ-sound}\AgdaSpace{}%
\AgdaSymbol{(}\AgdaBound{e}\AgdaSpace{}%
\AgdaOperator{\AgdaInductiveConstructor{∪}}\AgdaSpace{}%
\AgdaBound{e₁}\AgdaSymbol{)}\AgdaSpace{}%
\AgdaSymbol{(}\AgdaInductiveConstructor{inj₁}\AgdaSpace{}%
\AgdaBound{x}\AgdaSymbol{)}\AgdaSpace{}%
\AgdaSymbol{=}\AgdaSpace{}%
\AgdaInductiveConstructor{inj₁}\AgdaSpace{}%
\AgdaSymbol{(}\AgdaFunction{δ-sound}\AgdaSpace{}%
\AgdaBound{e}\AgdaSpace{}%
\AgdaBound{x}\AgdaSymbol{)}\<%
\\
%
\>[4]\AgdaFunction{δ-sound}\AgdaSpace{}%
\AgdaSymbol{(}\AgdaBound{e}\AgdaSpace{}%
\AgdaOperator{\AgdaInductiveConstructor{∪}}\AgdaSpace{}%
\AgdaBound{e₁}\AgdaSymbol{)}\AgdaSpace{}%
\AgdaSymbol{(}\AgdaInductiveConstructor{inj₂}\AgdaSpace{}%
\AgdaBound{y}\AgdaSymbol{)}\AgdaSpace{}%
\AgdaSymbol{=}\AgdaSpace{}%
\AgdaInductiveConstructor{inj₂}\AgdaSpace{}%
\AgdaSymbol{(}\AgdaFunction{δ-sound}\AgdaSpace{}%
\AgdaBound{e₁}\AgdaSpace{}%
\AgdaBound{y}\AgdaSymbol{)}\<%
\\
%
\>[4]\AgdaFunction{δ-sound}\AgdaSpace{}%
\AgdaSymbol{(}\AgdaBound{e}\AgdaSpace{}%
\AgdaOperator{\AgdaInductiveConstructor{∗}}\AgdaSpace{}%
\AgdaBound{e₁}\AgdaSymbol{)}\AgdaSpace{}%
\AgdaSymbol{(}\AgdaInductiveConstructor{inj₁}\AgdaSpace{}%
\AgdaSymbol{(}\AgdaBound{u}\AgdaSpace{}%
\AgdaOperator{\AgdaInductiveConstructor{,}}\AgdaSpace{}%
\AgdaBound{v}\AgdaSpace{}%
\AgdaOperator{\AgdaInductiveConstructor{,}}\AgdaSpace{}%
\AgdaInductiveConstructor{refl}\AgdaSpace{}%
\AgdaOperator{\AgdaInductiveConstructor{,}}\AgdaSpace{}%
\AgdaBound{x}\AgdaSpace{}%
\AgdaOperator{\AgdaInductiveConstructor{,}}\AgdaSpace{}%
\AgdaBound{y}\AgdaSymbol{))}\AgdaSpace{}%
\AgdaSymbol{=}\AgdaSpace{}%
\AgdaSymbol{\AgdaUnderscore{}}\AgdaSpace{}%
\AgdaOperator{\AgdaInductiveConstructor{∷}}\AgdaSpace{}%
\AgdaBound{u}\AgdaSpace{}%
\AgdaOperator{\AgdaInductiveConstructor{,}}\AgdaSpace{}%
\AgdaBound{v}\AgdaSpace{}%
\AgdaOperator{\AgdaInductiveConstructor{,}}\AgdaSpace{}%
\AgdaInductiveConstructor{refl}\AgdaSpace{}%
\AgdaOperator{\AgdaInductiveConstructor{,}}\AgdaSpace{}%
\AgdaFunction{δ-sound}\AgdaSpace{}%
\AgdaBound{e}\AgdaSpace{}%
\AgdaBound{x}\AgdaSpace{}%
\AgdaOperator{\AgdaInductiveConstructor{,}}\AgdaSpace{}%
\AgdaBound{y}\<%
\\
%
\>[4]\AgdaFunction{δ-sound}\AgdaSpace{}%
\AgdaSymbol{(}\AgdaBound{e}\AgdaSpace{}%
\AgdaOperator{\AgdaInductiveConstructor{∗}}\AgdaSpace{}%
\AgdaBound{e₁}\AgdaSymbol{)}\AgdaSpace{}%
\AgdaSymbol{(}\AgdaInductiveConstructor{inj₂}\AgdaSpace{}%
\AgdaSymbol{(}\AgdaBound{x}\AgdaSpace{}%
\AgdaOperator{\AgdaInductiveConstructor{,}}\AgdaSpace{}%
\AgdaBound{y}\AgdaSymbol{))}\AgdaSpace{}%
\AgdaSymbol{=}\AgdaSpace{}%
\AgdaInductiveConstructor{[]}\AgdaSpace{}%
\AgdaOperator{\AgdaInductiveConstructor{,}}\AgdaSpace{}%
\AgdaSymbol{\AgdaUnderscore{}}\AgdaSpace{}%
\AgdaOperator{\AgdaInductiveConstructor{,}}\AgdaSpace{}%
\AgdaInductiveConstructor{refl}\AgdaSpace{}%
\AgdaOperator{\AgdaInductiveConstructor{,}}\AgdaSpace{}%
\AgdaBound{x}\AgdaSpace{}%
\AgdaOperator{\AgdaInductiveConstructor{,}}\AgdaSpace{}%
\AgdaFunction{δ-sound}\AgdaSpace{}%
\AgdaBound{e₁}\AgdaSpace{}%
\AgdaBound{y}\<%
\end{code}

\begin{code}%
%
\>[4]\AgdaFunction{δ-complete}\AgdaSpace{}%
\AgdaSymbol{(}\AgdaOperator{\AgdaInductiveConstructor{`}}\AgdaSpace{}%
\AgdaBound{c}\AgdaSymbol{)}\AgdaSpace{}%
\AgdaInductiveConstructor{refl}\AgdaSpace{}%
\AgdaSymbol{=}\AgdaSpace{}%
\AgdaInductiveConstructor{refl}\AgdaSpace{}%
\AgdaOperator{\AgdaInductiveConstructor{,}}\AgdaSpace{}%
\AgdaInductiveConstructor{refl}\<%
\\
%
\>[4]\AgdaFunction{δ-complete}\AgdaSpace{}%
\AgdaSymbol{(}\AgdaBound{x₁}\AgdaSpace{}%
\AgdaOperator{\AgdaInductiveConstructor{·}}\AgdaSpace{}%
\AgdaBound{e}\AgdaSymbol{)}\AgdaSpace{}%
\AgdaSymbol{(}\AgdaBound{x}\AgdaSpace{}%
\AgdaOperator{\AgdaInductiveConstructor{,}}\AgdaSpace{}%
\AgdaBound{y}\AgdaSymbol{)}\AgdaSpace{}%
\AgdaSymbol{=}\AgdaSpace{}%
\AgdaBound{x}\AgdaSpace{}%
\AgdaOperator{\AgdaInductiveConstructor{,}}\AgdaSpace{}%
\AgdaFunction{δ-complete}\AgdaSpace{}%
\AgdaBound{e}\AgdaSpace{}%
\AgdaBound{y}\<%
\\
%
\>[4]\AgdaFunction{δ-complete}\AgdaSpace{}%
\AgdaSymbol{(}\AgdaBound{e}\AgdaSpace{}%
\AgdaOperator{\AgdaInductiveConstructor{∪}}\AgdaSpace{}%
\AgdaBound{e₁}\AgdaSymbol{)}\AgdaSpace{}%
\AgdaSymbol{(}\AgdaInductiveConstructor{inj₁}\AgdaSpace{}%
\AgdaBound{x}\AgdaSymbol{)}\AgdaSpace{}%
\AgdaSymbol{=}\AgdaSpace{}%
\AgdaInductiveConstructor{inj₁}\AgdaSpace{}%
\AgdaSymbol{(}\AgdaFunction{δ-complete}\AgdaSpace{}%
\AgdaBound{e}\AgdaSpace{}%
\AgdaBound{x}\AgdaSymbol{)}\<%
\\
%
\>[4]\AgdaFunction{δ-complete}\AgdaSpace{}%
\AgdaSymbol{(}\AgdaBound{e}\AgdaSpace{}%
\AgdaOperator{\AgdaInductiveConstructor{∪}}\AgdaSpace{}%
\AgdaBound{e₁}\AgdaSymbol{)}\AgdaSpace{}%
\AgdaSymbol{(}\AgdaInductiveConstructor{inj₂}\AgdaSpace{}%
\AgdaBound{y}\AgdaSymbol{)}\AgdaSpace{}%
\AgdaSymbol{=}\AgdaSpace{}%
\AgdaInductiveConstructor{inj₂}\AgdaSpace{}%
\AgdaSymbol{(}\AgdaFunction{δ-complete}\AgdaSpace{}%
\AgdaBound{e₁}\AgdaSpace{}%
\AgdaBound{y}\AgdaSymbol{)}\<%
\\
%
\>[4]\AgdaFunction{δ-complete}\AgdaSpace{}%
\AgdaSymbol{(}\AgdaBound{e}\AgdaSpace{}%
\AgdaOperator{\AgdaInductiveConstructor{∗}}\AgdaSpace{}%
\AgdaBound{e₁}\AgdaSymbol{)}\AgdaSpace{}%
\AgdaSymbol{(\AgdaUnderscore{}}\AgdaSpace{}%
\AgdaOperator{\AgdaInductiveConstructor{∷}}\AgdaSpace{}%
\AgdaSymbol{\AgdaUnderscore{}}\AgdaSpace{}%
\AgdaOperator{\AgdaInductiveConstructor{,}}\AgdaSpace{}%
\AgdaSymbol{\AgdaUnderscore{}}\AgdaSpace{}%
\AgdaOperator{\AgdaInductiveConstructor{,}}\AgdaSpace{}%
\AgdaInductiveConstructor{refl}\AgdaSpace{}%
\AgdaOperator{\AgdaInductiveConstructor{,}}\AgdaSpace{}%
\AgdaBound{x}\AgdaSpace{}%
\AgdaOperator{\AgdaInductiveConstructor{,}}\AgdaSpace{}%
\AgdaBound{y}\AgdaSymbol{)}\AgdaSpace{}%
\AgdaSymbol{=}\AgdaSpace{}%
\AgdaInductiveConstructor{inj₁}\AgdaSpace{}%
\AgdaSymbol{(\AgdaUnderscore{}}\AgdaSpace{}%
\AgdaOperator{\AgdaInductiveConstructor{,}}\AgdaSpace{}%
\AgdaSymbol{\AgdaUnderscore{}}\AgdaSpace{}%
\AgdaOperator{\AgdaInductiveConstructor{,}}\AgdaSpace{}%
\AgdaInductiveConstructor{refl}\AgdaSpace{}%
\AgdaOperator{\AgdaInductiveConstructor{,}}\AgdaSpace{}%
\AgdaFunction{δ-complete}\AgdaSpace{}%
\AgdaBound{e}\AgdaSpace{}%
\AgdaBound{x}\AgdaSpace{}%
\AgdaOperator{\AgdaInductiveConstructor{,}}\AgdaSpace{}%
\AgdaBound{y}\AgdaSymbol{)}\<%
\\
%
\>[4]\AgdaFunction{δ-complete}\AgdaSpace{}%
\AgdaSymbol{(}\AgdaBound{e}\AgdaSpace{}%
\AgdaOperator{\AgdaInductiveConstructor{∗}}\AgdaSpace{}%
\AgdaBound{e₁}\AgdaSymbol{)}\AgdaSpace{}%
\AgdaSymbol{(}\AgdaInductiveConstructor{[]}\AgdaSpace{}%
\AgdaOperator{\AgdaInductiveConstructor{,}}\AgdaSpace{}%
\AgdaSymbol{\AgdaUnderscore{}}\AgdaSpace{}%
\AgdaOperator{\AgdaInductiveConstructor{,}}\AgdaSpace{}%
\AgdaInductiveConstructor{refl}\AgdaSpace{}%
\AgdaOperator{\AgdaInductiveConstructor{,}}\AgdaSpace{}%
\AgdaBound{x}\AgdaSpace{}%
\AgdaOperator{\AgdaInductiveConstructor{,}}\AgdaSpace{}%
\AgdaBound{y}\AgdaSymbol{)}\AgdaSpace{}%
\AgdaSymbol{=}\AgdaSpace{}%
\AgdaInductiveConstructor{inj₂}\AgdaSpace{}%
\AgdaSymbol{(}\AgdaBound{x}\AgdaSpace{}%
\AgdaOperator{\AgdaInductiveConstructor{,}}\AgdaSpace{}%
\AgdaFunction{δ-complete}\AgdaSpace{}%
\AgdaBound{e₁}\AgdaSpace{}%
\AgdaBound{y}\AgdaSymbol{)}\<%
\end{code}

% \begin{code}[hide]
% module Simple where
% \end{code}
% \begin{code}
%     data Gram : Lang → Type₁ where
%         ∅     :                       Gram (λ _ → ⊥)
%         ε     :                       Gram (λ w → w ≡ [])
%         char  : (c : Char)         →  Gram (λ w → w ≡ c ∷ [])
%         _·_   : Dec A → Gram ℒ     →  Gram (λ w → A × ℒ w)
%         _∪_   : Gram ℒ₁ → Gram ℒ₂  →  Gram (λ w → ℒ₁ w ⊎ ℒ₂ w)
%         _∗_   : Gram ℒ₁ → Gram ℒ₂
%               → Gram (λ w → Σ String λ u → Σ String λ v → (w ≡ u ++ v) × ℒ₁ u × ℒ₂ v)
%         _◃_   : (ℒ₁ ⇔ ℒ₂) → Gram ℒ₁ → Gram ℒ₂
% \end{code}
% \begin{code}[hide]
%     variable G G₁ G₂ : Gram ℒ
% \end{code}
% \begin{remark}
% The \af{Gram} data type is parameterized by its language. This ties the constructors directly to their semantics.
% \end{remark}
% 
% By recursion over this data type of grammars, we can define a decision procedure for nullability and derivative function; both are correct by construction.
% \begin{code}
%     ν? : Gram ℒ → Dec (ν ℒ)
%     δ? : Gram ℒ → (c : Char) → Gram (δ ℒ c)
% \end{code}
% \begin{code}[hide]
%     ν∗ : (ν ℒ₁ × ν ℒ₂) ↔ Σ String λ u → Σ String λ v → ([] ≡ (u ++ v)) × ℒ₁ u × ℒ₂ v
%     to ν∗ (x , y) = [] , [] , refl , x , y
%     from ν∗ ([] , [] , refl , x , y) = x , y
% 
%     ν? ∅ = no λ ()
%     ν? ε = yes refl
%     ν? (char c) = no λ ()
%     ν? (x · G) = x ×? ν? G
%     ν? (G₁ ∪ G₂) = ν? G₁ ⊎? ν? G₂
%     ν? (G₁ ∗ G₂) = map? ν∗ (ν? G₁ ×? ν? G₂)
%     ν? (f ◃ G₂) = map? f (ν? G₂)
% \end{code}
% \begin{code}[hide]
%     δ? ∅ c = ∅
%     δ? ε c = record { to = λ () ; from = λ () } ◃ ∅
%     δ? (char c′) c with c ≟ c′
%     ... | yes refl = (λ { {[]} → record { to = λ _ → refl ; from = λ _ → refl } ; {_ ∷ _} → record { to = λ () ; from = λ () }}) ◃ ε
%     ... | no ¬c≡c′ = (λ { {[]} → record { to = λ () ; from = λ { refl → ¬c≡c′ refl }} ; {_ ∷ _} → record { to = λ () ; from = λ () }}) ◃ ∅
%     δ? (A · G) c = A · δ? G c
%     δ? (G₁ ∪ G₂) c = δ? G₁ c ∪ δ? G₂ c
%     δ? (G₁ ∗ G₂) c = (record { to = λ { (inl (u , v , refl , x , y)) → (c ∷ u) , v , refl , x , y ; (inr (x , y)) → [] , (c ∷ _) , refl , x , y } ; from = λ { ([] , _ , refl , x , y) → inr (x , y) ; ((_ ∷ u) , v , refl , x , y) → inl (u , v , refl , x , y) } } ) ◃ ((δ? G₁ c ∗ G₂) ∪ (ν? G₁ · δ? G₂ c))
%     δ? (f ◃ G₂) c = f ◃ δ? G₂ c
% 
%     -- δ?↔δ : ⟦ δ? c G ⟧ w ↔ δ c ⟦ G ⟧ w
% \end{code}
% \begin{code}[hide]
%     -- to (δ?↔δ {c} {G = ` c′}) x with c ≟ c′
%     -- to (δ?↔δ {c} {` .c}) refl | yes refl = refl
%     -- to (δ?↔δ {_} {` _}) () | no _
%     -- to (δ?↔δ {G = A · G}) (x , y) = x , to δ?↔δ y
%     -- to (δ?↔δ {G = G₁ ∪ G₂}) (inl x) = inl (to δ?↔δ x)
%     -- to (δ?↔δ {G = G₁ ∪ G₂}) (inr x) = inr (to δ?↔δ x)
%     -- to (δ?↔δ {c} {G = G₁ ▹ G₂}) (inl (u , v , refl , x , y)) = (c ∷ u) , v , refl , to δ?↔δ x , y
%     -- to (δ?↔δ {c} {G = G₁ ▹ G₂} {w}) (inr (π₁ , π₂)) = [] , (c ∷ w) , refl , π₁ , to δ?↔δ π₂
%     -- from (δ?↔δ {c} {G = ` c′}) x with c ≟ c′
%     -- from (δ?↔δ {c} {` c}) refl | yes refl = refl
%     -- from (δ?↔δ {c} {` .c}) refl | no ¬c≡c = ¬c≡c refl
%     -- from (δ?↔δ {G = A · G}) (π₁ , π₂) = π₁ , from δ?↔δ π₂
%     -- from (δ?↔δ {G = G ∪ G₁}) (inl x) = inl (from δ?↔δ x)
%     -- from (δ?↔δ {G = G ∪ G₁}) (inr x) = inr (from δ?↔δ x)
%     -- from (δ?↔δ {c} {G = G ▹ G₁}) ([] , (.c ∷ v) , refl , x , y) = inr (x , from δ?↔δ y)
%     -- from (δ?↔δ {c} {G = G ▹ G₁}) ((.c ∷ u) , v , refl , x , y) = inl (u , v , refl , from δ?↔δ x , y)
%     transport : {ℓ₁ : Level} {A : Set ℓ₁} {B : Set ℓ₁} → A ≡ B → A → B
%     transport refl x = x
% \end{code}
% Together, decidable nullability and the derivative function can be combined to decide whether any string is in the language described by a grammar.
% \begin{code}
%     parse : Gram ℒ → (w : String) → Dec (ℒ w)
%     parse G [] = ν? G
%     parse G (c ∷ w) = parse (δ? G c) w
% \end{code}
% Thus, we have defined a parser for our simple grammars.

% A language is a set of strings $\mathbb{2}^{(\af{List}~\af{Token})}$.
% 
% 
% \begin{code}[hide]
% Lang : Set₁
% \end{code}
% \begin{code}
% Lang = List Token → Set
% \end{code}
% 
% This type has a very rich structure. It forms an ... algebra with union and intersection and a semiring with union and sequential composition.
% 
% \begin{code}
% ∅ : Lang
% ∅ _ = ⊥
% \end{code}
% 
% Going beyond work by Elliot, we can try to define context-free grammars.
% Unfortunately, we quickly run into issues due to nontermination. It is not easy
% to show that a grammar defined in this way is well-founded. To solve this issue
% we can use guarded type theory, in our case provided by guarded cubical Agda.
% This allows us to define arbitrary fixed points of languages.
% 
% \begin{code}
% fueled : (Lang → Lang) → ℕ → Lang
% fueled f 0 = ∅
% fueled f (suc n) = f (fueled f n)
% \end{code}
% 
% \begin{code}
% fix : (Lang → Lang) → Lang
% fix f w = ∃[ n ] fueled f n w
% \end{code}

% \input{src/sections/3-syntax.tex}
% \input{src/sections/4-sharing.tex}
% \input{src/sections/6-case.tex}
\input{src/sections/7-related.tex}
% \input{src/sections/8-conclusion.tex}
%
% TODO:
%
% \begin{credits}
% \subsubsection{\ackname} A bold run-in heading in small font size at the end of the paper is
% used for general acknowledgments, for example: This study was funded
% by X (grant number Y).
% 
% \subsubsection{\discintname}
% It is now necessary to declare any competing interests or to specifically
% state that the authors have no competing interests. Please place the
% statement with a bold run-in heading in small font size beneath the
% (optional) acknowledgments\footnote{If EquinOCS, our proceedings submission
% system, is used, then the disclaimer can be provided directly in the system.},
% for example: The authors have no competing interests to declare that are
% relevant to the content of this article. Or: Author A has received research
% grants from Company W. Author B has received a speaker honorarium from
% Company X and owns stock in Company Y. Author C is a member of committee Z.
% \end{credits}
%
% ---- Bibliography ----
%
% BibTeX users should specify bibliography style 'splncs04'.
% References will then be sorted and formatted in the correct style.

\bibliographystyle{splncs04}
\bibliography{src/references.bib}

\end{document}
