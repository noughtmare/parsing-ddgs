% This is samplepaper.tex, a sample chapter demonstrating the
% LLNCS macro package for Springer Computer Science proceedings;
% Version 2.21 of 2022/01/12
%
\documentclass[runningheads]{llncs}
%
\usepackage[T1]{fontenc}
% T1 fonts will be used to generate the final print and online PDFs,
% so please use T1 fonts in your manuscript whenever possible.
% Other font encondings may result in incorrect characters.
%
\usepackage{graphicx}
% Used for displaying a sample figure. If possible, figure files should
% be included in EPS format.
%
% If you use the hyperref package, please uncomment the following two lines
% to display URLs in blue roman font according to Springer's eBook style:
%\usepackage{color}
%\renewcommand\UrlFont{\color{blue}\rmfamily}
%\urlstyle{rm}
\usepackage[utf8]{inputenc}
\usepackage{graphicx}
\usepackage{stmaryrd}
\usepackage{amsmath}
\usepackage{amssymb}
\usepackage{multicol}
\usepackage{xparse}
\usepackage{hyperref}
\usepackage[capitalise,noabbrev]{cleveref}
\usepackage{listings}
\usepackage{mathpartir}
\usepackage{xcolor}
\usepackage{mathtools}
\usepackage{wasysym}
\usepackage{agda}
\usepackage{src/agdadimmed}
% \usepackage{src/agdatheme}
\usepackage{newunicodechar}
\usepackage{mdframed}
\usepackage{wrapfig}
\usepackage{boxedminipage}
\usepackage{dsfont}
\usepackage{etoolbox}
\usepackage{scalerel}
\usepackage{graphicx}
\usepackage{mathbbol}
\usepackage{lineno}
\usepackage{setspace} % for GS course
\usepackage{ifthen}
\usepackage{tikz-cd} 
\usepackage{fontspec} 
\usepackage{unicode-math} 
%
\newtoggle{draft}
\toggletrue{draft} % enables notes

\iftoggle{draft}{
\linenumbers
% \doublespacing
}

% depends on toggle draft being set

\RequirePackage{geometry}
\iftoggle{draft}
{
  \newgeometry{twoside=true,
     includeheadfoot, head=13pt, foot=2pc,
     paperwidth=\dimexpr 6.75in + 6cm \relax, paperheight=10in,
     top=58pt, bottom=44pt, inner=46pt, outer=\dimexpr 46pt + 5cm\relax,
     marginparwidth=6cm,heightrounded
  }
  % \paperwidth=\dimexpr\paperwidth+6cm\relax
  % \evensidemargin=\dimexpr\evensidemargin+5cm\relax
  % \marginparwidth=\dimexpr\marginparwidth+5cm\relax
  % \marginparsep=1cm
}{}
\RequirePackage[colorinlistoftodos,prependcaption,textsize=tiny]{todonotes}
\RequirePackage{xcolor}
\newcommand{\comm}[3]{%
  \iftoggle{draft}%
    {\todo[linecolor=#1,backgroundcolor=#1!25,bordercolor=#1]{[#2] #3}}%
    {}}
\newcommand{\TODO}[1]{\comm{red}{TODO}{#1}}
\newcommand{\ITODO}[1]{{\color{red} #1}}
\newcommand{\jr}[1]{\comm{cyan}{JR}{#1}}
\newcommand{\cbp}[1]{\comm{green}{CBP}{#1}}
\newcommand{\nw}[1]{\comm{blue}{NW}{#1}}
\newcommand{\ba}[1]{\comm{orange}{BA}{#1}}
\newcommand{\cas}[1]{\comm{purple}{CAS}{#1}}
\newcommand{\resolved}[2][]{\comm{gray}{#1}{\sout{#2}}}
\newcommand{\wontfix}[3][]{\comm{gray}{#1}{WONTFIX: \sout{#2} #3}}

%%% Local Variables:
%%% mode: latex
%%% TeX-master: "../document"
%%% End:


%%
%% Agda typesetting commands shorthands, for
%% manual typesetting of inline code
%%

\newcommand{\af}{\AgdaFunction}
\newcommand{\un}{\AgdaUnderscore}
\newcommand{\ad}{\AgdaDatatype}
\newcommand{\ab}{\AgdaBound}
\newcommand{\ac}{\AgdaInductiveConstructor}
\newcommand{\aF}{\AgdaField}
\newcommand{\as}{\AgdaSymbol}
\newcommand{\aS}{\AgdaString}
\newcommand{\ak}{\AgdaKeyword}
\newcommand{\ap}{\AgdaPrimitiveType}
\newcommand{\an}{\AgdaNumber}
\newcommand{\aC}{\AgdaComment}
\newcommand{\am}{\AgdaModule}


%%
%% Unicode for typesetting Agda code
%%%%%%%%%%%%%%%%%%%
%% AGDA UNICODE %%%
%%%%%%%%%%%%%%%%%%%

% hacks

\newunicodechar{◂}{} % for repeating defs from stdlib for example

%%
%% Symbols
%%

\newcommand\superequiv{\mathrel{\rlap{\raisebox{\fontdimen22\textfont2}{$=$}}\raisebox{-0.5\fontdimen22\textfont2}{$ = $}}}

\newunicodechar{×}{\ensuremath{\mathnormal\times}}
\newunicodechar{→}{\ensuremath{\mathnormal\to}}
\newunicodechar{←}{\ensuremath{\mathnormal\leftarrow}}
\newunicodechar{⟦}{\ensuremath{\mathnormal\llbracket}}
\newunicodechar{⟧}{\ensuremath{\mathnormal\rrbracket}}
\newunicodechar{ℕ}{\ensuremath{\mathbb{N}}}
\newunicodechar{⊕}{\ensuremath{\mathnormal\oplus}}
\newunicodechar{⊗}{\ensuremath{\mathnormal\otimes}}
\newunicodechar{ẋ}{\ensuremath{\dot{\times}}}
\newunicodechar{∔}{\ensuremath{\dot{+}}}
\newunicodechar{⋎}{\ensuremath{\mathnormal\curlyvee}}
\newunicodechar{⊞}{\ensuremath{\mathnormal\boxplus}}
\newunicodechar{⇒}{\ensuremath{\mathnormal\Rightarrow}}
\newunicodechar{⇛}{\ensuremath{\mathnormal\Rrightarrow}}
\newunicodechar{⟨}{\ensuremath{\mathnormal\langle}}
\newunicodechar{⟩}{\ensuremath{\mathnormal\rangle}}
\newunicodechar{∪}{\ensuremath{\mathnormal\cup}}
\newunicodechar{⦃}{\ensuremath{\mathnormal\{\mskip-4.5mu\mid}}
\newunicodechar{⦄}{\ensuremath{\mathnormal\mid\mskip-4.5mu\}}}
\newunicodechar{⊎}{\ensuremath{\mathnormal\uplus}}
% \newunicodechar{∗}{\ensuremath{\mathnormal\ast}}
\newunicodechar{✴}{\ensuremath{\mathnormal\ast}}
\newunicodechar{↦}{\ensuremath{\mathnormal\mapsto}}
\newunicodechar{≡}{\ensuremath{\mathnormal\equiv}}
\newunicodechar{∀}{\ensuremath{\mathnormal\forall}}
\newunicodechar{∙}{\ensuremath{\mathnormal\bullet}}
\newunicodechar{≣}{\ensuremath{\mathnormal\superequiv}}
\newunicodechar{▿}{\ensuremath{\mathnormal\triangledown}}
\newunicodechar{▸}{\raisebox{0.25ex}{\scaleobj{0.75}{\ensuremath{\mathnormal\blacktriangleright}}}}
\newunicodechar{∼}{\ensuremath{\mathnormal\sim}}
\newunicodechar{≤}{\ensuremath{\mathnormal\leq}}
\newunicodechar{↔}{\ensuremath{\mathnormal\leftrightarrow}}
\newunicodechar{⇔}{\ensuremath{\mathnormal\Leftrightarrow}}
\newunicodechar{⊂}{\ensuremath{\mathnormal\subset}}
\newunicodechar{∘}{\ensuremath{\mathnormal\circ}}
\newunicodechar{∃}{\ensuremath{\mathnormal\exists}}
\newunicodechar{↓}{\ensuremath{\mathnormal\downarrow}}
\newunicodechar{↑}{\ensuremath{\mathnormal\uparrow}}
\newunicodechar{⊔}{\ensuremath{\mathnormal\sqcup}}
\newunicodechar{∐}{\ensuremath{\mathnormal\amalg}}
\newunicodechar{⊢}{\ensuremath{\mathnormal\vdash}}
\newunicodechar{◇}{\ensuremath{\mathnormal\diamond}}
\newunicodechar{◆}{}
\newunicodechar{⊙}{\ensuremath{\mathnormal\odot}}
\newunicodechar{⊤}{\ensuremath{\mathnormal\top}}
\newunicodechar{⊥}{\ensuremath{\mathnormal\bot}}
\newunicodechar{∣}{\ensuremath{\mathnormal\mid}}
% \newunicodechar{‵}{\ensuremath{^\backprime}}
\newunicodechar{′}{\ensuremath{^\prime}}
\newunicodechar{″}{\ensuremath{^{\prime\prime}}}
\newunicodechar{‴}{\ensuremath{^{\prime\prime\prime}}}
\newunicodechar{⅋}{}
\newunicodechar{∅}{\ensuremath{\emptyset}}
\newunicodechar{≺}{\ensuremath{\mathnormal{\prec}}}
\newunicodechar{≼}{\ensuremath{\mathnormal{\preceq}}}
\newunicodechar{∩}{\ensuremath{\mathnormal{\cap}}}
\newunicodechar{⟪}{\ensuremath{\langle\kern-.2em\langle}}
\newunicodechar{⟫}{\ensuremath{\rangle\kern-.2em\rangle}}
\newunicodechar{⊆}{\ensuremath{\mathnormal{\subseteq}}}
\newunicodechar{⊇}{\ensuremath{\mathnormal{\supseteq}}}
\newunicodechar{▻}{\ensuremath{\mathnormal\vartriangleright}}
\newunicodechar{∷}{\ensuremath{::}}
\newunicodechar{►}{\ensuremath{\mathnormal{\blacktriangleright}}}
\newunicodechar{▹}{\ensuremath{\mathnormal{\scaleobj{0.75}{\triangleright}}}}
\newunicodechar{□}{\ensuremath{\mathnormal{\square}}}
\newunicodechar{⋯}{\ensuremath{\mathnormal{\cdots}}}
\newunicodechar{▣}{\ensuremath{\mathnormal{\ldots}}}
\newunicodechar{⋮}{\ensuremath{\mathnormal{\quad\quad\vdots}}}
\newunicodechar{∈}{\ensuremath{\mathnormal{\in}}}
\newunicodechar{⊑}{\ensuremath{\mathbin{\sqsubseteq}}}
\newunicodechar{𝓑}{\ensuremath{\mathnormal{\gg\!\!=}}}
\newunicodechar{𝓢}{\ensuremath{\mathnormal{\gg}}}
\newunicodechar{𝓒}{\ensuremath{\mathbf{\mathscr{C}}}}
\newunicodechar{𝓓}{\ensuremath{\mathbf{\mathscr{D}}}}
\newunicodechar{ℒ}{\ensuremath{\mathscr{L}}}
\newunicodechar{♯}{\ensuremath{\sharp}}
\newunicodechar{∎}{\ensuremath{\qed}}
\newunicodechar{↣}{\ensuremath{\rightarrowtail}}
\newunicodechar{⟶}{\ensuremath{\longrightarrow}}
\newunicodechar{–}{--}
\newunicodechar{◃}{\ensuremath{\mathnormal{\scaleobj{0.75}{\triangleleft}}}}
\newunicodechar{≟}{\ensuremath{\stackrel{?}{=}}}
\newunicodechar{⌊}{\ensuremath{\lfloor}}
\newunicodechar{⌋}{\ensuremath{\rfloor}}

%%
%% Greek
%%

\newunicodechar{φ}{\ensuremath{\phi}}
\newunicodechar{Φ}{\ensuremath{\Phi}}
\newunicodechar{ψ}{\ensuremath{\psi}}
\newunicodechar{μ}{\ensuremath{\mu}}
\newunicodechar{α}{\ensuremath{\alpha}}
\newunicodechar{β}{\ensuremath{\beta}}
\newunicodechar{σ}{\ensuremath{\sigma}}
\newunicodechar{ξ}{\ensuremath{\xi}}
\newunicodechar{Ξ}{\ensuremath{\Xi}}
\newunicodechar{λ}{\ensuremath{\lambda}}
\newunicodechar{ε}{\ensuremath{\epsilon}}
\newunicodechar{γ}{\ensuremath{\gamma}}
\newunicodechar{Σ}{\ensuremath{\Sigma}}
\newunicodechar{Δ}{\ensuremath{\Delta}}
\newunicodechar{Π}{\ensuremath{\Pi}}
\newunicodechar{Γ}{\ensuremath{\Gamma}}
\newunicodechar{η}{\ensuremath{\eta}}
\newunicodechar{ζ}{\ensuremath{\zeta}}
\newunicodechar{ν}{\ensuremath{\nu}}
\newunicodechar{δ}{\ensuremath{\delta}}


%%
%% subscript/superscript
%%

\newunicodechar{₀}{\ensuremath{_{0}}}
\newunicodechar{₁}{\ensuremath{_{1}}}
\newunicodechar{₂}{\ensuremath{_{2}}}
\newunicodechar{₃}{\ensuremath{_{3}}}
\newunicodechar{₄}{\ensuremath{_{4}}}
\newunicodechar{₅}{\ensuremath{_{5}}}
\newunicodechar{₆}{\ensuremath{_{6}}}
\newunicodechar{₇}{\ensuremath{_{7}}}
\newunicodechar{₈}{\ensuremath{_{8}}}
\newunicodechar{₉}{\ensuremath{_{9}}}
\newunicodechar{¹}{\ensuremath{^{1}}}
\newunicodechar{⁻}{$^{-}$}
\newunicodechar{ᴬ}{$^{A}$}
\newunicodechar{ᴮ}{$^{B}$}
\newunicodechar{ᴱ}{$^{E}$}
\newunicodechar{ᴴ}{$^{H}$}
\newunicodechar{ˣ}{$^{×}$}
\newunicodechar{ᵈ}{$^{d}$}
\newunicodechar{ᵘ}{$^{u}$}
\newunicodechar{ᶠ}{$^{F}$}
\newunicodechar{ⁱ}{$^{i}$}
\newunicodechar{ᵒ}{$^{o}$}
\newunicodechar{ˢ}{$^{s}$}
\newunicodechar{ˡ}{$^{l}$}
\newunicodechar{ʳ}{$^r$}
\newunicodechar{ᴰ}{$^D$}
\newunicodechar{ᵇ}{$^{b}$}
\newunicodechar{ᵐ}{$_m$}
\newunicodechar{ⁿ}{$_n$}
\newunicodechar{ₚ}{$_p$}
\newunicodechar{ₒ}{$_o$}
\newunicodechar{ₖ}{$_k$}
\newunicodechar{ₙ}{$_n$}
\newunicodechar{ₘ}{$_m$}
\newunicodechar{ᵛ}{$_v$}
\newunicodechar{ᵍ}{$_f$}
\newunicodechar{ᵢ}{$_i$}
\newunicodechar{ⱼ}{$_j$}
\newunicodechar{ₗ}{$_l$}
\newunicodechar{ᵣ}{$_r$}
\newunicodechar{ₛ}{$_s$}
\newunicodechar{ₜ}{$_t$}
\newunicodechar{ᶜ}{$^c$}
\newunicodechar{ₐ}{$_a$}
\newunicodechar{∶}{$:$}
\newunicodechar{̅}{$^{\textit{d}}$}

\newunicodechar{𝑡}{\textit{t}}
\newunicodechar{ℎ}{\textit{h}}
\newunicodechar{𝑟}{\textit{r}}
\newunicodechar{𝑜}{\textit{o}}
\newunicodechar{𝑤}{\textit{w}}
\newunicodechar{𝑐}{\textit{c}}
\newunicodechar{𝑎}{\textit{a}}
\newunicodechar{𝑡}{\textit{t}}

\newunicodechar{𝓑}{\ensuremath{\mathnormal{\gg\!\!=}}}

\newunicodechar{𝟘}{\ensuremath{\mathbb{0}}}
\newunicodechar{𝟙}{\ensuremath{\mathbb{1}}}
\newunicodechar{⅀}{\ensuremath{\mathbb{\Sigma}}}
\newunicodechar{ℿ}{\ensuremath{\mathbb{\Pi}}}
\newunicodechar{ℕ}{\ensuremath{\mathbb{N}}}


\pagestyle{plain} % for page numbers
%
\begin{document}
%
\title{Context-free Languages, Type Theoretically}
%
%\titlerunning{Abbreviated paper title}
% If the paper title is too long for the running head, you can set
% an abbreviated paper title here
%
\author{Anonymous}
% \author{First Author\inst{1}\orcidID{0000-1111-2222-3333} \and
% Second Author\inst{2,3}\orcidID{1111-2222-3333-4444} \and
% Third Author\inst{3}\orcidID{2222--3333-4444-5555}}
%
% \authorrunning{F. Author et al.}
% First names are abbreviated in the running head.
% If there are more than two authors, 'et al.' is used.
%
% \institute{Princeton University, Princeton NJ 08544, USA \and
% Springer Heidelberg, Tiergartenstr. 17, 69121 Heidelberg, Germany
% \email{lncs@springer.com}\\
% \url{http://www.springer.com/gp/computer-science/lncs} \and
% ABC Institute, Rupert-Karls-University Heidelberg, Heidelberg, Germany\\
% \email{\{abc,lncs\}@uni-heidelberg.de}}

\maketitle              % typeset the header of the contribution

\begin{abstract}
Parsing is the process of recovering structure from strings, an essential part of implementing programming languages.
Previous work has shown that formalizing languages and parsers using an idiomatic type theoretic approach can be simple and enlightening.
Unfortunately, this approach has only been applied to regular languages, which are not expressive enough for many practical applications.
We are working on extending the type theoretic formalization to context-free languages which substantially more expressive.
We hope our formalization can serve as a foundation for reasoning about new disambiguation techniques and even more expressive formalisms such as data-dependent grammars.
\keywords{Language  \and Parsing \and Type Theory}
\end{abstract}

%
% the environments 'definition', 'lemma', 'proposition', 'corollary',
% 'remark', and 'example' are defined in the LLNCS documentclass as well.
%
\begin{code}[hide]%
\>[0]\AgdaSymbol{\{-\#}\AgdaSpace{}%
\AgdaKeyword{OPTIONS}\AgdaSpace{}%
\AgdaPragma{--cubical}\AgdaSpace{}%
\AgdaPragma{--guarded}\AgdaSpace{}%
\AgdaSymbol{\#-\}}\<%
\\
\>[0]\AgdaKeyword{module}\AgdaSpace{}%
\AgdaModule{1-introduction}\AgdaSpace{}%
\AgdaKeyword{where}\<%
\\
%
\\[\AgdaEmptyExtraSkip]%
\>[0]\AgdaKeyword{open}\AgdaSpace{}%
\AgdaKeyword{import}\AgdaSpace{}%
\AgdaModule{Data.Unit}\<%
\\
\>[0]\<%
\end{code}

\section{Introduction}

Parsing is the conversion of flat, human-readable text into a tree structure
that is easier for computers to manipulate.  As one of the central
pillars of compiler tooling since the 1960s, today almost every automated
transformation of computer programs requires a form of parsing.
Though it is a mature research subject, it is still actively studied, for example the question of how to resolve ambiguities in context-free grammars \cite{one-parser-to-rule-them-all}. 

Most parsing works mix the essence of the parsing technique with operational details \jr{such as... state machines, continuations, memoization?}. Our understanding and ability to improve upon these parsing techniques is hindered by the additional complexity of these inessential practical concerns. To address this issue, we are developing natural denotational semantics for traditional parsing techniques.

\jr{Elliot has kicked off this effort...}
Recent work by Elliot uses interactive theorem provers to state simple specifications of languages and that proofs of desirable properties of these language specifications transfer easily to their parsers \cite{conal-languages}. Unfortunately, this work only considers regular languages which are not powerful enough to describe practical programming languages.

\jr{Make the problem clear through an example: if we have a left-recursive grammar then naively unfolding it gets us into an infinite loop. }

In this paper, we formalize context-free languages and show how to parse them, extending Elliot’s type theoretic approach to language specfication.  One of the main challenges is that the recursive nature of context-free languages does not map directly onto interactive theorem provers as they do not support general recursion (for good reasons). We encode context-free languages as fixed points of functors (initial algebras).

\jr{Say something about the limitation that we only study acyclic grammars: there must be a total order on nonterminals and a nonterminal is not allowed to refer to nonterminals that come before it. We wanted to start by limiting ourselves to grammars with only one nonterminal, but those are not closed under derivatives.}

We make the following concrete contributions:
\begin{itemize}
\item We extend Elliot's type theoretic formalization of regular languages to context-free languages.
\end{itemize}

For this paper we have chosen Agda as our type theory and interactive theorem prover. We believe our definitions should transfer easily to other theories and tools. This paper itself is a literate Agda file; all highlighted Agda code has been accepted by Agda's type checker, giving us a high confidence of correctness.

% The goal is to give a denotational semantics to context-free languages
% And mechanize this in a proof assistants

% Challenge:
% Give a simple non-recursive example
% Expand to a recursive variant

% We could use full blown domain theory, but that is quite a big hammer

% In section 2:
% * Could: "This is a background section"
% * We recall Elliot's ...
% * To make it easier to add fixed points in the next section

% Look for a simple example in Section 2. Can be contrived
\begin{code}[hide]%
\>[0]\AgdaKeyword{module}\AgdaSpace{}%
\AgdaModule{2-overview}\AgdaSpace{}%
\AgdaKeyword{where}\<%
\\
%
\\[\AgdaEmptyExtraSkip]%
\>[0]\AgdaKeyword{open}\AgdaSpace{}%
\AgdaKeyword{import}\AgdaSpace{}%
\AgdaModule{Agda.Primitive}\AgdaSpace{}%
\AgdaKeyword{renaming}\AgdaSpace{}%
\AgdaSymbol{(}\AgdaPrimitive{Set}\AgdaSpace{}%
\AgdaSymbol{to}\AgdaSpace{}%
\AgdaPrimitive{Type}\AgdaSpace{}%
\AgdaSymbol{;}\AgdaSpace{}%
\AgdaPrimitive{Setω}\AgdaSpace{}%
\AgdaSymbol{to}\AgdaSpace{}%
\AgdaPrimitive{Typeω}\AgdaSymbol{)}\<%
\\
%
\\[\AgdaEmptyExtraSkip]%
\>[0]\AgdaKeyword{import}\AgdaSpace{}%
\AgdaModule{Function.Properties.Equivalence}\AgdaSpace{}%
\AgdaSymbol{as}\AgdaSpace{}%
\AgdaModule{⇔}\<%
\\
\>[0]\AgdaKeyword{import}\AgdaSpace{}%
\AgdaModule{Data.Bool}\AgdaSpace{}%
\AgdaSymbol{as}\AgdaSpace{}%
\AgdaModule{Bool}\<%
\\
\>[0]\AgdaKeyword{open}\AgdaSpace{}%
\AgdaKeyword{import}\AgdaSpace{}%
\AgdaModule{Data.Bool}\AgdaSpace{}%
\AgdaKeyword{using}\AgdaSpace{}%
\AgdaSymbol{(}\AgdaDatatype{Bool}\AgdaSpace{}%
\AgdaSymbol{;}\AgdaSpace{}%
\AgdaInductiveConstructor{true}\AgdaSpace{}%
\AgdaSymbol{;}\AgdaSpace{}%
\AgdaInductiveConstructor{false}\AgdaSymbol{)}\<%
\\
\>[0]\AgdaKeyword{open}\AgdaSpace{}%
\AgdaKeyword{import}\AgdaSpace{}%
\AgdaModule{Data.Char}\AgdaSpace{}%
\AgdaKeyword{using}\AgdaSpace{}%
\AgdaSymbol{(}\AgdaPostulate{Char}\AgdaSpace{}%
\AgdaSymbol{;}\AgdaSpace{}%
\AgdaOperator{\AgdaFunction{\AgdaUnderscore{}≟\AgdaUnderscore{}}}\AgdaSymbol{)}\<%
\\
\>[0]\AgdaKeyword{open}\AgdaSpace{}%
\AgdaKeyword{import}\AgdaSpace{}%
\AgdaModule{Data.List}\AgdaSpace{}%
\AgdaSymbol{as}\AgdaSpace{}%
\AgdaModule{List}\AgdaSpace{}%
\AgdaKeyword{hiding}\AgdaSpace{}%
\AgdaSymbol{(}\AgdaFunction{foldl}\AgdaSymbol{)}\<%
\\
\>[0]\AgdaKeyword{open}\AgdaSpace{}%
\AgdaKeyword{import}\AgdaSpace{}%
\AgdaModule{Data.Empty}\<%
\\
\>[0]\AgdaKeyword{open}\AgdaSpace{}%
\AgdaKeyword{import}\AgdaSpace{}%
\AgdaModule{Data.Product}\<%
\\
\>[0]\AgdaKeyword{open}\AgdaSpace{}%
\AgdaKeyword{import}\AgdaSpace{}%
\AgdaModule{Data.Sum}\AgdaSpace{}%
\AgdaSymbol{as}\AgdaSpace{}%
\AgdaModule{Sum}\<%
\\
\>[0]\AgdaKeyword{open}\AgdaSpace{}%
\AgdaKeyword{import}\AgdaSpace{}%
\AgdaModule{Data.Unit}\AgdaSpace{}%
\AgdaKeyword{hiding}\AgdaSpace{}%
\AgdaSymbol{(}\AgdaOperator{\AgdaFunction{\AgdaUnderscore{}≟\AgdaUnderscore{}}}\AgdaSymbol{)}\<%
\\
\>[0]\AgdaKeyword{open}\AgdaSpace{}%
\AgdaKeyword{import}\AgdaSpace{}%
\AgdaModule{Relation.Nullary.Decidable}\AgdaSpace{}%
\AgdaSymbol{as}\AgdaSpace{}%
\AgdaModule{Dec}\AgdaSpace{}%
\AgdaKeyword{hiding}\AgdaSpace{}%
\AgdaSymbol{(}\AgdaFunction{from-yes}\AgdaSpace{}%
\AgdaSymbol{;}\AgdaSpace{}%
\AgdaFunction{from-no}\AgdaSymbol{)}\<%
\\
\>[0]\AgdaKeyword{open}\AgdaSpace{}%
\AgdaKeyword{import}\AgdaSpace{}%
\AgdaModule{Level}\AgdaSpace{}%
\AgdaKeyword{hiding}\AgdaSpace{}%
\AgdaSymbol{(}\AgdaPrimitive{zero}\AgdaSpace{}%
\AgdaSymbol{;}\AgdaSpace{}%
\AgdaPrimitive{suc}\AgdaSymbol{)}\<%
\\
\>[0]\AgdaKeyword{open}\AgdaSpace{}%
\AgdaKeyword{import}\AgdaSpace{}%
\AgdaModule{Relation.Binary.PropositionalEquality}\<%
\\
\>[0]\AgdaKeyword{open}\AgdaSpace{}%
\AgdaKeyword{import}\AgdaSpace{}%
\AgdaModule{Function}\<%
\\
\>[0]\AgdaKeyword{open}\AgdaSpace{}%
\AgdaKeyword{import}\AgdaSpace{}%
\AgdaModule{Data.Fin}\AgdaSpace{}%
\AgdaKeyword{hiding}\AgdaSpace{}%
\AgdaSymbol{(}\AgdaOperator{\AgdaFunction{\AgdaUnderscore{}≟\AgdaUnderscore{}}}\AgdaSymbol{)}\<%
\\
\>[0]\AgdaKeyword{open}\AgdaSpace{}%
\AgdaKeyword{import}\AgdaSpace{}%
\AgdaModule{Data.Nat}\AgdaSpace{}%
\AgdaKeyword{hiding}\AgdaSpace{}%
\AgdaSymbol{(}\AgdaOperator{\AgdaPrimitive{\AgdaUnderscore{}*\AgdaUnderscore{}}}\AgdaSpace{}%
\AgdaSymbol{;}\AgdaSpace{}%
\AgdaOperator{\AgdaFunction{\AgdaUnderscore{}≟\AgdaUnderscore{}}}\AgdaSymbol{)}\<%
\\
\>[0]\AgdaKeyword{open}\AgdaSpace{}%
\AgdaKeyword{import}\AgdaSpace{}%
\AgdaModule{Relation.Nullary.Negation}\<%
\\
\>[0]\AgdaKeyword{import}\AgdaSpace{}%
\AgdaModule{Data.String}\AgdaSpace{}%
\AgdaSymbol{as}\AgdaSpace{}%
\AgdaModule{String}\<%
\\
\>[0]\AgdaKeyword{open}\AgdaSpace{}%
\AgdaKeyword{import}\AgdaSpace{}%
\AgdaModule{Agda.Builtin.FromString}\<%
\\
%
\\[\AgdaEmptyExtraSkip]%
\>[0]\AgdaFunction{transport}\AgdaSpace{}%
\AgdaSymbol{:}\AgdaSpace{}%
\AgdaSymbol{∀\{}\AgdaBound{A}\AgdaSpace{}%
\AgdaBound{B}\AgdaSpace{}%
\AgdaSymbol{:}\AgdaSpace{}%
\AgdaPrimitive{Type}\AgdaSymbol{\}}\AgdaSpace{}%
\AgdaSymbol{→}\AgdaSpace{}%
\AgdaBound{A}\AgdaSpace{}%
\AgdaOperator{\AgdaDatatype{≡}}\AgdaSpace{}%
\AgdaBound{B}\AgdaSpace{}%
\AgdaSymbol{→}\AgdaSpace{}%
\AgdaBound{A}\AgdaSpace{}%
\AgdaSymbol{→}\AgdaSpace{}%
\AgdaBound{B}\<%
\\
\>[0]\AgdaFunction{transport}\AgdaSpace{}%
\AgdaInductiveConstructor{refl}\AgdaSpace{}%
\AgdaBound{x}\AgdaSpace{}%
\AgdaSymbol{=}\AgdaSpace{}%
\AgdaBound{x}\<%
\\
%
\\[\AgdaEmptyExtraSkip]%
\>[0]\AgdaFunction{≡→⇔}\AgdaSpace{}%
\AgdaSymbol{:}\AgdaSpace{}%
\AgdaSymbol{∀}\AgdaSpace{}%
\AgdaSymbol{\{}\AgdaBound{A}\AgdaSpace{}%
\AgdaBound{B}\AgdaSpace{}%
\AgdaSymbol{:}\AgdaSpace{}%
\AgdaPrimitive{Type}\AgdaSymbol{\}}\AgdaSpace{}%
\AgdaSymbol{→}\AgdaSpace{}%
\AgdaBound{A}\AgdaSpace{}%
\AgdaOperator{\AgdaDatatype{≡}}\AgdaSpace{}%
\AgdaBound{B}\AgdaSpace{}%
\AgdaSymbol{→}\AgdaSpace{}%
\AgdaBound{A}\AgdaSpace{}%
\AgdaOperator{\AgdaFunction{⇔}}\AgdaSpace{}%
\AgdaBound{B}\<%
\\
\>[0]\AgdaFunction{≡→⇔}\AgdaSpace{}%
\AgdaInductiveConstructor{refl}\AgdaSpace{}%
\AgdaSymbol{=}\AgdaSpace{}%
\AgdaFunction{⇔.refl}\<%
\\
%
\\[\AgdaEmptyExtraSkip]%
\>[0]\AgdaFunction{lift⊎₂}\AgdaSpace{}%
\AgdaSymbol{:}\AgdaSpace{}%
\AgdaSymbol{∀\{}\AgdaBound{A}\AgdaSpace{}%
\AgdaBound{B}\AgdaSpace{}%
\AgdaBound{C}\AgdaSpace{}%
\AgdaBound{D}\AgdaSpace{}%
\AgdaSymbol{:}\AgdaSpace{}%
\AgdaPrimitive{Type}\AgdaSymbol{\}}\AgdaSpace{}%
\AgdaSymbol{→}\AgdaSpace{}%
\AgdaSymbol{(}\AgdaBound{A}\AgdaSpace{}%
\AgdaSymbol{→}\AgdaSpace{}%
\AgdaBound{B}\AgdaSpace{}%
\AgdaSymbol{→}\AgdaSpace{}%
\AgdaBound{C}\AgdaSymbol{)}\AgdaSpace{}%
\AgdaSymbol{→}\AgdaSpace{}%
\AgdaBound{A}\AgdaSpace{}%
\AgdaOperator{\AgdaDatatype{⊎}}\AgdaSpace{}%
\AgdaBound{D}\AgdaSpace{}%
\AgdaSymbol{→}\AgdaSpace{}%
\AgdaBound{B}\AgdaSpace{}%
\AgdaOperator{\AgdaDatatype{⊎}}\AgdaSpace{}%
\AgdaBound{D}\AgdaSpace{}%
\AgdaSymbol{→}\AgdaSpace{}%
\AgdaBound{C}\AgdaSpace{}%
\AgdaOperator{\AgdaDatatype{⊎}}\AgdaSpace{}%
\AgdaBound{D}\<%
\\
\>[0]\AgdaFunction{lift⊎₂}\AgdaSpace{}%
\AgdaBound{f}\AgdaSpace{}%
\AgdaSymbol{(}\AgdaInductiveConstructor{inj₁}\AgdaSpace{}%
\AgdaBound{x}\AgdaSymbol{)}\AgdaSpace{}%
\AgdaSymbol{(}\AgdaInductiveConstructor{inj₁}\AgdaSpace{}%
\AgdaBound{y}\AgdaSymbol{)}\AgdaSpace{}%
\AgdaSymbol{=}\AgdaSpace{}%
\AgdaInductiveConstructor{inj₁}\AgdaSpace{}%
\AgdaSymbol{(}\AgdaBound{f}\AgdaSpace{}%
\AgdaBound{x}\AgdaSpace{}%
\AgdaBound{y}\AgdaSymbol{)}\<%
\\
\>[0]\AgdaFunction{lift⊎₂}\AgdaSpace{}%
\AgdaSymbol{\AgdaUnderscore{}}\AgdaSpace{}%
\AgdaSymbol{(}\AgdaInductiveConstructor{inj₁}\AgdaSpace{}%
\AgdaSymbol{\AgdaUnderscore{})}\AgdaSpace{}%
\AgdaSymbol{(}\AgdaInductiveConstructor{inj₂}\AgdaSpace{}%
\AgdaBound{y}\AgdaSymbol{)}\AgdaSpace{}%
\AgdaSymbol{=}\AgdaSpace{}%
\AgdaInductiveConstructor{inj₂}\AgdaSpace{}%
\AgdaBound{y}\<%
\\
\>[0]\AgdaFunction{lift⊎₂}\AgdaSpace{}%
\AgdaSymbol{\AgdaUnderscore{}}\AgdaSpace{}%
\AgdaSymbol{(}\AgdaInductiveConstructor{inj₂}\AgdaSpace{}%
\AgdaBound{x}\AgdaSymbol{)}\AgdaSpace{}%
\AgdaSymbol{\AgdaUnderscore{}}\AgdaSpace{}%
\AgdaSymbol{=}\AgdaSpace{}%
\AgdaInductiveConstructor{inj₂}\AgdaSpace{}%
\AgdaBound{x}\<%
\\
%
\\[\AgdaEmptyExtraSkip]%
\>[0]\AgdaFunction{String}\AgdaSpace{}%
\AgdaSymbol{:}\AgdaSpace{}%
\AgdaPrimitive{Type}\<%
\\
\>[0]\AgdaFunction{String}\AgdaSpace{}%
\AgdaSymbol{=}\AgdaSpace{}%
\AgdaDatatype{List}\AgdaSpace{}%
\AgdaPostulate{Char}\<%
\\
\>[0]\AgdaKeyword{instance}\<%
\\
\>[0][@{}l@{\AgdaIndent{0}}]%
\>[2]\AgdaFunction{string}\AgdaSpace{}%
\AgdaSymbol{:}\AgdaSpace{}%
\AgdaRecord{IsString}\AgdaSpace{}%
\AgdaFunction{String}\<%
\\
%
\>[2]\AgdaField{IsString.Constraint}\AgdaSpace{}%
\AgdaFunction{string}\AgdaSpace{}%
\AgdaSymbol{\AgdaUnderscore{}}\AgdaSpace{}%
\AgdaSymbol{=}\AgdaSpace{}%
\AgdaRecord{⊤}\<%
\\
%
\>[2]\AgdaField{IsString.fromString}\AgdaSpace{}%
\AgdaFunction{string}\AgdaSpace{}%
\AgdaBound{xs}\AgdaSpace{}%
\AgdaSymbol{=}\AgdaSpace{}%
\AgdaPrimitive{String.toList}\AgdaSpace{}%
\AgdaBound{xs}\<%
\\
%
\\[\AgdaEmptyExtraSkip]%
\>[0]\AgdaFunction{foldl}\AgdaSpace{}%
\AgdaSymbol{:}\AgdaSpace{}%
\AgdaSymbol{∀}\AgdaSpace{}%
\AgdaSymbol{\{}\AgdaBound{ℓ₁}\AgdaSpace{}%
\AgdaBound{ℓ₂}\AgdaSymbol{\}}\AgdaSpace{}%
\AgdaSymbol{\{}\AgdaBound{A}\AgdaSpace{}%
\AgdaSymbol{:}\AgdaSpace{}%
\AgdaPrimitive{Type}\AgdaSpace{}%
\AgdaBound{ℓ₁}\AgdaSymbol{\}}\AgdaSpace{}%
\AgdaSymbol{\{}\AgdaBound{B}\AgdaSpace{}%
\AgdaSymbol{:}\AgdaSpace{}%
\AgdaPrimitive{Type}\AgdaSpace{}%
\AgdaBound{ℓ₂}\AgdaSymbol{\}}\AgdaSpace{}%
\AgdaSymbol{→}\AgdaSpace{}%
\AgdaSymbol{(}\AgdaBound{A}\AgdaSpace{}%
\AgdaSymbol{→}\AgdaSpace{}%
\AgdaBound{B}\AgdaSpace{}%
\AgdaSymbol{→}\AgdaSpace{}%
\AgdaBound{B}\AgdaSymbol{)}\AgdaSpace{}%
\AgdaSymbol{→}\AgdaSpace{}%
\AgdaBound{B}\AgdaSpace{}%
\AgdaSymbol{→}\AgdaSpace{}%
\AgdaDatatype{List}\AgdaSpace{}%
\AgdaBound{A}\AgdaSpace{}%
\AgdaSymbol{→}\AgdaSpace{}%
\AgdaBound{B}\<%
\\
\>[0]\AgdaFunction{foldl}\AgdaSpace{}%
\AgdaBound{k}\AgdaSpace{}%
\AgdaBound{z}\AgdaSpace{}%
\AgdaInductiveConstructor{[]}\AgdaSpace{}%
\AgdaSymbol{=}\AgdaSpace{}%
\AgdaBound{z}\<%
\\
\>[0]\AgdaFunction{foldl}\AgdaSpace{}%
\AgdaBound{k}\AgdaSpace{}%
\AgdaBound{z}\AgdaSpace{}%
\AgdaSymbol{(}\AgdaBound{c}\AgdaSpace{}%
\AgdaOperator{\AgdaInductiveConstructor{∷}}\AgdaSpace{}%
\AgdaBound{w}\AgdaSymbol{)}\AgdaSpace{}%
\AgdaSymbol{=}\AgdaSpace{}%
\AgdaFunction{foldl}\AgdaSpace{}%
\AgdaBound{k}\AgdaSpace{}%
\AgdaSymbol{(}\AgdaBound{k}\AgdaSpace{}%
\AgdaBound{c}\AgdaSpace{}%
\AgdaBound{z}\AgdaSymbol{)}\AgdaSpace{}%
\AgdaBound{w}\<%
\\
%
\\[\AgdaEmptyExtraSkip]%
\>[0]\AgdaKeyword{variable}\<%
\\
\>[0][@{}l@{\AgdaIndent{0}}]%
\>[4]\AgdaGeneralizable{ℓ}\AgdaSpace{}%
\AgdaGeneralizable{ℓ′}\AgdaSpace{}%
\AgdaSymbol{:}\AgdaSpace{}%
\AgdaPostulate{Level}\<%
\\
%
\>[4]\AgdaGeneralizable{A}\AgdaSpace{}%
\AgdaSymbol{:}\AgdaSpace{}%
\AgdaPrimitive{Type}\AgdaSpace{}%
\AgdaGeneralizable{ℓ}\<%
\\
%
\>[4]\AgdaGeneralizable{c}\AgdaSpace{}%
\AgdaSymbol{:}\AgdaSpace{}%
\AgdaPostulate{Char}\<%
\\
%
\>[4]\AgdaGeneralizable{w}\AgdaSpace{}%
\AgdaSymbol{:}\AgdaSpace{}%
\AgdaFunction{String}\<%
\end{code}

\subsection{Languages}

\begin{code}[hide]%
\>[0]\AgdaKeyword{postulate}\AgdaSpace{}%
\AgdaPostulate{foo‵}\AgdaSpace{}%
\AgdaSymbol{:}\AgdaSpace{}%
\AgdaRecord{⊤}\<%
\end{code}

We define languages as being functions from strings to types.\footnote{We use \af{Type} as a synonym for Agda's \af{Set} to avoid confusion.}
\begin{code}[hide]%
\>[0]\AgdaFunction{Lang}\AgdaSpace{}%
\AgdaSymbol{:}\AgdaSpace{}%
\AgdaPrimitive{Set₁}\<%
\end{code}
\begin{code}%
\>[0]\AgdaFunction{Lang}\AgdaSpace{}%
\AgdaSymbol{=}\AgdaSpace{}%
\AgdaFunction{String}\AgdaSpace{}%
\AgdaSymbol{→}\AgdaSpace{}%
\AgdaPrimitive{Type}\<%
\end{code}
The result type can be thought of as the type of proofs that the string is in the language.
\begin{remark}
Note that a language may admit multiple different proofs for the same string. That is an important difference between the type theoretic approach and the more common set theoretic approach, which models languages as sets of strings.
\end{remark}
This is a broad definition of what a language is; it includes languages that are outside the class of context-free languages. 
\begin{example}\label{ex:non-context-free}
The language $a^n b^n c^n$ can be specified as follows:
\begin{code}[hide]%
\>[0]\AgdaFunction{repeat}\AgdaSpace{}%
\AgdaSymbol{:}\AgdaSpace{}%
\AgdaDatatype{ℕ}\AgdaSpace{}%
\AgdaSymbol{→}\AgdaSpace{}%
\AgdaPostulate{Char}\AgdaSpace{}%
\AgdaSymbol{→}\AgdaSpace{}%
\AgdaFunction{String}\<%
\\
\>[0]\AgdaFunction{repeat}\AgdaSpace{}%
\AgdaInductiveConstructor{zero}\AgdaSpace{}%
\AgdaSymbol{\AgdaUnderscore{}}\AgdaSpace{}%
\AgdaSymbol{=}\AgdaSpace{}%
\AgdaInductiveConstructor{[]}\<%
\\
\>[0]\AgdaFunction{repeat}\AgdaSpace{}%
\AgdaSymbol{(}\AgdaInductiveConstructor{suc}\AgdaSpace{}%
\AgdaBound{n}\AgdaSymbol{)}\AgdaSpace{}%
\AgdaBound{c}\AgdaSpace{}%
\AgdaSymbol{=}\AgdaSpace{}%
\AgdaBound{c}\AgdaSpace{}%
\AgdaOperator{\AgdaInductiveConstructor{∷}}\AgdaSpace{}%
\AgdaFunction{repeat}\AgdaSpace{}%
\AgdaBound{n}\AgdaSpace{}%
\AgdaBound{c}\<%
\end{code}
\begin{code}%
\>[0]\AgdaFunction{abc}\AgdaSpace{}%
\AgdaSymbol{:}\AgdaSpace{}%
\AgdaFunction{Lang}\<%
\\
\>[0]\AgdaFunction{abc}\AgdaSpace{}%
\AgdaBound{w}\AgdaSpace{}%
\AgdaSymbol{=}\AgdaSpace{}%
\AgdaFunction{Σ[}\AgdaSpace{}%
\AgdaBound{n}\AgdaSpace{}%
\AgdaFunction{∈}\AgdaSpace{}%
\AgdaDatatype{ℕ}\AgdaSpace{}%
\AgdaFunction{]}\AgdaSpace{}%
\AgdaBound{w}\AgdaSpace{}%
\AgdaOperator{\AgdaDatatype{≡}}\AgdaSpace{}%
\AgdaFunction{repeat}\AgdaSpace{}%
\AgdaBound{n}\AgdaSpace{}%
\AgdaString{'a'}\AgdaSpace{}%
\AgdaOperator{\AgdaFunction{++}}\AgdaSpace{}%
\AgdaFunction{repeat}\AgdaSpace{}%
\AgdaBound{n}\AgdaSpace{}%
\AgdaString{'b'}\AgdaSpace{}%
\AgdaOperator{\AgdaFunction{++}}\AgdaSpace{}%
\AgdaFunction{repeat}\AgdaSpace{}%
\AgdaBound{n}\AgdaSpace{}%
\AgdaString{'c'}\<%
\end{code}
We can show that the string $aabbcc$ is in this language by choosing $n$ to be $2$, from which the required equality follows by reflexivity after normalization:
\begin{code}%
\>[0]\AgdaFunction{aabbcc}\AgdaSpace{}%
\AgdaSymbol{:}\AgdaSpace{}%
\AgdaFunction{abc}\AgdaSpace{}%
\AgdaString{"aabbcc"}\<%
\\
\>[0]\AgdaFunction{aabbcc}\AgdaSpace{}%
\AgdaSymbol{=}\AgdaSpace{}%
\AgdaNumber{2}\AgdaSpace{}%
\AgdaOperator{\AgdaInductiveConstructor{,}}\AgdaSpace{}%
\AgdaInductiveConstructor{refl}\<%
\end{code}
\end{example}
\cref{ex:non-context-free} shows that it is possible to specify languages and prove that certain strings are in those languages, but for practical applications we do not want to be burdened with writing such proofs ourselves. The compiler should be able to decide whether or not your program is valid by itself.

\begin{itemize}
\item Agda is too powerful: it can specify undecidable languages \jr{do I need to give an example?}
\item So, we need to define a simpler language which still supports all the features we need.
\end{itemize}


\begin{grammar}
<permissions>  ::= <read> <write> <execute>

<read>         ::= `-' | `r'

<write>        ::= `-' | `w'

<execute>      ::= `-' | `x'
\end{grammar}

\begin{code}[hide]%
\>[0]\AgdaKeyword{variable}\AgdaSpace{}%
\AgdaGeneralizable{ℒ}\AgdaSpace{}%
\AgdaGeneralizable{ℒ₁}\AgdaSpace{}%
\AgdaGeneralizable{ℒ₂}\AgdaSpace{}%
\AgdaSymbol{:}\AgdaSpace{}%
\AgdaFunction{Lang}\<%
\\
%
\\[\AgdaEmptyExtraSkip]%
\>[0]\AgdaKeyword{module}\AgdaSpace{}%
\AgdaModule{◇}\AgdaSpace{}%
\AgdaKeyword{where}\<%
\end{code}

\begin{figure}
\begin{minipage}{.36\textwidth}
\begin{code}%
\>[0][@{}l@{\AgdaIndent{1}}]%
\>[4]\AgdaFunction{∅}\AgdaSpace{}%
\AgdaSymbol{:}\AgdaSpace{}%
\AgdaFunction{Lang}\<%
\\
%
\>[4]\AgdaFunction{∅}\AgdaSpace{}%
\AgdaSymbol{\AgdaUnderscore{}}\AgdaSpace{}%
\AgdaSymbol{=}\AgdaSpace{}%
\AgdaFunction{⊥}\<%
\end{code}
\begin{code}%
%
\>[4]\AgdaFunction{ε}\AgdaSpace{}%
\AgdaSymbol{:}\AgdaSpace{}%
\AgdaFunction{Lang}\<%
\\
%
\>[4]\AgdaFunction{ε}\AgdaSpace{}%
\AgdaBound{w}\AgdaSpace{}%
\AgdaSymbol{=}\AgdaSpace{}%
\AgdaBound{w}\AgdaSpace{}%
\AgdaOperator{\AgdaDatatype{≡}}\AgdaSpace{}%
\AgdaInductiveConstructor{[]}\<%
\end{code}
\begin{code}%
%
\>[4]\AgdaOperator{\AgdaFunction{\AgdaUnderscore{}∪\AgdaUnderscore{}}}\AgdaSpace{}%
\AgdaSymbol{:}\AgdaSpace{}%
\AgdaFunction{Lang}\AgdaSpace{}%
\AgdaSymbol{→}\AgdaSpace{}%
\AgdaFunction{Lang}\AgdaSpace{}%
\AgdaSymbol{→}\AgdaSpace{}%
\AgdaFunction{Lang}\<%
\\
%
\>[4]\AgdaSymbol{(}\AgdaBound{P}\AgdaSpace{}%
\AgdaOperator{\AgdaFunction{∪}}\AgdaSpace{}%
\AgdaBound{Q}\AgdaSymbol{)}\AgdaSpace{}%
\AgdaBound{w}\AgdaSpace{}%
\AgdaSymbol{=}\AgdaSpace{}%
\AgdaBound{P}\AgdaSpace{}%
\AgdaBound{w}\AgdaSpace{}%
\AgdaOperator{\AgdaDatatype{⊎}}\AgdaSpace{}%
\AgdaBound{Q}\AgdaSpace{}%
\AgdaBound{w}\<%
\end{code}
\end{minipage} \quad
\begin{minipage}{.63\textwidth}
\begin{code}%
%
\>[4]\AgdaOperator{\AgdaFunction{\AgdaUnderscore{}*\AgdaUnderscore{}}}\AgdaSpace{}%
\AgdaSymbol{:}\AgdaSpace{}%
\AgdaFunction{Lang}\AgdaSpace{}%
\AgdaSymbol{→}\AgdaSpace{}%
\AgdaFunction{Lang}\AgdaSpace{}%
\AgdaSymbol{→}\AgdaSpace{}%
\AgdaFunction{Lang}\<%
\\
%
\>[4]\AgdaSymbol{(}\AgdaBound{P}\AgdaSpace{}%
\AgdaOperator{\AgdaFunction{*}}\AgdaSpace{}%
\AgdaBound{Q}\AgdaSymbol{)}\AgdaSpace{}%
\AgdaBound{w}\AgdaSpace{}%
\AgdaSymbol{=}\AgdaSpace{}%
\AgdaFunction{∃[}\AgdaSpace{}%
\AgdaBound{u}\AgdaSpace{}%
\AgdaFunction{]}\AgdaSpace{}%
\AgdaFunction{∃[}\AgdaSpace{}%
\AgdaBound{v}\AgdaSpace{}%
\AgdaFunction{]}\AgdaSpace{}%
\AgdaBound{w}\AgdaSpace{}%
\AgdaOperator{\AgdaDatatype{≡}}\AgdaSpace{}%
\AgdaBound{u}\AgdaSpace{}%
\AgdaOperator{\AgdaFunction{++}}\AgdaSpace{}%
\AgdaBound{v}\AgdaSpace{}%
\AgdaOperator{\AgdaFunction{×}}\AgdaSpace{}%
\AgdaBound{P}\AgdaSpace{}%
\AgdaBound{u}\AgdaSpace{}%
\AgdaOperator{\AgdaFunction{×}}\AgdaSpace{}%
\AgdaBound{Q}\AgdaSpace{}%
\AgdaBound{v}\<%
\end{code}
\begin{code}%
%
\>[4]\AgdaOperator{\AgdaFunction{`\AgdaUnderscore{}}}\AgdaSpace{}%
\AgdaSymbol{:}\AgdaSpace{}%
\AgdaPostulate{Char}\AgdaSpace{}%
\AgdaSymbol{→}\AgdaSpace{}%
\AgdaFunction{Lang}\<%
\\
%
\>[4]\AgdaSymbol{(}\AgdaOperator{\AgdaFunction{`}}\AgdaSpace{}%
\AgdaBound{c}\AgdaSymbol{)}\AgdaSpace{}%
\AgdaBound{w}\AgdaSpace{}%
\AgdaSymbol{=}\AgdaSpace{}%
\AgdaBound{w}\AgdaSpace{}%
\AgdaOperator{\AgdaDatatype{≡}}\AgdaSpace{}%
\AgdaBound{c}\AgdaSpace{}%
\AgdaOperator{\AgdaInductiveConstructor{∷}}\AgdaSpace{}%
\AgdaInductiveConstructor{[]}\<%
\end{code}
\begin{code}%
%
\>[4]\AgdaOperator{\AgdaFunction{\AgdaUnderscore{}·\AgdaUnderscore{}}}\AgdaSpace{}%
\AgdaSymbol{:}\AgdaSpace{}%
\AgdaSymbol{\{}\AgdaBound{A}\AgdaSpace{}%
\AgdaSymbol{:}\AgdaSpace{}%
\AgdaPrimitive{Type}\AgdaSymbol{\}}\AgdaSpace{}%
\AgdaSymbol{→}\AgdaSpace{}%
\AgdaRecord{Dec}\AgdaSpace{}%
\AgdaBound{A}\AgdaSpace{}%
\AgdaSymbol{→}\AgdaSpace{}%
\AgdaFunction{Lang}\AgdaSpace{}%
\AgdaSymbol{→}\AgdaSpace{}%
\AgdaFunction{Lang}\<%
\\
%
\>[4]\AgdaOperator{\AgdaFunction{\AgdaUnderscore{}·\AgdaUnderscore{}}}\AgdaSpace{}%
\AgdaSymbol{\{}\AgdaBound{A}\AgdaSymbol{\}}\AgdaSpace{}%
\AgdaSymbol{\AgdaUnderscore{}}\AgdaSpace{}%
\AgdaBound{P}\AgdaSpace{}%
\AgdaBound{w}\AgdaSpace{}%
\AgdaSymbol{=}\AgdaSpace{}%
\AgdaBound{A}\AgdaSpace{}%
\AgdaOperator{\AgdaFunction{×}}\AgdaSpace{}%
\AgdaBound{P}\AgdaSpace{}%
\AgdaBound{w}\<%
\end{code}
\begin{code}[hide]%
%
\>[4]\AgdaKeyword{infix}\AgdaSpace{}%
\AgdaNumber{22}\AgdaSpace{}%
\AgdaOperator{\AgdaFunction{`\AgdaUnderscore{}}}\<%
\\
%
\>[4]\AgdaKeyword{infixr}\AgdaSpace{}%
\AgdaNumber{21}\AgdaSpace{}%
\AgdaOperator{\AgdaFunction{\AgdaUnderscore{}*\AgdaUnderscore{}}}\<%
\\
%
\>[4]\AgdaKeyword{infixr}\AgdaSpace{}%
\AgdaNumber{20}\AgdaSpace{}%
\AgdaOperator{\AgdaFunction{\AgdaUnderscore{}∪\AgdaUnderscore{}}}\<%
\end{code}
\end{minipage}
\caption{Basic language combinators.}\label{fig:combinators}
\end{figure}

This grammar uses three important features: sequencing, choice, and matching character literals.

\begin{code}[hide]%
%
\>[4]\AgdaFunction{permissions}\AgdaSpace{}%
\AgdaFunction{read}\AgdaSpace{}%
\AgdaFunction{write}\AgdaSpace{}%
\AgdaFunction{execute}\AgdaSpace{}%
\AgdaSymbol{:}\AgdaSpace{}%
\AgdaFunction{Lang}\<%
\end{code}
\begin{code}%
%
\>[4]\AgdaFunction{permissions}%
\>[17]\AgdaSymbol{=}\AgdaSpace{}%
\AgdaFunction{read}\AgdaSpace{}%
\AgdaOperator{\AgdaFunction{*}}\AgdaSpace{}%
\AgdaFunction{write}\AgdaSpace{}%
\AgdaOperator{\AgdaFunction{*}}\AgdaSpace{}%
\AgdaFunction{execute}\<%
\\
%
\>[4]\AgdaFunction{read}%
\>[17]\AgdaSymbol{=}\AgdaSpace{}%
\AgdaOperator{\AgdaFunction{`}}\AgdaSpace{}%
\AgdaString{'-'}\AgdaSpace{}%
\AgdaOperator{\AgdaFunction{∪}}\AgdaSpace{}%
\AgdaOperator{\AgdaFunction{`}}\AgdaSpace{}%
\AgdaString{'r'}\<%
\\
%
\>[4]\AgdaFunction{write}%
\>[17]\AgdaSymbol{=}\AgdaSpace{}%
\AgdaOperator{\AgdaFunction{`}}\AgdaSpace{}%
\AgdaString{'-'}\AgdaSpace{}%
\AgdaOperator{\AgdaFunction{∪}}\AgdaSpace{}%
\AgdaOperator{\AgdaFunction{`}}\AgdaSpace{}%
\AgdaString{'w'}\<%
\\
%
\>[4]\AgdaFunction{execute}%
\>[17]\AgdaSymbol{=}\AgdaSpace{}%
\AgdaOperator{\AgdaFunction{`}}\AgdaSpace{}%
\AgdaString{'-'}\AgdaSpace{}%
\AgdaOperator{\AgdaFunction{∪}}\AgdaSpace{}%
\AgdaOperator{\AgdaFunction{`}}\AgdaSpace{}%
\AgdaString{'x'}\<%
\end{code}

We want to write a program which can prove for us that a given string is in the language. What should this program return for strings that are not in the language? We want to make sure our program does find a proof if it exists, so if it does not exist then we want a proof that the string is not in the language. We can capture this using a type called \af{Dec} from the Agda standard library. It can be defined as follows:

\begin{code}%
%
\>[4]\AgdaKeyword{data}\AgdaSpace{}%
\AgdaDatatype{◂Dec}\AgdaSpace{}%
\AgdaSymbol{(}\AgdaBound{A}\AgdaSpace{}%
\AgdaSymbol{:}\AgdaSpace{}%
\AgdaPrimitive{Type}\AgdaSymbol{)}\AgdaSpace{}%
\AgdaSymbol{:}\AgdaSpace{}%
\AgdaPrimitive{Type}\AgdaSpace{}%
\AgdaKeyword{where}\<%
\\
\>[4][@{}l@{\AgdaIndent{0}}]%
\>[8]\AgdaInductiveConstructor{◂yes}\AgdaSpace{}%
\AgdaSymbol{:}\AgdaSpace{}%
\AgdaBound{A}\AgdaSpace{}%
\AgdaSymbol{→}\AgdaSpace{}%
\AgdaDatatype{◂Dec}\AgdaSpace{}%
\AgdaBound{A}\<%
\\
%
\>[8]\AgdaInductiveConstructor{◂no}\AgdaSpace{}%
\AgdaSymbol{:}\AgdaSpace{}%
\AgdaOperator{\AgdaFunction{¬}}\AgdaSpace{}%
\AgdaBound{A}\AgdaSpace{}%
\AgdaSymbol{→}\AgdaSpace{}%
\AgdaDatatype{◂Dec}\AgdaSpace{}%
\AgdaBound{A}\<%
\end{code}

\begin{code}%
%
\>[4]\AgdaFunction{parseChar}\AgdaSpace{}%
\AgdaSymbol{:}\AgdaSpace{}%
\AgdaSymbol{(}\AgdaBound{c}\AgdaSpace{}%
\AgdaSymbol{:}\AgdaSpace{}%
\AgdaPostulate{Char}\AgdaSymbol{)}\AgdaSpace{}%
\AgdaSymbol{→}\AgdaSpace{}%
\AgdaSymbol{(}\AgdaBound{w}\AgdaSpace{}%
\AgdaSymbol{:}\AgdaSpace{}%
\AgdaFunction{String}\AgdaSymbol{)}\AgdaSpace{}%
\AgdaSymbol{→}\AgdaSpace{}%
\AgdaRecord{Dec}\AgdaSpace{}%
\AgdaSymbol{((}\AgdaOperator{\AgdaFunction{`}}\AgdaSpace{}%
\AgdaBound{c}\AgdaSymbol{)}\AgdaSpace{}%
\AgdaBound{w}\AgdaSymbol{)}\<%
\\
%
\>[4]\AgdaFunction{parseChar}\AgdaSpace{}%
\AgdaSymbol{\AgdaUnderscore{}}\AgdaSpace{}%
\AgdaInductiveConstructor{[]}\AgdaSpace{}%
\AgdaSymbol{=}\AgdaSpace{}%
\AgdaInductiveConstructor{no}\AgdaSpace{}%
\AgdaSymbol{λ}\AgdaSpace{}%
\AgdaSymbol{()}\<%
\\
%
\>[4]\AgdaFunction{parseChar}\AgdaSpace{}%
\AgdaBound{c}\AgdaSpace{}%
\AgdaSymbol{(}\AgdaBound{x}\AgdaSpace{}%
\AgdaOperator{\AgdaInductiveConstructor{∷}}\AgdaSpace{}%
\AgdaInductiveConstructor{[]}\AgdaSymbol{)}\AgdaSpace{}%
\AgdaSymbol{=}\AgdaSpace{}%
\AgdaFunction{Dec.map}\AgdaSpace{}%
\AgdaSymbol{(}\AgdaFunction{mk⇔}\AgdaSpace{}%
\AgdaSymbol{(λ}\AgdaSpace{}%
\AgdaSymbol{\{}\AgdaSpace{}%
\AgdaInductiveConstructor{refl}\AgdaSpace{}%
\AgdaSymbol{→}\AgdaSpace{}%
\AgdaInductiveConstructor{refl}\AgdaSpace{}%
\AgdaSymbol{\})}\AgdaSpace{}%
\AgdaSymbol{(λ}\AgdaSpace{}%
\AgdaSymbol{\{}\AgdaSpace{}%
\AgdaInductiveConstructor{refl}\AgdaSpace{}%
\AgdaSymbol{→}\AgdaSpace{}%
\AgdaInductiveConstructor{refl}\AgdaSpace{}%
\AgdaSymbol{\}))}\AgdaSpace{}%
\AgdaSymbol{(}\AgdaBound{x}\AgdaSpace{}%
\AgdaOperator{\AgdaFunction{≟}}\AgdaSpace{}%
\AgdaBound{c}\AgdaSymbol{)}\<%
\\
%
\>[4]\AgdaFunction{parseChar}\AgdaSpace{}%
\AgdaSymbol{\AgdaUnderscore{}}\AgdaSpace{}%
\AgdaSymbol{(\AgdaUnderscore{}}\AgdaSpace{}%
\AgdaOperator{\AgdaInductiveConstructor{∷}}\AgdaSpace{}%
\AgdaSymbol{\AgdaUnderscore{}}\AgdaSpace{}%
\AgdaOperator{\AgdaInductiveConstructor{∷}}\AgdaSpace{}%
\AgdaSymbol{\AgdaUnderscore{})}\AgdaSpace{}%
\AgdaSymbol{=}\AgdaSpace{}%
\AgdaInductiveConstructor{no}\AgdaSpace{}%
\AgdaSymbol{λ}\AgdaSpace{}%
\AgdaSymbol{()}\<%
\end{code}

\begin{code}[hide]%
%
\>[4]\AgdaKeyword{variable}\AgdaSpace{}%
\AgdaGeneralizable{P}\AgdaSpace{}%
\AgdaGeneralizable{Q}\AgdaSpace{}%
\AgdaSymbol{:}\AgdaSpace{}%
\AgdaFunction{Lang}\<%
\end{code}
\begin{code}%
%
\>[4]\AgdaFunction{parseUnion}\AgdaSpace{}%
\AgdaSymbol{:}\AgdaSpace{}%
\AgdaSymbol{((}\AgdaBound{w}\AgdaSpace{}%
\AgdaSymbol{:}\AgdaSpace{}%
\AgdaFunction{String}\AgdaSymbol{)}\AgdaSpace{}%
\AgdaSymbol{→}\AgdaSpace{}%
\AgdaRecord{Dec}\AgdaSpace{}%
\AgdaSymbol{(}\AgdaGeneralizable{P}\AgdaSpace{}%
\AgdaBound{w}\AgdaSymbol{))}\AgdaSpace{}%
\AgdaSymbol{→}\AgdaSpace{}%
\AgdaSymbol{((}\AgdaBound{w}\AgdaSpace{}%
\AgdaSymbol{:}\AgdaSpace{}%
\AgdaFunction{String}\AgdaSymbol{)}\AgdaSpace{}%
\AgdaSymbol{→}\AgdaSpace{}%
\AgdaRecord{Dec}\AgdaSpace{}%
\AgdaSymbol{(}\AgdaGeneralizable{Q}\AgdaSpace{}%
\AgdaBound{w}\AgdaSymbol{))}\AgdaSpace{}%
\AgdaSymbol{→}\AgdaSpace{}%
\AgdaSymbol{(}\AgdaBound{w}\AgdaSpace{}%
\AgdaSymbol{:}\AgdaSpace{}%
\AgdaFunction{String}\AgdaSymbol{)}\AgdaSpace{}%
\AgdaSymbol{→}\AgdaSpace{}%
\AgdaRecord{Dec}\AgdaSpace{}%
\AgdaSymbol{((}\AgdaGeneralizable{P}\AgdaSpace{}%
\AgdaOperator{\AgdaFunction{∪}}\AgdaSpace{}%
\AgdaGeneralizable{Q}\AgdaSymbol{)}\AgdaSpace{}%
\AgdaBound{w}\AgdaSymbol{)}\<%
\\
%
\>[4]\AgdaFunction{parseUnion}\AgdaSpace{}%
\AgdaBound{x}\AgdaSpace{}%
\AgdaBound{y}\AgdaSpace{}%
\AgdaBound{w}\AgdaSpace{}%
\AgdaSymbol{=}\AgdaSpace{}%
\AgdaBound{x}\AgdaSpace{}%
\AgdaBound{w}\AgdaSpace{}%
\AgdaOperator{\AgdaFunction{⊎-dec}}\AgdaSpace{}%
\AgdaBound{y}\AgdaSpace{}%
\AgdaBound{w}\<%
\end{code}

\begin{code}%
%
\>[4]\AgdaFunction{parseRead}\AgdaSpace{}%
\AgdaSymbol{:}\AgdaSpace{}%
\AgdaSymbol{(}\AgdaBound{w}\AgdaSpace{}%
\AgdaSymbol{:}\AgdaSpace{}%
\AgdaFunction{String}\AgdaSymbol{)}\AgdaSpace{}%
\AgdaSymbol{→}\AgdaSpace{}%
\AgdaRecord{Dec}\AgdaSpace{}%
\AgdaSymbol{(}\AgdaFunction{read}\AgdaSpace{}%
\AgdaBound{w}\AgdaSymbol{)}\<%
\\
%
\>[4]\AgdaFunction{parseRead}\AgdaSpace{}%
\AgdaSymbol{=}\AgdaSpace{}%
\AgdaFunction{parseUnion}\AgdaSpace{}%
\AgdaSymbol{(}\AgdaFunction{parseChar}\AgdaSpace{}%
\AgdaString{'-'}\AgdaSymbol{)}\AgdaSpace{}%
\AgdaSymbol{(}\AgdaFunction{parseChar}\AgdaSpace{}%
\AgdaString{'r'}\AgdaSymbol{)}\<%
\\
%
\>[4]\AgdaFunction{parseWrite}\AgdaSpace{}%
\AgdaSymbol{:}\AgdaSpace{}%
\AgdaSymbol{(}\AgdaBound{w}\AgdaSpace{}%
\AgdaSymbol{:}\AgdaSpace{}%
\AgdaFunction{String}\AgdaSymbol{)}\AgdaSpace{}%
\AgdaSymbol{→}\AgdaSpace{}%
\AgdaRecord{Dec}\AgdaSpace{}%
\AgdaSymbol{(}\AgdaFunction{write}\AgdaSpace{}%
\AgdaBound{w}\AgdaSymbol{)}\<%
\\
%
\>[4]\AgdaFunction{parseWrite}\AgdaSpace{}%
\AgdaSymbol{=}\AgdaSpace{}%
\AgdaFunction{parseUnion}\AgdaSpace{}%
\AgdaSymbol{(}\AgdaFunction{parseChar}\AgdaSpace{}%
\AgdaString{'-'}\AgdaSymbol{)}\AgdaSpace{}%
\AgdaSymbol{(}\AgdaFunction{parseChar}\AgdaSpace{}%
\AgdaString{'w'}\AgdaSymbol{)}\<%
\\
%
\>[4]\AgdaFunction{parseExecute}\AgdaSpace{}%
\AgdaSymbol{:}\AgdaSpace{}%
\AgdaSymbol{(}\AgdaBound{w}\AgdaSpace{}%
\AgdaSymbol{:}\AgdaSpace{}%
\AgdaFunction{String}\AgdaSymbol{)}\AgdaSpace{}%
\AgdaSymbol{→}\AgdaSpace{}%
\AgdaRecord{Dec}\AgdaSpace{}%
\AgdaSymbol{(}\AgdaFunction{execute}\AgdaSpace{}%
\AgdaBound{w}\AgdaSymbol{)}\<%
\\
%
\>[4]\AgdaFunction{parseExecute}\AgdaSpace{}%
\AgdaSymbol{=}\AgdaSpace{}%
\AgdaFunction{parseUnion}\AgdaSpace{}%
\AgdaSymbol{(}\AgdaFunction{parseChar}\AgdaSpace{}%
\AgdaString{'-'}\AgdaSymbol{)}\AgdaSpace{}%
\AgdaSymbol{(}\AgdaFunction{parseChar}\AgdaSpace{}%
\AgdaString{'x'}\AgdaSymbol{)}\<%
\end{code}

\begin{code}%
%
\>[4]\AgdaFunction{parseSequence}\AgdaSpace{}%
\AgdaSymbol{:}\AgdaSpace{}%
\AgdaSymbol{((}\AgdaBound{u}\AgdaSpace{}%
\AgdaSymbol{:}\AgdaSpace{}%
\AgdaFunction{String}\AgdaSymbol{)}\AgdaSpace{}%
\AgdaSymbol{→}\AgdaSpace{}%
\AgdaRecord{Dec}\AgdaSpace{}%
\AgdaSymbol{(}\AgdaGeneralizable{P}\AgdaSpace{}%
\AgdaBound{u}\AgdaSymbol{))}\AgdaSpace{}%
\AgdaSymbol{→}\AgdaSpace{}%
\AgdaSymbol{((}\AgdaBound{v}\AgdaSpace{}%
\AgdaSymbol{:}\AgdaSpace{}%
\AgdaFunction{String}\AgdaSymbol{)}\AgdaSpace{}%
\AgdaSymbol{→}\AgdaSpace{}%
\AgdaRecord{Dec}\AgdaSpace{}%
\AgdaSymbol{(}\AgdaGeneralizable{Q}\AgdaSpace{}%
\AgdaBound{v}\AgdaSymbol{))}\AgdaSpace{}%
\AgdaSymbol{→}\AgdaSpace{}%
\AgdaSymbol{(}\AgdaBound{w}\AgdaSpace{}%
\AgdaSymbol{:}\AgdaSpace{}%
\AgdaFunction{String}\AgdaSymbol{)}\AgdaSpace{}%
\AgdaSymbol{→}\AgdaSpace{}%
\AgdaRecord{Dec}\AgdaSpace{}%
\AgdaSymbol{((}\AgdaGeneralizable{P}\AgdaSpace{}%
\AgdaOperator{\AgdaFunction{*}}\AgdaSpace{}%
\AgdaGeneralizable{Q}\AgdaSymbol{)}\AgdaSpace{}%
\AgdaBound{w}\AgdaSymbol{)}\<%
\\
%
\>[4]\AgdaFunction{parseSequence}\AgdaSpace{}%
\AgdaBound{φ}\AgdaSpace{}%
\AgdaBound{ψ}\AgdaSpace{}%
\AgdaInductiveConstructor{[]}\AgdaSpace{}%
\AgdaSymbol{=}\AgdaSpace{}%
\AgdaFunction{Dec.map}\AgdaSpace{}%
\AgdaSymbol{(}\AgdaFunction{mk⇔}\AgdaSpace{}%
\AgdaSymbol{(λ}\AgdaSpace{}%
\AgdaBound{x}\AgdaSpace{}%
\AgdaSymbol{→}\AgdaSpace{}%
\AgdaInductiveConstructor{[]}\AgdaSpace{}%
\AgdaOperator{\AgdaInductiveConstructor{,}}\AgdaSpace{}%
\AgdaInductiveConstructor{[]}\AgdaSpace{}%
\AgdaOperator{\AgdaInductiveConstructor{,}}\AgdaSpace{}%
\AgdaInductiveConstructor{refl}\AgdaSpace{}%
\AgdaOperator{\AgdaInductiveConstructor{,}}\AgdaSpace{}%
\AgdaBound{x}\AgdaSymbol{)}\AgdaSpace{}%
\AgdaSymbol{(λ}\AgdaSpace{}%
\AgdaSymbol{\{}\AgdaSpace{}%
\AgdaSymbol{(}\AgdaInductiveConstructor{[]}\AgdaSpace{}%
\AgdaOperator{\AgdaInductiveConstructor{,}}\AgdaSpace{}%
\AgdaInductiveConstructor{[]}\AgdaSpace{}%
\AgdaOperator{\AgdaInductiveConstructor{,}}\AgdaSpace{}%
\AgdaInductiveConstructor{refl}\AgdaSpace{}%
\AgdaOperator{\AgdaInductiveConstructor{,}}\AgdaSpace{}%
\AgdaBound{x}\AgdaSymbol{)}\AgdaSpace{}%
\AgdaSymbol{→}\AgdaSpace{}%
\AgdaBound{x}\AgdaSpace{}%
\AgdaSymbol{\}))}\AgdaSpace{}%
\AgdaSymbol{(}\AgdaBound{φ}\AgdaSpace{}%
\AgdaInductiveConstructor{[]}\AgdaSpace{}%
\AgdaOperator{\AgdaFunction{×-dec}}\AgdaSpace{}%
\AgdaBound{ψ}\AgdaSpace{}%
\AgdaInductiveConstructor{[]}\AgdaSymbol{)}\<%
\\
%
\>[4]\AgdaFunction{parseSequence}\AgdaSpace{}%
\AgdaBound{φ}\AgdaSpace{}%
\AgdaBound{ψ}\AgdaSpace{}%
\AgdaSymbol{(}\AgdaBound{c}\AgdaSpace{}%
\AgdaOperator{\AgdaInductiveConstructor{∷}}\AgdaSpace{}%
\AgdaBound{w}\AgdaSymbol{)}\AgdaSpace{}%
\AgdaSymbol{=}\AgdaSpace{}%
\AgdaFunction{Dec.map}\<%
\\
\>[4][@{}l@{\AgdaIndent{0}}]%
\>[6]\AgdaSymbol{(}\AgdaFunction{mk⇔}\<%
\\
\>[6][@{}l@{\AgdaIndent{0}}]%
\>[8]\AgdaSymbol{(λ}\AgdaSpace{}%
\AgdaKeyword{where}\<%
\\
\>[8][@{}l@{\AgdaIndent{0}}]%
\>[10]\AgdaSymbol{(}\AgdaInductiveConstructor{inj₁}\AgdaSpace{}%
\AgdaBound{x}\AgdaSymbol{)}\AgdaSpace{}%
\AgdaSymbol{→}\AgdaSpace{}%
\AgdaInductiveConstructor{[]}\AgdaSpace{}%
\AgdaOperator{\AgdaInductiveConstructor{,}}\AgdaSpace{}%
\AgdaSymbol{\AgdaUnderscore{}}\AgdaSpace{}%
\AgdaOperator{\AgdaInductiveConstructor{,}}\AgdaSpace{}%
\AgdaInductiveConstructor{refl}\AgdaSpace{}%
\AgdaOperator{\AgdaInductiveConstructor{,}}\AgdaSpace{}%
\AgdaBound{x}\<%
\\
%
\>[10]\AgdaSymbol{(}\AgdaInductiveConstructor{inj₂}\AgdaSpace{}%
\AgdaSymbol{(}\AgdaBound{u}\AgdaSpace{}%
\AgdaOperator{\AgdaInductiveConstructor{,}}\AgdaSpace{}%
\AgdaBound{v}\AgdaSpace{}%
\AgdaOperator{\AgdaInductiveConstructor{,}}\AgdaSpace{}%
\AgdaInductiveConstructor{refl}\AgdaSpace{}%
\AgdaOperator{\AgdaInductiveConstructor{,}}\AgdaSpace{}%
\AgdaBound{x}\AgdaSymbol{))}\AgdaSpace{}%
\AgdaSymbol{→}\AgdaSpace{}%
\AgdaSymbol{\AgdaUnderscore{}}\AgdaSpace{}%
\AgdaOperator{\AgdaInductiveConstructor{∷}}\AgdaSpace{}%
\AgdaBound{u}\AgdaSpace{}%
\AgdaOperator{\AgdaInductiveConstructor{,}}\AgdaSpace{}%
\AgdaBound{v}\AgdaSpace{}%
\AgdaOperator{\AgdaInductiveConstructor{,}}\AgdaSpace{}%
\AgdaInductiveConstructor{refl}\AgdaSpace{}%
\AgdaOperator{\AgdaInductiveConstructor{,}}\AgdaSpace{}%
\AgdaBound{x}\AgdaSymbol{)}\<%
\\
%
\>[8]\AgdaSymbol{λ}%
\>[662I]\AgdaKeyword{where}\<%
\\
\>[.][@{}l@{}]\<[662I]%
\>[10]\AgdaSymbol{(}\AgdaInductiveConstructor{[]}\AgdaSpace{}%
\AgdaOperator{\AgdaInductiveConstructor{,}}\AgdaSpace{}%
\AgdaSymbol{\AgdaUnderscore{}}\AgdaSpace{}%
\AgdaOperator{\AgdaInductiveConstructor{,}}\AgdaSpace{}%
\AgdaInductiveConstructor{refl}\AgdaSpace{}%
\AgdaOperator{\AgdaInductiveConstructor{,}}\AgdaSpace{}%
\AgdaBound{x}\AgdaSymbol{)}\AgdaSpace{}%
\AgdaSymbol{→}\AgdaSpace{}%
\AgdaInductiveConstructor{inj₁}\AgdaSpace{}%
\AgdaBound{x}\<%
\\
%
\>[10]\AgdaSymbol{(\AgdaUnderscore{}}\AgdaSpace{}%
\AgdaOperator{\AgdaInductiveConstructor{∷}}\AgdaSpace{}%
\AgdaBound{u}\AgdaSpace{}%
\AgdaOperator{\AgdaInductiveConstructor{,}}\AgdaSpace{}%
\AgdaBound{v}\AgdaSpace{}%
\AgdaOperator{\AgdaInductiveConstructor{,}}\AgdaSpace{}%
\AgdaInductiveConstructor{refl}\AgdaSpace{}%
\AgdaOperator{\AgdaInductiveConstructor{,}}\AgdaSpace{}%
\AgdaBound{x}\AgdaSymbol{)}\AgdaSpace{}%
\AgdaSymbol{→}\AgdaSpace{}%
\AgdaInductiveConstructor{inj₂}\AgdaSpace{}%
\AgdaSymbol{(}\AgdaBound{u}\AgdaSpace{}%
\AgdaOperator{\AgdaInductiveConstructor{,}}\AgdaSpace{}%
\AgdaBound{v}\AgdaSpace{}%
\AgdaOperator{\AgdaInductiveConstructor{,}}\AgdaSpace{}%
\AgdaInductiveConstructor{refl}\AgdaSpace{}%
\AgdaOperator{\AgdaInductiveConstructor{,}}\AgdaSpace{}%
\AgdaBound{x}\AgdaSymbol{))}\<%
\\
%
\>[6]\AgdaSymbol{(}\AgdaBound{φ}\AgdaSpace{}%
\AgdaInductiveConstructor{[]}\AgdaSpace{}%
\AgdaOperator{\AgdaFunction{×-dec}}\AgdaSpace{}%
\AgdaBound{ψ}\AgdaSpace{}%
\AgdaSymbol{(}\AgdaBound{c}\AgdaSpace{}%
\AgdaOperator{\AgdaInductiveConstructor{∷}}\AgdaSpace{}%
\AgdaBound{w}\AgdaSymbol{)}\AgdaSpace{}%
\AgdaOperator{\AgdaFunction{⊎-dec}}\AgdaSpace{}%
\AgdaFunction{parseSequence}\AgdaSpace{}%
\AgdaSymbol{(}\AgdaBound{φ}\AgdaSpace{}%
\AgdaOperator{\AgdaFunction{∘}}\AgdaSpace{}%
\AgdaSymbol{(}\AgdaBound{c}\AgdaSpace{}%
\AgdaOperator{\AgdaInductiveConstructor{∷\AgdaUnderscore{}}}\AgdaSymbol{))}\AgdaSpace{}%
\AgdaBound{ψ}\AgdaSpace{}%
\AgdaBound{w}\AgdaSymbol{)}\<%
\end{code}

\begin{code}%
%
\>[4]\AgdaFunction{parsePermissions}\AgdaSpace{}%
\AgdaSymbol{:}\AgdaSpace{}%
\AgdaSymbol{(}\AgdaBound{w}\AgdaSpace{}%
\AgdaSymbol{:}\AgdaSpace{}%
\AgdaFunction{String}\AgdaSymbol{)}\AgdaSpace{}%
\AgdaSymbol{→}\AgdaSpace{}%
\AgdaRecord{Dec}\AgdaSpace{}%
\AgdaSymbol{(}\AgdaFunction{permissions}\AgdaSpace{}%
\AgdaBound{w}\AgdaSymbol{)}\<%
\\
%
\>[4]\AgdaFunction{parsePermissions}\AgdaSpace{}%
\AgdaSymbol{=}\AgdaSpace{}%
\AgdaFunction{parseSequence}\AgdaSpace{}%
\AgdaFunction{parseRead}\AgdaSpace{}%
\AgdaSymbol{(}\AgdaFunction{parseSequence}\AgdaSpace{}%
\AgdaFunction{parseWrite}\AgdaSpace{}%
\AgdaFunction{parseExecute}\AgdaSymbol{)}\<%
\end{code}

\jr{transition}

For starters, we define some structure on this definition of language in
\cref{fig:combinators}. First, Languages form a semiring, with union
$\af{\un{}∪\un{}}$, concatenation $\af{\un{}*\un{}}$, the empty language
$\af{∅}$ which is the unit of union, and the language which only includes the
empty string $\af{ε}$ which is the unit of concatenation. Furthermore the
$\af{`\un}$ combinator defines a language which contains exactly the string
consisting of a single given character. Finally, the scalar multiplication
$\af{\un{}·\un{}}$ combinator injects an Agda type into a language. The purpose
of this combinator will become clearer in later sections\jr{mention specific sections}.

% \subsection{Decidability}

% From our type theoretic perspective, parsing a string is the same thing as producing an element of the result type of a language for that given input string, or showing that no such element can exist. In Agda, we encode this using the following \af{Dec} data type which is parameterized by a type \ab{A} and contains a constructor \ac{yes} for when you can produce an element of \ab{A} or \ac{no} if you can show that no such element exists.
% \begin{code}
% data Dec (A : Type) : Type where
    % yes : A → Dec A
    % no : (A → ⊥) → Dec A
% \end{code}
% Sometimes we want to change the parameter type of a \af{Dec}. For that we need to provide conversion functions between the old and the new type in both ways.
% \begin{code}
% map? : (A ↔ B) → Dec A → Dec B
% map? f (yes x) = yes (to f x)
% map? f (no ¬A) = no λ x → ¬A (from f x)
% \end{code}
% \begin{code}[hide]
% ⌊_⌋ : Dec A → Type
% ⌊_⌋ {A} _ = A

% _×?_ : Dec A → Dec B → Dec (A × B)
% yes x ×? yes y = yes (x , y)
% yes _ ×? no ¬y = no λ (_ , y) → ¬y y
% no ¬x ×? _ = no λ (x , _) → ¬x x

% _⊎?_ : Dec A → Dec B → Dec (A ⊎ B)
% yes x ⊎? y = yes (inl x)
% no x ⊎? yes y = yes (inr y)
% no ¬x ⊎? no ¬y = no λ where
    % (inl x) → ¬x x
    % (inr y) → ¬y y

% _≟_ : (c : Char) → (c′ : Char) → Dec (c ≡ c′)
% `a ≟ `a = yes refl
% `a ≟ `b = no λ ()
% `a ≟ `c = no λ ()
% `a ≟ `0 = no λ ()
% `a ≟ `1 = no λ ()
% `b ≟ `a = no λ ()
% `b ≟ `b = yes refl
% `b ≟ `c = no λ ()
% `b ≟ `0 = no λ ()
% `b ≟ `1 = no λ ()
% `c ≟ `a = no λ ()
% `c ≟ `b = no λ ()
% `c ≟ `c = yes refl
% `c ≟ `0 = no λ ()
% `c ≟ `1 = no λ ()
% `0 ≟ `a = no λ ()
% `0 ≟ `b = no λ ()
% `0 ≟ `c = no λ ()
% `0 ≟ `0 = yes refl
% `0 ≟ `1 = no λ ()
% `1 ≟ `a = no λ ()
% `1 ≟ `b = no λ ()
% `1 ≟ `c = no λ ()
% `1 ≟ `0 = no λ ()
% `1 ≟ `1 = yes refl
% `a ≟ `[ = no λ ()
% `a ≟ `] = no λ ()
% `b ≟ `[ = no λ ()
% `b ≟ `] = no λ ()
% `c ≟ `[ = no λ ()
% `c ≟ `] = no λ ()
% `0 ≟ `[ = no λ ()
% `0 ≟ `] = no λ ()
% `1 ≟ `[ = no λ ()
% `1 ≟ `] = no λ ()
% `[ ≟ `a = no λ ()
% `[ ≟ `b = no λ ()
% `[ ≟ `c = no λ ()
% `[ ≟ `0 = no λ ()
% `[ ≟ `1 = no λ ()
% `[ ≟ `[ = yes refl
% `[ ≟ `] = no λ ()
% `] ≟ `a = no λ ()
% `] ≟ `b = no λ ()
% `] ≟ `c = no λ ()
% `] ≟ `0 = no λ ()
% `] ≟ `1 = no λ ()
% `] ≟ `[ = no λ ()
% `] ≟ `] = yes refl

% \end{code}

\section{Grammars}\label{sec:gram-and-parsing}

We have seen in \cref{ex:non-context-free} that our definition of language is very general, comprising even context-sensitive languages. Parsing such languages automatically poses a significant challenge. Hence, we side-step this problem by restricting the scope of our parsers to a smaller well-defined subset of languages. In this subsection, we consider a subset of regular languages without Kleene star (i.e., closure under concatenation). In \cref{sec:context-free}, we extend this class of languages to include fixed points which subsume the Kleene star.

\begin{code}[hide]%
\>[0]\AgdaKeyword{module}\AgdaSpace{}%
\AgdaModule{◆}\AgdaSpace{}%
\AgdaKeyword{where}\<%
\end{code}
\begin{code}%
\>[0][@{}l@{\AgdaIndent{1}}]%
\>[4]\AgdaKeyword{data}\AgdaSpace{}%
\AgdaDatatype{Exp}\AgdaSpace{}%
\AgdaSymbol{:}\AgdaSpace{}%
\AgdaPrimitive{Type₁}\AgdaSpace{}%
\AgdaKeyword{where}\<%
\\
\>[4][@{}l@{\AgdaIndent{0}}]%
\>[8]\AgdaInductiveConstructor{∅}\AgdaSpace{}%
\AgdaSymbol{:}\AgdaSpace{}%
\AgdaDatatype{Exp}\<%
\\
%
\>[8]\AgdaInductiveConstructor{ε}\AgdaSpace{}%
\AgdaSymbol{:}\AgdaSpace{}%
\AgdaDatatype{Exp}\<%
\\
%
\>[8]\AgdaOperator{\AgdaInductiveConstructor{`\AgdaUnderscore{}}}\AgdaSpace{}%
\AgdaSymbol{:}\AgdaSpace{}%
\AgdaSymbol{(}\AgdaBound{c}\AgdaSpace{}%
\AgdaSymbol{:}\AgdaSpace{}%
\AgdaPostulate{Char}\AgdaSymbol{)}\AgdaSpace{}%
\AgdaSymbol{→}\AgdaSpace{}%
\AgdaDatatype{Exp}\<%
\\
%
\>[8]\AgdaOperator{\AgdaInductiveConstructor{\AgdaUnderscore{}·\AgdaUnderscore{}}}\AgdaSpace{}%
\AgdaSymbol{:}\AgdaSpace{}%
\AgdaSymbol{\{}\AgdaBound{A}\AgdaSpace{}%
\AgdaSymbol{:}\AgdaSpace{}%
\AgdaPrimitive{Type}\AgdaSymbol{\}}\AgdaSpace{}%
\AgdaSymbol{→}\AgdaSpace{}%
\AgdaRecord{Dec}\AgdaSpace{}%
\AgdaBound{A}\AgdaSpace{}%
\AgdaSymbol{→}\AgdaSpace{}%
\AgdaDatatype{Exp}\AgdaSpace{}%
\AgdaSymbol{→}\AgdaSpace{}%
\AgdaDatatype{Exp}\<%
\\
%
\>[8]\AgdaOperator{\AgdaInductiveConstructor{\AgdaUnderscore{}∪\AgdaUnderscore{}}}\AgdaSpace{}%
\AgdaSymbol{:}\AgdaSpace{}%
\AgdaDatatype{Exp}\AgdaSpace{}%
\AgdaSymbol{→}\AgdaSpace{}%
\AgdaDatatype{Exp}\AgdaSpace{}%
\AgdaSymbol{→}\AgdaSpace{}%
\AgdaDatatype{Exp}\<%
\\
%
\>[8]\AgdaOperator{\AgdaInductiveConstructor{\AgdaUnderscore{}*\AgdaUnderscore{}}}\AgdaSpace{}%
\AgdaSymbol{:}\AgdaSpace{}%
\AgdaDatatype{Exp}\AgdaSpace{}%
\AgdaSymbol{→}\AgdaSpace{}%
\AgdaDatatype{Exp}\AgdaSpace{}%
\AgdaSymbol{→}\AgdaSpace{}%
\AgdaDatatype{Exp}\<%
\end{code}

This syntax maps directly onto the semantics we defined in \cref{fig:combinators}.

\begin{code}%
%
\>[4]\AgdaOperator{\AgdaFunction{⟦\AgdaUnderscore{}⟧}}\AgdaSpace{}%
\AgdaSymbol{:}\AgdaSpace{}%
\AgdaDatatype{Exp}\AgdaSpace{}%
\AgdaSymbol{→}\AgdaSpace{}%
\AgdaFunction{Lang}\<%
\\
%
\>[4]\AgdaOperator{\AgdaFunction{⟦}}\AgdaSpace{}%
\AgdaInductiveConstructor{∅}\AgdaSpace{}%
\AgdaOperator{\AgdaFunction{⟧}}\AgdaSpace{}%
\AgdaSymbol{=}\AgdaSpace{}%
\AgdaFunction{◇.∅}\<%
\\
%
\>[4]\AgdaOperator{\AgdaFunction{⟦}}\AgdaSpace{}%
\AgdaInductiveConstructor{ε}\AgdaSpace{}%
\AgdaOperator{\AgdaFunction{⟧}}\AgdaSpace{}%
\AgdaSymbol{=}\AgdaSpace{}%
\AgdaFunction{◇.ε}\<%
\\
%
\>[4]\AgdaOperator{\AgdaFunction{⟦}}\AgdaSpace{}%
\AgdaOperator{\AgdaInductiveConstructor{`}}\AgdaSpace{}%
\AgdaBound{c}\AgdaSpace{}%
\AgdaOperator{\AgdaFunction{⟧}}\AgdaSpace{}%
\AgdaSymbol{=}\AgdaSpace{}%
\AgdaOperator{\AgdaFunction{◇.`}}\AgdaSpace{}%
\AgdaBound{c}\<%
\\
%
\>[4]\AgdaOperator{\AgdaFunction{⟦}}\AgdaSpace{}%
\AgdaBound{x}\AgdaSpace{}%
\AgdaOperator{\AgdaInductiveConstructor{·}}\AgdaSpace{}%
\AgdaBound{e}\AgdaSpace{}%
\AgdaOperator{\AgdaFunction{⟧}}\AgdaSpace{}%
\AgdaSymbol{=}\AgdaSpace{}%
\AgdaBound{x}\AgdaSpace{}%
\AgdaOperator{\AgdaFunction{◇.·}}\AgdaSpace{}%
\AgdaOperator{\AgdaFunction{⟦}}\AgdaSpace{}%
\AgdaBound{e}\AgdaSpace{}%
\AgdaOperator{\AgdaFunction{⟧}}\<%
\\
%
\>[4]\AgdaOperator{\AgdaFunction{⟦}}\AgdaSpace{}%
\AgdaBound{e}\AgdaSpace{}%
\AgdaOperator{\AgdaInductiveConstructor{∪}}\AgdaSpace{}%
\AgdaBound{e₁}\AgdaSpace{}%
\AgdaOperator{\AgdaFunction{⟧}}\AgdaSpace{}%
\AgdaSymbol{=}\AgdaSpace{}%
\AgdaOperator{\AgdaFunction{⟦}}\AgdaSpace{}%
\AgdaBound{e}\AgdaSpace{}%
\AgdaOperator{\AgdaFunction{⟧}}\AgdaSpace{}%
\AgdaOperator{\AgdaFunction{◇.∪}}\AgdaSpace{}%
\AgdaOperator{\AgdaFunction{⟦}}\AgdaSpace{}%
\AgdaBound{e₁}\AgdaSpace{}%
\AgdaOperator{\AgdaFunction{⟧}}\<%
\\
%
\>[4]\AgdaOperator{\AgdaFunction{⟦}}\AgdaSpace{}%
\AgdaBound{e}\AgdaSpace{}%
\AgdaOperator{\AgdaInductiveConstructor{*}}\AgdaSpace{}%
\AgdaBound{e₁}\AgdaSpace{}%
\AgdaOperator{\AgdaFunction{⟧}}\AgdaSpace{}%
\AgdaSymbol{=}\AgdaSpace{}%
\AgdaOperator{\AgdaFunction{⟦}}\AgdaSpace{}%
\AgdaBound{e}\AgdaSpace{}%
\AgdaOperator{\AgdaFunction{⟧}}\AgdaSpace{}%
\AgdaOperator{\AgdaFunction{◇.*}}\AgdaSpace{}%
\AgdaOperator{\AgdaFunction{⟦}}\AgdaSpace{}%
\AgdaBound{e₁}\AgdaSpace{}%
\AgdaOperator{\AgdaFunction{⟧}}\<%
\end{code}

\subsection{Parsing}

To facilitate proving the inclusion of strings in a language, we start by decomposing the problem. A string is either empty or a character followed by the tail of the string. We can decompose the problem of string inclusion along the same dimensions. First, we define nullability $ν$ as the inclusion of the empty string in a language as follows:
\begin{code}%
%
\>[4]\AgdaFunction{◇ν}\AgdaSpace{}%
\AgdaSymbol{:}\AgdaSpace{}%
\AgdaFunction{Lang}\AgdaSpace{}%
\AgdaSymbol{→}\AgdaSpace{}%
\AgdaPrimitive{Type}\<%
\\
%
\>[4]\AgdaFunction{◇ν}\AgdaSpace{}%
\AgdaBound{ℒ}\AgdaSpace{}%
\AgdaSymbol{=}\AgdaSpace{}%
\AgdaBound{ℒ}\AgdaSpace{}%
\AgdaInductiveConstructor{[]}\<%
\end{code}
Second, we define the derivative $δ$ of a language $ℒ$ with respect to the character $c$ to be all the suffixes of the words in $ℒ$ which start with the $c$.
\begin{code}%
%
\>[4]\AgdaFunction{◇δ}\AgdaSpace{}%
\AgdaSymbol{:}\AgdaSpace{}%
\AgdaPostulate{Char}\AgdaSpace{}%
\AgdaSymbol{→}\AgdaSpace{}%
\AgdaFunction{Lang}\AgdaSpace{}%
\AgdaSymbol{→}\AgdaSpace{}%
\AgdaFunction{Lang}\<%
\\
%
\>[4]\AgdaFunction{◇δ}\AgdaSpace{}%
\AgdaBound{c}\AgdaSpace{}%
\AgdaBound{ℒ}\AgdaSpace{}%
\AgdaSymbol{=}\AgdaSpace{}%
\AgdaSymbol{λ}\AgdaSpace{}%
\AgdaBound{w}\AgdaSpace{}%
\AgdaSymbol{→}\AgdaSpace{}%
\AgdaBound{ℒ}\AgdaSpace{}%
\AgdaSymbol{(}\AgdaBound{c}\AgdaSpace{}%
\AgdaOperator{\AgdaInductiveConstructor{∷}}\AgdaSpace{}%
\AgdaBound{w}\AgdaSymbol{)}\<%
\end{code}
The relevance of these definitions is shown by \cref{thm:nullability-after-derivatives}.
\begin{theorem}\label{thm:nullability-after-derivatives}
Nullability after repeated derivatives fully captures what a language is. Formally, we state this as follows:
\begin{code}[hide]%
%
\>[4]\AgdaFunction{ν∘foldlδℒ≡ℒ}\AgdaSpace{}%
\AgdaSymbol{:}\<%
\end{code}
\begin{code}%
\>[4][@{}l@{\AgdaIndent{1}}]%
\>[8]\AgdaFunction{◇ν}\AgdaSpace{}%
\AgdaOperator{\AgdaFunction{∘}}\AgdaSpace{}%
\AgdaFunction{foldl}\AgdaSpace{}%
\AgdaFunction{◇δ}\AgdaSpace{}%
\AgdaGeneralizable{ℒ}\AgdaSpace{}%
\AgdaOperator{\AgdaDatatype{≡}}\AgdaSpace{}%
\AgdaGeneralizable{ℒ}\<%
\end{code}
\begin{code}[hide]%
%
\>[4]\AgdaFunction{ν∘foldlδℒ≡ℒ′}\AgdaSpace{}%
\AgdaSymbol{:}\AgdaSpace{}%
\AgdaSymbol{(}\AgdaBound{ℒ}\AgdaSpace{}%
\AgdaSymbol{:}\AgdaSpace{}%
\AgdaFunction{Lang}\AgdaSymbol{)}\AgdaSpace{}%
\AgdaSymbol{(}\AgdaBound{w}\AgdaSpace{}%
\AgdaSymbol{:}\AgdaSpace{}%
\AgdaFunction{String}\AgdaSymbol{)}\AgdaSpace{}%
\AgdaSymbol{→}\AgdaSpace{}%
\AgdaFunction{◇ν}\AgdaSpace{}%
\AgdaSymbol{(}\AgdaFunction{foldl}\AgdaSpace{}%
\AgdaFunction{◇δ}\AgdaSpace{}%
\AgdaBound{ℒ}\AgdaSpace{}%
\AgdaBound{w}\AgdaSymbol{)}\AgdaSpace{}%
\AgdaOperator{\AgdaDatatype{≡}}\AgdaSpace{}%
\AgdaBound{ℒ}\AgdaSpace{}%
\AgdaBound{w}\<%
\\
%
\>[4]\AgdaFunction{ν∘foldlδℒ≡ℒ′}\AgdaSpace{}%
\AgdaBound{ℒ}\AgdaSpace{}%
\AgdaInductiveConstructor{[]}\AgdaSpace{}%
\AgdaSymbol{=}\AgdaSpace{}%
\AgdaInductiveConstructor{refl}\<%
\\
%
\>[4]\AgdaFunction{ν∘foldlδℒ≡ℒ′}\AgdaSpace{}%
\AgdaBound{ℒ}\AgdaSpace{}%
\AgdaSymbol{(}\AgdaBound{c}\AgdaSpace{}%
\AgdaOperator{\AgdaInductiveConstructor{∷}}\AgdaSpace{}%
\AgdaBound{w}\AgdaSymbol{)}\AgdaSpace{}%
\AgdaSymbol{=}\AgdaSpace{}%
\AgdaFunction{ν∘foldlδℒ≡ℒ′}\AgdaSpace{}%
\AgdaSymbol{(}\AgdaFunction{◇δ}\AgdaSpace{}%
\AgdaBound{c}\AgdaSpace{}%
\AgdaBound{ℒ}\AgdaSymbol{)}\AgdaSpace{}%
\AgdaBound{w}\<%
\\
%
\\[\AgdaEmptyExtraSkip]%
%
\>[4]\AgdaKeyword{postulate}\AgdaSpace{}%
\AgdaPostulate{funext}\AgdaSpace{}%
\AgdaSymbol{:}\AgdaSpace{}%
\AgdaSymbol{\{}\AgdaBound{ℓ₁}\AgdaSpace{}%
\AgdaBound{ℓ₂}\AgdaSpace{}%
\AgdaSymbol{:}\AgdaSpace{}%
\AgdaPostulate{Level}\AgdaSymbol{\}}\AgdaSpace{}%
\AgdaSymbol{\{}\AgdaBound{A}\AgdaSpace{}%
\AgdaSymbol{:}\AgdaSpace{}%
\AgdaPrimitive{Set}\AgdaSpace{}%
\AgdaBound{ℓ₁}\AgdaSymbol{\}}\AgdaSpace{}%
\AgdaSymbol{\{}\AgdaBound{B}\AgdaSpace{}%
\AgdaSymbol{:}\AgdaSpace{}%
\AgdaPrimitive{Set}\AgdaSpace{}%
\AgdaBound{ℓ₂}\AgdaSymbol{\}}\AgdaSpace{}%
\AgdaSymbol{\{}\AgdaBound{P}\AgdaSpace{}%
\AgdaBound{Q}\AgdaSpace{}%
\AgdaSymbol{:}\AgdaSpace{}%
\AgdaBound{A}\AgdaSpace{}%
\AgdaSymbol{→}\AgdaSpace{}%
\AgdaBound{B}\AgdaSymbol{\}}\AgdaSpace{}%
\AgdaSymbol{→}\AgdaSpace{}%
\AgdaSymbol{((}\AgdaBound{x}\AgdaSpace{}%
\AgdaSymbol{:}\AgdaSpace{}%
\AgdaBound{A}\AgdaSymbol{)}\AgdaSpace{}%
\AgdaSymbol{→}\AgdaSpace{}%
\AgdaBound{P}\AgdaSpace{}%
\AgdaBound{x}\AgdaSpace{}%
\AgdaOperator{\AgdaDatatype{≡}}\AgdaSpace{}%
\AgdaBound{Q}\AgdaSpace{}%
\AgdaBound{x}\AgdaSymbol{)}\AgdaSpace{}%
\AgdaSymbol{→}\AgdaSpace{}%
\AgdaBound{P}\AgdaSpace{}%
\AgdaOperator{\AgdaDatatype{≡}}\AgdaSpace{}%
\AgdaBound{Q}\<%
\\
%
\\[\AgdaEmptyExtraSkip]%
%
\>[4]\AgdaFunction{ν∘foldlδℒ≡ℒ}\AgdaSpace{}%
\AgdaSymbol{\{}\AgdaArgument{ℒ}\AgdaSpace{}%
\AgdaSymbol{=}\AgdaSpace{}%
\AgdaBound{ℒ}\AgdaSymbol{\}}\AgdaSpace{}%
\AgdaSymbol{=}\AgdaSpace{}%
\AgdaPostulate{funext}\AgdaSpace{}%
\AgdaSymbol{(}\AgdaFunction{ν∘foldlδℒ≡ℒ′}\AgdaSpace{}%
\AgdaBound{ℒ}\AgdaSymbol{)}\<%
\end{code}
\end{theorem}

\begin{code}%
%
\>[4]\AgdaFunction{ν}\AgdaSpace{}%
\AgdaSymbol{:}\AgdaSpace{}%
\AgdaSymbol{(}\AgdaBound{e}\AgdaSpace{}%
\AgdaSymbol{:}\AgdaSpace{}%
\AgdaDatatype{Exp}\AgdaSymbol{)}\AgdaSpace{}%
\AgdaSymbol{→}\AgdaSpace{}%
\AgdaRecord{Dec}\AgdaSpace{}%
\AgdaSymbol{(}\AgdaFunction{◇ν}\AgdaSpace{}%
\AgdaOperator{\AgdaFunction{⟦}}\AgdaSpace{}%
\AgdaBound{e}\AgdaSpace{}%
\AgdaOperator{\AgdaFunction{⟧}}\AgdaSymbol{)}\<%
\\
%
\>[4]\AgdaFunction{δ}\AgdaSpace{}%
\AgdaSymbol{:}\AgdaSpace{}%
\AgdaPostulate{Char}\AgdaSpace{}%
\AgdaSymbol{→}\AgdaSpace{}%
\AgdaDatatype{Exp}\AgdaSpace{}%
\AgdaSymbol{→}\AgdaSpace{}%
\AgdaDatatype{Exp}\<%
\\
%
\>[4]\AgdaFunction{δ-sound}\AgdaSpace{}%
\AgdaSymbol{:}\AgdaSpace{}%
\AgdaSymbol{∀}\AgdaSpace{}%
\AgdaBound{e}\AgdaSpace{}%
\AgdaSymbol{→}\AgdaSpace{}%
\AgdaOperator{\AgdaFunction{⟦}}\AgdaSpace{}%
\AgdaFunction{δ}\AgdaSpace{}%
\AgdaGeneralizable{c}\AgdaSpace{}%
\AgdaBound{e}\AgdaSpace{}%
\AgdaOperator{\AgdaFunction{⟧}}\AgdaSpace{}%
\AgdaGeneralizable{w}\AgdaSpace{}%
\AgdaSymbol{→}\AgdaSpace{}%
\AgdaFunction{◇δ}\AgdaSpace{}%
\AgdaGeneralizable{c}\AgdaSpace{}%
\AgdaOperator{\AgdaFunction{⟦}}\AgdaSpace{}%
\AgdaBound{e}\AgdaSpace{}%
\AgdaOperator{\AgdaFunction{⟧}}\AgdaSpace{}%
\AgdaGeneralizable{w}\<%
\\
%
\>[4]\AgdaFunction{δ-complete}\AgdaSpace{}%
\AgdaSymbol{:}\AgdaSpace{}%
\AgdaSymbol{∀}\AgdaSpace{}%
\AgdaBound{e}\AgdaSpace{}%
\AgdaSymbol{→}\AgdaSpace{}%
\AgdaFunction{◇δ}\AgdaSpace{}%
\AgdaGeneralizable{c}\AgdaSpace{}%
\AgdaOperator{\AgdaFunction{⟦}}\AgdaSpace{}%
\AgdaBound{e}\AgdaSpace{}%
\AgdaOperator{\AgdaFunction{⟧}}\AgdaSpace{}%
\AgdaGeneralizable{w}\AgdaSpace{}%
\AgdaSymbol{→}\AgdaSpace{}%
\AgdaOperator{\AgdaFunction{⟦}}\AgdaSpace{}%
\AgdaFunction{δ}\AgdaSpace{}%
\AgdaGeneralizable{c}\AgdaSpace{}%
\AgdaBound{e}\AgdaSpace{}%
\AgdaOperator{\AgdaFunction{⟧}}\AgdaSpace{}%
\AgdaGeneralizable{w}\<%
\end{code}

\begin{code}[hide]%
%
\>[4]\AgdaFunction{map'}\AgdaSpace{}%
\AgdaSymbol{:}\AgdaSpace{}%
\AgdaSymbol{∀\{}\AgdaBound{A}\AgdaSpace{}%
\AgdaBound{B}\AgdaSymbol{\}}\AgdaSpace{}%
\AgdaSymbol{→}\AgdaSpace{}%
\AgdaSymbol{(}\AgdaBound{A}\AgdaSpace{}%
\AgdaSymbol{→}\AgdaSpace{}%
\AgdaBound{B}\AgdaSymbol{)}\AgdaSpace{}%
\AgdaSymbol{→}\AgdaSpace{}%
\AgdaSymbol{(}\AgdaBound{B}\AgdaSpace{}%
\AgdaSymbol{→}\AgdaSpace{}%
\AgdaBound{A}\AgdaSymbol{)}\AgdaSpace{}%
\AgdaSymbol{→}\AgdaSpace{}%
\AgdaRecord{Dec}\AgdaSpace{}%
\AgdaBound{A}\AgdaSpace{}%
\AgdaSymbol{→}\AgdaSpace{}%
\AgdaRecord{Dec}\AgdaSpace{}%
\AgdaBound{B}\<%
\\
%
\>[4]\AgdaFunction{map'}\AgdaSpace{}%
\AgdaSymbol{=}\AgdaSpace{}%
\AgdaFunction{map′}\<%
\end{code}
\begin{code}%
%
\>[4]\AgdaFunction{parse}\AgdaSpace{}%
\AgdaSymbol{:}\AgdaSpace{}%
\AgdaSymbol{(}\AgdaBound{e}\AgdaSpace{}%
\AgdaSymbol{:}\AgdaSpace{}%
\AgdaDatatype{Exp}\AgdaSymbol{)}\AgdaSpace{}%
\AgdaSymbol{(}\AgdaBound{w}\AgdaSpace{}%
\AgdaSymbol{:}\AgdaSpace{}%
\AgdaFunction{String}\AgdaSymbol{)}\AgdaSpace{}%
\AgdaSymbol{→}\AgdaSpace{}%
\AgdaRecord{Dec}\AgdaSpace{}%
\AgdaSymbol{(}\AgdaOperator{\AgdaFunction{⟦}}\AgdaSpace{}%
\AgdaBound{e}\AgdaSpace{}%
\AgdaOperator{\AgdaFunction{⟧}}\AgdaSpace{}%
\AgdaBound{w}\AgdaSymbol{)}\<%
\\
%
\>[4]\AgdaFunction{parse}\AgdaSpace{}%
\AgdaBound{e}\AgdaSpace{}%
\AgdaInductiveConstructor{[]}\AgdaSpace{}%
\AgdaSymbol{=}\AgdaSpace{}%
\AgdaFunction{ν}\AgdaSpace{}%
\AgdaBound{e}\<%
\\
%
\>[4]\AgdaFunction{parse}\AgdaSpace{}%
\AgdaBound{e}\AgdaSpace{}%
\AgdaSymbol{(}\AgdaBound{c}\AgdaSpace{}%
\AgdaOperator{\AgdaInductiveConstructor{∷}}\AgdaSpace{}%
\AgdaBound{w}\AgdaSymbol{)}\AgdaSpace{}%
\AgdaSymbol{=}\AgdaSpace{}%
\AgdaFunction{map'}\AgdaSpace{}%
\AgdaSymbol{(}\AgdaFunction{δ-sound}\AgdaSpace{}%
\AgdaBound{e}\AgdaSymbol{)}\AgdaSpace{}%
\AgdaSymbol{(}\AgdaFunction{δ-complete}\AgdaSpace{}%
\AgdaBound{e}\AgdaSymbol{)}\AgdaSpace{}%
\AgdaSymbol{(}\AgdaFunction{parse}\AgdaSpace{}%
\AgdaSymbol{(}\AgdaFunction{δ}\AgdaSpace{}%
\AgdaBound{c}\AgdaSpace{}%
\AgdaBound{e}\AgdaSymbol{)}\AgdaSpace{}%
\AgdaBound{w}\AgdaSymbol{)}\<%
\end{code}

\subsection{Nullability}

\begin{lemma}
Two languages, $\ab{ℒ₁}$ and $\ab{ℒ₂}$, are nullable if and only if their concatenation, $\ab{ℒ₁}~\af{◇.*}~\ab{ℒ₂}$, is nullable. 
\begin{code}%
%
\>[4]\AgdaFunction{ν*}\AgdaSpace{}%
\AgdaSymbol{:}\AgdaSpace{}%
\AgdaSymbol{(}\AgdaFunction{◇ν}\AgdaSpace{}%
\AgdaGeneralizable{ℒ₁}\AgdaSpace{}%
\AgdaOperator{\AgdaFunction{×}}\AgdaSpace{}%
\AgdaFunction{◇ν}\AgdaSpace{}%
\AgdaGeneralizable{ℒ₂}\AgdaSymbol{)}\AgdaSpace{}%
\AgdaOperator{\AgdaFunction{⇔}}\AgdaSpace{}%
\AgdaFunction{◇ν}\AgdaSpace{}%
\AgdaSymbol{(}\AgdaGeneralizable{ℒ₁}\AgdaSpace{}%
\AgdaOperator{\AgdaFunction{◇.*}}\AgdaSpace{}%
\AgdaGeneralizable{ℒ₂}\AgdaSymbol{)}\<%
\end{code}
\begin{code}[hide]%
%
\>[4]\AgdaFunction{ν*}\AgdaSpace{}%
\AgdaSymbol{=}\AgdaSpace{}%
\AgdaFunction{mk⇔}\AgdaSpace{}%
\AgdaSymbol{(λ}\AgdaSpace{}%
\AgdaBound{x}\AgdaSpace{}%
\AgdaSymbol{→}\AgdaSpace{}%
\AgdaInductiveConstructor{[]}\AgdaSpace{}%
\AgdaOperator{\AgdaInductiveConstructor{,}}\AgdaSpace{}%
\AgdaInductiveConstructor{[]}\AgdaSpace{}%
\AgdaOperator{\AgdaInductiveConstructor{,}}\AgdaSpace{}%
\AgdaInductiveConstructor{refl}\AgdaSpace{}%
\AgdaOperator{\AgdaInductiveConstructor{,}}\AgdaSpace{}%
\AgdaBound{x}\AgdaSymbol{)}\AgdaSpace{}%
\AgdaSymbol{λ}\AgdaSpace{}%
\AgdaSymbol{\{}\AgdaSpace{}%
\AgdaSymbol{(}\AgdaInductiveConstructor{[]}\AgdaSpace{}%
\AgdaOperator{\AgdaInductiveConstructor{,}}\AgdaSpace{}%
\AgdaInductiveConstructor{[]}\AgdaSpace{}%
\AgdaOperator{\AgdaInductiveConstructor{,}}\AgdaSpace{}%
\AgdaInductiveConstructor{refl}\AgdaSpace{}%
\AgdaOperator{\AgdaInductiveConstructor{,}}\AgdaSpace{}%
\AgdaBound{x}\AgdaSymbol{)}\AgdaSpace{}%
\AgdaSymbol{→}\AgdaSpace{}%
\AgdaBound{x}\AgdaSpace{}%
\AgdaSymbol{\}}\<%
\end{code}
\end{lemma}

\begin{code}%
%
\>[4]\AgdaFunction{ν}\AgdaSpace{}%
\AgdaInductiveConstructor{∅}\AgdaSpace{}%
\AgdaSymbol{=}\AgdaSpace{}%
\AgdaInductiveConstructor{no}\AgdaSpace{}%
\AgdaSymbol{λ}\AgdaSpace{}%
\AgdaSymbol{()}\<%
\\
%
\>[4]\AgdaFunction{ν}\AgdaSpace{}%
\AgdaInductiveConstructor{ε}\AgdaSpace{}%
\AgdaSymbol{=}\AgdaSpace{}%
\AgdaInductiveConstructor{yes}\AgdaSpace{}%
\AgdaInductiveConstructor{refl}\<%
\\
%
\>[4]\AgdaFunction{ν}\AgdaSpace{}%
\AgdaSymbol{(}\AgdaOperator{\AgdaInductiveConstructor{`}}\AgdaSpace{}%
\AgdaBound{c}\AgdaSymbol{)}\AgdaSpace{}%
\AgdaSymbol{=}\AgdaSpace{}%
\AgdaInductiveConstructor{no}\AgdaSpace{}%
\AgdaSymbol{λ}\AgdaSpace{}%
\AgdaSymbol{()}\<%
\\
%
\>[4]\AgdaFunction{ν}\AgdaSpace{}%
\AgdaSymbol{(}\AgdaBound{x}\AgdaSpace{}%
\AgdaOperator{\AgdaInductiveConstructor{·}}\AgdaSpace{}%
\AgdaBound{e}\AgdaSymbol{)}\AgdaSpace{}%
\AgdaSymbol{=}\AgdaSpace{}%
\AgdaBound{x}\AgdaSpace{}%
\AgdaOperator{\AgdaFunction{×-dec}}\AgdaSpace{}%
\AgdaFunction{ν}\AgdaSpace{}%
\AgdaBound{e}\<%
\\
%
\>[4]\AgdaFunction{ν}\AgdaSpace{}%
\AgdaSymbol{(}\AgdaBound{e}\AgdaSpace{}%
\AgdaOperator{\AgdaInductiveConstructor{∪}}\AgdaSpace{}%
\AgdaBound{e₁}\AgdaSymbol{)}\AgdaSpace{}%
\AgdaSymbol{=}\AgdaSpace{}%
\AgdaFunction{ν}\AgdaSpace{}%
\AgdaBound{e}\AgdaSpace{}%
\AgdaOperator{\AgdaFunction{⊎-dec}}\AgdaSpace{}%
\AgdaFunction{ν}\AgdaSpace{}%
\AgdaBound{e₁}\<%
\\
%
\>[4]\AgdaFunction{ν}\AgdaSpace{}%
\AgdaSymbol{(}\AgdaBound{e}\AgdaSpace{}%
\AgdaOperator{\AgdaInductiveConstructor{*}}\AgdaSpace{}%
\AgdaBound{e₁}\AgdaSymbol{)}\AgdaSpace{}%
\AgdaSymbol{=}\AgdaSpace{}%
\AgdaFunction{Dec.map}\AgdaSpace{}%
\AgdaFunction{ν*}\AgdaSpace{}%
\AgdaSymbol{(}\AgdaFunction{ν}\AgdaSpace{}%
\AgdaBound{e}\AgdaSpace{}%
\AgdaOperator{\AgdaFunction{×-dec}}\AgdaSpace{}%
\AgdaFunction{ν}\AgdaSpace{}%
\AgdaBound{e₁}\AgdaSymbol{)}\<%
\end{code}

\subsection{Derivation}

\begin{code}%
%
\>[4]\AgdaFunction{δ}\AgdaSpace{}%
\AgdaBound{c}\AgdaSpace{}%
\AgdaInductiveConstructor{∅}\AgdaSpace{}%
\AgdaSymbol{=}\AgdaSpace{}%
\AgdaInductiveConstructor{∅}\<%
\\
%
\>[4]\AgdaFunction{δ}\AgdaSpace{}%
\AgdaBound{c}\AgdaSpace{}%
\AgdaInductiveConstructor{ε}\AgdaSpace{}%
\AgdaSymbol{=}\AgdaSpace{}%
\AgdaInductiveConstructor{∅}\<%
\\
%
\>[4]\AgdaFunction{δ}\AgdaSpace{}%
\AgdaBound{c}\AgdaSpace{}%
\AgdaSymbol{(}\AgdaOperator{\AgdaInductiveConstructor{`}}\AgdaSpace{}%
\AgdaBound{c₁}\AgdaSymbol{)}\AgdaSpace{}%
\AgdaSymbol{=}\AgdaSpace{}%
\AgdaSymbol{(}\AgdaBound{c}\AgdaSpace{}%
\AgdaOperator{\AgdaFunction{≟}}\AgdaSpace{}%
\AgdaBound{c₁}\AgdaSymbol{)}\AgdaSpace{}%
\AgdaOperator{\AgdaInductiveConstructor{·}}\AgdaSpace{}%
\AgdaInductiveConstructor{ε}\AgdaSpace{}%
\AgdaComment{--\ a\ bit\ interesting}\<%
\\
%
\>[4]\AgdaFunction{δ}\AgdaSpace{}%
\AgdaBound{c}\AgdaSpace{}%
\AgdaSymbol{(}\AgdaBound{x}\AgdaSpace{}%
\AgdaOperator{\AgdaInductiveConstructor{·}}\AgdaSpace{}%
\AgdaBound{e}\AgdaSymbol{)}\AgdaSpace{}%
\AgdaSymbol{=}\AgdaSpace{}%
\AgdaBound{x}\AgdaSpace{}%
\AgdaOperator{\AgdaInductiveConstructor{·}}\AgdaSpace{}%
\AgdaFunction{δ}\AgdaSpace{}%
\AgdaBound{c}\AgdaSpace{}%
\AgdaBound{e}\<%
\\
%
\>[4]\AgdaFunction{δ}\AgdaSpace{}%
\AgdaBound{c}\AgdaSpace{}%
\AgdaSymbol{(}\AgdaBound{e}\AgdaSpace{}%
\AgdaOperator{\AgdaInductiveConstructor{∪}}\AgdaSpace{}%
\AgdaBound{e₁}\AgdaSymbol{)}\AgdaSpace{}%
\AgdaSymbol{=}\AgdaSpace{}%
\AgdaFunction{δ}\AgdaSpace{}%
\AgdaBound{c}\AgdaSpace{}%
\AgdaBound{e}\AgdaSpace{}%
\AgdaOperator{\AgdaInductiveConstructor{∪}}\AgdaSpace{}%
\AgdaFunction{δ}\AgdaSpace{}%
\AgdaBound{c}\AgdaSpace{}%
\AgdaBound{e₁}\<%
\\
%
\>[4]\AgdaFunction{δ}\AgdaSpace{}%
\AgdaBound{c}\AgdaSpace{}%
\AgdaSymbol{(}\AgdaBound{e}\AgdaSpace{}%
\AgdaOperator{\AgdaInductiveConstructor{*}}\AgdaSpace{}%
\AgdaBound{e₁}\AgdaSymbol{)}\AgdaSpace{}%
\AgdaSymbol{=}\AgdaSpace{}%
\AgdaSymbol{(}\AgdaFunction{δ}\AgdaSpace{}%
\AgdaBound{c}\AgdaSpace{}%
\AgdaBound{e}\AgdaSpace{}%
\AgdaOperator{\AgdaInductiveConstructor{*}}\AgdaSpace{}%
\AgdaBound{e₁}\AgdaSymbol{)}\AgdaSpace{}%
\AgdaOperator{\AgdaInductiveConstructor{∪}}\AgdaSpace{}%
\AgdaSymbol{(}\AgdaFunction{ν}\AgdaSpace{}%
\AgdaBound{e}\AgdaSpace{}%
\AgdaOperator{\AgdaInductiveConstructor{·}}\AgdaSpace{}%
\AgdaFunction{δ}\AgdaSpace{}%
\AgdaBound{c}\AgdaSpace{}%
\AgdaBound{e₁}\AgdaSymbol{)}\AgdaSpace{}%
\AgdaComment{--\ interesting}\<%
\end{code}

The proofs are very straightforward:

% \begin{code}
%     open Equivalence
%     δ-correct ∅ = ⇔.refl
%     δ-correct ε = mk⇔ (λ ()) (λ ())
%     δ-correct (` c) = mk⇔ (λ { (refl , refl) → refl }) λ { refl → refl , refl }
%     δ-correct (x · e) = mk⇔ (λ (x , y) → x , δ-correct e .to y) λ (x , y) → x , δ-correct e .from y
%     δ-correct (e ∪ e₁) = mk⇔ (λ { (inj₁ x) → inj₁ (δ-correct e .to x) ; (inj₂ x) → inj₂ (δ-correct e₁ .to x) }) (λ { (inj₁ x) → inj₁ (δ-correct e .from x) ; (inj₂ x) → inj₂ (δ-correct e₁ .from x) })
%     δ-correct (e * e₁) = {!   !}
% \end{code}

\begin{code}%
%
\>[4]\AgdaFunction{δ-sound}\AgdaSpace{}%
\AgdaSymbol{(}\AgdaOperator{\AgdaInductiveConstructor{`}}\AgdaSpace{}%
\AgdaBound{c}\AgdaSymbol{)}\AgdaSpace{}%
\AgdaSymbol{(}\AgdaInductiveConstructor{refl}\AgdaSpace{}%
\AgdaOperator{\AgdaInductiveConstructor{,}}\AgdaSpace{}%
\AgdaInductiveConstructor{refl}\AgdaSymbol{)}\AgdaSpace{}%
\AgdaSymbol{=}\AgdaSpace{}%
\AgdaInductiveConstructor{refl}\<%
\\
%
\>[4]\AgdaFunction{δ-sound}\AgdaSpace{}%
\AgdaSymbol{(}\AgdaBound{x₁}\AgdaSpace{}%
\AgdaOperator{\AgdaInductiveConstructor{·}}\AgdaSpace{}%
\AgdaBound{e}\AgdaSymbol{)}\AgdaSpace{}%
\AgdaSymbol{(}\AgdaBound{x}\AgdaSpace{}%
\AgdaOperator{\AgdaInductiveConstructor{,}}\AgdaSpace{}%
\AgdaBound{y}\AgdaSymbol{)}\AgdaSpace{}%
\AgdaSymbol{=}\AgdaSpace{}%
\AgdaBound{x}\AgdaSpace{}%
\AgdaOperator{\AgdaInductiveConstructor{,}}\AgdaSpace{}%
\AgdaFunction{δ-sound}\AgdaSpace{}%
\AgdaBound{e}\AgdaSpace{}%
\AgdaBound{y}\<%
\\
%
\>[4]\AgdaFunction{δ-sound}\AgdaSpace{}%
\AgdaSymbol{(}\AgdaBound{e}\AgdaSpace{}%
\AgdaOperator{\AgdaInductiveConstructor{∪}}\AgdaSpace{}%
\AgdaBound{e₁}\AgdaSymbol{)}\AgdaSpace{}%
\AgdaSymbol{(}\AgdaInductiveConstructor{inj₁}\AgdaSpace{}%
\AgdaBound{x}\AgdaSymbol{)}\AgdaSpace{}%
\AgdaSymbol{=}\AgdaSpace{}%
\AgdaInductiveConstructor{inj₁}\AgdaSpace{}%
\AgdaSymbol{(}\AgdaFunction{δ-sound}\AgdaSpace{}%
\AgdaBound{e}\AgdaSpace{}%
\AgdaBound{x}\AgdaSymbol{)}\<%
\\
%
\>[4]\AgdaFunction{δ-sound}\AgdaSpace{}%
\AgdaSymbol{(}\AgdaBound{e}\AgdaSpace{}%
\AgdaOperator{\AgdaInductiveConstructor{∪}}\AgdaSpace{}%
\AgdaBound{e₁}\AgdaSymbol{)}\AgdaSpace{}%
\AgdaSymbol{(}\AgdaInductiveConstructor{inj₂}\AgdaSpace{}%
\AgdaBound{y}\AgdaSymbol{)}\AgdaSpace{}%
\AgdaSymbol{=}\AgdaSpace{}%
\AgdaInductiveConstructor{inj₂}\AgdaSpace{}%
\AgdaSymbol{(}\AgdaFunction{δ-sound}\AgdaSpace{}%
\AgdaBound{e₁}\AgdaSpace{}%
\AgdaBound{y}\AgdaSymbol{)}\<%
\\
%
\>[4]\AgdaFunction{δ-sound}\AgdaSpace{}%
\AgdaSymbol{(}\AgdaBound{e}\AgdaSpace{}%
\AgdaOperator{\AgdaInductiveConstructor{*}}\AgdaSpace{}%
\AgdaBound{e₁}\AgdaSymbol{)}\AgdaSpace{}%
\AgdaSymbol{(}\AgdaInductiveConstructor{inj₁}\AgdaSpace{}%
\AgdaSymbol{(}\AgdaBound{u}\AgdaSpace{}%
\AgdaOperator{\AgdaInductiveConstructor{,}}\AgdaSpace{}%
\AgdaBound{v}\AgdaSpace{}%
\AgdaOperator{\AgdaInductiveConstructor{,}}\AgdaSpace{}%
\AgdaInductiveConstructor{refl}\AgdaSpace{}%
\AgdaOperator{\AgdaInductiveConstructor{,}}\AgdaSpace{}%
\AgdaBound{x}\AgdaSpace{}%
\AgdaOperator{\AgdaInductiveConstructor{,}}\AgdaSpace{}%
\AgdaBound{y}\AgdaSymbol{))}\AgdaSpace{}%
\AgdaSymbol{=}\AgdaSpace{}%
\AgdaSymbol{\AgdaUnderscore{}}\AgdaSpace{}%
\AgdaOperator{\AgdaInductiveConstructor{∷}}\AgdaSpace{}%
\AgdaBound{u}\AgdaSpace{}%
\AgdaOperator{\AgdaInductiveConstructor{,}}\AgdaSpace{}%
\AgdaBound{v}\AgdaSpace{}%
\AgdaOperator{\AgdaInductiveConstructor{,}}\AgdaSpace{}%
\AgdaInductiveConstructor{refl}\AgdaSpace{}%
\AgdaOperator{\AgdaInductiveConstructor{,}}\AgdaSpace{}%
\AgdaFunction{δ-sound}\AgdaSpace{}%
\AgdaBound{e}\AgdaSpace{}%
\AgdaBound{x}\AgdaSpace{}%
\AgdaOperator{\AgdaInductiveConstructor{,}}\AgdaSpace{}%
\AgdaBound{y}\<%
\\
%
\>[4]\AgdaFunction{δ-sound}\AgdaSpace{}%
\AgdaSymbol{(}\AgdaBound{e}\AgdaSpace{}%
\AgdaOperator{\AgdaInductiveConstructor{*}}\AgdaSpace{}%
\AgdaBound{e₁}\AgdaSymbol{)}\AgdaSpace{}%
\AgdaSymbol{(}\AgdaInductiveConstructor{inj₂}\AgdaSpace{}%
\AgdaSymbol{(}\AgdaBound{x}\AgdaSpace{}%
\AgdaOperator{\AgdaInductiveConstructor{,}}\AgdaSpace{}%
\AgdaBound{y}\AgdaSymbol{))}\AgdaSpace{}%
\AgdaSymbol{=}\AgdaSpace{}%
\AgdaInductiveConstructor{[]}\AgdaSpace{}%
\AgdaOperator{\AgdaInductiveConstructor{,}}\AgdaSpace{}%
\AgdaSymbol{\AgdaUnderscore{}}\AgdaSpace{}%
\AgdaOperator{\AgdaInductiveConstructor{,}}\AgdaSpace{}%
\AgdaInductiveConstructor{refl}\AgdaSpace{}%
\AgdaOperator{\AgdaInductiveConstructor{,}}\AgdaSpace{}%
\AgdaBound{x}\AgdaSpace{}%
\AgdaOperator{\AgdaInductiveConstructor{,}}\AgdaSpace{}%
\AgdaFunction{δ-sound}\AgdaSpace{}%
\AgdaBound{e₁}\AgdaSpace{}%
\AgdaBound{y}\<%
\end{code}

\begin{code}%
%
\>[4]\AgdaFunction{δ-complete}\AgdaSpace{}%
\AgdaSymbol{(}\AgdaOperator{\AgdaInductiveConstructor{`}}\AgdaSpace{}%
\AgdaBound{c}\AgdaSymbol{)}\AgdaSpace{}%
\AgdaInductiveConstructor{refl}\AgdaSpace{}%
\AgdaSymbol{=}\AgdaSpace{}%
\AgdaInductiveConstructor{refl}\AgdaSpace{}%
\AgdaOperator{\AgdaInductiveConstructor{,}}\AgdaSpace{}%
\AgdaInductiveConstructor{refl}\<%
\\
%
\>[4]\AgdaFunction{δ-complete}\AgdaSpace{}%
\AgdaSymbol{(}\AgdaBound{x₁}\AgdaSpace{}%
\AgdaOperator{\AgdaInductiveConstructor{·}}\AgdaSpace{}%
\AgdaBound{e}\AgdaSymbol{)}\AgdaSpace{}%
\AgdaSymbol{(}\AgdaBound{x}\AgdaSpace{}%
\AgdaOperator{\AgdaInductiveConstructor{,}}\AgdaSpace{}%
\AgdaBound{y}\AgdaSymbol{)}\AgdaSpace{}%
\AgdaSymbol{=}\AgdaSpace{}%
\AgdaBound{x}\AgdaSpace{}%
\AgdaOperator{\AgdaInductiveConstructor{,}}\AgdaSpace{}%
\AgdaFunction{δ-complete}\AgdaSpace{}%
\AgdaBound{e}\AgdaSpace{}%
\AgdaBound{y}\<%
\\
%
\>[4]\AgdaFunction{δ-complete}\AgdaSpace{}%
\AgdaSymbol{(}\AgdaBound{e}\AgdaSpace{}%
\AgdaOperator{\AgdaInductiveConstructor{∪}}\AgdaSpace{}%
\AgdaBound{e₁}\AgdaSymbol{)}\AgdaSpace{}%
\AgdaSymbol{(}\AgdaInductiveConstructor{inj₁}\AgdaSpace{}%
\AgdaBound{x}\AgdaSymbol{)}\AgdaSpace{}%
\AgdaSymbol{=}\AgdaSpace{}%
\AgdaInductiveConstructor{inj₁}\AgdaSpace{}%
\AgdaSymbol{(}\AgdaFunction{δ-complete}\AgdaSpace{}%
\AgdaBound{e}\AgdaSpace{}%
\AgdaBound{x}\AgdaSymbol{)}\<%
\\
%
\>[4]\AgdaFunction{δ-complete}\AgdaSpace{}%
\AgdaSymbol{(}\AgdaBound{e}\AgdaSpace{}%
\AgdaOperator{\AgdaInductiveConstructor{∪}}\AgdaSpace{}%
\AgdaBound{e₁}\AgdaSymbol{)}\AgdaSpace{}%
\AgdaSymbol{(}\AgdaInductiveConstructor{inj₂}\AgdaSpace{}%
\AgdaBound{y}\AgdaSymbol{)}\AgdaSpace{}%
\AgdaSymbol{=}\AgdaSpace{}%
\AgdaInductiveConstructor{inj₂}\AgdaSpace{}%
\AgdaSymbol{(}\AgdaFunction{δ-complete}\AgdaSpace{}%
\AgdaBound{e₁}\AgdaSpace{}%
\AgdaBound{y}\AgdaSymbol{)}\<%
\\
%
\>[4]\AgdaFunction{δ-complete}\AgdaSpace{}%
\AgdaSymbol{(}\AgdaBound{e}\AgdaSpace{}%
\AgdaOperator{\AgdaInductiveConstructor{*}}\AgdaSpace{}%
\AgdaBound{e₁}\AgdaSymbol{)}\AgdaSpace{}%
\AgdaSymbol{(\AgdaUnderscore{}}\AgdaSpace{}%
\AgdaOperator{\AgdaInductiveConstructor{∷}}\AgdaSpace{}%
\AgdaSymbol{\AgdaUnderscore{}}\AgdaSpace{}%
\AgdaOperator{\AgdaInductiveConstructor{,}}\AgdaSpace{}%
\AgdaSymbol{\AgdaUnderscore{}}\AgdaSpace{}%
\AgdaOperator{\AgdaInductiveConstructor{,}}\AgdaSpace{}%
\AgdaInductiveConstructor{refl}\AgdaSpace{}%
\AgdaOperator{\AgdaInductiveConstructor{,}}\AgdaSpace{}%
\AgdaBound{x}\AgdaSpace{}%
\AgdaOperator{\AgdaInductiveConstructor{,}}\AgdaSpace{}%
\AgdaBound{y}\AgdaSymbol{)}\AgdaSpace{}%
\AgdaSymbol{=}\AgdaSpace{}%
\AgdaInductiveConstructor{inj₁}\AgdaSpace{}%
\AgdaSymbol{(\AgdaUnderscore{}}\AgdaSpace{}%
\AgdaOperator{\AgdaInductiveConstructor{,}}\AgdaSpace{}%
\AgdaSymbol{\AgdaUnderscore{}}\AgdaSpace{}%
\AgdaOperator{\AgdaInductiveConstructor{,}}\AgdaSpace{}%
\AgdaInductiveConstructor{refl}\AgdaSpace{}%
\AgdaOperator{\AgdaInductiveConstructor{,}}\AgdaSpace{}%
\AgdaFunction{δ-complete}\AgdaSpace{}%
\AgdaBound{e}\AgdaSpace{}%
\AgdaBound{x}\AgdaSpace{}%
\AgdaOperator{\AgdaInductiveConstructor{,}}\AgdaSpace{}%
\AgdaBound{y}\AgdaSymbol{)}\<%
\\
%
\>[4]\AgdaFunction{δ-complete}\AgdaSpace{}%
\AgdaSymbol{(}\AgdaBound{e}\AgdaSpace{}%
\AgdaOperator{\AgdaInductiveConstructor{*}}\AgdaSpace{}%
\AgdaBound{e₁}\AgdaSymbol{)}\AgdaSpace{}%
\AgdaSymbol{(}\AgdaInductiveConstructor{[]}\AgdaSpace{}%
\AgdaOperator{\AgdaInductiveConstructor{,}}\AgdaSpace{}%
\AgdaSymbol{\AgdaUnderscore{}}\AgdaSpace{}%
\AgdaOperator{\AgdaInductiveConstructor{,}}\AgdaSpace{}%
\AgdaInductiveConstructor{refl}\AgdaSpace{}%
\AgdaOperator{\AgdaInductiveConstructor{,}}\AgdaSpace{}%
\AgdaBound{x}\AgdaSpace{}%
\AgdaOperator{\AgdaInductiveConstructor{,}}\AgdaSpace{}%
\AgdaBound{y}\AgdaSymbol{)}\AgdaSpace{}%
\AgdaSymbol{=}\AgdaSpace{}%
\AgdaInductiveConstructor{inj₂}\AgdaSpace{}%
\AgdaSymbol{(}\AgdaBound{x}\AgdaSpace{}%
\AgdaOperator{\AgdaInductiveConstructor{,}}\AgdaSpace{}%
\AgdaFunction{δ-complete}\AgdaSpace{}%
\AgdaBound{e₁}\AgdaSpace{}%
\AgdaBound{y}\AgdaSymbol{)}\<%
\end{code}

% \begin{code}[hide]
% module Simple where
% \end{code}
% \begin{code}
%     data Gram : Lang → Type₁ where
%         ∅     :                       Gram (λ _ → ⊥)
%         ε     :                       Gram (λ w → w ≡ [])
%         char  : (c : Char)         →  Gram (λ w → w ≡ c ∷ [])
%         _·_   : Dec A → Gram ℒ     →  Gram (λ w → A × ℒ w)
%         _∪_   : Gram ℒ₁ → Gram ℒ₂  →  Gram (λ w → ℒ₁ w ⊎ ℒ₂ w)
%         _∗_   : Gram ℒ₁ → Gram ℒ₂
%               → Gram (λ w → Σ String λ u → Σ String λ v → (w ≡ u ++ v) × ℒ₁ u × ℒ₂ v)
%         _◃_   : (ℒ₁ ⇔ ℒ₂) → Gram ℒ₁ → Gram ℒ₂
% \end{code}
% \begin{code}[hide]
%     variable G G₁ G₂ : Gram ℒ
% \end{code}
% \begin{remark}
% The \af{Gram} data type is parameterized by its language. This ties the constructors directly to their semantics.
% \end{remark}
% 
% By recursion over this data type of grammars, we can define a decision procedure for nullability and derivative function; both are correct by construction.
% \begin{code}
%     ν? : Gram ℒ → Dec (ν ℒ)
%     δ? : Gram ℒ → (c : Char) → Gram (δ ℒ c)
% \end{code}
% \begin{code}[hide]
%     ν∗ : (ν ℒ₁ × ν ℒ₂) ↔ Σ String λ u → Σ String λ v → ([] ≡ (u ++ v)) × ℒ₁ u × ℒ₂ v
%     to ν∗ (x , y) = [] , [] , refl , x , y
%     from ν∗ ([] , [] , refl , x , y) = x , y
% 
%     ν? ∅ = no λ ()
%     ν? ε = yes refl
%     ν? (char c) = no λ ()
%     ν? (x · G) = x ×? ν? G
%     ν? (G₁ ∪ G₂) = ν? G₁ ⊎? ν? G₂
%     ν? (G₁ ∗ G₂) = map? ν∗ (ν? G₁ ×? ν? G₂)
%     ν? (f ◃ G₂) = map? f (ν? G₂)
% \end{code}
% \begin{code}[hide]
%     δ? ∅ c = ∅
%     δ? ε c = record { to = λ () ; from = λ () } ◃ ∅
%     δ? (char c′) c with c ≟ c′
%     ... | yes refl = (λ { {[]} → record { to = λ _ → refl ; from = λ _ → refl } ; {_ ∷ _} → record { to = λ () ; from = λ () }}) ◃ ε
%     ... | no ¬c≡c′ = (λ { {[]} → record { to = λ () ; from = λ { refl → ¬c≡c′ refl }} ; {_ ∷ _} → record { to = λ () ; from = λ () }}) ◃ ∅
%     δ? (A · G) c = A · δ? G c
%     δ? (G₁ ∪ G₂) c = δ? G₁ c ∪ δ? G₂ c
%     δ? (G₁ ∗ G₂) c = (record { to = λ { (inl (u , v , refl , x , y)) → (c ∷ u) , v , refl , x , y ; (inr (x , y)) → [] , (c ∷ _) , refl , x , y } ; from = λ { ([] , _ , refl , x , y) → inr (x , y) ; ((_ ∷ u) , v , refl , x , y) → inl (u , v , refl , x , y) } } ) ◃ ((δ? G₁ c ∗ G₂) ∪ (ν? G₁ · δ? G₂ c))
%     δ? (f ◃ G₂) c = f ◃ δ? G₂ c
% 
%     -- δ?↔δ : ⟦ δ? c G ⟧ w ↔ δ c ⟦ G ⟧ w
% \end{code}
% \begin{code}[hide]
%     -- to (δ?↔δ {c} {G = ` c′}) x with c ≟ c′
%     -- to (δ?↔δ {c} {` .c}) refl | yes refl = refl
%     -- to (δ?↔δ {_} {` _}) () | no _
%     -- to (δ?↔δ {G = A · G}) (x , y) = x , to δ?↔δ y
%     -- to (δ?↔δ {G = G₁ ∪ G₂}) (inl x) = inl (to δ?↔δ x)
%     -- to (δ?↔δ {G = G₁ ∪ G₂}) (inr x) = inr (to δ?↔δ x)
%     -- to (δ?↔δ {c} {G = G₁ ▹ G₂}) (inl (u , v , refl , x , y)) = (c ∷ u) , v , refl , to δ?↔δ x , y
%     -- to (δ?↔δ {c} {G = G₁ ▹ G₂} {w}) (inr (π₁ , π₂)) = [] , (c ∷ w) , refl , π₁ , to δ?↔δ π₂
%     -- from (δ?↔δ {c} {G = ` c′}) x with c ≟ c′
%     -- from (δ?↔δ {c} {` c}) refl | yes refl = refl
%     -- from (δ?↔δ {c} {` .c}) refl | no ¬c≡c = ¬c≡c refl
%     -- from (δ?↔δ {G = A · G}) (π₁ , π₂) = π₁ , from δ?↔δ π₂
%     -- from (δ?↔δ {G = G ∪ G₁}) (inl x) = inl (from δ?↔δ x)
%     -- from (δ?↔δ {G = G ∪ G₁}) (inr x) = inr (from δ?↔δ x)
%     -- from (δ?↔δ {c} {G = G ▹ G₁}) ([] , (.c ∷ v) , refl , x , y) = inr (x , from δ?↔δ y)
%     -- from (δ?↔δ {c} {G = G ▹ G₁}) ((.c ∷ u) , v , refl , x , y) = inl (u , v , refl , from δ?↔δ x , y)
%     transport : {ℓ₁ : Level} {A : Set ℓ₁} {B : Set ℓ₁} → A ≡ B → A → B
%     transport refl x = x
% \end{code}
% Together, decidable nullability and the derivative function can be combined to decide whether any string is in the language described by a grammar.
% \begin{code}
%     parse : Gram ℒ → (w : String) → Dec (ℒ w)
%     parse G [] = ν? G
%     parse G (c ∷ w) = parse (δ? G c) w
% \end{code}
% Thus, we have defined a parser for our simple grammars.

% A language is a set of strings $\mathbb{2}^{(\af{List}~\af{Token})}$.
% 
% 
% \begin{code}[hide]
% Lang : Set₁
% \end{code}
% \begin{code}
% Lang = List Token → Set
% \end{code}
% 
% This type has a very rich structure. It forms an ... algebra with union and intersection and a semiring with union and sequential composition.
% 
% \begin{code}
% ∅ : Lang
% ∅ _ = ⊥
% \end{code}
% 
% Going beyond work by Elliot, we can try to define context-free grammars.
% Unfortunately, we quickly run into issues due to nontermination. It is not easy
% to show that a grammar defined in this way is well-founded. To solve this issue
% we can use guarded type theory, in our case provided by guarded cubical Agda.
% This allows us to define arbitrary fixed points of languages.
% 
% \begin{code}
% fueled : (Lang → Lang) → ℕ → Lang
% fueled f 0 = ∅
% fueled f (suc n) = f (fueled f n)
% \end{code}
% 
% \begin{code}
% fix : (Lang → Lang) → Lang
% fix f w = ∃[ n ] fueled f n w
% \end{code}

\section{Context-free Languages}\label{sec:context-free}

\begin{code}[hide]%
\>[0]\AgdaKeyword{module}\AgdaSpace{}%
\AgdaModule{3-cfg}\AgdaSpace{}%
\AgdaKeyword{where}\<%
\\
%
\\[\AgdaEmptyExtraSkip]%
\>[0]\AgdaKeyword{open}\AgdaSpace{}%
\AgdaKeyword{import}\AgdaSpace{}%
\AgdaModule{Agda.Primitive}\AgdaSpace{}%
\AgdaKeyword{renaming}\AgdaSpace{}%
\AgdaSymbol{(}\AgdaPrimitive{Set}\AgdaSpace{}%
\AgdaSymbol{to}\AgdaSpace{}%
\AgdaPrimitive{Type}\AgdaSpace{}%
\AgdaSymbol{;}\AgdaSpace{}%
\AgdaPrimitive{Setω}\AgdaSpace{}%
\AgdaSymbol{to}\AgdaSpace{}%
\AgdaPrimitive{Typeω}\AgdaSymbol{)}\<%
\\
%
\\[\AgdaEmptyExtraSkip]%
\>[0]\AgdaKeyword{import}\AgdaSpace{}%
\AgdaModule{Function.Properties.Equivalence}\AgdaSpace{}%
\AgdaSymbol{as}\AgdaSpace{}%
\AgdaModule{⇔}\<%
\\
\>[0]\AgdaKeyword{import}\AgdaSpace{}%
\AgdaModule{Data.Bool}\AgdaSpace{}%
\AgdaSymbol{as}\AgdaSpace{}%
\AgdaModule{Bool}\<%
\\
\>[0]\AgdaKeyword{open}\AgdaSpace{}%
\AgdaKeyword{import}\AgdaSpace{}%
\AgdaModule{Data.Bool}\AgdaSpace{}%
\AgdaKeyword{using}\AgdaSpace{}%
\AgdaSymbol{(}\AgdaDatatype{Bool}\AgdaSpace{}%
\AgdaSymbol{;}\AgdaSpace{}%
\AgdaInductiveConstructor{true}\AgdaSpace{}%
\AgdaSymbol{;}\AgdaSpace{}%
\AgdaInductiveConstructor{false}\AgdaSymbol{)}\<%
\\
\>[0]\AgdaKeyword{open}\AgdaSpace{}%
\AgdaKeyword{import}\AgdaSpace{}%
\AgdaModule{Data.Char}\AgdaSpace{}%
\AgdaKeyword{using}\AgdaSpace{}%
\AgdaSymbol{(}\AgdaPostulate{Char}\AgdaSpace{}%
\AgdaSymbol{;}\AgdaSpace{}%
\AgdaOperator{\AgdaFunction{\AgdaUnderscore{}≟\AgdaUnderscore{}}}\AgdaSymbol{)}\<%
\\
\>[0]\AgdaKeyword{open}\AgdaSpace{}%
\AgdaKeyword{import}\AgdaSpace{}%
\AgdaModule{Data.List}\AgdaSpace{}%
\AgdaSymbol{as}\AgdaSpace{}%
\AgdaModule{List}\AgdaSpace{}%
\AgdaKeyword{hiding}\AgdaSpace{}%
\AgdaSymbol{(}\AgdaFunction{foldl}\AgdaSymbol{)}\<%
\\
\>[0]\AgdaKeyword{open}\AgdaSpace{}%
\AgdaKeyword{import}\AgdaSpace{}%
\AgdaModule{Data.Empty}\<%
\\
\>[0]\AgdaKeyword{open}\AgdaSpace{}%
\AgdaKeyword{import}\AgdaSpace{}%
\AgdaModule{Data.Product}\AgdaSpace{}%
\AgdaSymbol{as}\AgdaSpace{}%
\AgdaModule{Prod}\<%
\\
\>[0]\AgdaKeyword{open}\AgdaSpace{}%
\AgdaKeyword{import}\AgdaSpace{}%
\AgdaModule{Data.Sum}\AgdaSpace{}%
\AgdaSymbol{as}\AgdaSpace{}%
\AgdaModule{Sum}\<%
\\
\>[0]\AgdaKeyword{open}\AgdaSpace{}%
\AgdaKeyword{import}\AgdaSpace{}%
\AgdaModule{Data.Unit}\AgdaSpace{}%
\AgdaKeyword{hiding}\AgdaSpace{}%
\AgdaSymbol{(}\AgdaOperator{\AgdaFunction{\AgdaUnderscore{}≟\AgdaUnderscore{}}}\AgdaSymbol{)}\<%
\\
\>[0]\AgdaKeyword{open}\AgdaSpace{}%
\AgdaKeyword{import}\AgdaSpace{}%
\AgdaModule{Relation.Nullary.Decidable}\AgdaSpace{}%
\AgdaSymbol{as}\AgdaSpace{}%
\AgdaModule{Dec}\AgdaSpace{}%
\AgdaKeyword{hiding}\AgdaSpace{}%
\AgdaSymbol{(}\AgdaFunction{from-yes}\AgdaSpace{}%
\AgdaSymbol{;}\AgdaSpace{}%
\AgdaFunction{from-no}\AgdaSymbol{)}\<%
\\
\>[0]\AgdaKeyword{open}\AgdaSpace{}%
\AgdaKeyword{import}\AgdaSpace{}%
\AgdaModule{Level}\AgdaSpace{}%
\AgdaKeyword{hiding}\AgdaSpace{}%
\AgdaSymbol{(}\AgdaPrimitive{zero}\AgdaSpace{}%
\AgdaSymbol{;}\AgdaSpace{}%
\AgdaPrimitive{suc}\AgdaSymbol{)}\<%
\\
\>[0]\AgdaKeyword{open}\AgdaSpace{}%
\AgdaKeyword{import}\AgdaSpace{}%
\AgdaModule{Relation.Binary.PropositionalEquality}\<%
\\
\>[0]\AgdaKeyword{open}\AgdaSpace{}%
\AgdaKeyword{import}\AgdaSpace{}%
\AgdaModule{Function}\<%
\\
\>[0]\AgdaKeyword{open}\AgdaSpace{}%
\AgdaKeyword{import}\AgdaSpace{}%
\AgdaModule{Data.Fin}\AgdaSpace{}%
\AgdaKeyword{hiding}\AgdaSpace{}%
\AgdaSymbol{(}\AgdaOperator{\AgdaFunction{\AgdaUnderscore{}≟\AgdaUnderscore{}}}\AgdaSymbol{)}\<%
\\
\>[0]\AgdaKeyword{open}\AgdaSpace{}%
\AgdaKeyword{import}\AgdaSpace{}%
\AgdaModule{Data.Nat}\AgdaSpace{}%
\AgdaKeyword{hiding}\AgdaSpace{}%
\AgdaSymbol{(}\AgdaOperator{\AgdaPrimitive{\AgdaUnderscore{}*\AgdaUnderscore{}}}\AgdaSpace{}%
\AgdaSymbol{;}\AgdaSpace{}%
\AgdaOperator{\AgdaFunction{\AgdaUnderscore{}≟\AgdaUnderscore{}}}\AgdaSymbol{)}\<%
\\
\>[0]\AgdaKeyword{open}\AgdaSpace{}%
\AgdaKeyword{import}\AgdaSpace{}%
\AgdaModule{Relation.Nullary.Negation}\<%
\\
\>[0]\AgdaKeyword{import}\AgdaSpace{}%
\AgdaModule{Data.String}\AgdaSpace{}%
\AgdaSymbol{as}\AgdaSpace{}%
\AgdaModule{String}\<%
\\
\>[0]\AgdaKeyword{open}\AgdaSpace{}%
\AgdaKeyword{import}\AgdaSpace{}%
\AgdaModule{Agda.Builtin.FromString}\<%
\\
%
\\[\AgdaEmptyExtraSkip]%
\>[0]\AgdaKeyword{open}\AgdaSpace{}%
\AgdaKeyword{import}\AgdaSpace{}%
\AgdaModule{2-overview}\AgdaSpace{}%
\AgdaSymbol{as}\AgdaSpace{}%
\AgdaModule{◇}\<%
\end{code}

\subsection{Fixed Points}

\jr{Make it clear that we depart from Elliott's work at this point.}
\begin{itemize}
\item If $\ab{F}~\as{:}~\af{Type}~\as{→}~\af{Type}$ is a strictly positive functor, then we know its fixed point is well-defined.
\item So we could restrict the argument of our fixed point combinator to only accept strictly positive functors.
\item We are dealing with languages and not types directly, but luckily our definition of language is based on types and our basic combinators are strictly positive.
\item One catch is that Agda currently has no way of directly expressing that a functor is strictly positive.\footnote{There is work on implementing positivity annotations.}
\item We can still make this evident to Agda by defining a data type of descriptions such as those used in the paper "gentle art of levitation".\jr{todo: cite this}
\end{itemize}

\begin{code}[hide]%
\>[0]\AgdaKeyword{module}\AgdaSpace{}%
\AgdaModule{F}\AgdaSpace{}%
\AgdaKeyword{where}\<%
\end{code}

\begin{code}%
\>[0][@{}l@{\AgdaIndent{1}}]%
\>[4]\AgdaKeyword{data}\AgdaSpace{}%
\AgdaDatatype{Desc}\AgdaSpace{}%
\AgdaSymbol{:}\AgdaSpace{}%
\AgdaPrimitive{Type₁}\AgdaSpace{}%
\AgdaKeyword{where}\<%
\\
\>[4][@{}l@{\AgdaIndent{0}}]%
\>[8]\AgdaInductiveConstructor{∅}%
\>[13]\AgdaSymbol{:}\AgdaSpace{}%
\AgdaDatatype{Desc}\<%
\\
%
\>[8]\AgdaInductiveConstructor{ε}%
\>[13]\AgdaSymbol{:}\AgdaSpace{}%
\AgdaDatatype{Desc}\<%
\\
%
\>[8]\AgdaOperator{\AgdaInductiveConstructor{`\AgdaUnderscore{}}}%
\>[13]\AgdaSymbol{:}\AgdaSpace{}%
\AgdaPostulate{Char}\AgdaSpace{}%
\AgdaSymbol{→}\AgdaSpace{}%
\AgdaDatatype{Desc}\<%
\\
%
\>[8]\AgdaOperator{\AgdaInductiveConstructor{\AgdaUnderscore{}∪\AgdaUnderscore{}}}%
\>[13]\AgdaSymbol{:}\AgdaSpace{}%
\AgdaDatatype{Desc}\AgdaSpace{}%
\AgdaSymbol{→}\AgdaSpace{}%
\AgdaDatatype{Desc}\AgdaSpace{}%
\AgdaSymbol{→}\AgdaSpace{}%
\AgdaDatatype{Desc}\<%
\\
%
\>[8]\AgdaOperator{\AgdaInductiveConstructor{\AgdaUnderscore{}∗\AgdaUnderscore{}}}%
\>[13]\AgdaSymbol{:}\AgdaSpace{}%
\AgdaDatatype{Desc}\AgdaSpace{}%
\AgdaSymbol{→}\AgdaSpace{}%
\AgdaDatatype{Desc}\AgdaSpace{}%
\AgdaSymbol{→}\AgdaSpace{}%
\AgdaDatatype{Desc}\<%
\\
%
\>[8]\AgdaComment{--\ We\ need\ Dec\ if\ we\ want\ to\ be\ able\ to\ write\ parsers}\<%
\\
%
\>[8]\AgdaComment{--\ but\ for\ specifiction\ it\ is\ not\ really\ needed}\<%
\\
%
\>[8]\AgdaOperator{\AgdaInductiveConstructor{\AgdaUnderscore{}·\AgdaUnderscore{}}}%
\>[13]\AgdaSymbol{:}\AgdaSpace{}%
\AgdaSymbol{\{}\AgdaBound{A}\AgdaSpace{}%
\AgdaSymbol{:}\AgdaSpace{}%
\AgdaPrimitive{Type}\AgdaSymbol{\}}\AgdaSpace{}%
\AgdaSymbol{→}\AgdaSpace{}%
\AgdaRecord{Dec}\AgdaSpace{}%
\AgdaBound{A}\AgdaSpace{}%
\AgdaSymbol{→}\AgdaSpace{}%
\AgdaDatatype{Desc}\AgdaSpace{}%
\AgdaSymbol{→}\AgdaSpace{}%
\AgdaDatatype{Desc}\<%
\\
%
\>[8]\AgdaInductiveConstructor{var}%
\>[13]\AgdaSymbol{:}\AgdaSpace{}%
\AgdaDatatype{Desc}\<%
\end{code}

\begin{code}[hide]%
%
\>[4]\AgdaKeyword{infix}\AgdaSpace{}%
\AgdaNumber{22}\AgdaSpace{}%
\AgdaOperator{\AgdaInductiveConstructor{`\AgdaUnderscore{}}}\<%
\\
%
\>[4]\AgdaKeyword{infixr}\AgdaSpace{}%
\AgdaNumber{21}\AgdaSpace{}%
\AgdaOperator{\AgdaInductiveConstructor{\AgdaUnderscore{}∗\AgdaUnderscore{}}}\<%
\\
%
\>[4]\AgdaKeyword{infix}\AgdaSpace{}%
\AgdaNumber{21}\AgdaSpace{}%
\AgdaOperator{\AgdaInductiveConstructor{\AgdaUnderscore{}·\AgdaUnderscore{}}}\<%
\\
%
\>[4]\AgdaKeyword{infixr}\AgdaSpace{}%
\AgdaNumber{20}\AgdaSpace{}%
\AgdaOperator{\AgdaInductiveConstructor{\AgdaUnderscore{}∪\AgdaUnderscore{}}}\<%
\end{code}

We can give semantics to our descriptions in terms of languages that we defined in the previous section.\jr{todo: proper ref}

\begin{code}%
%
\>[4]\AgdaOperator{\AgdaFunction{⟦\AgdaUnderscore{}⟧ₒ}}\AgdaSpace{}%
\AgdaSymbol{:}\AgdaSpace{}%
\AgdaDatatype{Desc}\AgdaSpace{}%
\AgdaSymbol{→}\AgdaSpace{}%
\AgdaFunction{◇.Lang}\AgdaSpace{}%
\AgdaSymbol{→}\AgdaSpace{}%
\AgdaFunction{◇.Lang}\<%
\\
%
\>[4]\AgdaOperator{\AgdaFunction{⟦}}\AgdaSpace{}%
\AgdaInductiveConstructor{∅}\AgdaSpace{}%
\AgdaOperator{\AgdaFunction{⟧ₒ}}%
\>[22]\AgdaSymbol{\AgdaUnderscore{}}\AgdaSpace{}%
\AgdaSymbol{=}\AgdaSpace{}%
\AgdaFunction{◇.∅}\<%
\\
%
\>[4]\AgdaOperator{\AgdaFunction{⟦}}\AgdaSpace{}%
\AgdaInductiveConstructor{ε}\AgdaSpace{}%
\AgdaOperator{\AgdaFunction{⟧ₒ}}%
\>[22]\AgdaSymbol{\AgdaUnderscore{}}\AgdaSpace{}%
\AgdaSymbol{=}\AgdaSpace{}%
\AgdaFunction{◇.ε}\<%
\\
%
\>[4]\AgdaOperator{\AgdaFunction{⟦}}\AgdaSpace{}%
\AgdaOperator{\AgdaInductiveConstructor{`}}\AgdaSpace{}%
\AgdaBound{c}\AgdaSpace{}%
\AgdaOperator{\AgdaFunction{⟧ₒ}}%
\>[22]\AgdaSymbol{\AgdaUnderscore{}}\AgdaSpace{}%
\AgdaSymbol{=}\AgdaSpace{}%
\AgdaOperator{\AgdaFunction{◇.`}}\AgdaSpace{}%
\AgdaBound{c}\<%
\\
%
\>[4]\AgdaOperator{\AgdaFunction{⟦}}\AgdaSpace{}%
\AgdaBound{D₁}\AgdaSpace{}%
\AgdaOperator{\AgdaInductiveConstructor{∪}}\AgdaSpace{}%
\AgdaBound{D₂}\AgdaSpace{}%
\AgdaOperator{\AgdaFunction{⟧ₒ}}%
\>[22]\AgdaBound{P}\AgdaSpace{}%
\AgdaSymbol{=}\AgdaSpace{}%
\AgdaOperator{\AgdaFunction{⟦}}\AgdaSpace{}%
\AgdaBound{D₁}\AgdaSpace{}%
\AgdaOperator{\AgdaFunction{⟧ₒ}}\AgdaSpace{}%
\AgdaBound{P}\AgdaSpace{}%
\AgdaOperator{\AgdaFunction{◇.∪}}\AgdaSpace{}%
\AgdaOperator{\AgdaFunction{⟦}}\AgdaSpace{}%
\AgdaBound{D₂}\AgdaSpace{}%
\AgdaOperator{\AgdaFunction{⟧ₒ}}\AgdaSpace{}%
\AgdaBound{P}\<%
\\
%
\>[4]\AgdaOperator{\AgdaFunction{⟦}}\AgdaSpace{}%
\AgdaBound{D₁}\AgdaSpace{}%
\AgdaOperator{\AgdaInductiveConstructor{∗}}\AgdaSpace{}%
\AgdaBound{D₂}\AgdaSpace{}%
\AgdaOperator{\AgdaFunction{⟧ₒ}}%
\>[22]\AgdaBound{P}\AgdaSpace{}%
\AgdaSymbol{=}\AgdaSpace{}%
\AgdaOperator{\AgdaFunction{⟦}}\AgdaSpace{}%
\AgdaBound{D₁}\AgdaSpace{}%
\AgdaOperator{\AgdaFunction{⟧ₒ}}\AgdaSpace{}%
\AgdaBound{P}\AgdaSpace{}%
\AgdaOperator{\AgdaFunction{◇.∗}}\AgdaSpace{}%
\AgdaOperator{\AgdaFunction{⟦}}\AgdaSpace{}%
\AgdaBound{D₂}\AgdaSpace{}%
\AgdaOperator{\AgdaFunction{⟧ₒ}}\AgdaSpace{}%
\AgdaBound{P}\<%
\\
%
\>[4]\AgdaOperator{\AgdaFunction{⟦}}\AgdaSpace{}%
\AgdaOperator{\AgdaInductiveConstructor{\AgdaUnderscore{}·\AgdaUnderscore{}}}\AgdaSpace{}%
\AgdaSymbol{\{}\AgdaBound{A}\AgdaSymbol{\}}\AgdaSpace{}%
\AgdaSymbol{\AgdaUnderscore{}}\AgdaSpace{}%
\AgdaBound{D}\AgdaSpace{}%
\AgdaOperator{\AgdaFunction{⟧ₒ}}%
\>[22]\AgdaBound{P}\AgdaSpace{}%
\AgdaSymbol{=}\AgdaSpace{}%
\AgdaBound{A}\AgdaSpace{}%
\AgdaOperator{\AgdaFunction{◇.·}}\AgdaSpace{}%
\AgdaOperator{\AgdaFunction{⟦}}\AgdaSpace{}%
\AgdaBound{D}\AgdaSpace{}%
\AgdaOperator{\AgdaFunction{⟧ₒ}}\AgdaSpace{}%
\AgdaBound{P}\<%
\\
%
\>[4]\AgdaOperator{\AgdaFunction{⟦}}\AgdaSpace{}%
\AgdaInductiveConstructor{var}\AgdaSpace{}%
\AgdaOperator{\AgdaFunction{⟧ₒ}}%
\>[22]\AgdaBound{P}\AgdaSpace{}%
\AgdaSymbol{=}\AgdaSpace{}%
\AgdaBound{P}\<%
\end{code}

Using these descriptions, we can define a fixed point as follows:

\begin{code}%
%
\>[4]\AgdaKeyword{data}\AgdaSpace{}%
\AgdaOperator{\AgdaDatatype{⟦\AgdaUnderscore{}⟧}}\AgdaSpace{}%
\AgdaSymbol{(}\AgdaBound{D}\AgdaSpace{}%
\AgdaSymbol{:}\AgdaSpace{}%
\AgdaDatatype{Desc}\AgdaSymbol{)}\AgdaSpace{}%
\AgdaSymbol{:}\AgdaSpace{}%
\AgdaFunction{◇.Lang}\AgdaSpace{}%
\AgdaKeyword{where}\<%
\\
\>[4][@{}l@{\AgdaIndent{0}}]%
\>[8]\AgdaInductiveConstructor{roll}\AgdaSpace{}%
\AgdaSymbol{:}\AgdaSpace{}%
\AgdaOperator{\AgdaFunction{⟦}}\AgdaSpace{}%
\AgdaBound{D}\AgdaSpace{}%
\AgdaOperator{\AgdaFunction{⟧ₒ}}\AgdaSpace{}%
\AgdaOperator{\AgdaDatatype{⟦}}\AgdaSpace{}%
\AgdaBound{D}\AgdaSpace{}%
\AgdaOperator{\AgdaDatatype{⟧}}\AgdaSpace{}%
\AgdaGeneralizable{w}\AgdaSpace{}%
\AgdaSymbol{→}\AgdaSpace{}%
\AgdaOperator{\AgdaDatatype{⟦}}\AgdaSpace{}%
\AgdaBound{D}\AgdaSpace{}%
\AgdaOperator{\AgdaDatatype{⟧}}\AgdaSpace{}%
\AgdaGeneralizable{w}\<%
\end{code}
\begin{code}[hide]%
%
\>[4]\AgdaKeyword{variable}\AgdaSpace{}%
\AgdaGeneralizable{D}\AgdaSpace{}%
\AgdaSymbol{:}\AgdaSpace{}%
\AgdaDatatype{Desc}\<%
\end{code}
\begin{code}%
%
\>[4]\AgdaFunction{unroll}\AgdaSpace{}%
\AgdaSymbol{:}\AgdaSpace{}%
\AgdaOperator{\AgdaDatatype{⟦}}\AgdaSpace{}%
\AgdaGeneralizable{D}\AgdaSpace{}%
\AgdaOperator{\AgdaDatatype{⟧}}\AgdaSpace{}%
\AgdaGeneralizable{w}\AgdaSpace{}%
\AgdaSymbol{→}\AgdaSpace{}%
\AgdaOperator{\AgdaFunction{⟦}}\AgdaSpace{}%
\AgdaGeneralizable{D}\AgdaSpace{}%
\AgdaOperator{\AgdaFunction{⟧ₒ}}\AgdaSpace{}%
\AgdaOperator{\AgdaDatatype{⟦}}\AgdaSpace{}%
\AgdaGeneralizable{D}\AgdaSpace{}%
\AgdaOperator{\AgdaDatatype{⟧}}\AgdaSpace{}%
\AgdaGeneralizable{w}\<%
\\
%
\>[4]\AgdaFunction{unroll}\AgdaSpace{}%
\AgdaSymbol{(}\AgdaInductiveConstructor{roll}\AgdaSpace{}%
\AgdaBound{x}\AgdaSymbol{)}\AgdaSpace{}%
\AgdaSymbol{=}\AgdaSpace{}%
\AgdaBound{x}\<%
\end{code}

So we can finally define the brackets language.\footnote{We split this definition into two because we want to separately reuse the description later.}\jr{Brackets is one example, but can we characterise the whole class of languages we can define using these descriptions?}

\begin{code}%
%
\>[4]\AgdaFunction{bracketsD}\AgdaSpace{}%
\AgdaSymbol{=}\AgdaSpace{}%
\AgdaInductiveConstructor{ε}\AgdaSpace{}%
\AgdaOperator{\AgdaInductiveConstructor{∪}}\AgdaSpace{}%
\AgdaOperator{\AgdaInductiveConstructor{`}}\AgdaSpace{}%
\AgdaString{'['}\AgdaSpace{}%
\AgdaOperator{\AgdaInductiveConstructor{∗}}\AgdaSpace{}%
\AgdaInductiveConstructor{var}\AgdaSpace{}%
\AgdaOperator{\AgdaInductiveConstructor{∗}}\AgdaSpace{}%
\AgdaOperator{\AgdaInductiveConstructor{`}}\AgdaSpace{}%
\AgdaString{']'}\AgdaSpace{}%
\AgdaOperator{\AgdaInductiveConstructor{∪}}\AgdaSpace{}%
\AgdaInductiveConstructor{var}\AgdaSpace{}%
\AgdaOperator{\AgdaInductiveConstructor{∗}}\AgdaSpace{}%
\AgdaInductiveConstructor{var}\<%
\\
%
\>[4]\AgdaFunction{brackets}\AgdaSpace{}%
\AgdaSymbol{=}\AgdaSpace{}%
\AgdaOperator{\AgdaDatatype{⟦}}\AgdaSpace{}%
\AgdaFunction{bracketsD}\AgdaSpace{}%
\AgdaOperator{\AgdaDatatype{⟧}}\<%
\end{code}

This representation is not modular, however. We cannot nest fixed points in
descriptions.\jr{This modularity and nesting is not clear enough.} This problem comes up naturally when considering reduction, which we discuss next.

\subsection{Reduction by Example}\label{sec:reduction-by-example}

As we have seen with finite languages in \cref{sec:finite-languages}, when writing parsers it is useful to consider how a language changes after one character has been parsed. We will call this \emph{reduction}. For example, we could consider what happens to our brackets languages after one opening brackets has been parsed: $\af{δ}~\aS{'['}~\af{brackets}$. In this section, we search for a description of this reduced language (the \emph{reduct}).

We can mechanically derive this new language from the brackets definition by
going over each of the disjuncts. The first disjunct, $ε$, does not play a
role because we know the string contains at least the opening bracket. The
second disjunct, brackets surrounding a self-reference, is trickier. The opening
bracket clearly matches, but it would be a mistake to say the new disjunct
should be a self-reference followed by a closing bracket: $\ac{var}~\ac{∗}~\ac{`}~\aS{']'}$.

Note that the self-reference in the new language would refer to the derivative
of the old language, not to the old language itself. We would like to refer to
the original bracket language:
$\af{brackets}~\ac{∗}~\ac{`}~\aS{']'}$, but we cannot nest the brackets language
into another description.

There are cases where we do want to use self-reference in the new language.
Consider the third disjunct, $\ac{var}~\ac{∗}~\ac{var}$. It is a sequence so we
expect from the finite case of \cref{sec:finite-languages} that matching one character results in
two new disjuncts: one where the first sequent matches the empty string and the
second is reduced and one where the first is reduced and the second is
unchanged.\jr{Why? That is what we saw in Section 2} In this case both sequents are self-references, so we need to know
three things: 
%
\begin{enumerate}
\item Does the original language match the empty string?
\item What is the reduct of the language? (With reduct I mean the new language that results after one character is matched.)
\item What does it mean for the language to remain the same?
\end{enumerate}
%
At first glance, the last point seems obvious, but remember that we are reducing
the language, so self-references will change meaning even if they remain
unchanged. Similarly to the previous disjunct, we want to refer to the original
brackets in this case. To resolve this issue of referring to the original brackets expression, we introduce a new combinator $\ac{μ}$, which has the meaning of locally taking a fixed point of a subexpression.
%
\begin{code}[hide]%
\>[0]\AgdaKeyword{module}\AgdaSpace{}%
\AgdaModule{F2}\AgdaSpace{}%
\AgdaKeyword{where}\<%
\end{code}
\begin{AgdaAlign}
\vspace{\abovedisplayskip}
\AgdaNoSpaceAroundCode{}
\begin{code}%
\>[0][@{}l@{\AgdaIndent{1}}]%
\>[4]\AgdaKeyword{data}\AgdaSpace{}%
\AgdaDatatype{Desc}\AgdaSpace{}%
\AgdaSymbol{:}\AgdaSpace{}%
\AgdaPrimitive{Type₁}\AgdaSpace{}%
\AgdaKeyword{where}\<%
\\
\>[4][@{}l@{\AgdaIndent{0}}]%
\>[8]\AgdaComment{--\ ...}\<%
\end{code}%
\begin{code}[hide]%
%
\>[8]\AgdaInductiveConstructor{∅}%
\>[13]\AgdaSymbol{:}\AgdaSpace{}%
\AgdaDatatype{Desc}\<%
\\
%
\>[8]\AgdaInductiveConstructor{ε}%
\>[13]\AgdaSymbol{:}\AgdaSpace{}%
\AgdaDatatype{Desc}\<%
\\
%
\>[8]\AgdaOperator{\AgdaInductiveConstructor{`\AgdaUnderscore{}}}%
\>[13]\AgdaSymbol{:}\AgdaSpace{}%
\AgdaPostulate{Char}\AgdaSpace{}%
\AgdaSymbol{→}\AgdaSpace{}%
\AgdaDatatype{Desc}\<%
\\
%
\>[8]\AgdaOperator{\AgdaInductiveConstructor{\AgdaUnderscore{}∪\AgdaUnderscore{}}}%
\>[13]\AgdaSymbol{:}\AgdaSpace{}%
\AgdaDatatype{Desc}\AgdaSpace{}%
\AgdaSymbol{→}\AgdaSpace{}%
\AgdaDatatype{Desc}\AgdaSpace{}%
\AgdaSymbol{→}\AgdaSpace{}%
\AgdaDatatype{Desc}\<%
\\
%
\>[8]\AgdaOperator{\AgdaInductiveConstructor{\AgdaUnderscore{}∗\AgdaUnderscore{}}}%
\>[13]\AgdaSymbol{:}\AgdaSpace{}%
\AgdaDatatype{Desc}\AgdaSpace{}%
\AgdaSymbol{→}\AgdaSpace{}%
\AgdaDatatype{Desc}\AgdaSpace{}%
\AgdaSymbol{→}\AgdaSpace{}%
\AgdaDatatype{Desc}\<%
\\
%
\>[8]\AgdaComment{--\ We\ need\ Dec\ if\ we\ want\ to\ be\ able\ to\ write\ parsers}\<%
\\
%
\>[8]\AgdaComment{--\ but\ for\ specifiction\ it\ is\ not\ really\ needed}\<%
\\
%
\>[8]\AgdaOperator{\AgdaInductiveConstructor{\AgdaUnderscore{}·\AgdaUnderscore{}}}%
\>[13]\AgdaSymbol{:}\AgdaSpace{}%
\AgdaSymbol{\{}\AgdaBound{A}\AgdaSpace{}%
\AgdaSymbol{:}\AgdaSpace{}%
\AgdaPrimitive{Type}\AgdaSymbol{\}}\AgdaSpace{}%
\AgdaSymbol{→}\AgdaSpace{}%
\AgdaRecord{Dec}\AgdaSpace{}%
\AgdaBound{A}\AgdaSpace{}%
\AgdaSymbol{→}\AgdaSpace{}%
\AgdaDatatype{Desc}\AgdaSpace{}%
\AgdaSymbol{→}\AgdaSpace{}%
\AgdaDatatype{Desc}\<%
\\
%
\>[8]\AgdaInductiveConstructor{var}%
\>[13]\AgdaSymbol{:}\AgdaSpace{}%
\AgdaDatatype{Desc}\<%
\end{code}%
\begin{code}%
%
\>[8]\AgdaInductiveConstructor{μ}\AgdaSpace{}%
\AgdaSymbol{:}\AgdaSpace{}%
\AgdaDatatype{Desc}\AgdaSpace{}%
\AgdaSymbol{→}\AgdaSpace{}%
\AgdaDatatype{Desc}\<%
\end{code}
\AgdaSpaceAroundCode{}
\end{AgdaAlign}
%
\begin{code}[hide]%
%
\>[4]\AgdaKeyword{variable}\AgdaSpace{}%
\AgdaGeneralizable{D}\AgdaSpace{}%
\AgdaGeneralizable{D₀}\AgdaSpace{}%
\AgdaGeneralizable{D₁}\AgdaSpace{}%
\AgdaGeneralizable{D₂}\AgdaSpace{}%
\AgdaSymbol{:}\AgdaSpace{}%
\AgdaDatatype{Desc}\<%
\\
%
\>[4]\AgdaKeyword{infix}\AgdaSpace{}%
\AgdaNumber{22}\AgdaSpace{}%
\AgdaOperator{\AgdaInductiveConstructor{`\AgdaUnderscore{}}}\<%
\\
%
\>[4]\AgdaKeyword{infixr}\AgdaSpace{}%
\AgdaNumber{21}\AgdaSpace{}%
\AgdaOperator{\AgdaInductiveConstructor{\AgdaUnderscore{}∗\AgdaUnderscore{}}}\<%
\\
%
\>[4]\AgdaKeyword{infix}\AgdaSpace{}%
\AgdaNumber{21}\AgdaSpace{}%
\AgdaOperator{\AgdaInductiveConstructor{\AgdaUnderscore{}·\AgdaUnderscore{}}}\<%
\\
%
\>[4]\AgdaKeyword{infixr}\AgdaSpace{}%
\AgdaNumber{20}\AgdaSpace{}%
\AgdaOperator{\AgdaInductiveConstructor{\AgdaUnderscore{}∪\AgdaUnderscore{}}}\<%
\end{code}
%
\begin{AgdaAlign}
\AgdaNoSpaceAroundCode{}
\begin{code}%
%
\>[4]\AgdaOperator{\AgdaFunction{⟦\AgdaUnderscore{}⟧ₒ}}\AgdaSpace{}%
\AgdaSymbol{:}\AgdaSpace{}%
\AgdaDatatype{Desc}\AgdaSpace{}%
\AgdaSymbol{→}\AgdaSpace{}%
\AgdaFunction{◇.Lang}\AgdaSpace{}%
\AgdaSymbol{→}\AgdaSpace{}%
\AgdaFunction{◇.Lang}\<%
\\
%
\>[4]\AgdaComment{--\ ...}\<%
\end{code}
\begin{code}[hide]%
%
\>[4]\AgdaKeyword{data}\AgdaSpace{}%
\AgdaOperator{\AgdaDatatype{⟦\AgdaUnderscore{}⟧}}\AgdaSpace{}%
\AgdaSymbol{(}\AgdaBound{X}\AgdaSpace{}%
\AgdaSymbol{:}\AgdaSpace{}%
\AgdaDatatype{Desc}\AgdaSymbol{)}\AgdaSpace{}%
\AgdaSymbol{:}\AgdaSpace{}%
\AgdaFunction{◇.Lang}\AgdaSpace{}%
\AgdaKeyword{where}\<%
\\
\>[4][@{}l@{\AgdaIndent{0}}]%
\>[8]\AgdaInductiveConstructor{roll}\AgdaSpace{}%
\AgdaSymbol{:}\AgdaSpace{}%
\AgdaOperator{\AgdaFunction{⟦}}\AgdaSpace{}%
\AgdaBound{X}\AgdaSpace{}%
\AgdaOperator{\AgdaFunction{⟧ₒ}}\AgdaSpace{}%
\AgdaOperator{\AgdaDatatype{⟦}}\AgdaSpace{}%
\AgdaBound{X}\AgdaSpace{}%
\AgdaOperator{\AgdaDatatype{⟧}}\AgdaSpace{}%
\AgdaGeneralizable{w}\AgdaSpace{}%
\AgdaSymbol{→}\AgdaSpace{}%
\AgdaOperator{\AgdaDatatype{⟦}}\AgdaSpace{}%
\AgdaBound{X}\AgdaSpace{}%
\AgdaOperator{\AgdaDatatype{⟧}}\AgdaSpace{}%
\AgdaGeneralizable{w}\<%
\\
%
\>[4]\AgdaFunction{unroll}\AgdaSpace{}%
\AgdaSymbol{:}\AgdaSpace{}%
\AgdaOperator{\AgdaDatatype{⟦}}\AgdaSpace{}%
\AgdaGeneralizable{D}\AgdaSpace{}%
\AgdaOperator{\AgdaDatatype{⟧}}\AgdaSpace{}%
\AgdaGeneralizable{w}\AgdaSpace{}%
\AgdaSymbol{→}\AgdaSpace{}%
\AgdaOperator{\AgdaFunction{⟦}}\AgdaSpace{}%
\AgdaGeneralizable{D}\AgdaSpace{}%
\AgdaOperator{\AgdaFunction{⟧ₒ}}\AgdaSpace{}%
\AgdaOperator{\AgdaDatatype{⟦}}\AgdaSpace{}%
\AgdaGeneralizable{D}\AgdaSpace{}%
\AgdaOperator{\AgdaDatatype{⟧}}\AgdaSpace{}%
\AgdaGeneralizable{w}\<%
\\
%
\>[4]\AgdaFunction{unroll}\AgdaSpace{}%
\AgdaSymbol{(}\AgdaInductiveConstructor{roll}\AgdaSpace{}%
\AgdaBound{x}\AgdaSymbol{)}\AgdaSpace{}%
\AgdaSymbol{=}\AgdaSpace{}%
\AgdaBound{x}\<%
\\
%
\>[4]\AgdaOperator{\AgdaFunction{⟦}}\AgdaSpace{}%
\AgdaInductiveConstructor{∅}\AgdaSpace{}%
\AgdaOperator{\AgdaFunction{⟧ₒ}}%
\>[18]\AgdaSymbol{=}\AgdaSpace{}%
\AgdaFunction{const}\AgdaSpace{}%
\AgdaFunction{◇.∅}\<%
\\
%
\>[4]\AgdaOperator{\AgdaFunction{⟦}}\AgdaSpace{}%
\AgdaInductiveConstructor{ε}\AgdaSpace{}%
\AgdaOperator{\AgdaFunction{⟧ₒ}}%
\>[18]\AgdaSymbol{=}\AgdaSpace{}%
\AgdaFunction{const}\AgdaSpace{}%
\AgdaFunction{◇.ε}\<%
\\
%
\>[4]\AgdaOperator{\AgdaFunction{⟦}}\AgdaSpace{}%
\AgdaOperator{\AgdaInductiveConstructor{`}}\AgdaSpace{}%
\AgdaBound{c}\AgdaSpace{}%
\AgdaOperator{\AgdaFunction{⟧ₒ}}%
\>[18]\AgdaSymbol{=}\AgdaSpace{}%
\AgdaFunction{const}\AgdaSpace{}%
\AgdaSymbol{(}\AgdaOperator{\AgdaFunction{◇.`}}\AgdaSpace{}%
\AgdaBound{c}\AgdaSymbol{)}\<%
\\
%
\>[4]\AgdaOperator{\AgdaFunction{⟦}}\AgdaSpace{}%
\AgdaBound{X}\AgdaSpace{}%
\AgdaOperator{\AgdaInductiveConstructor{∪}}\AgdaSpace{}%
\AgdaBound{Y}\AgdaSpace{}%
\AgdaOperator{\AgdaFunction{⟧ₒ}}\AgdaSpace{}%
\AgdaBound{P}%
\>[18]\AgdaSymbol{=}\AgdaSpace{}%
\AgdaOperator{\AgdaFunction{⟦}}\AgdaSpace{}%
\AgdaBound{X}\AgdaSpace{}%
\AgdaOperator{\AgdaFunction{⟧ₒ}}\AgdaSpace{}%
\AgdaBound{P}\AgdaSpace{}%
\AgdaOperator{\AgdaFunction{◇.∪}}\AgdaSpace{}%
\AgdaOperator{\AgdaFunction{⟦}}\AgdaSpace{}%
\AgdaBound{Y}\AgdaSpace{}%
\AgdaOperator{\AgdaFunction{⟧ₒ}}\AgdaSpace{}%
\AgdaBound{P}\<%
\\
%
\>[4]\AgdaOperator{\AgdaFunction{⟦}}\AgdaSpace{}%
\AgdaBound{X}\AgdaSpace{}%
\AgdaOperator{\AgdaInductiveConstructor{∗}}\AgdaSpace{}%
\AgdaBound{Y}\AgdaSpace{}%
\AgdaOperator{\AgdaFunction{⟧ₒ}}\AgdaSpace{}%
\AgdaBound{P}%
\>[18]\AgdaSymbol{=}\AgdaSpace{}%
\AgdaOperator{\AgdaFunction{⟦}}\AgdaSpace{}%
\AgdaBound{X}\AgdaSpace{}%
\AgdaOperator{\AgdaFunction{⟧ₒ}}\AgdaSpace{}%
\AgdaBound{P}\AgdaSpace{}%
\AgdaOperator{\AgdaFunction{◇.∗}}\AgdaSpace{}%
\AgdaOperator{\AgdaFunction{⟦}}\AgdaSpace{}%
\AgdaBound{Y}\AgdaSpace{}%
\AgdaOperator{\AgdaFunction{⟧ₒ}}\AgdaSpace{}%
\AgdaBound{P}\<%
\\
%
\>[4]\AgdaOperator{\AgdaFunction{⟦}}\AgdaSpace{}%
\AgdaOperator{\AgdaInductiveConstructor{\AgdaUnderscore{}·\AgdaUnderscore{}}}\AgdaSpace{}%
\AgdaSymbol{\{}\AgdaBound{A}\AgdaSymbol{\}}\AgdaSpace{}%
\AgdaSymbol{\AgdaUnderscore{}}\AgdaSpace{}%
\AgdaBound{X}\AgdaSpace{}%
\AgdaOperator{\AgdaFunction{⟧ₒ}}\AgdaSpace{}%
\AgdaBound{P}%
\>[24]\AgdaSymbol{=}\AgdaSpace{}%
\AgdaBound{A}\AgdaSpace{}%
\AgdaOperator{\AgdaFunction{◇.·}}\AgdaSpace{}%
\AgdaOperator{\AgdaFunction{⟦}}\AgdaSpace{}%
\AgdaBound{X}\AgdaSpace{}%
\AgdaOperator{\AgdaFunction{⟧ₒ}}\AgdaSpace{}%
\AgdaBound{P}\<%
\\
%
\>[4]\AgdaOperator{\AgdaFunction{⟦}}\AgdaSpace{}%
\AgdaInductiveConstructor{var}\AgdaSpace{}%
\AgdaOperator{\AgdaFunction{⟧ₒ}}\AgdaSpace{}%
\AgdaBound{P}%
\>[18]\AgdaSymbol{=}\AgdaSpace{}%
\AgdaBound{P}\<%
\end{code}
\begin{code}%
%
\>[4]\AgdaOperator{\AgdaFunction{⟦}}\AgdaSpace{}%
\AgdaInductiveConstructor{μ}\AgdaSpace{}%
\AgdaBound{D}\AgdaSpace{}%
\AgdaOperator{\AgdaFunction{⟧ₒ}}\AgdaSpace{}%
\AgdaSymbol{\AgdaUnderscore{}}%
\>[18]\AgdaSymbol{=}\AgdaSpace{}%
\AgdaOperator{\AgdaDatatype{⟦}}\AgdaSpace{}%
\AgdaBound{D}\AgdaSpace{}%
\AgdaOperator{\AgdaDatatype{⟧}}\<%
\end{code}
\AgdaSpaceAroundCode{}
\vspace{\belowdisplayskip}
\end{AgdaAlign}
\jr{How is this used in our example?}

The first question is easy to answer: yes, the first disjunct of brackets is epsilon which matches the empty string.
%
\begin{code}[hide]%
%
\>[4]\AgdaFunction{bracketsD}\AgdaSpace{}%
\AgdaSymbol{=}\AgdaSpace{}%
\AgdaInductiveConstructor{ε}\AgdaSpace{}%
\AgdaOperator{\AgdaInductiveConstructor{∪}}\AgdaSpace{}%
\AgdaOperator{\AgdaInductiveConstructor{`}}\AgdaSpace{}%
\AgdaString{'['}\AgdaSpace{}%
\AgdaOperator{\AgdaInductiveConstructor{∗}}\AgdaSpace{}%
\AgdaInductiveConstructor{var}\AgdaSpace{}%
\AgdaOperator{\AgdaInductiveConstructor{∗}}\AgdaSpace{}%
\AgdaOperator{\AgdaInductiveConstructor{`}}\AgdaSpace{}%
\AgdaString{']'}\AgdaSpace{}%
\AgdaOperator{\AgdaInductiveConstructor{∪}}\AgdaSpace{}%
\AgdaInductiveConstructor{var}\AgdaSpace{}%
\AgdaOperator{\AgdaInductiveConstructor{∗}}\AgdaSpace{}%
\AgdaInductiveConstructor{var}\<%
\\
%
\>[4]\AgdaFunction{brackets}\AgdaSpace{}%
\AgdaSymbol{=}\AgdaSpace{}%
\AgdaOperator{\AgdaDatatype{⟦}}\AgdaSpace{}%
\AgdaFunction{bracketsD}\AgdaSpace{}%
\AgdaOperator{\AgdaDatatype{⟧}}\<%
\end{code}
\begin{code}%
%
\>[4]\AgdaFunction{νbrackets}\AgdaSpace{}%
\AgdaSymbol{:}\AgdaSpace{}%
\AgdaRecord{Dec}\AgdaSpace{}%
\AgdaSymbol{(}\AgdaFunction{◇.ν}\AgdaSpace{}%
\AgdaFunction{brackets}\AgdaSymbol{)}\<%
\\
%
\>[4]\AgdaFunction{νbrackets}\AgdaSpace{}%
\AgdaSymbol{=}\AgdaSpace{}%
\AgdaInductiveConstructor{yes}\AgdaSpace{}%
\AgdaSymbol{(}\AgdaInductiveConstructor{roll}\AgdaSpace{}%
\AgdaSymbol{(}\AgdaInductiveConstructor{inj₁}\AgdaSpace{}%
\AgdaInductiveConstructor{refl}\AgdaSymbol{))}\<%
\end{code}

The second question is where having a self-reference in the new language is useful. We can refer to the reduct of brackets by using self-reference.

This enables us to write the reduct of brackets with respect to the opening bracket.

\begin{code}%
%
\>[4]\AgdaFunction{bracketsD'}%
\>[16]\AgdaSymbol{=}\AgdaSpace{}%
\AgdaInductiveConstructor{μ}\AgdaSpace{}%
\AgdaFunction{bracketsD}\AgdaSpace{}%
\AgdaOperator{\AgdaInductiveConstructor{∗}}\AgdaSpace{}%
\AgdaOperator{\AgdaInductiveConstructor{`}}\AgdaSpace{}%
\AgdaString{']'}\AgdaSpace{}%
\AgdaOperator{\AgdaInductiveConstructor{∪}}\AgdaSpace{}%
\AgdaFunction{νbrackets}\AgdaSpace{}%
\AgdaOperator{\AgdaInductiveConstructor{·}}\AgdaSpace{}%
\AgdaInductiveConstructor{var}\AgdaSpace{}%
\AgdaOperator{\AgdaInductiveConstructor{∪}}\AgdaSpace{}%
\AgdaInductiveConstructor{var}\AgdaSpace{}%
\AgdaOperator{\AgdaInductiveConstructor{∗}}\AgdaSpace{}%
\AgdaInductiveConstructor{μ}\AgdaSpace{}%
\AgdaFunction{bracketsD}\<%
\\
%
\>[4]\AgdaFunction{brackets'}%
\>[16]\AgdaSymbol{=}\AgdaSpace{}%
\AgdaOperator{\AgdaDatatype{⟦}}\AgdaSpace{}%
\AgdaFunction{bracketsD'}\AgdaSpace{}%
\AgdaOperator{\AgdaDatatype{⟧}}\<%
\end{code}

Conclusion:
\begin{itemize}
\item We can reuse many of the results of finite languages (\cref{sec:finite-languages}).
\item We need a new $\ac{μ}$ combinator to nest fixed points in descriptions. This is necessary to refer back to the original language before reduction.
\item Reducing a self-reference simply results in a self-reference again, because self-references in the reduct refer to the reduct.
\end{itemize}
Again, we do not want to have to do this reduction manually. Instead, we show
how to do it in general for any description in the next section.

\subsection{Parsing in General}

Our goal is to define:

\begin{code}%
%
\>[4]\AgdaFunction{parse}\AgdaSpace{}%
\AgdaSymbol{:}\AgdaSpace{}%
\AgdaSymbol{∀}\AgdaSpace{}%
\AgdaBound{D}\AgdaSpace{}%
\AgdaSymbol{→}\AgdaSpace{}%
\AgdaFunction{◇.Parser}\AgdaSpace{}%
\AgdaOperator{\AgdaDatatype{⟦}}\AgdaSpace{}%
\AgdaBound{D}\AgdaSpace{}%
\AgdaOperator{\AgdaDatatype{⟧}}\<%
\end{code}

We approach this by decomposing parsing into $\af{ν}$ and $\af{δ}$.

\begin{code}%
%
\>[4]\AgdaFunction{νD}\AgdaSpace{}%
\AgdaSymbol{:}\AgdaSpace{}%
\AgdaSymbol{∀}\AgdaSpace{}%
\AgdaBound{D}\AgdaSpace{}%
\AgdaSymbol{→}\AgdaSpace{}%
\AgdaRecord{Dec}\AgdaSpace{}%
\AgdaSymbol{(}\AgdaFunction{◇.ν}\AgdaSpace{}%
\AgdaOperator{\AgdaDatatype{⟦}}\AgdaSpace{}%
\AgdaBound{D}\AgdaSpace{}%
\AgdaOperator{\AgdaDatatype{⟧}}\AgdaSymbol{)}\<%
\\
%
\>[4]\AgdaFunction{δD}\AgdaSpace{}%
\AgdaSymbol{:}\AgdaSpace{}%
\AgdaPostulate{Char}\AgdaSpace{}%
\AgdaSymbol{→}\AgdaSpace{}%
\AgdaDatatype{Desc}\AgdaSpace{}%
\AgdaSymbol{→}\AgdaSpace{}%
\AgdaDatatype{Desc}\<%
\end{code}

The $\af{νD}$ function can easily be written to be correct by construction, however $\af{δD}$ must be proven correct separately as follows:

\begin{code}%
%
\>[4]\AgdaFunction{δD-correct}\AgdaSpace{}%
\AgdaSymbol{:}\AgdaSpace{}%
\AgdaOperator{\AgdaDatatype{⟦}}\AgdaSpace{}%
\AgdaFunction{δD}\AgdaSpace{}%
\AgdaGeneralizable{c}\AgdaSpace{}%
\AgdaGeneralizable{D}\AgdaSpace{}%
\AgdaOperator{\AgdaDatatype{⟧}}\AgdaSpace{}%
\AgdaOperator{\AgdaFunction{◇.⟺}}\AgdaSpace{}%
\AgdaFunction{◇.δ}\AgdaSpace{}%
\AgdaGeneralizable{c}\AgdaSpace{}%
\AgdaOperator{\AgdaDatatype{⟦}}\AgdaSpace{}%
\AgdaGeneralizable{D}\AgdaSpace{}%
\AgdaOperator{\AgdaDatatype{⟧}}\<%
\end{code}

The actual parsing can now be done character by character:

\begin{code}%
%
\>[4]\AgdaFunction{parse}\AgdaSpace{}%
\AgdaBound{D}\AgdaSpace{}%
\AgdaInductiveConstructor{[]}\AgdaSpace{}%
\AgdaSymbol{=}\AgdaSpace{}%
\AgdaFunction{νD}\AgdaSpace{}%
\AgdaBound{D}\<%
\\
%
\>[4]\AgdaFunction{parse}\AgdaSpace{}%
\AgdaBound{D}\AgdaSpace{}%
\AgdaSymbol{(}\AgdaBound{c}\AgdaSpace{}%
\AgdaOperator{\AgdaInductiveConstructor{∷}}\AgdaSpace{}%
\AgdaBound{w}\AgdaSymbol{)}\AgdaSpace{}%
\AgdaSymbol{=}\AgdaSpace{}%
\AgdaFunction{Dec.map}\AgdaSpace{}%
\AgdaFunction{δD-correct}\AgdaSpace{}%
\AgdaSymbol{(}\AgdaFunction{parse}\AgdaSpace{}%
\AgdaSymbol{(}\AgdaFunction{δD}\AgdaSpace{}%
\AgdaBound{c}\AgdaSpace{}%
\AgdaBound{D}\AgdaSymbol{)}\AgdaSpace{}%
\AgdaBound{w}\AgdaSymbol{)}\<%
\end{code}

That is the main result of this paper. The remainder of the paper concerns
the implementation of $\af{νD}$, $\af{δD}$, $\af{δD-correct}$.

\subsection{Nullability}

If we know the nullability of a language, $\ab{P}$, then the nullability of a description functor applied to $\ab{P}$ is the same as the empty string parsers for our finite languages, but with the nullability of the variables given by the nullability of $\ab{P}$. For the $\ac{μ}$ case we use the nullability of the fixed point, which we will implement shortly.\jr{Reiterate that the cases for the basic combinators are the same as in \cref{fig:null-delta}.}
%
\begin{code}[hide]%
%
\>[4]\AgdaKeyword{variable}\AgdaSpace{}%
\AgdaGeneralizable{P}\AgdaSpace{}%
\AgdaSymbol{:}\AgdaSpace{}%
\AgdaFunction{◇.Lang}\<%
\end{code}
\begin{code}%
%
\>[4]\AgdaFunction{νₒ}\AgdaSpace{}%
\AgdaSymbol{:}\AgdaSpace{}%
\AgdaRecord{Dec}\AgdaSpace{}%
\AgdaSymbol{(}\AgdaFunction{◇.ν}\AgdaSpace{}%
\AgdaGeneralizable{P}\AgdaSymbol{)}\AgdaSpace{}%
\AgdaSymbol{→}\AgdaSpace{}%
\AgdaSymbol{∀}\AgdaSpace{}%
\AgdaBound{D}\AgdaSpace{}%
\AgdaSymbol{→}\AgdaSpace{}%
\AgdaRecord{Dec}\AgdaSpace{}%
\AgdaSymbol{(}\AgdaFunction{◇.ν}\AgdaSpace{}%
\AgdaSymbol{(}\AgdaOperator{\AgdaFunction{⟦}}\AgdaSpace{}%
\AgdaBound{D}\AgdaSpace{}%
\AgdaOperator{\AgdaFunction{⟧ₒ}}\AgdaSpace{}%
\AgdaGeneralizable{P}\AgdaSymbol{))}\<%
\\
%
\>[4]\AgdaFunction{νₒ}\AgdaSpace{}%
\AgdaBound{νP}\AgdaSpace{}%
\AgdaInductiveConstructor{∅}%
\>[20]\AgdaSymbol{=}\AgdaSpace{}%
\AgdaInductiveConstructor{no}\AgdaSpace{}%
\AgdaSymbol{λ}\AgdaSpace{}%
\AgdaSymbol{()}\<%
\\
%
\>[4]\AgdaFunction{νₒ}\AgdaSpace{}%
\AgdaBound{νP}\AgdaSpace{}%
\AgdaInductiveConstructor{ε}%
\>[20]\AgdaSymbol{=}\AgdaSpace{}%
\AgdaInductiveConstructor{yes}\AgdaSpace{}%
\AgdaInductiveConstructor{refl}\<%
\\
%
\>[4]\AgdaFunction{νₒ}\AgdaSpace{}%
\AgdaBound{νP}\AgdaSpace{}%
\AgdaSymbol{(}\AgdaOperator{\AgdaInductiveConstructor{`}}\AgdaSpace{}%
\AgdaBound{c}\AgdaSymbol{)}%
\>[20]\AgdaSymbol{=}\AgdaSpace{}%
\AgdaInductiveConstructor{no}\AgdaSpace{}%
\AgdaSymbol{λ}\AgdaSpace{}%
\AgdaSymbol{()}\<%
\\
%
\>[4]\AgdaFunction{νₒ}\AgdaSpace{}%
\AgdaBound{νP}\AgdaSpace{}%
\AgdaSymbol{(}\AgdaBound{D}\AgdaSpace{}%
\AgdaOperator{\AgdaInductiveConstructor{∪}}\AgdaSpace{}%
\AgdaBound{D₁}\AgdaSymbol{)}%
\>[20]\AgdaSymbol{=}\AgdaSpace{}%
\AgdaFunction{νₒ}\AgdaSpace{}%
\AgdaBound{νP}\AgdaSpace{}%
\AgdaBound{D}\AgdaSpace{}%
\AgdaOperator{\AgdaFunction{⊎-dec}}\AgdaSpace{}%
\AgdaFunction{νₒ}\AgdaSpace{}%
\AgdaBound{νP}\AgdaSpace{}%
\AgdaBound{D₁}\<%
\\
%
\>[4]\AgdaFunction{νₒ}\AgdaSpace{}%
\AgdaBound{νP}\AgdaSpace{}%
\AgdaSymbol{(}\AgdaBound{D}\AgdaSpace{}%
\AgdaOperator{\AgdaInductiveConstructor{∗}}\AgdaSpace{}%
\AgdaBound{D₁}\AgdaSymbol{)}%
\>[20]\AgdaSymbol{=}\AgdaSpace{}%
\AgdaFunction{Dec.map}\AgdaSpace{}%
\AgdaFunction{◇.ν∗}\AgdaSpace{}%
\AgdaSymbol{(}\AgdaFunction{νₒ}\AgdaSpace{}%
\AgdaBound{νP}\AgdaSpace{}%
\AgdaBound{D}\AgdaSpace{}%
\AgdaOperator{\AgdaFunction{×-dec}}\AgdaSpace{}%
\AgdaFunction{νₒ}\AgdaSpace{}%
\AgdaBound{νP}\AgdaSpace{}%
\AgdaBound{D₁}\AgdaSymbol{)}\<%
\\
%
\>[4]\AgdaFunction{νₒ}\AgdaSpace{}%
\AgdaBound{νP}\AgdaSpace{}%
\AgdaSymbol{(}\AgdaBound{x}\AgdaSpace{}%
\AgdaOperator{\AgdaInductiveConstructor{·}}\AgdaSpace{}%
\AgdaBound{D}\AgdaSymbol{)}%
\>[20]\AgdaSymbol{=}\AgdaSpace{}%
\AgdaBound{x}\AgdaSpace{}%
\AgdaOperator{\AgdaFunction{×-dec}}\AgdaSpace{}%
\AgdaFunction{νₒ}\AgdaSpace{}%
\AgdaBound{νP}\AgdaSpace{}%
\AgdaBound{D}\<%
\\
%
\>[4]\AgdaFunction{νₒ}\AgdaSpace{}%
\AgdaBound{νP}\AgdaSpace{}%
\AgdaInductiveConstructor{var}%
\>[20]\AgdaSymbol{=}\AgdaSpace{}%
\AgdaBound{νP}\<%
\\
%
\>[4]\AgdaFunction{νₒ}\AgdaSpace{}%
\AgdaBound{νP}\AgdaSpace{}%
\AgdaSymbol{(}\AgdaInductiveConstructor{μ}\AgdaSpace{}%
\AgdaBound{D}\AgdaSymbol{)}%
\>[20]\AgdaSymbol{=}\AgdaSpace{}%
\AgdaFunction{νD}\AgdaSpace{}%
\AgdaBound{D}\<%
\end{code}

\begin{itemize}
\item Naively we might try $\af{νD}~\ab{D}~\as{=}~\af{νₒ}~\as{(}\af{νD}~\ab{D}\as{)}~\ab{D}$
\item But that obviously will not terminate (consider the language $\af{⟦}~\ac{var}~\af{⟧}$). 
\item Instead we use \cref{lem:null-split}
\end{itemize}
%
\begin{lemma}\label{lem:null-split}
The nullability of a fixed point is determined completely by a single application of the underlying functor to the empty language.
\begin{code}%
%
\>[4]\AgdaFunction{νD∅⇔νD}\AgdaSpace{}%
\AgdaSymbol{:}\AgdaSpace{}%
\AgdaFunction{◇.ν}\AgdaSpace{}%
\AgdaSymbol{(}\AgdaOperator{\AgdaFunction{⟦}}\AgdaSpace{}%
\AgdaGeneralizable{D}\AgdaSpace{}%
\AgdaOperator{\AgdaFunction{⟧ₒ}}\AgdaSpace{}%
\AgdaFunction{◇.∅}\AgdaSymbol{)}\AgdaSpace{}%
\AgdaOperator{\AgdaFunction{⇔}}\AgdaSpace{}%
\AgdaFunction{◇.ν}\AgdaSpace{}%
\AgdaOperator{\AgdaDatatype{⟦}}\AgdaSpace{}%
\AgdaGeneralizable{D}\AgdaSpace{}%
\AgdaOperator{\AgdaDatatype{⟧}}\<%
\end{code}
\end{lemma}
\begin{proof}
The forward direction is easily proven by noting that nullability and the
semantics of a description are functors and that the empty language is initial.
It is also straightforward to write the proof directly.
\begin{code}%
%
\>[4]\AgdaFunction{νD∅→νD}\AgdaSpace{}%
\AgdaSymbol{:}\AgdaSpace{}%
\AgdaSymbol{∀}\AgdaSpace{}%
\AgdaBound{D}\AgdaSpace{}%
\AgdaSymbol{→}\AgdaSpace{}%
\AgdaFunction{◇.ν}\AgdaSpace{}%
\AgdaSymbol{(}\AgdaOperator{\AgdaFunction{⟦}}\AgdaSpace{}%
\AgdaBound{D}\AgdaSpace{}%
\AgdaOperator{\AgdaFunction{⟧ₒ}}\AgdaSpace{}%
\AgdaFunction{◇.∅}\AgdaSymbol{)}\AgdaSpace{}%
\AgdaSymbol{→}\AgdaSpace{}%
\AgdaFunction{◇.ν}\AgdaSpace{}%
\AgdaSymbol{(}\AgdaOperator{\AgdaFunction{⟦}}\AgdaSpace{}%
\AgdaBound{D}\AgdaSpace{}%
\AgdaOperator{\AgdaFunction{⟧ₒ}}\AgdaSpace{}%
\AgdaOperator{\AgdaDatatype{⟦}}\AgdaSpace{}%
\AgdaGeneralizable{D₀}\AgdaSpace{}%
\AgdaOperator{\AgdaDatatype{⟧}}\AgdaSymbol{)}\<%
\end{code}
\begin{code}[hide]%
%
\>[4]\AgdaFunction{νD∅→νD}\AgdaSpace{}%
\AgdaInductiveConstructor{ε}\AgdaSpace{}%
\AgdaSymbol{\AgdaUnderscore{}}\AgdaSpace{}%
\AgdaSymbol{=}\AgdaSpace{}%
\AgdaInductiveConstructor{refl}\<%
\\
%
\>[4]\AgdaFunction{νD∅→νD}\AgdaSpace{}%
\AgdaSymbol{(}\AgdaBound{D}\AgdaSpace{}%
\AgdaOperator{\AgdaInductiveConstructor{∪}}\AgdaSpace{}%
\AgdaBound{D₁}\AgdaSymbol{)}\AgdaSpace{}%
\AgdaSymbol{(}\AgdaInductiveConstructor{inj₁}\AgdaSpace{}%
\AgdaBound{x}\AgdaSymbol{)}\AgdaSpace{}%
\AgdaSymbol{=}\AgdaSpace{}%
\AgdaInductiveConstructor{inj₁}\AgdaSpace{}%
\AgdaSymbol{(}\AgdaFunction{νD∅→νD}\AgdaSpace{}%
\AgdaBound{D}\AgdaSpace{}%
\AgdaBound{x}\AgdaSymbol{)}\<%
\\
%
\>[4]\AgdaFunction{νD∅→νD}\AgdaSpace{}%
\AgdaSymbol{(}\AgdaBound{D}\AgdaSpace{}%
\AgdaOperator{\AgdaInductiveConstructor{∪}}\AgdaSpace{}%
\AgdaBound{D₁}\AgdaSymbol{)}\AgdaSpace{}%
\AgdaSymbol{(}\AgdaInductiveConstructor{inj₂}\AgdaSpace{}%
\AgdaBound{y}\AgdaSymbol{)}\AgdaSpace{}%
\AgdaSymbol{=}\AgdaSpace{}%
\AgdaInductiveConstructor{inj₂}\AgdaSpace{}%
\AgdaSymbol{(}\AgdaFunction{νD∅→νD}\AgdaSpace{}%
\AgdaBound{D₁}\AgdaSpace{}%
\AgdaBound{y}\AgdaSymbol{)}\<%
\\
%
\>[4]\AgdaFunction{νD∅→νD}\AgdaSpace{}%
\AgdaSymbol{(}\AgdaBound{D}\AgdaSpace{}%
\AgdaOperator{\AgdaInductiveConstructor{∗}}\AgdaSpace{}%
\AgdaBound{D₁}\AgdaSymbol{)}\AgdaSpace{}%
\AgdaSymbol{(}\AgdaInductiveConstructor{[]}\AgdaSpace{}%
\AgdaOperator{\AgdaInductiveConstructor{,}}\AgdaSpace{}%
\AgdaInductiveConstructor{[]}\AgdaSpace{}%
\AgdaOperator{\AgdaInductiveConstructor{,}}\AgdaSpace{}%
\AgdaInductiveConstructor{refl}\AgdaSpace{}%
\AgdaOperator{\AgdaInductiveConstructor{,}}\AgdaSpace{}%
\AgdaBound{x}\AgdaSpace{}%
\AgdaOperator{\AgdaInductiveConstructor{,}}\AgdaSpace{}%
\AgdaBound{y}\AgdaSymbol{)}\AgdaSpace{}%
\AgdaSymbol{=}\AgdaSpace{}%
\AgdaInductiveConstructor{[]}\AgdaSpace{}%
\AgdaOperator{\AgdaInductiveConstructor{,}}\AgdaSpace{}%
\AgdaInductiveConstructor{[]}\AgdaSpace{}%
\AgdaOperator{\AgdaInductiveConstructor{,}}\AgdaSpace{}%
\AgdaInductiveConstructor{refl}\AgdaSpace{}%
\AgdaOperator{\AgdaInductiveConstructor{,}}\AgdaSpace{}%
\AgdaFunction{νD∅→νD}\AgdaSpace{}%
\AgdaBound{D}\AgdaSpace{}%
\AgdaBound{x}\AgdaSpace{}%
\AgdaOperator{\AgdaInductiveConstructor{,}}\AgdaSpace{}%
\AgdaFunction{νD∅→νD}\AgdaSpace{}%
\AgdaBound{D₁}\AgdaSpace{}%
\AgdaBound{y}\<%
\\
%
\>[4]\AgdaFunction{νD∅→νD}\AgdaSpace{}%
\AgdaSymbol{(}\AgdaBound{A}\AgdaSpace{}%
\AgdaOperator{\AgdaInductiveConstructor{·}}\AgdaSpace{}%
\AgdaBound{D}\AgdaSymbol{)}\AgdaSpace{}%
\AgdaSymbol{(}\AgdaBound{x}\AgdaSpace{}%
\AgdaOperator{\AgdaInductiveConstructor{,}}\AgdaSpace{}%
\AgdaBound{y}\AgdaSymbol{)}\AgdaSpace{}%
\AgdaSymbol{=}\AgdaSpace{}%
\AgdaBound{x}\AgdaSpace{}%
\AgdaOperator{\AgdaInductiveConstructor{,}}\AgdaSpace{}%
\AgdaFunction{νD∅→νD}\AgdaSpace{}%
\AgdaBound{D}\AgdaSpace{}%
\AgdaBound{y}\<%
\\
%
\>[4]\AgdaFunction{νD∅→νD}\AgdaSpace{}%
\AgdaSymbol{(}\AgdaInductiveConstructor{μ}\AgdaSpace{}%
\AgdaBound{D}\AgdaSymbol{)}\AgdaSpace{}%
\AgdaBound{x}\AgdaSpace{}%
\AgdaSymbol{=}\AgdaSpace{}%
\AgdaBound{x}\<%
\end{code}
The backwards direction is more difficult. We prove a more general lemma from which our disired result follows. The generalized lemma states that, if the application of a descriptor functor to a fixed point of another descriptor is nullable, then either the fixed point plays no role and the descriptor functor is also nullable if applied to the empty language, or the other descriptor (that we took the fixed point of) is nullable when applied to the empty language.
\begin{code}%
%
\>[4]\AgdaFunction{νD∅←νD}\AgdaSpace{}%
\AgdaSymbol{:}\AgdaSpace{}%
\AgdaSymbol{∀}\AgdaSpace{}%
\AgdaBound{D}\AgdaSpace{}%
\AgdaSymbol{→}\AgdaSpace{}%
\AgdaFunction{◇.ν}\AgdaSpace{}%
\AgdaSymbol{(}\AgdaOperator{\AgdaFunction{⟦}}\AgdaSpace{}%
\AgdaBound{D}\AgdaSpace{}%
\AgdaOperator{\AgdaFunction{⟧ₒ}}\AgdaSpace{}%
\AgdaOperator{\AgdaDatatype{⟦}}\AgdaSpace{}%
\AgdaGeneralizable{D₀}\AgdaSpace{}%
\AgdaOperator{\AgdaDatatype{⟧}}\AgdaSymbol{)}\AgdaSpace{}%
\AgdaSymbol{→}\AgdaSpace{}%
\AgdaFunction{◇.ν}\AgdaSpace{}%
\AgdaSymbol{(}\AgdaOperator{\AgdaFunction{⟦}}\AgdaSpace{}%
\AgdaBound{D}\AgdaSpace{}%
\AgdaOperator{\AgdaFunction{⟧ₒ}}\AgdaSpace{}%
\AgdaFunction{◇.∅}\AgdaSymbol{)}\AgdaSpace{}%
\AgdaOperator{\AgdaDatatype{⊎}}\AgdaSpace{}%
\AgdaFunction{◇.ν}\AgdaSpace{}%
\AgdaSymbol{(}\AgdaOperator{\AgdaFunction{⟦}}\AgdaSpace{}%
\AgdaGeneralizable{D₀}\AgdaSpace{}%
\AgdaOperator{\AgdaFunction{⟧ₒ}}\AgdaSpace{}%
\AgdaFunction{◇.∅}\AgdaSymbol{)}\<%
\end{code}
\begin{code}[hide]%
%
\>[4]\AgdaFunction{νD∅←νD}\AgdaSpace{}%
\AgdaInductiveConstructor{ε}\AgdaSpace{}%
\AgdaBound{x}\AgdaSpace{}%
\AgdaSymbol{=}\AgdaSpace{}%
\AgdaInductiveConstructor{inj₁}\AgdaSpace{}%
\AgdaInductiveConstructor{refl}\<%
\\
%
\>[4]\AgdaFunction{νD∅←νD}\AgdaSpace{}%
\AgdaSymbol{(}\AgdaBound{D}\AgdaSpace{}%
\AgdaOperator{\AgdaInductiveConstructor{∪}}\AgdaSpace{}%
\AgdaBound{D₁}\AgdaSymbol{)}\AgdaSpace{}%
\AgdaSymbol{(}\AgdaInductiveConstructor{inj₁}\AgdaSpace{}%
\AgdaBound{x}\AgdaSymbol{)}\AgdaSpace{}%
\AgdaSymbol{=}\AgdaSpace{}%
\AgdaFunction{Sum.map₁}\AgdaSpace{}%
\AgdaInductiveConstructor{inj₁}\AgdaSpace{}%
\AgdaSymbol{(}\AgdaFunction{νD∅←νD}\AgdaSpace{}%
\AgdaBound{D}\AgdaSpace{}%
\AgdaBound{x}\AgdaSymbol{)}\<%
\\
%
\>[4]\AgdaFunction{νD∅←νD}\AgdaSpace{}%
\AgdaSymbol{(}\AgdaBound{D}\AgdaSpace{}%
\AgdaOperator{\AgdaInductiveConstructor{∪}}\AgdaSpace{}%
\AgdaBound{D₁}\AgdaSymbol{)}\AgdaSpace{}%
\AgdaSymbol{(}\AgdaInductiveConstructor{inj₂}\AgdaSpace{}%
\AgdaBound{y}\AgdaSymbol{)}\AgdaSpace{}%
\AgdaSymbol{=}\AgdaSpace{}%
\AgdaFunction{Sum.map₁}\AgdaSpace{}%
\AgdaInductiveConstructor{inj₂}\AgdaSpace{}%
\AgdaSymbol{(}\AgdaFunction{νD∅←νD}\AgdaSpace{}%
\AgdaBound{D₁}\AgdaSpace{}%
\AgdaBound{y}\AgdaSymbol{)}\<%
\\
%
\>[4]\AgdaFunction{νD∅←νD}\AgdaSpace{}%
\AgdaSymbol{(}\AgdaBound{D}\AgdaSpace{}%
\AgdaOperator{\AgdaInductiveConstructor{∗}}\AgdaSpace{}%
\AgdaBound{D₁}\AgdaSymbol{)}\AgdaSpace{}%
\AgdaSymbol{(}\AgdaInductiveConstructor{[]}\AgdaSpace{}%
\AgdaOperator{\AgdaInductiveConstructor{,}}\AgdaSpace{}%
\AgdaInductiveConstructor{[]}\AgdaSpace{}%
\AgdaOperator{\AgdaInductiveConstructor{,}}\AgdaSpace{}%
\AgdaInductiveConstructor{refl}\AgdaSpace{}%
\AgdaOperator{\AgdaInductiveConstructor{,}}\AgdaSpace{}%
\AgdaBound{x}\AgdaSpace{}%
\AgdaOperator{\AgdaInductiveConstructor{,}}\AgdaSpace{}%
\AgdaBound{y}\AgdaSymbol{)}\AgdaSpace{}%
\AgdaSymbol{=}\AgdaSpace{}%
\AgdaFunction{lift⊎₂}\AgdaSpace{}%
\AgdaSymbol{(λ}\AgdaSpace{}%
\AgdaBound{x}\AgdaSpace{}%
\AgdaBound{y}\AgdaSpace{}%
\AgdaSymbol{→}\AgdaSpace{}%
\AgdaInductiveConstructor{[]}\AgdaSpace{}%
\AgdaOperator{\AgdaInductiveConstructor{,}}\AgdaSpace{}%
\AgdaInductiveConstructor{[]}\AgdaSpace{}%
\AgdaOperator{\AgdaInductiveConstructor{,}}\AgdaSpace{}%
\AgdaInductiveConstructor{refl}\AgdaSpace{}%
\AgdaOperator{\AgdaInductiveConstructor{,}}\AgdaSpace{}%
\AgdaBound{x}\AgdaSpace{}%
\AgdaOperator{\AgdaInductiveConstructor{,}}\AgdaSpace{}%
\AgdaBound{y}\AgdaSymbol{)}\AgdaSpace{}%
\AgdaSymbol{(}\AgdaFunction{νD∅←νD}\AgdaSpace{}%
\AgdaBound{D}\AgdaSpace{}%
\AgdaBound{x}\AgdaSymbol{)}\AgdaSpace{}%
\AgdaSymbol{(}\AgdaFunction{νD∅←νD}\AgdaSpace{}%
\AgdaBound{D₁}\AgdaSpace{}%
\AgdaBound{y}\AgdaSymbol{)}\<%
\\
%
\>[4]\AgdaFunction{νD∅←νD}\AgdaSpace{}%
\AgdaSymbol{(}\AgdaBound{x₁}\AgdaSpace{}%
\AgdaOperator{\AgdaInductiveConstructor{·}}\AgdaSpace{}%
\AgdaBound{D}\AgdaSymbol{)}\AgdaSpace{}%
\AgdaSymbol{(}\AgdaBound{x}\AgdaSpace{}%
\AgdaOperator{\AgdaInductiveConstructor{,}}\AgdaSpace{}%
\AgdaBound{y}\AgdaSymbol{)}\AgdaSpace{}%
\AgdaSymbol{=}\AgdaSpace{}%
\AgdaFunction{Sum.map₁}\AgdaSpace{}%
\AgdaSymbol{(}\AgdaBound{x}\AgdaSpace{}%
\AgdaOperator{\AgdaInductiveConstructor{,\AgdaUnderscore{}}}\AgdaSymbol{)}\AgdaSpace{}%
\AgdaSymbol{(}\AgdaFunction{νD∅←νD}\AgdaSpace{}%
\AgdaBound{D}\AgdaSpace{}%
\AgdaBound{y}\AgdaSymbol{)}\<%
\\
%
\>[4]\AgdaFunction{νD∅←νD}\AgdaSpace{}%
\AgdaSymbol{\{}\AgdaArgument{D₀}\AgdaSpace{}%
\AgdaSymbol{=}\AgdaSpace{}%
\AgdaBound{D₀}\AgdaSymbol{\}}\AgdaSpace{}%
\AgdaInductiveConstructor{var}\AgdaSpace{}%
\AgdaSymbol{(}\AgdaInductiveConstructor{roll}\AgdaSpace{}%
\AgdaBound{x}\AgdaSymbol{)}\AgdaSpace{}%
\AgdaSymbol{=}\AgdaSpace{}%
\AgdaInductiveConstructor{inj₂}\AgdaSpace{}%
\AgdaSymbol{(}\AgdaFunction{reduce}\AgdaSpace{}%
\AgdaSymbol{(}\AgdaFunction{νD∅←νD}\AgdaSpace{}%
\AgdaBound{D₀}\AgdaSpace{}%
\AgdaBound{x}\AgdaSymbol{))}\<%
\\
%
\>[4]\AgdaFunction{νD∅←νD}\AgdaSpace{}%
\AgdaSymbol{(}\AgdaInductiveConstructor{μ}\AgdaSpace{}%
\AgdaBound{D}\AgdaSymbol{)}\AgdaSpace{}%
\AgdaBound{x}\AgdaSpace{}%
\AgdaSymbol{=}\AgdaSpace{}%
\AgdaInductiveConstructor{inj₁}\AgdaSpace{}%
\AgdaBound{x}\<%
\end{code}
If we choose $\ab{D₀}~\as{=}~\ab{D}$ then both cases of the resulting disjoint union have the same type, so we can just pick whichever of the two we get as a result using the $\af{reduce}~\as{:}~\ab{A}~\af{⊎}~\ab{A}~\as{→}~\ab{A}$ function. Modulo wrapping and unwrapping of the fixed point (using the $\ac{roll}$ constructor), we now have the two functions which prove the lemma:
\begin{code}%
%
\>[4]\AgdaFunction{νD∅⇔νD}\AgdaSpace{}%
\AgdaSymbol{\{}\AgdaBound{D}\AgdaSymbol{\}}\AgdaSpace{}%
\AgdaSymbol{=}\AgdaSpace{}%
\AgdaFunction{mk⇔}\AgdaSpace{}%
\AgdaSymbol{(}\AgdaInductiveConstructor{roll}\AgdaSpace{}%
\AgdaOperator{\AgdaFunction{∘}}\AgdaSpace{}%
\AgdaFunction{νD∅→νD}\AgdaSpace{}%
\AgdaBound{D}\AgdaSymbol{)}\AgdaSpace{}%
\AgdaSymbol{(}\AgdaFunction{reduce}\AgdaSpace{}%
\AgdaOperator{\AgdaFunction{∘}}\AgdaSpace{}%
\AgdaFunction{νD∅←νD}\AgdaSpace{}%
\AgdaSymbol{\{}\AgdaArgument{D₀}\AgdaSpace{}%
\AgdaSymbol{=}\AgdaSpace{}%
\AgdaBound{D}\AgdaSymbol{\}}\AgdaSpace{}%
\AgdaBound{D}\AgdaSpace{}%
\AgdaOperator{\AgdaFunction{∘}}\AgdaSpace{}%
\AgdaFunction{unroll}\AgdaSymbol{)}\<%
\end{code}
\end{proof}

Using \cref{lem:null-split}, we can easily define nullability for our description functors.

\begin{code}%
%
\>[4]\AgdaFunction{νD}\AgdaSpace{}%
\AgdaSymbol{=}\AgdaSpace{}%
\AgdaFunction{Dec.map}\AgdaSpace{}%
\AgdaFunction{νD∅⇔νD}\AgdaSpace{}%
\AgdaOperator{\AgdaFunction{∘}}\AgdaSpace{}%
\AgdaFunction{νₒ}\AgdaSpace{}%
\AgdaSymbol{(}\AgdaInductiveConstructor{no}\AgdaSpace{}%
\AgdaSymbol{λ}\AgdaSpace{}%
\AgdaSymbol{())}\<%
\end{code}

\begin{remark}
\Cref{lem:null-split} does not define an isomorphism on types. In particular, the backwards direction is not injective. Consider the brackets language. It has the following null element, where we first choose the third disjunct, $\ac{var}~\ac{∗}~\ac{var}$, and then the first disjunct $\ac{ε}$ for both branches.
%
\begin{code}%
%
\>[4]\AgdaFunction{brackets₀}\AgdaSpace{}%
\AgdaSymbol{:}\AgdaSpace{}%
\AgdaFunction{◇.ν}\AgdaSpace{}%
\AgdaFunction{brackets}\<%
\\
%
\>[4]\AgdaFunction{brackets₀}\AgdaSpace{}%
\AgdaSymbol{=}\AgdaSpace{}%
\AgdaInductiveConstructor{roll}\AgdaSpace{}%
\AgdaSymbol{(}\AgdaInductiveConstructor{inj₂}\AgdaSpace{}%
\AgdaSymbol{(}\AgdaInductiveConstructor{inj₂}\AgdaSpace{}%
\AgdaSymbol{(}\AgdaInductiveConstructor{[]}\AgdaSpace{}%
\AgdaOperator{\AgdaInductiveConstructor{,}}\AgdaSpace{}%
\AgdaInductiveConstructor{[]}\AgdaSpace{}%
\AgdaOperator{\AgdaInductiveConstructor{,}}\AgdaSpace{}%
\AgdaInductiveConstructor{refl}\AgdaSpace{}%
\AgdaOperator{\AgdaInductiveConstructor{,}}\AgdaSpace{}%
\AgdaInductiveConstructor{roll}\AgdaSpace{}%
\AgdaSymbol{(}\AgdaInductiveConstructor{inj₁}\AgdaSpace{}%
\AgdaInductiveConstructor{refl}\AgdaSymbol{)}\AgdaSpace{}%
\AgdaOperator{\AgdaInductiveConstructor{,}}\AgdaSpace{}%
\AgdaInductiveConstructor{roll}\AgdaSpace{}%
\AgdaSymbol{(}\AgdaInductiveConstructor{inj₁}\AgdaSpace{}%
\AgdaInductiveConstructor{refl}\AgdaSymbol{))))}\<%
\end{code}
%
When we round-trip this through our lemma, we get a different result:
%
\begin{code}[hide]%
%
\>[4]\AgdaKeyword{open}\AgdaSpace{}%
\AgdaModule{Equivalence}\<%
\end{code}
\begin{code}%
%
\>[4]\AgdaFunction{brackets₀'}%
\>[16]\AgdaSymbol{:}\AgdaSpace{}%
\AgdaFunction{νD∅⇔νD}\AgdaSpace{}%
\AgdaSymbol{\{}\AgdaFunction{bracketsD}\AgdaSymbol{\}}\AgdaSpace{}%
\AgdaSymbol{.}\AgdaField{to}\AgdaSpace{}%
\AgdaSymbol{(}\AgdaFunction{νD∅⇔νD}\AgdaSpace{}%
\AgdaSymbol{\{}\AgdaFunction{bracketsD}\AgdaSymbol{\}}\AgdaSpace{}%
\AgdaSymbol{.}\AgdaField{from}\AgdaSpace{}%
\AgdaFunction{brackets₀}\AgdaSymbol{)}\<%
\\
%
\>[16]\AgdaOperator{\AgdaDatatype{≡}}\AgdaSpace{}%
\AgdaInductiveConstructor{roll}\AgdaSpace{}%
\AgdaSymbol{(}\AgdaInductiveConstructor{inj₁}\AgdaSpace{}%
\AgdaInductiveConstructor{refl}\AgdaSymbol{)}\<%
\\
%
\>[4]\AgdaFunction{brackets₀'}\AgdaSpace{}%
\AgdaSymbol{=}\AgdaSpace{}%
\AgdaInductiveConstructor{refl}\<%
\end{code}
%
It now directly takes the first disjunct, $\ac{ε}$.

In practice, such problems should be avoided by using unambiguous languages, ensuring that there is only one valid parse result for each string.
\jr{todo: give recommendations for future work, for example to use data-dependent grammars.}
\end{remark}

\subsection{Reduction}

The final piece of the puzzle is reduction. This tells us how the language descriptions change after parsing each input character.

In \cref{sec:reduction-by-example}, we established that the meaning of self-references changes and thus they need to be replaced by local fixed points of the original language. We define a function $\af{σD}$ to perform this substitution. It is a simple recursive function which replaces the $\ac{var}$ constructor with a given $\ab{D'}$ description.
%
\begin{code}%
%
\>[4]\AgdaFunction{σ}\AgdaSpace{}%
\AgdaSymbol{:}\AgdaSpace{}%
\AgdaDatatype{Desc}\AgdaSpace{}%
\AgdaSymbol{→}\AgdaSpace{}%
\AgdaDatatype{Desc}\AgdaSpace{}%
\AgdaSymbol{→}\AgdaSpace{}%
\AgdaDatatype{Desc}\<%
\\
%
\>[4]\AgdaFunction{σ}\AgdaSpace{}%
\AgdaInductiveConstructor{∅}%
\>[16]\AgdaBound{D'}\AgdaSpace{}%
\AgdaSymbol{=}\AgdaSpace{}%
\AgdaInductiveConstructor{∅}\<%
\\
%
\>[4]\AgdaFunction{σ}\AgdaSpace{}%
\AgdaInductiveConstructor{ε}%
\>[16]\AgdaBound{D'}\AgdaSpace{}%
\AgdaSymbol{=}\AgdaSpace{}%
\AgdaInductiveConstructor{ε}\<%
\\
%
\>[4]\AgdaFunction{σ}\AgdaSpace{}%
\AgdaSymbol{(}\AgdaOperator{\AgdaInductiveConstructor{`}}\AgdaSpace{}%
\AgdaBound{c}\AgdaSymbol{)}%
\>[16]\AgdaBound{D'}\AgdaSpace{}%
\AgdaSymbol{=}\AgdaSpace{}%
\AgdaOperator{\AgdaInductiveConstructor{`}}\AgdaSpace{}%
\AgdaBound{c}\<%
\\
%
\>[4]\AgdaFunction{σ}\AgdaSpace{}%
\AgdaSymbol{(}\AgdaBound{D}\AgdaSpace{}%
\AgdaOperator{\AgdaInductiveConstructor{∪}}\AgdaSpace{}%
\AgdaBound{D₁}\AgdaSymbol{)}%
\>[16]\AgdaBound{D'}\AgdaSpace{}%
\AgdaSymbol{=}\AgdaSpace{}%
\AgdaFunction{σ}\AgdaSpace{}%
\AgdaBound{D}\AgdaSpace{}%
\AgdaBound{D'}\AgdaSpace{}%
\AgdaOperator{\AgdaInductiveConstructor{∪}}\AgdaSpace{}%
\AgdaFunction{σ}\AgdaSpace{}%
\AgdaBound{D₁}\AgdaSpace{}%
\AgdaBound{D'}\<%
\\
%
\>[4]\AgdaFunction{σ}\AgdaSpace{}%
\AgdaSymbol{(}\AgdaBound{D}\AgdaSpace{}%
\AgdaOperator{\AgdaInductiveConstructor{∗}}\AgdaSpace{}%
\AgdaBound{D₁}\AgdaSymbol{)}%
\>[16]\AgdaBound{D'}\AgdaSpace{}%
\AgdaSymbol{=}\AgdaSpace{}%
\AgdaFunction{σ}\AgdaSpace{}%
\AgdaBound{D}\AgdaSpace{}%
\AgdaBound{D'}\AgdaSpace{}%
\AgdaOperator{\AgdaInductiveConstructor{∗}}\AgdaSpace{}%
\AgdaFunction{σ}\AgdaSpace{}%
\AgdaBound{D₁}\AgdaSpace{}%
\AgdaBound{D'}\<%
\\
%
\>[4]\AgdaFunction{σ}\AgdaSpace{}%
\AgdaSymbol{(}\AgdaBound{x}\AgdaSpace{}%
\AgdaOperator{\AgdaInductiveConstructor{·}}\AgdaSpace{}%
\AgdaBound{D}\AgdaSymbol{)}%
\>[16]\AgdaBound{D'}\AgdaSpace{}%
\AgdaSymbol{=}\AgdaSpace{}%
\AgdaBound{x}\AgdaSpace{}%
\AgdaOperator{\AgdaInductiveConstructor{·}}\AgdaSpace{}%
\AgdaFunction{σ}\AgdaSpace{}%
\AgdaBound{D}\AgdaSpace{}%
\AgdaBound{D'}\<%
\\
%
\>[4]\AgdaFunction{σ}\AgdaSpace{}%
\AgdaInductiveConstructor{var}%
\>[16]\AgdaBound{D'}\AgdaSpace{}%
\AgdaSymbol{=}\AgdaSpace{}%
\AgdaBound{D'}\<%
\\
%
\>[4]\AgdaFunction{σ}\AgdaSpace{}%
\AgdaSymbol{(}\AgdaInductiveConstructor{μ}\AgdaSpace{}%
\AgdaBound{D}\AgdaSymbol{)}%
\>[16]\AgdaBound{D'}\AgdaSpace{}%
\AgdaSymbol{=}\AgdaSpace{}%
\AgdaInductiveConstructor{μ}\AgdaSpace{}%
\AgdaBound{D}\<%
\end{code}

\begin{lemma}
Substitution of a local fixed point into a description is the same as applying the corresponding functor to the semantic fixed point.
\begin{code}%
%
\>[4]\AgdaFunction{σμ}\AgdaSpace{}%
\AgdaSymbol{:}\AgdaSpace{}%
\AgdaSymbol{∀}\AgdaSpace{}%
\AgdaBound{D}\AgdaSpace{}%
\AgdaSymbol{→}\AgdaSpace{}%
\AgdaOperator{\AgdaFunction{⟦}}\AgdaSpace{}%
\AgdaFunction{σ}\AgdaSpace{}%
\AgdaBound{D}\AgdaSpace{}%
\AgdaSymbol{(}\AgdaInductiveConstructor{μ}\AgdaSpace{}%
\AgdaGeneralizable{D₀}\AgdaSymbol{)}\AgdaSpace{}%
\AgdaOperator{\AgdaFunction{⟧ₒ}}\AgdaSpace{}%
\AgdaGeneralizable{P}\AgdaSpace{}%
\AgdaGeneralizable{w}\AgdaSpace{}%
\AgdaOperator{\AgdaDatatype{≡}}\AgdaSpace{}%
\AgdaOperator{\AgdaFunction{⟦}}\AgdaSpace{}%
\AgdaBound{D}\AgdaSpace{}%
\AgdaOperator{\AgdaFunction{⟧ₒ}}\AgdaSpace{}%
\AgdaOperator{\AgdaDatatype{⟦}}\AgdaSpace{}%
\AgdaGeneralizable{D₀}\AgdaSpace{}%
\AgdaOperator{\AgdaDatatype{⟧}}\AgdaSpace{}%
\AgdaGeneralizable{w}\<%
\end{code}
\end{lemma}
The proof follows directly by induction and computation.
\begin{code}[hide]%
%
\>[4]\AgdaFunction{σμ'}\AgdaSpace{}%
\AgdaSymbol{:}\AgdaSpace{}%
\AgdaSymbol{∀}\AgdaSpace{}%
\AgdaBound{D}\AgdaSpace{}%
\AgdaSymbol{→}\AgdaSpace{}%
\AgdaOperator{\AgdaFunction{⟦}}\AgdaSpace{}%
\AgdaFunction{σ}\AgdaSpace{}%
\AgdaBound{D}\AgdaSpace{}%
\AgdaSymbol{(}\AgdaInductiveConstructor{μ}\AgdaSpace{}%
\AgdaGeneralizable{D₀}\AgdaSymbol{)}\AgdaSpace{}%
\AgdaOperator{\AgdaFunction{⟧ₒ}}\AgdaSpace{}%
\AgdaGeneralizable{P}\AgdaSpace{}%
\AgdaOperator{\AgdaDatatype{≡}}\AgdaSpace{}%
\AgdaOperator{\AgdaFunction{⟦}}\AgdaSpace{}%
\AgdaBound{D}\AgdaSpace{}%
\AgdaOperator{\AgdaFunction{⟧ₒ}}\AgdaSpace{}%
\AgdaOperator{\AgdaDatatype{⟦}}\AgdaSpace{}%
\AgdaGeneralizable{D₀}\AgdaSpace{}%
\AgdaOperator{\AgdaDatatype{⟧}}\<%
\\
%
\>[4]\AgdaFunction{σμ'}\AgdaSpace{}%
\AgdaInductiveConstructor{∅}\AgdaSpace{}%
\AgdaSymbol{=}\AgdaSpace{}%
\AgdaInductiveConstructor{refl}\<%
\\
%
\>[4]\AgdaFunction{σμ'}\AgdaSpace{}%
\AgdaInductiveConstructor{ε}\AgdaSpace{}%
\AgdaSymbol{=}\AgdaSpace{}%
\AgdaInductiveConstructor{refl}\<%
\\
%
\>[4]\AgdaFunction{σμ'}\AgdaSpace{}%
\AgdaSymbol{(}\AgdaOperator{\AgdaInductiveConstructor{`}}\AgdaSpace{}%
\AgdaBound{x}\AgdaSymbol{)}\AgdaSpace{}%
\AgdaSymbol{=}\AgdaSpace{}%
\AgdaInductiveConstructor{refl}\<%
\\
%
\>[4]\AgdaFunction{σμ'}\AgdaSpace{}%
\AgdaSymbol{(}\AgdaBound{D}\AgdaSpace{}%
\AgdaOperator{\AgdaInductiveConstructor{∪}}\AgdaSpace{}%
\AgdaBound{D₁}\AgdaSymbol{)}\AgdaSpace{}%
\AgdaSymbol{=}\AgdaSpace{}%
\AgdaFunction{cong₂}\AgdaSpace{}%
\AgdaOperator{\AgdaFunction{◇.\AgdaUnderscore{}∪\AgdaUnderscore{}}}\AgdaSpace{}%
\AgdaSymbol{(}\AgdaFunction{σμ'}\AgdaSpace{}%
\AgdaBound{D}\AgdaSymbol{)}\AgdaSpace{}%
\AgdaSymbol{(}\AgdaFunction{σμ'}\AgdaSpace{}%
\AgdaBound{D₁}\AgdaSymbol{)}\<%
\\
%
\>[4]\AgdaFunction{σμ'}\AgdaSpace{}%
\AgdaSymbol{(}\AgdaBound{D}\AgdaSpace{}%
\AgdaOperator{\AgdaInductiveConstructor{∗}}\AgdaSpace{}%
\AgdaBound{D₁}\AgdaSymbol{)}\AgdaSpace{}%
\AgdaSymbol{=}\AgdaSpace{}%
\AgdaFunction{cong₂}\AgdaSpace{}%
\AgdaOperator{\AgdaFunction{◇.\AgdaUnderscore{}∗\AgdaUnderscore{}}}\AgdaSpace{}%
\AgdaSymbol{(}\AgdaFunction{σμ'}\AgdaSpace{}%
\AgdaBound{D}\AgdaSymbol{)}\AgdaSpace{}%
\AgdaSymbol{(}\AgdaFunction{σμ'}\AgdaSpace{}%
\AgdaBound{D₁}\AgdaSymbol{)}\<%
\\
%
\>[4]\AgdaFunction{σμ'}\AgdaSpace{}%
\AgdaSymbol{(\AgdaUnderscore{}}\AgdaSpace{}%
\AgdaOperator{\AgdaInductiveConstructor{·}}\AgdaSpace{}%
\AgdaBound{D}\AgdaSymbol{)}\AgdaSpace{}%
\AgdaSymbol{=}\AgdaSpace{}%
\AgdaFunction{cong}\AgdaSpace{}%
\AgdaSymbol{(\AgdaUnderscore{}}\AgdaSpace{}%
\AgdaOperator{\AgdaFunction{◇.·\AgdaUnderscore{}}}\AgdaSymbol{)}\AgdaSpace{}%
\AgdaSymbol{(}\AgdaFunction{σμ'}\AgdaSpace{}%
\AgdaBound{D}\AgdaSymbol{)}\<%
\\
%
\>[4]\AgdaFunction{σμ'}\AgdaSpace{}%
\AgdaInductiveConstructor{var}\AgdaSpace{}%
\AgdaSymbol{=}\AgdaSpace{}%
\AgdaInductiveConstructor{refl}\<%
\\
%
\>[4]\AgdaFunction{σμ'}\AgdaSpace{}%
\AgdaSymbol{(}\AgdaInductiveConstructor{μ}\AgdaSpace{}%
\AgdaBound{D}\AgdaSymbol{)}\AgdaSpace{}%
\AgdaSymbol{=}\AgdaSpace{}%
\AgdaInductiveConstructor{refl}\<%
\\
\>[0]\<%
\\
%
\>[4]\AgdaFunction{σμ}\AgdaSpace{}%
\AgdaBound{D}\AgdaSpace{}%
\AgdaSymbol{=}\AgdaSpace{}%
\AgdaFunction{cong}\AgdaSpace{}%
\AgdaSymbol{(λ}\AgdaSpace{}%
\AgdaBound{f}\AgdaSpace{}%
\AgdaSymbol{→}\AgdaSpace{}%
\AgdaBound{f}\AgdaSpace{}%
\AgdaSymbol{\AgdaUnderscore{})}\AgdaSpace{}%
\AgdaSymbol{(}\AgdaFunction{σμ'}\AgdaSpace{}%
\AgdaBound{D}\AgdaSymbol{)}\<%
\end{code}


\begin{code}[hide]%
%
\>[4]\AgdaFunction{◂νₒ}\AgdaSpace{}%
\AgdaSymbol{:}\AgdaSpace{}%
\AgdaRecord{Dec}\AgdaSpace{}%
\AgdaSymbol{(}\AgdaFunction{◇.ν}\AgdaSpace{}%
\AgdaOperator{\AgdaDatatype{⟦}}\AgdaSpace{}%
\AgdaGeneralizable{D₀}\AgdaSpace{}%
\AgdaOperator{\AgdaDatatype{⟧}}\AgdaSymbol{)}\AgdaSpace{}%
\AgdaSymbol{→}\AgdaSpace{}%
\AgdaSymbol{∀}\AgdaSpace{}%
\AgdaBound{D}\AgdaSpace{}%
\AgdaSymbol{→}\AgdaSpace{}%
\AgdaRecord{Dec}\AgdaSpace{}%
\AgdaSymbol{(}\AgdaFunction{◇.ν}\AgdaSpace{}%
\AgdaSymbol{(}\AgdaOperator{\AgdaFunction{⟦}}\AgdaSpace{}%
\AgdaBound{D}\AgdaSpace{}%
\AgdaOperator{\AgdaFunction{⟧ₒ}}\AgdaSpace{}%
\AgdaOperator{\AgdaDatatype{⟦}}\AgdaSpace{}%
\AgdaGeneralizable{D₀}\AgdaSpace{}%
\AgdaOperator{\AgdaDatatype{⟧}}\AgdaSymbol{))}\<%
\\
%
\>[4]\AgdaFunction{◂νₒ}\AgdaSpace{}%
\AgdaSymbol{=}\AgdaSpace{}%
\AgdaFunction{νₒ}\AgdaSpace{}%
\AgdaSymbol{\{}\AgdaArgument{P}\AgdaSpace{}%
\AgdaSymbol{=}\AgdaSpace{}%
\AgdaOperator{\AgdaDatatype{⟦}}\AgdaSpace{}%
\AgdaSymbol{\AgdaUnderscore{}}\AgdaSpace{}%
\AgdaOperator{\AgdaDatatype{⟧}}\AgdaSymbol{\}}\<%
\end{code}
\begin{code}%
%
\>[4]\AgdaFunction{δₒ}\AgdaSpace{}%
\AgdaSymbol{:}\AgdaSpace{}%
\AgdaDatatype{Desc}\AgdaSpace{}%
\AgdaSymbol{→}\AgdaSpace{}%
\AgdaPostulate{Char}\AgdaSpace{}%
\AgdaSymbol{→}\AgdaSpace{}%
\AgdaDatatype{Desc}\AgdaSpace{}%
\AgdaSymbol{→}\AgdaSpace{}%
\AgdaDatatype{Desc}\<%
\\
%
\>[4]\AgdaFunction{δₒ}\AgdaSpace{}%
\AgdaBound{D₀}\AgdaSpace{}%
\AgdaBound{c}\AgdaSpace{}%
\AgdaInductiveConstructor{∅}%
\>[22]\AgdaSymbol{=}\AgdaSpace{}%
\AgdaInductiveConstructor{∅}\<%
\\
%
\>[4]\AgdaFunction{δₒ}\AgdaSpace{}%
\AgdaBound{D₀}\AgdaSpace{}%
\AgdaBound{c}\AgdaSpace{}%
\AgdaInductiveConstructor{ε}%
\>[22]\AgdaSymbol{=}\AgdaSpace{}%
\AgdaInductiveConstructor{∅}\<%
\\
%
\>[4]\AgdaFunction{δₒ}\AgdaSpace{}%
\AgdaBound{D₀}\AgdaSpace{}%
\AgdaBound{c}\AgdaSpace{}%
\AgdaSymbol{(}\AgdaOperator{\AgdaInductiveConstructor{`}}\AgdaSpace{}%
\AgdaBound{c'}\AgdaSymbol{)}%
\>[22]\AgdaSymbol{=}\AgdaSpace{}%
\AgdaSymbol{(}\AgdaBound{c}\AgdaSpace{}%
\AgdaOperator{\AgdaFunction{≟}}\AgdaSpace{}%
\AgdaBound{c'}\AgdaSymbol{)}\AgdaSpace{}%
\AgdaOperator{\AgdaInductiveConstructor{·}}\AgdaSpace{}%
\AgdaInductiveConstructor{ε}\<%
\\
%
\>[4]\AgdaFunction{δₒ}\AgdaSpace{}%
\AgdaBound{D₀}\AgdaSpace{}%
\AgdaBound{c}\AgdaSpace{}%
\AgdaSymbol{(}\AgdaBound{D}\AgdaSpace{}%
\AgdaOperator{\AgdaInductiveConstructor{∪}}\AgdaSpace{}%
\AgdaBound{D₁}\AgdaSymbol{)}%
\>[22]\AgdaSymbol{=}\AgdaSpace{}%
\AgdaFunction{δₒ}\AgdaSpace{}%
\AgdaBound{D₀}\AgdaSpace{}%
\AgdaBound{c}\AgdaSpace{}%
\AgdaBound{D}\AgdaSpace{}%
\AgdaOperator{\AgdaInductiveConstructor{∪}}\AgdaSpace{}%
\AgdaFunction{δₒ}\AgdaSpace{}%
\AgdaBound{D₀}\AgdaSpace{}%
\AgdaBound{c}\AgdaSpace{}%
\AgdaBound{D₁}\<%
\\
%
\>[4]\AgdaFunction{δₒ}\AgdaSpace{}%
\AgdaBound{D₀}\AgdaSpace{}%
\AgdaBound{c}\AgdaSpace{}%
\AgdaSymbol{(}\AgdaBound{D}\AgdaSpace{}%
\AgdaOperator{\AgdaInductiveConstructor{∗}}\AgdaSpace{}%
\AgdaBound{D₁}\AgdaSymbol{)}%
\>[22]\AgdaSymbol{=}\AgdaSpace{}%
\AgdaFunction{◂νₒ}\AgdaSpace{}%
\AgdaSymbol{(}\AgdaFunction{νD}\AgdaSpace{}%
\AgdaBound{D₀}\AgdaSymbol{)}\AgdaSpace{}%
\AgdaBound{D}\AgdaSpace{}%
\AgdaOperator{\AgdaInductiveConstructor{·}}\AgdaSpace{}%
\AgdaFunction{δₒ}\AgdaSpace{}%
\AgdaBound{D₀}\AgdaSpace{}%
\AgdaBound{c}\AgdaSpace{}%
\AgdaBound{D₁}\AgdaSpace{}%
\AgdaOperator{\AgdaInductiveConstructor{∪}}\AgdaSpace{}%
\AgdaFunction{δₒ}\AgdaSpace{}%
\AgdaBound{D₀}\AgdaSpace{}%
\AgdaBound{c}\AgdaSpace{}%
\AgdaBound{D}\AgdaSpace{}%
\AgdaOperator{\AgdaInductiveConstructor{∗}}\AgdaSpace{}%
\AgdaFunction{σ}\AgdaSpace{}%
\AgdaBound{D₁}\AgdaSpace{}%
\AgdaSymbol{(}\AgdaInductiveConstructor{μ}\AgdaSpace{}%
\AgdaBound{D₀}\AgdaSymbol{)}\<%
\\
%
\>[4]\AgdaFunction{δₒ}\AgdaSpace{}%
\AgdaBound{D₀}\AgdaSpace{}%
\AgdaBound{c}\AgdaSpace{}%
\AgdaSymbol{(}\AgdaBound{x}\AgdaSpace{}%
\AgdaOperator{\AgdaInductiveConstructor{·}}\AgdaSpace{}%
\AgdaBound{D}\AgdaSymbol{)}%
\>[22]\AgdaSymbol{=}\AgdaSpace{}%
\AgdaBound{x}\AgdaSpace{}%
\AgdaOperator{\AgdaInductiveConstructor{·}}\AgdaSpace{}%
\AgdaFunction{δₒ}\AgdaSpace{}%
\AgdaBound{D₀}\AgdaSpace{}%
\AgdaBound{c}\AgdaSpace{}%
\AgdaBound{D}\<%
\\
%
\>[4]\AgdaFunction{δₒ}\AgdaSpace{}%
\AgdaBound{D₀}\AgdaSpace{}%
\AgdaBound{c}\AgdaSpace{}%
\AgdaInductiveConstructor{var}%
\>[22]\AgdaSymbol{=}\AgdaSpace{}%
\AgdaInductiveConstructor{var}\<%
\\
%
\>[4]\AgdaFunction{δₒ}\AgdaSpace{}%
\AgdaBound{D₀}\AgdaSpace{}%
\AgdaBound{c}\AgdaSpace{}%
\AgdaSymbol{(}\AgdaInductiveConstructor{μ}\AgdaSpace{}%
\AgdaBound{D}\AgdaSymbol{)}%
\>[22]\AgdaSymbol{=}\AgdaSpace{}%
\AgdaInductiveConstructor{μ}\AgdaSpace{}%
\AgdaSymbol{(}\AgdaFunction{δD}\AgdaSpace{}%
\AgdaBound{c}\AgdaSpace{}%
\AgdaBound{D}\AgdaSymbol{)}\<%
\end{code}
\begin{code}%
%
\>[4]\AgdaFunction{δD}\AgdaSpace{}%
\AgdaBound{c}\AgdaSpace{}%
\AgdaBound{D}\AgdaSpace{}%
\AgdaSymbol{=}\AgdaSpace{}%
\AgdaFunction{δₒ}\AgdaSpace{}%
\AgdaBound{D}\AgdaSpace{}%
\AgdaBound{c}\AgdaSpace{}%
\AgdaBound{D}\<%
\end{code}
\begin{code}%
%
\>[4]\AgdaFunction{δD-to}\AgdaSpace{}%
\AgdaSymbol{:}\AgdaSpace{}%
\AgdaSymbol{∀}\AgdaSpace{}%
\AgdaBound{D}\AgdaSpace{}%
\AgdaSymbol{→}\AgdaSpace{}%
\AgdaOperator{\AgdaFunction{⟦}}\AgdaSpace{}%
\AgdaFunction{δₒ}\AgdaSpace{}%
\AgdaGeneralizable{D₀}\AgdaSpace{}%
\AgdaGeneralizable{c}\AgdaSpace{}%
\AgdaBound{D}\AgdaSpace{}%
\AgdaOperator{\AgdaFunction{⟧ₒ}}\AgdaSpace{}%
\AgdaOperator{\AgdaDatatype{⟦}}\AgdaSpace{}%
\AgdaFunction{δD}\AgdaSpace{}%
\AgdaGeneralizable{c}\AgdaSpace{}%
\AgdaGeneralizable{D₀}\AgdaSpace{}%
\AgdaOperator{\AgdaDatatype{⟧}}\AgdaSpace{}%
\AgdaGeneralizable{w}\AgdaSpace{}%
\AgdaSymbol{→}\AgdaSpace{}%
\AgdaFunction{◇.δ}\AgdaSpace{}%
\AgdaGeneralizable{c}\AgdaSpace{}%
\AgdaSymbol{(}\AgdaOperator{\AgdaFunction{⟦}}\AgdaSpace{}%
\AgdaBound{D}\AgdaSpace{}%
\AgdaOperator{\AgdaFunction{⟧ₒ}}\AgdaSpace{}%
\AgdaOperator{\AgdaDatatype{⟦}}\AgdaSpace{}%
\AgdaGeneralizable{D₀}\AgdaSpace{}%
\AgdaOperator{\AgdaDatatype{⟧}}\AgdaSymbol{)}\AgdaSpace{}%
\AgdaGeneralizable{w}\<%
\\
%
\>[4]\AgdaFunction{δD-to}\AgdaSpace{}%
\AgdaSymbol{(}\AgdaOperator{\AgdaInductiveConstructor{`}}\AgdaSpace{}%
\AgdaBound{c'}\AgdaSymbol{)}\AgdaSpace{}%
\AgdaSymbol{=}\AgdaSpace{}%
\AgdaFunction{◇.δ`}\AgdaSpace{}%
\AgdaSymbol{.}\AgdaField{to}\<%
\\
%
\>[4]\AgdaFunction{δD-to}\AgdaSpace{}%
\AgdaSymbol{(}\AgdaBound{D}\AgdaSpace{}%
\AgdaOperator{\AgdaInductiveConstructor{∪}}\AgdaSpace{}%
\AgdaBound{D₁}\AgdaSymbol{)}\AgdaSpace{}%
\AgdaSymbol{(}\AgdaInductiveConstructor{inj₁}\AgdaSpace{}%
\AgdaBound{x}\AgdaSymbol{)}\AgdaSpace{}%
\AgdaSymbol{=}\AgdaSpace{}%
\AgdaInductiveConstructor{inj₁}\AgdaSpace{}%
\AgdaSymbol{(}\AgdaFunction{δD-to}\AgdaSpace{}%
\AgdaBound{D}\AgdaSpace{}%
\AgdaBound{x}\AgdaSymbol{)}\<%
\\
%
\>[4]\AgdaFunction{δD-to}\AgdaSpace{}%
\AgdaSymbol{(}\AgdaBound{D}\AgdaSpace{}%
\AgdaOperator{\AgdaInductiveConstructor{∪}}\AgdaSpace{}%
\AgdaBound{D₁}\AgdaSymbol{)}\AgdaSpace{}%
\AgdaSymbol{(}\AgdaInductiveConstructor{inj₂}\AgdaSpace{}%
\AgdaBound{y}\AgdaSymbol{)}\AgdaSpace{}%
\AgdaSymbol{=}\AgdaSpace{}%
\AgdaInductiveConstructor{inj₂}\AgdaSpace{}%
\AgdaSymbol{(}\AgdaFunction{δD-to}\AgdaSpace{}%
\AgdaBound{D₁}\AgdaSpace{}%
\AgdaBound{y}\AgdaSymbol{)}\<%
\\
%
\>[4]\AgdaFunction{δD-to}\AgdaSpace{}%
\AgdaSymbol{(}\AgdaBound{D}\AgdaSpace{}%
\AgdaOperator{\AgdaInductiveConstructor{∗}}\AgdaSpace{}%
\AgdaBound{D₁}\AgdaSymbol{)}\AgdaSpace{}%
\AgdaSymbol{(}\AgdaInductiveConstructor{inj₁}\AgdaSpace{}%
\AgdaSymbol{(}\AgdaBound{x}\AgdaSpace{}%
\AgdaOperator{\AgdaInductiveConstructor{,}}\AgdaSpace{}%
\AgdaBound{y}\AgdaSymbol{))}\AgdaSpace{}%
\AgdaSymbol{=}\AgdaSpace{}%
\AgdaInductiveConstructor{[]}\AgdaSpace{}%
\AgdaOperator{\AgdaInductiveConstructor{,}}\AgdaSpace{}%
\AgdaSymbol{\AgdaUnderscore{}}\AgdaSpace{}%
\AgdaOperator{\AgdaInductiveConstructor{,}}\AgdaSpace{}%
\AgdaInductiveConstructor{refl}\AgdaSpace{}%
\AgdaOperator{\AgdaInductiveConstructor{,}}\AgdaSpace{}%
\AgdaBound{x}\AgdaSpace{}%
\AgdaOperator{\AgdaInductiveConstructor{,}}\AgdaSpace{}%
\AgdaFunction{δD-to}\AgdaSpace{}%
\AgdaBound{D₁}\AgdaSpace{}%
\AgdaBound{y}\<%
\\
%
\>[4]\AgdaFunction{δD-to}\AgdaSpace{}%
\AgdaSymbol{(}\AgdaBound{D}\AgdaSpace{}%
\AgdaOperator{\AgdaInductiveConstructor{∗}}\AgdaSpace{}%
\AgdaBound{D₁}\AgdaSymbol{)}\AgdaSpace{}%
\AgdaSymbol{(}\AgdaInductiveConstructor{inj₂}\AgdaSpace{}%
\AgdaSymbol{(}\AgdaBound{u}\AgdaSpace{}%
\AgdaOperator{\AgdaInductiveConstructor{,}}\AgdaSpace{}%
\AgdaBound{v}\AgdaSpace{}%
\AgdaOperator{\AgdaInductiveConstructor{,}}\AgdaSpace{}%
\AgdaInductiveConstructor{refl}\AgdaSpace{}%
\AgdaOperator{\AgdaInductiveConstructor{,}}\AgdaSpace{}%
\AgdaBound{x}\AgdaSpace{}%
\AgdaOperator{\AgdaInductiveConstructor{,}}\AgdaSpace{}%
\AgdaBound{y}\AgdaSymbol{))}\AgdaSpace{}%
\AgdaSymbol{=}\AgdaSpace{}%
\AgdaSymbol{(\AgdaUnderscore{}}\AgdaSpace{}%
\AgdaOperator{\AgdaInductiveConstructor{∷}}\AgdaSpace{}%
\AgdaSymbol{\AgdaUnderscore{})}\AgdaSpace{}%
\AgdaOperator{\AgdaInductiveConstructor{,}}\AgdaSpace{}%
\AgdaSymbol{\AgdaUnderscore{}}\AgdaSpace{}%
\AgdaOperator{\AgdaInductiveConstructor{,}}\AgdaSpace{}%
\AgdaInductiveConstructor{refl}\AgdaSpace{}%
\AgdaOperator{\AgdaInductiveConstructor{,}}\AgdaSpace{}%
\AgdaFunction{δD-to}\AgdaSpace{}%
\AgdaBound{D}\AgdaSpace{}%
\AgdaBound{x}\AgdaSpace{}%
\AgdaOperator{\AgdaInductiveConstructor{,}}\AgdaSpace{}%
\AgdaFunction{subst}\AgdaSpace{}%
\AgdaFunction{id}\AgdaSpace{}%
\AgdaSymbol{(}\AgdaFunction{σμ}\AgdaSpace{}%
\AgdaBound{D₁}\AgdaSymbol{)}\AgdaSpace{}%
\AgdaBound{y}\<%
\\
%
\>[4]\AgdaFunction{δD-to}\AgdaSpace{}%
\AgdaSymbol{(}\AgdaBound{A}\AgdaSpace{}%
\AgdaOperator{\AgdaInductiveConstructor{·}}\AgdaSpace{}%
\AgdaBound{D}\AgdaSymbol{)}\AgdaSpace{}%
\AgdaSymbol{(}\AgdaBound{x}\AgdaSpace{}%
\AgdaOperator{\AgdaInductiveConstructor{,}}\AgdaSpace{}%
\AgdaBound{y}\AgdaSymbol{)}\AgdaSpace{}%
\AgdaSymbol{=}\AgdaSpace{}%
\AgdaBound{x}\AgdaSpace{}%
\AgdaOperator{\AgdaInductiveConstructor{,}}\AgdaSpace{}%
\AgdaFunction{δD-to}\AgdaSpace{}%
\AgdaBound{D}\AgdaSpace{}%
\AgdaBound{y}\<%
\\
%
\>[4]\AgdaFunction{δD-to}\AgdaSpace{}%
\AgdaSymbol{\{}\AgdaArgument{D₀}\AgdaSpace{}%
\AgdaSymbol{=}\AgdaSpace{}%
\AgdaBound{D}\AgdaSymbol{\}}\AgdaSpace{}%
\AgdaInductiveConstructor{var}\AgdaSpace{}%
\AgdaSymbol{(}\AgdaInductiveConstructor{roll}\AgdaSpace{}%
\AgdaBound{x}\AgdaSymbol{)}\AgdaSpace{}%
\AgdaSymbol{=}\AgdaSpace{}%
\AgdaInductiveConstructor{roll}\AgdaSpace{}%
\AgdaSymbol{(}\AgdaFunction{δD-to}\AgdaSpace{}%
\AgdaBound{D}\AgdaSpace{}%
\AgdaBound{x}\AgdaSymbol{)}\<%
\\
%
\>[4]\AgdaFunction{δD-to}\AgdaSpace{}%
\AgdaSymbol{(}\AgdaInductiveConstructor{μ}\AgdaSpace{}%
\AgdaBound{D}\AgdaSymbol{)}\AgdaSpace{}%
\AgdaSymbol{(}\AgdaInductiveConstructor{roll}\AgdaSpace{}%
\AgdaBound{x}\AgdaSymbol{)}\AgdaSpace{}%
\AgdaSymbol{=}\AgdaSpace{}%
\AgdaInductiveConstructor{roll}\AgdaSpace{}%
\AgdaSymbol{(}\AgdaFunction{δD-to}\AgdaSpace{}%
\AgdaBound{D}\AgdaSpace{}%
\AgdaBound{x}\AgdaSymbol{)}\<%
\end{code}
\begin{code}%
%
\>[4]\AgdaFunction{δD-from}\AgdaSpace{}%
\AgdaSymbol{:}\AgdaSpace{}%
\AgdaSymbol{∀}\AgdaSpace{}%
\AgdaBound{D}\AgdaSpace{}%
\AgdaSymbol{→}\AgdaSpace{}%
\AgdaFunction{◇.δ}\AgdaSpace{}%
\AgdaGeneralizable{c}\AgdaSpace{}%
\AgdaSymbol{(}\AgdaOperator{\AgdaFunction{⟦}}\AgdaSpace{}%
\AgdaBound{D}\AgdaSpace{}%
\AgdaOperator{\AgdaFunction{⟧ₒ}}\AgdaSpace{}%
\AgdaOperator{\AgdaDatatype{⟦}}\AgdaSpace{}%
\AgdaGeneralizable{D₀}\AgdaSpace{}%
\AgdaOperator{\AgdaDatatype{⟧}}\AgdaSymbol{)}\AgdaSpace{}%
\AgdaGeneralizable{w}\AgdaSpace{}%
\AgdaSymbol{→}\AgdaSpace{}%
\AgdaOperator{\AgdaFunction{⟦}}\AgdaSpace{}%
\AgdaFunction{δₒ}\AgdaSpace{}%
\AgdaGeneralizable{D₀}\AgdaSpace{}%
\AgdaGeneralizable{c}\AgdaSpace{}%
\AgdaBound{D}\AgdaSpace{}%
\AgdaOperator{\AgdaFunction{⟧ₒ}}\AgdaSpace{}%
\AgdaOperator{\AgdaDatatype{⟦}}\AgdaSpace{}%
\AgdaFunction{δD}\AgdaSpace{}%
\AgdaGeneralizable{c}\AgdaSpace{}%
\AgdaGeneralizable{D₀}\AgdaSpace{}%
\AgdaOperator{\AgdaDatatype{⟧}}\AgdaSpace{}%
\AgdaGeneralizable{w}\<%
\\
%
\>[4]\AgdaFunction{δD-from}\AgdaSpace{}%
\AgdaSymbol{(}\AgdaOperator{\AgdaInductiveConstructor{`}}\AgdaSpace{}%
\AgdaBound{c'}\AgdaSymbol{)}\AgdaSpace{}%
\AgdaSymbol{=}\AgdaSpace{}%
\AgdaFunction{◇.δ`}\AgdaSpace{}%
\AgdaSymbol{.}\AgdaField{from}\<%
\\
%
\>[4]\AgdaFunction{δD-from}\AgdaSpace{}%
\AgdaSymbol{(}\AgdaBound{D}\AgdaSpace{}%
\AgdaOperator{\AgdaInductiveConstructor{∪}}\AgdaSpace{}%
\AgdaBound{D₁}\AgdaSymbol{)}\AgdaSpace{}%
\AgdaSymbol{(}\AgdaInductiveConstructor{inj₁}\AgdaSpace{}%
\AgdaBound{x}\AgdaSymbol{)}\AgdaSpace{}%
\AgdaSymbol{=}\AgdaSpace{}%
\AgdaInductiveConstructor{inj₁}\AgdaSpace{}%
\AgdaSymbol{(}\AgdaFunction{δD-from}\AgdaSpace{}%
\AgdaBound{D}\AgdaSpace{}%
\AgdaBound{x}\AgdaSymbol{)}\<%
\\
%
\>[4]\AgdaFunction{δD-from}\AgdaSpace{}%
\AgdaSymbol{(}\AgdaBound{D}\AgdaSpace{}%
\AgdaOperator{\AgdaInductiveConstructor{∪}}\AgdaSpace{}%
\AgdaBound{D₁}\AgdaSymbol{)}\AgdaSpace{}%
\AgdaSymbol{(}\AgdaInductiveConstructor{inj₂}\AgdaSpace{}%
\AgdaBound{y}\AgdaSymbol{)}\AgdaSpace{}%
\AgdaSymbol{=}\AgdaSpace{}%
\AgdaInductiveConstructor{inj₂}\AgdaSpace{}%
\AgdaSymbol{(}\AgdaFunction{δD-from}\AgdaSpace{}%
\AgdaBound{D₁}\AgdaSpace{}%
\AgdaBound{y}\AgdaSymbol{)}\<%
\\
%
\>[4]\AgdaFunction{δD-from}\AgdaSpace{}%
\AgdaSymbol{(}\AgdaBound{D}\AgdaSpace{}%
\AgdaOperator{\AgdaInductiveConstructor{∗}}\AgdaSpace{}%
\AgdaBound{D₁}\AgdaSymbol{)}\AgdaSpace{}%
\AgdaSymbol{(}\AgdaInductiveConstructor{[]}\AgdaSpace{}%
\AgdaOperator{\AgdaInductiveConstructor{,}}\AgdaSpace{}%
\AgdaBound{w}\AgdaSpace{}%
\AgdaOperator{\AgdaInductiveConstructor{,}}\AgdaSpace{}%
\AgdaInductiveConstructor{refl}\AgdaSpace{}%
\AgdaOperator{\AgdaInductiveConstructor{,}}\AgdaSpace{}%
\AgdaBound{x}\AgdaSpace{}%
\AgdaOperator{\AgdaInductiveConstructor{,}}\AgdaSpace{}%
\AgdaBound{y}\AgdaSymbol{)}\AgdaSpace{}%
\AgdaSymbol{=}\AgdaSpace{}%
\AgdaInductiveConstructor{inj₁}\AgdaSpace{}%
\AgdaSymbol{(}\AgdaBound{x}\AgdaSpace{}%
\AgdaOperator{\AgdaInductiveConstructor{,}}\AgdaSpace{}%
\AgdaFunction{δD-from}\AgdaSpace{}%
\AgdaBound{D₁}\AgdaSpace{}%
\AgdaBound{y}\AgdaSymbol{)}\<%
\\
%
\>[4]\AgdaFunction{δD-from}\AgdaSpace{}%
\AgdaSymbol{(}\AgdaBound{D}\AgdaSpace{}%
\AgdaOperator{\AgdaInductiveConstructor{∗}}\AgdaSpace{}%
\AgdaBound{D₁}\AgdaSymbol{)}\AgdaSpace{}%
\AgdaSymbol{(}\AgdaBound{c}\AgdaSpace{}%
\AgdaOperator{\AgdaInductiveConstructor{∷}}\AgdaSpace{}%
\AgdaBound{u}\AgdaSpace{}%
\AgdaOperator{\AgdaInductiveConstructor{,}}\AgdaSpace{}%
\AgdaBound{v}\AgdaSpace{}%
\AgdaOperator{\AgdaInductiveConstructor{,}}\AgdaSpace{}%
\AgdaInductiveConstructor{refl}\AgdaSpace{}%
\AgdaOperator{\AgdaInductiveConstructor{,}}\AgdaSpace{}%
\AgdaBound{x}\AgdaSpace{}%
\AgdaOperator{\AgdaInductiveConstructor{,}}\AgdaSpace{}%
\AgdaBound{y}\AgdaSymbol{)}\AgdaSpace{}%
\AgdaSymbol{=}\AgdaSpace{}%
\AgdaInductiveConstructor{inj₂}\AgdaSpace{}%
\AgdaSymbol{(}\AgdaBound{u}\AgdaSpace{}%
\AgdaOperator{\AgdaInductiveConstructor{,}}\AgdaSpace{}%
\AgdaBound{v}\AgdaSpace{}%
\AgdaOperator{\AgdaInductiveConstructor{,}}\AgdaSpace{}%
\AgdaInductiveConstructor{refl}\AgdaSpace{}%
\AgdaOperator{\AgdaInductiveConstructor{,}}\AgdaSpace{}%
\AgdaFunction{δD-from}\AgdaSpace{}%
\AgdaBound{D}\AgdaSpace{}%
\AgdaBound{x}\AgdaSpace{}%
\AgdaOperator{\AgdaInductiveConstructor{,}}\AgdaSpace{}%
\AgdaFunction{subst}\AgdaSpace{}%
\AgdaFunction{id}\AgdaSpace{}%
\AgdaSymbol{(}\AgdaFunction{sym}\AgdaSpace{}%
\AgdaSymbol{(}\AgdaFunction{σμ}\AgdaSpace{}%
\AgdaBound{D₁}\AgdaSymbol{))}\AgdaSpace{}%
\AgdaBound{y}\AgdaSymbol{)}\<%
\\
%
\>[4]\AgdaFunction{δD-from}\AgdaSpace{}%
\AgdaSymbol{(}\AgdaBound{A}\AgdaSpace{}%
\AgdaOperator{\AgdaInductiveConstructor{·}}\AgdaSpace{}%
\AgdaBound{D}\AgdaSymbol{)}\AgdaSpace{}%
\AgdaSymbol{(}\AgdaBound{x}\AgdaSpace{}%
\AgdaOperator{\AgdaInductiveConstructor{,}}\AgdaSpace{}%
\AgdaBound{y}\AgdaSymbol{)}\AgdaSpace{}%
\AgdaSymbol{=}\AgdaSpace{}%
\AgdaBound{x}\AgdaSpace{}%
\AgdaOperator{\AgdaInductiveConstructor{,}}\AgdaSpace{}%
\AgdaFunction{δD-from}\AgdaSpace{}%
\AgdaBound{D}\AgdaSpace{}%
\AgdaBound{y}\<%
\\
%
\>[4]\AgdaFunction{δD-from}\AgdaSpace{}%
\AgdaSymbol{\{}\AgdaArgument{D₀}\AgdaSpace{}%
\AgdaSymbol{=}\AgdaSpace{}%
\AgdaBound{D}\AgdaSymbol{\}}\AgdaSpace{}%
\AgdaInductiveConstructor{var}\AgdaSpace{}%
\AgdaSymbol{(}\AgdaInductiveConstructor{roll}\AgdaSpace{}%
\AgdaBound{x}\AgdaSymbol{)}\AgdaSpace{}%
\AgdaSymbol{=}\AgdaSpace{}%
\AgdaInductiveConstructor{roll}\AgdaSpace{}%
\AgdaSymbol{(}\AgdaFunction{δD-from}\AgdaSpace{}%
\AgdaBound{D}\AgdaSpace{}%
\AgdaBound{x}\AgdaSymbol{)}\<%
\\
%
\>[4]\AgdaFunction{δD-from}\AgdaSpace{}%
\AgdaSymbol{(}\AgdaInductiveConstructor{μ}\AgdaSpace{}%
\AgdaBound{D}\AgdaSymbol{)}\AgdaSpace{}%
\AgdaSymbol{(}\AgdaInductiveConstructor{roll}\AgdaSpace{}%
\AgdaBound{x}\AgdaSymbol{)}\AgdaSpace{}%
\AgdaSymbol{=}\AgdaSpace{}%
\AgdaInductiveConstructor{roll}\AgdaSpace{}%
\AgdaSymbol{(}\AgdaFunction{δD-from}\AgdaSpace{}%
\AgdaBound{D}\AgdaSpace{}%
\AgdaBound{x}\AgdaSymbol{)}\<%
\end{code}
\begin{code}%
%
\>[4]\AgdaFunction{δD-correct}\AgdaSpace{}%
\AgdaSymbol{\{}\AgdaArgument{D}\AgdaSpace{}%
\AgdaSymbol{=}\AgdaSpace{}%
\AgdaBound{D}\AgdaSymbol{\}}\AgdaSpace{}%
\AgdaSymbol{=}\AgdaSpace{}%
\AgdaFunction{mk⇔}\AgdaSpace{}%
\AgdaSymbol{(}\AgdaInductiveConstructor{roll}\AgdaSpace{}%
\AgdaOperator{\AgdaFunction{∘}}\AgdaSpace{}%
\AgdaFunction{δD-to}\AgdaSpace{}%
\AgdaBound{D}\AgdaSpace{}%
\AgdaOperator{\AgdaFunction{∘}}\AgdaSpace{}%
\AgdaFunction{unroll}\AgdaSymbol{)}\AgdaSpace{}%
\AgdaSymbol{(}\AgdaInductiveConstructor{roll}\AgdaSpace{}%
\AgdaOperator{\AgdaFunction{∘}}\AgdaSpace{}%
\AgdaFunction{δD-from}\AgdaSpace{}%
\AgdaBound{D}\AgdaSpace{}%
\AgdaOperator{\AgdaFunction{∘}}\AgdaSpace{}%
\AgdaFunction{unroll}\AgdaSymbol{)}\<%
\end{code}

% \begin{code}
%     variable D D₀ : Desc
%     variable P : ◇.Lang
% 
%     νD : (D : Desc) → Dec (◇.ν ⟦ D ⟧)
%     νD D = {!   !}
% 
%     δD : ∀ c → (D : Desc) → Σ Desc (λ D' → ⟦ D' ⟧ₒ (◇.δ c ⟦ D ⟧) ◇.⟺ ◇.δ c (⟦ D ⟧ₒ ⟦ D ⟧))
%     δD _ ∅ = ∅ , ⇔.refl
%     δD _ ε = ∅ , ◇.δε
%     δD c (` c') = (c ≟ c') · ε , ◇.δ`
%     δD c (D ∪ D₁) = {!   !}
%     δD c (D ∗ D₁) = {!   !}
%     δD c (x · D) = {! !}
%     δD c var = var , ⇔.refl
% 
%     -- -- This does not work out. Consider the union case. Which split should it
%     -- -- report?  We have to choose one, but that excludes the other which could
%     -- -- be the one we actually want.
%     -- -- If the parser is required to parse the whole input, then it is
%     -- -- not possible to make the wrong choice.
%     -- Parser : ◇.Lang → Type
%     -- Parser P = (w : String) → Dec (∃[ u ] ∃[ v ] w ≡ u ++ v × P u)
% 
%     -- ⟦_⟧ₒ-parse : ∀ D → Parser (⟦ D ⟧ₒ P)
%     -- ⟦ ∅ ⟧ₒ-parse _ = no λ { (_ , _ , ()) }
%     -- ⟦ ε ⟧ₒ-parse w = yes ([] , w , refl , refl)
%     -- ⟦ ` c' ⟧ₒ-parse [] = no λ { ([] , [] , refl , ()) }
%     -- ⟦ ` c' ⟧ₒ-parse (c ∷ w) = Dec.map
%     --   (mk⇔ 
%     --     (λ { refl → c ∷ [] , w , refl , refl })
%     --     (λ { (.(c ∷ []) , .w , refl , refl) → refl }))
%     --   (c ≟ c')
%     -- ⟦ D ∪ D₁ ⟧ₒ-parse w = Dec.map {!   !} (⟦ D ⟧ₒ-parse w ×-dec ⟦ D₁ ⟧ₒ-parse w)
%     -- ⟦ D ∗ D₁ ⟧ₒ-parse = {!   !}
%     -- ⟦ x · D ⟧ₒ-parse = {!   !}
%     -- ⟦ var ⟧ₒ-parse = {!   !}
% 
%     -- ⟦_⟧ₒ-parse : ∀ D → ◇.Parser (⟦ D ⟧ₒ ⟦ D ⟧)
%     -- ⟦ ∅ ⟧ₒ-parse = ◇.∅-parse
%     -- ⟦ ε ⟧ₒ-parse = ◇.ε-parse
%     -- ⟦ ` c ⟧ₒ-parse = ◇.`-parse c
%     -- ⟦ D ∪ D₁ ⟧ₒ-parse = ⟦ D ⟧ₒ-parse ◇.∪-parse ⟦ D₁ ⟧ₒ-parse
%     -- ⟦ D ∗ D₁ ⟧ₒ-parse = ⟦ D ⟧ₒ-parse ◇.∗-parse ⟦ D₁ ⟧ₒ-parse
%     -- ⟦ x · D ⟧ₒ-parse = x ◇.·-parse ⟦ D ⟧ₒ-parse
%     -- ⟦ var ⟧ₒ-parse [] = {!   !}
%     -- ⟦ var ⟧ₒ-parse (c ∷ w) = ⟦ var ⟧ₒ-parse w
% 
%     -- ⟦_⟧-parse : ∀ D → ◇.Parser ⟦ D ⟧
%     -- ⟦ D ⟧-parse = Dec.map (mk⇔ roll (λ { (roll x) → x })) ∘ ⟦ D ⟧ₒ-parse
% \end{code}
% 
% % Practice for the indexed thing
% \begin{code}[hide]
%     open Equivalence using (to ; from)
% 
%     distrib : ∀{ℓ₁ ℓ₂ ℓ₃} {A : Type ℓ₁} {B : Type ℓ₂} {C : Type ℓ₃} 
%             → (A ⊎ C) × (B ⊎ C) → (A × B) ⊎ C
%     distrib (inj₁ x , inj₁ y) = inj₁ (x , y)
%     distrib (inj₁ _ , inj₂ y) = inj₂ y
%     distrib (inj₂ x , _) = inj₂ x
% 
%     ν⟦⟧ : ◇.ν (⟦ D ⟧ₒ ◇.∅) ⇔ ◇.ν ⟦ D ⟧
%     ν⟦⟧ {D = D} = mk⇔ (λ { x → roll (go→ {D₀ = D} D x) }) λ { (roll x) → reduce (go← {D₀ = D} D x) } where
% 
%       go→ : ∀ D → ◇.ν (⟦ D ⟧ₒ ◇.∅) → ◇.ν (⟦ D ⟧ₒ ⟦ D₀ ⟧)
%       go→ ε refl = refl
%       go→ (D ∪ D₁) (inj₁ x) = inj₁ (go→ D x)
%       go→ (D ∪ D₁) (inj₂ y) = inj₂ (go→ D₁ y)
%       go→ (D ∗ D₁) ([] , [] , refl , x , y) = [] , [] , refl , go→ D x , go→ D₁ y
%       go→ (A · D) (x , y) = x , go→ D y
% 
%       go← : ∀ D → ◇.ν (⟦ D ⟧ₒ ⟦ D₀ ⟧) → ◇.ν (⟦ D ⟧ₒ ◇.∅) ⊎ ◇.ν (⟦ D₀ ⟧ₒ ◇.∅)
%       go← ε refl = inj₁ refl
%       go← (D ∪ D₁) (inj₁ x) = Sum.map₁ inj₁ (go← D x)
%       go← (D ∪ D₁) (inj₂ y) = Sum.map₁ inj₂ (go← D₁ y)
%       go← (D ∗ D₁) ([] , [] , refl , x , y) = Sum.map₁ (λ x → [] , [] , refl , x) (distrib (go← D x , go← D₁ y))
%       go← (A · D) (x , y) = Sum.map₁ (x ,_) (go← D y)
%       go← {D₀ = D} var (roll x) = inj₂ (reduce (go← D x))
% 
%     -- there is not a simple way to define this
%     -- δ⟦⟧ : ? ◇.⟺ ◇.δ c ⟦ D ⟧
% \end{code}
% 
% However, we often want more than just one nonterminal, so we need to generalize to indexed descriptors.
% 
% \begin{code}[hide]
% module IDesc where
%     variable I : Type
% \end{code}
% 
% \begin{code}
%     data IDesc (I : Type) : Type₁ where
%         ∅    : IDesc I
%         ε    : IDesc I
%         `_   : Char → IDesc I
%         _∪_  : IDesc I → IDesc I → IDesc I
%         _∗_  : IDesc I → IDesc I → IDesc I
%         _·_  : Type → IDesc I → IDesc I
%         var  : I → IDesc I
% \end{code}
% 
% \begin{code}
%     ⟦_⟧ₒ : IDesc I → (I → ◇.Lang) → ◇.Lang
%     ⟦ ∅ ⟧ₒ        = const ◇.∅
%     ⟦ ε ⟧ₒ        = const ◇.ε
%     ⟦ ` c ⟧ₒ      = const (◇.` c) 
%     ⟦ X ∪ Y ⟧ₒ P  = ⟦ X ⟧ₒ P ◇.∪ ⟦ Y ⟧ₒ P
%     ⟦ X ∗ Y ⟧ₒ P  = ⟦ X ⟧ₒ P ◇.∗ ⟦ Y ⟧ₒ P
%     ⟦ x · X ⟧ₒ P  = x ◇.· ⟦ X ⟧ₒ P 
%     ⟦ var i ⟧ₒ P  = P i
% \end{code}
% 
% 
% 
% \begin{code}[hide]
%     data 𝟏+_ (I : Type) : Type where
%         here : 𝟏+ I
%         there : I → 𝟏+ I
% \end{code}
% 
% \begin{code}
%     variable D D₀ : IDesc I
%     variable Γ : I → ◇.Lang
%     cons : A → (I → A) → (𝟏+ I → A)
%     cons x Γ here = x
%     cons x Γ (there i) = Γ i
%     data ⟦_⟧ (D : IDesc (𝟏+ I)) (Γ : I → ◇.Lang) : ◇.Lang where
%         roll : ⟦ D ⟧ₒ (cons (⟦ D ⟧ Γ) Γ) w → ⟦ D ⟧ Γ w
% \end{code}
% 
% \begin{code}[hide]
%     distrib : ∀{ℓ₁ ℓ₂ ℓ₃} {A : Type ℓ₁} {B : Type ℓ₂} {C : Type ℓ₃} 
%             → (A ⊎ C) × (B ⊎ C) → (A × B) ⊎ C
%     distrib (inj₁ x , inj₁ y) = inj₁ (x , y)
%     distrib (inj₁ _ , inj₂ y) = inj₂ y
%     distrib (inj₂ x , _) = inj₂ x
% 
%     ν⟦⟧ : ◇.ν (⟦ D ⟧ₒ (cons ◇.∅ Γ)) ⇔ ◇.ν (⟦ D ⟧ Γ)
%     ν⟦⟧ {D = D} = mk⇔ (λ { x → roll (go→ {D₀ = D} D x) }) λ { (roll x) → reduce (go← {D₀ = D} D x) } where
% 
%       go→ : ∀ D → ◇.ν (⟦ D ⟧ₒ (cons ◇.∅ Γ)) → ◇.ν (⟦ D ⟧ₒ (cons (⟦ D₀ ⟧ Γ) Γ))
%       go→ ε refl = refl
%       go→ (D ∪ D₁) (inj₁ x) = inj₁ (go→ D x)
%       go→ (D ∪ D₁) (inj₂ y) = inj₂ (go→ D₁ y)
%       go→ (D ∗ D₁) ([] , [] , refl , x , y) = [] , [] , refl , go→ D x , go→ D₁ y
%       go→ (A · D) (x , y) = x , go→ D y
%       go→ (var (there i)) x = x
% 
%       go← : ∀ D → ◇.ν (⟦ D ⟧ₒ (cons (⟦ D₀ ⟧ Γ) Γ)) → ◇.ν (⟦ D ⟧ₒ (cons ◇.∅ Γ)) ⊎ ◇.ν (⟦ D₀ ⟧ₒ (cons ◇.∅ Γ))
%       go← ε refl = inj₁ refl
%       go← (D ∪ D₁) (inj₁ x) = Sum.map₁ inj₁ (go← D x)
%       go← (D ∪ D₁) (inj₂ y) = Sum.map₁ inj₂ (go← D₁ y)
%       go← (D ∗ D₁) ([] , [] , refl , x , y) = Sum.map₁ (λ x → [] , [] , refl , x) (distrib (go← D x , go← D₁ y))
%       go← (A · D) (x , y) = Sum.map₁ (x ,_) (go← D y)
%       go← {D₀ = D} (var here) (roll x) = inj₂ (reduce (go← D x))
%       go← {D₀ = D} (var (there i)) x = inj₁ x
% 
%     data Δ (I : Type) : Type where
%         normal : I → Δ I
%         delta : I → Δ I
% 
%     δ_⟦_⟧ₒ : Char → IDesc I → (Δ I → ◇.Lang) → ◇.Lang
%     δ c ⟦ ∅ ⟧ₒ        = const ◇.∅
%     δ c ⟦ ε ⟧ₒ        = const ◇.∅
%     δ c ⟦ ` c' ⟧ₒ     = const ((c ≡ c') ◇.· ◇.ε)
%     (δ c ⟦ X ∪ Y ⟧ₒ) P  = δ c ⟦ X ⟧ₒ P ◇.∪ δ c ⟦ Y ⟧ₒ P
%     (δ c ⟦ X ∗ Y ⟧ₒ) P  = ◇.ν (⟦ X ⟧ₒ (P ∘ normal)) ◇.· (δ c ⟦ Y ⟧ₒ) P ◇.∪ (δ c ⟦ X ⟧ₒ) P ◇.∗ ⟦ Y ⟧ₒ (P ∘ normal)
%     (δ c ⟦ x · X ⟧ₒ) P  = x ◇.· δ c ⟦ X ⟧ₒ P 
%     (δ c ⟦ var i ⟧ₒ) P  = P (delta i)
% 
%     withΔ : Char → (I → ◇.Lang) → (Δ I → ◇.Lang)
%     withΔ _ Γ (normal i) = Γ i
%     withΔ c Γ (delta i) = ◇.δ c (Γ i)
% 
%     δ⟦⟧ : {I : Type} {Γ : I → ◇.Lang} {D : IDesc (𝟏+ I)} {c : Char}
%         → (δ c ⟦ D ⟧ₒ) (withΔ c (cons (⟦ D ⟧ Γ) Γ)) ◇.⟺ ◇.δ c (⟦ D ⟧ Γ)
%     δ⟦⟧ = {!   !}
% \end{code}
% 
% 
% % \subsection{Syntax}
% % 
% % \begin{code}
% % data Exp : Type₁ where
% %     ∅ : Exp
% %     ε : Exp
% %     `_ : (c : Char) → Exp
% %     _·_ : {a : Type} → Dec a → Exp → Exp
% %     _∪_ : Exp → Exp → Exp
% %     _*_ : Exp → Exp → Exp
% %     i : Exp
% %     μ : Exp → Exp -- explain later
% % \end{code}
% % \begin{code}[hide]
% % infix 22 `_
% % infixr 21 _*_
% % infixr 20 _∪_
% % 
% % variable
% %     n m : ℕ
% %     l : Lang
% %     e e₀ : Exp
% % \end{code}
% % 
% % Mapping syntax onto semantics:
% % 
% % \begin{code}
% % ⟦_⟧₁ : Exp → Lang → Lang
% % \end{code}
% % \begin{code}
% % data ⟦_⟧ (e : Exp) : Lang where
% %     ∞ : ⟦ e ⟧₁ ⟦ e ⟧ w → ⟦ e ⟧ w
% % ! : ⟦ e ⟧ w → ⟦ e ⟧₁ ⟦ e ⟧ w
% % ! (∞ x) = x
% % \end{code}
% % \begin{code}
% % ⟦ ∅ ⟧₁ _ = ◇.∅
% % ⟦ ε ⟧₁ _ = ◇.ε
% % ⟦ ` c ⟧₁ _ = ◇.` c
% % ⟦ x · e ⟧₁ l = x ◇.· ⟦ e ⟧₁ l
% % ⟦ e ∪ e₁ ⟧₁ l = ⟦ e ⟧₁ l ◇.∪ ⟦ e₁ ⟧₁ l
% % ⟦ e * e₁ ⟧₁ l = ⟦ e ⟧₁ l ◇.* ⟦ e₁ ⟧₁ l
% % ⟦ i ⟧₁ l = l
% % ⟦ μ e ⟧₁ _ = ⟦ e ⟧ -- explain this later
% % \end{code}
% % 
% % \subsection{Goal}
% % 
% % Our goal is to define:
% % 
% % \begin{code}
% % parse : (e : Exp) (w : String) → Dec (⟦ e ⟧ w)
% % \end{code}
% % 
% % Our approach uses the decomposition of languages into $\af{ν}$ and $\af{δ}$.
% % \jr{Now we should explain the $\af{◇ν}$ and $\af{◇δ}$}
% % 
% % \begin{code}
% % ν : (e : Exp) → Dec (◆.◇ν ⟦ e ⟧)
% % δ : Char → Exp → Exp
% % \end{code}
% % 
% % The ν function can easily be written to be correct by construction, however δ must be proven correct separately as follows:
% % 
% % \begin{code}
% % δ-sound    : ⟦ δ c e ⟧ w   → ◆.◇δ c ⟦ e ⟧ w
% % δ-complete : ◆.◇δ c ⟦ e ⟧ w → ⟦ δ c e ⟧ w
% % \end{code}
% % 
% % The actual parsing follows the $\af{ν∘foldlδ}$ decomposition.
% % 
% % \begin{code}[hide]
% % map' : ∀{A B} → (A → B) → (B → A) → Dec A → Dec B
% % map' = map′
% % \end{code}
% % \begin{code}
% % parse e [] = ν e
% % parse e (c ∷ w) = map' δ-sound δ-complete (parse (δ c e) w)
% % \end{code}
% % 
% % That is the main result of this paper. The remainder of the paper concerns
% % the implementation of $\af{ν}$, $af{δ}$, $\af{δ-sound}$, and $\af{δ-commplete}$.
% % 
% % \subsection{Nullability correctness}
% % 
% % \begin{lemma}\label{lem:null-sub}
% % nullability of e substituted in e is the same as the
% % nullability of e by itself
% % \begin{code}
% % νe∅→νee : (e : Exp) → ◆.◇ν (⟦ e ⟧₁ ◇.∅) → ◆.◇ν (⟦ e ⟧₁ ⟦ e₀ ⟧) -- more general than we need, but easy
% % νee→νe∅ : (e : Exp) → ◆.◇ν (⟦ e ⟧₁ ⟦ e ⟧) → ◆.◇ν (⟦ e ⟧₁ ◇.∅)
% % \end{code}
% % \end{lemma}
% % 
% % Syntactic nullability (correct by construction):
% % 
% % \begin{code}
% % ν₁ : (e : Exp) → Dec (◆.◇ν (⟦ e ⟧₁ ◇.∅))
% % ν₁ ∅ = no λ ()
% % ν₁ ε = yes refl
% % ν₁ (` c) = no λ ()
% % ν₁ (x · e) = x ×-dec ν₁ e
% % ν₁ (e ∪ e₁) = ν₁ e ⊎-dec ν₁ e₁
% % ν₁ (e * e₁) = map' (λ x → [] , [] , refl , x) (λ { ([] , [] , refl , x) → x }) (ν₁ e ×-dec ν₁ e₁)
% % ν₁ i = no λ ()
% % ν₁ (μ e) = map' (∞ ∘ νe∅→νee e) (νee→νe∅ e ∘ !) (ν₁ e)
% % \end{code}
% % 
% % Using \cref{lem:null-sub} we can define $\af{ν}$ in terms of $\af{ν₁}$:
% % \begin{code}
% % ν e = map' (∞ ∘ νe∅→νee e) (νee→νe∅ e ∘ !) (ν₁ e)
% % \end{code}
% % 
% % \jr{TODO: show how ν works through examples}
% % 
% % The forward direction is proven using straightforward induction.
% % 
% % \begin{code}
% % νe∅→νee ε x = x
% % νe∅→νee (x₁ · e) (x , y) = x , νe∅→νee e y
% % νe∅→νee (e ∪ e₁) (inj₁ x) = inj₁ (νe∅→νee e x)
% % νe∅→νee (e ∪ e₁) (inj₂ y) = inj₂ (νe∅→νee e₁ y)
% % νe∅→νee (e * e₁) ([] , [] , refl , x , y) = [] , [] , refl , νe∅→νee e x , νe∅→νee e₁ y
% % νe∅→νee i ()
% % νe∅→νee (μ e) x = x
% % \end{code}
% % 
% % The backwards direction requires a bit more work. We use the
% % following lemma:
% % 
% % \begin{lemma}\label{lem:null-split}
% % If substituting e₀ into e is nullable then that must mean:
% % \begin{enumerate}
% % \item e  by itself was already nullable, or
% % \item e₀ by itself is nullable
% % \end{enumerate}
% % 
% % Proof:
% % 
% % \begin{code}
% % ν-split : (e : Exp) → ◆.◇ν (⟦ e ⟧₁ ⟦ e₀ ⟧) → ◆.◇ν (⟦ e ⟧₁ ◇.∅) ⊎ ◆.◇ν (⟦ e₀ ⟧₁ ◇.∅)
% % ν-split ε x = inj₁ x
% % ν-split (_ · e) (x , y) = Sum.map₁ (x ,_) (ν-split e y)
% % ν-split (e ∪ e₁) (inj₁ x) = Sum.map₁ inj₁ (ν-split e x)
% % ν-split (e ∪ e₁) (inj₂ y) = Sum.map₁ inj₂ (ν-split e₁ y)
% % ν-split (e * e₁) ([] , [] , refl , x , y) = lift⊎₂ (λ x y → [] , [] , refl , x , y) (ν-split e x) (ν-split e₁ y)
% % ν-split {e₀ = e} i (∞ x) = inj₂ (reduce (ν-split e x))
% % ν-split (μ e) x = inj₁ x
% % \end{code}
% % \end{lemma}
% % 
% % The backwards direction of \cref{lem:null-sub} is now simply a result of
% % \cref{lem:null-split} where both sides of the disjoint union are equal and thus
% % we can reduce it to a single value.
% % 
% % \begin{code}
% % νee→νe∅ e x = reduce (ν-split {e₀ = e} e x)
% % \end{code}
% % 
% % \subsection{Derivative correctness}
% % 
% % \jr{At this point (specifically the $\af{\un{}*\un{}}$ case of $\af{δ₁}$) we need to introduce $\ac{μ}$}
% % 
% % Internal/syntactic substitution:
% % 
% % \begin{code}
% % sub : Exp → Exp → Exp
% % sub _ ∅ = ∅
% % sub _ ε = ε
% % sub _ (` c) = ` c
% % sub e₀ (x · e) = x · sub e₀ e
% % sub e₀ (e ∪ e₁) = sub e₀ e ∪ sub e₀ e₁
% % sub e₀ (e * e₁) = sub e₀ e * sub e₀ e₁
% % sub e₀ i = e₀
% % sub _ (μ e) = μ e
% % \end{code}
% % 
% % We would like to be able to say \verb|⟦ sub e₀ e ⟧ ≡ ⟦ e ⟧₁ ⟦ e₀ ⟧\verb|, but
% % we can't because $\ab{e₀}$'s free variable would get (implicitly)
% % captured. $\ac{μ}$ closes off an expression and thus prevents capture.
% % 
% % \begin{lemma}\label{lem:sub-sem}
% % (Internal) syntactic substitution is the same as
% % (external) semantic substitution. This is the raison d'être of μ.
% % 
% % Proof:
% % 
% % \begin{code}
% % sub-sem' : (e : Exp) → ⟦ sub (μ e₀) e ⟧₁ l ≡ ⟦ e ⟧₁ ⟦ e₀ ⟧
% % sub-sem' ∅ = refl
% % sub-sem' ε = refl
% % sub-sem' (` _) = refl
% % sub-sem' (x · e) = cong (x ◇.·_) (sub-sem' e) 
% % sub-sem' (e ∪ e₁) = cong₂ ◇._∪_ (sub-sem' e) (sub-sem' e₁)
% % sub-sem' (e * e₁) = cong₂ ◇._*_ (sub-sem' e) (sub-sem' e₁)
% % sub-sem' i = refl
% % sub-sem' (μ _) = refl
% % \end{code}
% % 
% % We only need to use this proof in its expanded form:
% % 
% % \begin{code}
% % sub-sem : (e : Exp) → ⟦ sub (μ e₀) e ⟧₁ l w ≡ ⟦ e ⟧₁ ⟦ e₀ ⟧ w
% % sub-sem e = cong (λ l → l _) (sub-sem' e)
% % \end{code}
% % \end{lemma}
% % 
% % This is the syntactic derivative (the $\ab{e₀}$ argument stands for the whole expression):
% % 
% % \begin{code}
% % δ₁ : (c : Char) → Exp → Exp → Exp
% % δ₁ c _ ∅ = ∅
% % δ₁ c _ ε = ∅
% % δ₁ c _ (` c₁) = (c ≟ c₁) · ε
% % δ₁ c e₀ (x · e) = x · δ₁ c e₀ e
% % δ₁ c e₀ (e ∪ e₁) = δ₁ c e₀ e ∪ δ₁ c e₀ e₁
% % δ₁ c e₀ (e * e₁) = (δ₁ c e₀ e * sub (μ e₀) e₁) ∪ (Dec.map (⇔.trans (mk⇔ ! ∞) (≡→⇔ (sub-sem e))) (ν (sub (μ e₀) e)) · δ₁ c e₀ e₁)
% % δ₁ c e₀ i = i
% % δ₁ c _ (μ e) = μ (δ₁ c e e)
% % \end{code}
% % 
% % For a top-level expression the derivative is just the open $\af{δ₁}$ where $\ab{e₀}$ is $\ab{e}$ itself:
% % 
% % \begin{code}
% % δ c e = δ₁ c e e
% % \end{code}
% % 
% % \jr{todo: show how δ works through examples.}
% % 
% % The proofs are by induction and the \cref{lem:sub-sem}:
% % 
% % \begin{code}
% % δ-sound' : (e : Exp) → ⟦ δ₁ c e₀ e ⟧₁ ⟦ δ c e₀ ⟧ w → ◆.◇δ c (⟦ e ⟧₁ ⟦ e₀ ⟧) w
% % δ-sound' (` _) (refl , refl) = refl
% % δ-sound' (x₁ · e) (x , y) = x , δ-sound' e y
% % δ-sound' (e ∪ e₁) (inj₁ x) = inj₁ (δ-sound' e x)
% % δ-sound' (e ∪ e₁) (inj₂ y) = inj₂ (δ-sound' e₁ y)
% % δ-sound' {c = c} (e * e₁) (inj₁ (u , v , refl , x , y)) = c ∷ u , v , refl , δ-sound' e x , transport (sub-sem e₁) y
% % δ-sound' {c = c} {w = w} (e * e₁) (inj₂ (x , y)) = [] , c ∷ w , refl , x , δ-sound' e₁ y
% % δ-sound' {e₀ = e} i (∞ x) = ∞ (δ-sound' e x)
% % δ-sound' (μ e) (∞ x) = ∞ (δ-sound' e x)
% % \end{code}
% % 
% % \begin{code}
% % δ-sound {e = e} (∞ x) = ∞ (δ-sound' e x)
% % \end{code}
% % 
% % \begin{code}
% % δ-complete' : (e : Exp) → ◆.◇δ c (⟦ e ⟧₁ ⟦ e₀ ⟧) w → ⟦ δ₁ c e₀ e ⟧₁ ⟦ δ c e₀ ⟧ w
% % δ-complete' (` _) refl = refl , refl
% % δ-complete' (_ · e) (x , y) = x , δ-complete' e y
% % δ-complete' (e ∪ e₁) (inj₁ x) = inj₁ (δ-complete' e x)
% % δ-complete' (e ∪ e₁) (inj₂ y) = inj₂ (δ-complete' e₁ y)
% % δ-complete' (e * e₁) (c ∷ u , v , refl , x , y) = inj₁ (u , v , refl , δ-complete' e x , transport (sym (sub-sem e₁)) y)
% % δ-complete' (e * e₁) ([] , c ∷ w , refl , x , y) = inj₂ (x , δ-complete' e₁ y)
% % δ-complete' {e₀ = e} i (∞ x) = ∞ (δ-complete' e x)
% % δ-complete' (μ e) (∞ x) = ∞ (δ-complete' e x)
% % \end{code}
% % 
% % \begin{code}
% % δ-complete {e = e} (∞ x) = ∞ (δ-complete' e x)
% % \end{code}
% % 
% % That's the end of the proof.
% % 
% % 
% % % \begin{code}[hide]
% % % variable V V₁ V₂ V' : Set
% % % variable k k' n m : ℕ
% % % 
% % % data Fin : ℕ → Set where
% % %     zero : Fin (suc n)
% % %     suc : Fin n → Fin (suc n)
% % % 
% % % ∃-syntax : {A : Set} → (A → Set) → Set
% % % ∃-syntax {A} B = Σ A B
% % % 
% % % syntax ∃-syntax (λ x → A) = ∃[ x ] A
% % % \end{code}
% % % 
% % % Regular languages can be useful for describing patterns in program text, but they are not sufficient to model the full language of a programming language.
% % % For example, balanced brackets are a common syntactic element in programming languages. 
% % % 
% % % \begin{example}
% % % We can boil the problem down to the following language which consists only of balanced brackets:
% % % 
% % % \begin{code}
% % % bracketsₖ : ℕ → Lang
% % % bracketsₖ zero _ = ⊥
% % % bracketsₖ (suc k) w  = (w ≡ [])
% % %                      ⊎ (∃[ u ] (w ≡ `[ ∷ [] ++ u ++ `] ∷ []) × bracketsₖ k u)
% % %                      ⊎ (∃[ u ] ∃[ v ] (w ≡ u ++ v) × bracketsₖ k u × bracketsₖ k v)
% % % \end{code}
% % % \begin{code}
% % % brackets : Lang
% % % brackets w = ∃[ k ] bracketsₖ k w
% % % \end{code}
% % % 
% % % \begin{remark}\label{rem:truncation}
% % % The \af{bracketsₖ} function is truncated after \ab{k} recursive calls to ensure termination, which is required for all functions in Type theory. The proper language \af{brackets} asserts that, for a string to be in the language, there must exist a \ab{k} which is large enough that the truncation becomes irrelevant for that particular string.
% % % \end{remark}
% % % \end{example}
% % % 
% % % \subsection{Context-free Grammars}
% % % 
% % % This language of balanced brackets is famously context-free. To support languages such as these we add variables, \ac{var}, and fixed points, \ac{μ}, to our grammars.
% % % \begin{code}
% % % data Gram (n : ℕ) : Set₁ where
% % %     ∅ ε : Gram n
% % %     char : Char → Gram n
% % %     _·_ : Dec A → Gram n → Gram n
% % %     _∪_ _∗_ : Gram n → Gram n → Gram n
% % %     var : Fin n → Gram n
% % %     μ : Gram (suc n) → Gram n
% % % \end{code}
% % % \begin{code}[hide]
% % % infixr 21 _∗_
% % % infixr 20 _∪_
% % % \end{code}
% % % 
% % % % TODO: this probably needs more explanation
% % % 
% % % \begin{code}[hide]
% % % variable G G₁ G₂ : Gram n
% % % variable Γ Γ' : Fin n → Lang
% % % 
% % % _∷>_ : {ℓ : Level} {A : Fin (suc n) → Set ℓ} → A zero → ((i : Fin n) → A (suc i)) → ((i : Fin (suc n)) → A i)
% % % (x ∷> xs) zero = x
% % % (x ∷> xs) (suc i) = xs i
% % % \end{code}
% % % 
% % % \begin{code}
% % % ⟦_⟧ₖ : Gram n → (Fin n → Lang) → ℕ → Lang
% % % \end{code}
% % % \begin{code}[hide]
% % % ⟦ ∅ ⟧ₖ Γ _ _ = ⊥
% % % ⟦ ε ⟧ₖ Γ _ w = w ≡ []
% % % ⟦ x · G ⟧ₖ Γ k w = ⌊ x ⌋ × ⟦ G ⟧ₖ Γ k w
% % % ⟦ G₁ ∪ G₂ ⟧ₖ Γ k w = ⟦ G₁ ⟧ₖ Γ k w ⊎ ⟦ G₂ ⟧ₖ Γ k w
% % % ⟦ G₁ ∗ G₂ ⟧ₖ Γ k w = ∃[ u ] ∃[ v ] (w ≡ (u ++ v)) × ⟦ G₁ ⟧ₖ Γ k u × ⟦ G₂ ⟧ₖ Γ k v
% % % ⟦ char x ⟧ₖ Γ _ w = w ≡ (x ∷ [])
% % % \end{code}
% % % \begin{code}
% % % ⟦ var i ⟧ₖ Γ k w = Γ i w
% % % ⟦ μ G ⟧ₖ Γ zero w = ⊥
% % % ⟦ μ G ⟧ₖ Γ (suc k) w = ⟦ G ⟧ₖ (⟦ μ G ⟧ₖ Γ k ∷> Γ) k w
% % % \end{code}
% % % \begin{code}
% % % ⟦_⟧ : Gram n → (Fin n → Lang) → Lang
% % % ⟦ G ⟧ Γ w = ∃[ k ] ⟦ G ⟧ₖ Γ k w
% % % \end{code}
% % % 
% % % \begin{example}
% % % This allows us to write a grammar for the language of balanced brackets.
% % % \begin{code}
% % % bracketsG : Gram n
% % % bracketsG = μ (ε ∪ char `[ ∗ var zero ∗ char `] ∪ var zero ∗ var zero)
% % % \end{code}
% % % \end{example}
% % % 
% % % \begin{lemma}
% % % We can map over context and the fuel of the truncated semantics.
% % % \begin{code}[hide]
% % % max : ℕ → ℕ → ℕ
% % % max zero k' = k'
% % % max (suc k) zero = suc k
% % % max (suc k) (suc k') = suc (max k k')
% % % 
% % % data _≤_ : ℕ → ℕ → Set where
% % %     z≤m : zero ≤ m
% % %     s≤s : n ≤ m → suc n ≤ suc m
% % % 
% % % ≤refl : n ≤ n
% % % ≤refl {n = zero} = z≤m
% % % ≤refl {n = suc n} = s≤s ≤refl
% % % 
% % % n≤maxnm : (n m : ℕ) → n ≤ max n m
% % % n≤maxnm zero m = z≤m
% % % n≤maxnm (suc n) zero = ≤refl
% % % n≤maxnm (suc n) (suc m) = s≤s (n≤maxnm n m)
% % % 
% % % m≤maxnm : (n m : ℕ) → m ≤ max n m
% % % m≤maxnm n zero = z≤m
% % % m≤maxnm zero (suc m) = ≤refl
% % % m≤maxnm (suc n) (suc m) = s≤s (m≤maxnm n m)
% % % 
% % % \end{code}
% % % \begin{code}
% % % mapΓ  : (G : Gram n) (Γ Γ' : Fin n → Lang) 
% % %       → ((i : Fin n) → {w : String} → Γ i w → Γ' i w)
% % %       → ⟦ G ⟧ₖ Γ k w → ⟦ G ⟧ₖ Γ' k w
% % % \end{code}
% % % \begin{code}[hide]
% % % mapΓ ε Γ Γ' f x = x
% % % mapΓ (char x₁) Γ Γ' f x = x
% % % mapΓ (x₁ · G) Γ Γ' f (x , y) = x , mapΓ G Γ Γ' f y
% % % mapΓ (G₁ ∪ G₂) Γ Γ' f (inl x) = inl (mapΓ G₁ Γ Γ' f x)
% % % mapΓ (G₁ ∪ G₂) Γ Γ' f (inr x) = inr (mapΓ G₂ Γ Γ' f x)
% % % mapΓ (G₁ ∗ G₂) Γ Γ' f (u , v , refl , x , y) = u , v , refl , mapΓ G₁ Γ Γ' f x , mapΓ G₂ Γ Γ' f y
% % % mapΓ (var i) Γ Γ' f x = f i x
% % % mapΓ {k = suc k} (μ G) Γ Γ' f x = mapΓ G _ _ (λ { zero → mapΓ {k = k} (μ G) Γ Γ' f ; (suc i) → f i }) x
% % % 
% % % \end{code}
% % % \begin{code}
% % % mapk : k ≤ k' → ⟦ G ⟧ₖ Γ k w → ⟦ G ⟧ₖ Γ k' w
% % % \end{code}
% % % \begin{code}[hide]
% % % mapk {G = ε} k≤k' x = x
% % % mapk {G = char x₁} k≤k' x = x
% % % mapk {G = x₁ · G} k≤k' (x , y) = x , mapk {G = G} k≤k' y
% % % mapk {G = G₁ ∪ G₂} k≤k' (inl x) = inl (mapk {G = G₁} k≤k' x)
% % % mapk {G = G₁ ∪ G₂} k≤k' (inr x) = inr (mapk {G = G₂} k≤k' x)
% % % mapk {G = G₁ ∗ G₂} k≤k' (u , v , refl , x , y) = u , v , refl , mapk {G = G₁} k≤k' x , mapk {G = G₂} k≤k' y
% % % mapk {G = var i} k≤k' x = x
% % % mapk {G = μ G} (s≤s k≤k') x = mapk {G = G} k≤k' (mapΓ G _ _ (λ { zero → mapk {G = μ G} k≤k' ; (suc i) → λ z → z}) x)
% % % 
% % % weakenˡ : ⟦ G ⟧ₖ Γ k w → ⟦ G ⟧ₖ Γ (max k k') w
% % % weakenˡ {G = G} {k = k} {k' = k'} = mapk {G = G} (n≤maxnm k k')
% % % 
% % % weakenʳ : ⟦ G ⟧ₖ Γ k' w → ⟦ G ⟧ₖ Γ (max k k') w
% % % weakenʳ {G = G} {k' = k'} {k = k} = mapk {G = G} (m≤maxnm k k')
% % % \end{code}
% % % \end{lemma}
% % % 
% % % \begin{lemma}
% % % We can map a change of variables over a grammar and we can substitute variables. This essentially shows that grammars form a relative monad.
% % % \begin{code}
% % % rename : (Fin n → Fin m) → Gram n → Gram m
% % % \end{code}
% % % \begin{code}[hide]
% % % rename _ ∅ = ∅
% % % rename _ ε = ε
% % % rename _ (char c) = char c
% % % rename f (x · G) = x · rename f G
% % % rename f (G₁ ∪ G₂) = rename f G₁ ∪ rename f G₂
% % % rename f (G₁ ∗ G₂) = rename f G₁ ∗ rename f G₂
% % % rename f (var i) = var (f i)
% % % rename f (μ G) = μ (rename (λ { zero → zero ; (suc i) → suc (f i) }) G)
% % % \end{code}
% % % \begin{code}
% % % subst : Gram n → (Fin n → Gram m) → Gram m
% % % \end{code}
% % % \begin{code}[hide]
% % % subst ∅ σ = ∅
% % % subst ε σ = ε
% % % subst (char c) σ = char c
% % % subst (x · G) σ = x · subst G σ
% % % subst (G ∪ G₁) σ = subst G σ ∪ subst G₁ σ
% % % subst (G ∗ G₁) σ = subst G σ ∗ subst G₁ σ
% % % subst (var x) σ = σ x
% % % subst (μ G) σ = μ (subst G λ { zero → var zero ; (suc i) → rename suc (σ i) })
% % % \end{code}
% % % \end{lemma}
% % % 
% % % \subsection{Parsing}\label{sec:cfg-parsing}
% % % 
% % % Parsing our context-free grammar follows the same structure as the simple grammars from \cref{sec:gram-and-parsing}. Concretely, we define functions that compute the nullability, \af{ν?}, and derivatives, \af{δ?}. For this section we have taken inspiration from a blog post by Grenrus~\cite{fix-ing-regular-expressions}.
% % % 
% % % \begin{example}\label{ex:cfg-parsing}
% % % Let us consider the balanced bracket grammar example. We can see that it is nullable because it contains an \ac{ε} in the fixed point. It is also possible to parse the empty string by taking one iteration of the fixed point using the \ac{var}~\ac{zero}~∗~\ac{var}~\ac{zero} part and then the \ac{ε} for both recursive calls, but note that we always need to end in an empty base case. Thus, for a fixed point to be nullable, it must be nullable even if we do not consider the recursive calls.
% % % 
% % % The derivative of the balanced bracket grammar can be taken with respect to any character, but only the character \ac{`[} results in anything interesting because any string in the balanced bracket language needs to start with an opening bracket. The first thing we might try is to unroll the fixed point one roll, yielding the following grammar:
% % % \begin{code}
% % % bracketsG₁ : Gram n
% % % bracketsG₁ = ε ∪ char `[ ∗ bracketsG ∗ char `] ∪ bracketsG ∗ bracketsG
% % % \end{code}
% % % We know how to take the derivative of the first two parts, but \af{bracketsG}~\ac{∗}~\af{bracketsG} seems problematic because its derivative depends on the derivative of \af{bracketsG} itself. Luckily, we can introduce a new fixed point when describing the derivative to refer to the derivative itself.
% % % \begin{code}
% % % bracketsG' : Gram n
% % % bracketsG' = μ (bracketsG ∗ char `] ∪ var zero ∗ bracketsG)
% % % \end{code}
% % % \end{example}
% % % 
% % % \subsubsection{Nullability}
% % % 
% % % Computing the nullability now requires us to deal with grammars that contain free variables, but we can make use of a context \ab{Γν} which tells us how to compute the nullability of those variables.
% % % 
% % % \begin{code}
% % % ν? : (G : Gram n) (Γν : (i : Fin n) → Dec (ν (Γ i))) → Dec (ν (⟦ G ⟧ Γ))
% % % \end{code}
% % % The simple cases remain the same except that \ab{Γν} now has to be passed properly to recursive calls. We skip to the two new cases: variables and fixed points.
% % % \begin{code}[hide]
% % % ν▹ : (ν (⟦ G₁ ⟧ Γ) × ν (⟦ G₂ ⟧ Γ)) ↔ ν (⟦ G₁ ∗ G₂ ⟧ Γ)
% % % to (ν▹ {G₁ = G₁} {G₂ = G₂}) ((n , x) , (m , y)) = max n m , [] , [] , refl , weakenˡ {G = G₁} x , weakenʳ {G = G₂} y
% % % from ν▹ (n , [] , [] , refl , x , y) = (n , x) , (n , y)
% % % 
% % % -- refold : (G : Gram (suc n)) → ⟦ G ⟧ (⟦ μ G ⟧ Γ ∷> Γ) ⇔ ⟦ μ G ⟧ Γ
% % % -- to (refold G) x = {!!}
% % % -- from (refold G) (suc k , x) = k , mapΓ G _ _ (λ { zero → k ,_ ; (suc i) → λ z → z }) x
% % % n≤sucn : n ≤ suc n
% % % n≤sucn {zero} = z≤m
% % % n≤sucn {suc n} = s≤s n≤sucn
% % % 
% % % variable i : Fin n
% % % \end{code}
% % % For both cases we need a helper. In the case of variables this helper just deals with converting between the truncated semantics and the proper semantics.
% % % \begin{code}
% % % νΓi↔ν⟦vari⟧Γ : ν (Γ i) ↔ ν (⟦ var i ⟧ Γ)
% % % to νΓi↔ν⟦vari⟧Γ x = zero , x
% % % from νΓi↔ν⟦vari⟧Γ (_ , x) = x
% % % \end{code}
% % % For the fixed point, we need to formalize the intuition from \cref{ex:cfg-parsing}. Recall that we noted how determining the nullability of a fixed point only requires unrolling it once and no more.
% % % \begin{code}
% % % νG⊥↔νμG  : ν (⟦ G ⟧ ((λ _ → ⊥) ∷> Γ)) ↔ ν (⟦ μ G ⟧ Γ)
% % % \end{code}
% % % We are still working on a proof of this property, but we have been able to reduce it to the following postulate which states that, if a grammar with free variables is nullable, either the nullability is independent of that variable, or that variable itself needs to be nullable.
% % % \begin{code}
% % % postulate νGℒ→νG⊥⊎νℒ  : ν (⟦ G ⟧ₖ (ℒ ∷> Γ) k) → ν (⟦ G ⟧ₖ ((λ _ → ⊥) ∷> Γ) k) ⊎ ν ℒ
% % % \end{code}
% % % \begin{code}[hide]
% % % νGμG→νG⊥  : ν (⟦ G ⟧ₖ (⟦ μ G ⟧ₖ Γ k ∷> Γ) k) → ν (⟦ G ⟧ₖ ((λ _ → ⊥) ∷> Γ) k)
% % % νGμG→νG⊥ {G = G} x with νGℒ→νG⊥⊎νℒ {G = G} x
% % % ... | inl x = x
% % % νGμG→νG⊥ {G = G} {k = suc k} _ | inr x = mapk {G = G} n≤sucn (νGμG→νG⊥ {G = G} {k = k} x)
% % % \end{code}
% % % \begin{code}[hide]
% % % to (νG⊥↔νμG {G = G}) (k , x) = suc k , mapΓ G _ _ (λ { zero → λ () ; (suc _) → λ z → z }) x
% % % from (νG⊥↔νμG {G = G}) (suc k , x) = k , νGμG→νG⊥ {G = G} x
% % % 
% % % \end{code}
% % % \begin{code}[hide]
% % % ν? ∅ _ = no λ ()
% % % ν? ε _ = yes (zero , refl)
% % % ν? (char c) _ = no λ ()
% % % ν? (x · G) Γν = map? (record { to = λ (x , n , y) → (n , x , y) ; from = λ (n , x , y) → (x , n , y) }) (x ×? ν? G Γν)
% % % ν? (G₁ ∪ G₂) Γν = map? (record { to = λ { (inl (n , x)) → n , inl x ; (inr (n , x)) → n , inr x } ; from = λ { (n , inl x) → inl (n , x) ; (n , inr x) → inr (n , x) } }) (ν? G₁ Γν ⊎? ν? G₂ Γν)
% % % ν? (G₁ ∗ G₂) Γν = map? (ν▹ {G₁ = G₁} {G₂ = G₂}) (ν? G₁ Γν ×? ν? G₂ Γν)
% % % \end{code}
% % % Using these two helpers, we can define the nullability of variables and fixed points as follows:
% % % \begin{code}
% % % ν? {Γ = Γ} (var i) Γν = map? (νΓi↔ν⟦vari⟧Γ {Γ = Γ}) (Γν i)
% % % ν? (μ G) Γν = map? νG⊥↔νμG (ν? G (no (λ ()) ∷> Γν))
% % % \end{code}
% % % 
% % % \subsubsection{Derivatives}
% % % 
% % % Computing the derivative also requires us to deal with free variables in our grammar. For derivatives, we need to keep track of four different environments:
% % % 
% % % \begin{enumerate}
% % % \item The language environment, \ab{Γ}, which contains the language of each variable.
% % % \item The nullability environment, \ab{Γν}, which tells us the nullability of all variables.
% % % \item The derivative environment, \ab{Γδ}, which keeps track of the derivative of each variable.
% % % \item The unrolling environment, \ab{Γσ}, which allows us to replace each variable by the fixed point that bound it, thus unrolling the fixed point.
% % % \end{enumerate}
% % % 
% % % The \af{Gram} data Type is no longer parameterized by its semantics, so we first define a syntactic derivative function \af{δ?} and afterwards prove that it corresponds to the semantic derivative.
% % % \begin{code}
% % % δ?  : (Γ : Fin n → Lang) (Γν : (i : Fin n) → Dec (ν (Γ i))) (Γδ : Fin n → Gram m) 
% % %       (Γσ : Fin n → Gram m) 
% % %     → Gram n → Char → Gram m
% % % \end{code}
% % % Again, all simple cases are the same except for passing around the environments correctly to recursive calls, so we skip to the two new cases for variables and fixed points.
% % % \begin{code}[hide]
% % % δ? _ _ _ _ ∅ c = ∅
% % % δ? _ _ _ _ ε c = ∅
% % % δ? _ _ _ _ (char c') c with c ≟ c'
% % % ... | yes _ = ε
% % % ... | no _ = ∅
% % % δ? Γ Γν Γδ Γσ (A · G) c = A · δ? Γ Γν Γδ Γσ G c
% % % δ? Γ Γν Γδ Γσ (G₁ ∪ G₂) c = δ? Γ Γν Γδ Γσ G₁ c ∪ δ? Γ Γν Γδ Γσ G₂ c
% % % δ? Γ Γν Γδ Γσ (G₁ ∗ G₂) c =  (δ? Γ Γν Γδ Γσ G₁ c ∗ subst G₂ Γσ)
% % %                           ∪  (ν? {Γ = Γ} G₁ Γν · δ? Γ Γν Γδ Γσ G₂ c)
% % % \end{code}
% % % For variables, we simply look up their derivative in the derivative environment. For fixed points, we need to show how to extend each of the four environments. Here we apply the same trick as we discovered in \cref{ex:cfg-parsing}, namely that we introduce a new fixed point which allows us to refer to the derivative itself.
% % % \begin{code}
% % % δ? _ _ Γδ _ (var i) _ = Γδ i
% % % δ? Γ Γν Γδ Γσ (μ G) c =
% % %   μ (δ?  (⟦ μ G ⟧ Γ                      ∷> Γ)
% % %          (ν? {Γ = Γ} (μ G) Γν            ∷> Γν)
% % %          (var zero                       ∷> (rename suc ∘ Γδ))
% % %          (subst (μ G) (rename suc ∘ Γσ)  ∷> (rename suc ∘ Γσ))
% % %          G c)
% % % \end{code}
% % % \begin{code}[hide]
% % % 
% % % ↔refl : A ↔ A
% % % ↔refl = record { to = λ x → x ; from = λ z → z }
% % % 
% % % \end{code}
% % % 
% % % We show the correctness of the syntactic derivative by showing that every string accepted by the result of taking the syntactic derivative of a grammar is also accepted by the semantic derivative of the original grammar and vice versa. The last two arguments specify that the unrolling and derivative environment actually contain what they are supposed to contain.
% % % \begin{code}
% % % δ?↔δ : (G : Gram n) {Γ : Fin n → Lang} {Γ' : Fin m → Lang} 
% % %        {Γν : (i : Fin n) → Dec (ν (Γ i))} {Γδ : Fin n → Gram m} {Γσ : Fin n → Gram m}
% % %      → ((i : Fin n) → ⟦ Γσ i ⟧ Γ' ⇔ Γ i)
% % %      → ((i : Fin n) → ⟦ Γδ i ⟧ Γ' ⇔ δ (Γ i) c)
% % %      → ⟦ δ? Γ Γν Γδ Γσ G c ⟧ Γ' ⇔ δ (⟦ G ⟧ Γ) c
% % % \end{code}
% % % We are still working on proofs for two parts of this correspondence. First, if a substitution corresponds pointwise to a change of environment, substituting all variables in a grammar also corresponds to a change of environment.
% % % \begin{code}
% % % postulate substΓσ  : {Γσ : Fin n → Gram m} (G : Gram n)
% % %                    → ((i : Fin n) → ⟦ Γσ i ⟧ Γ' ⇔ Γ i) → ⟦ subst G Γσ ⟧ Γ' ⇔ ⟦ G ⟧ Γ
% % % \end{code}
% % % Second, we are still working on proving the correctness of the syntactic derivative of fixed points.
% % % \begin{code}
% % % postulate
% % %   δ?↔δμ  : (G : Gram (suc n)) {Γ : Fin n → Lang} {Γ' : Fin m → Lang} 
% % %            {Γν : (i : Fin n) → Dec (ν (Γ i))} {Γδ : Fin n → Gram m} {Γσ : Fin n → Gram m}
% % %          → ((i : Fin n) → ⟦ Γσ i ⟧ Γ' ⇔ Γ i)
% % %          → ((i : Fin n) → ⟦ Γδ i ⟧ Γ' ⇔ δ (Γ i) c)
% % %          → ⟦ δ? Γ Γν Γδ Γσ (μ G) c ⟧ Γ' ⇔ δ (⟦ μ G ⟧ Γ) c
% % % \end{code}
% % % \begin{code}[hide]
% % % δ?↔δ ∅ eσ eδ = ↔refl
% % % to (δ?↔δ ε eσ eδ) ()
% % % from (δ?↔δ ε eσ eδ) ()
% % % 
% % % to (δ?↔δ {c = c}     (char c') eσ eδ) x with c ≟ c'
% % % to (δ?↔δ {c = c}     (char .c) eσ eδ) (k , refl) | yes refl = k , refl
% % % to (δ?↔δ             (char _)  eσ eδ) () | no _
% % % to (δ?↔δ             (A · G)   eσ eδ) (k , x , y) with to (δ?↔δ G eσ eδ) (k , y)
% % % ... | k , y = k , x , y 
% % % to (δ?↔δ             (G₁ ∪ G₂) eσ eδ) (k , inl x) with to (δ?↔δ G₁ eσ eδ) (k , x)
% % % ... | k , x = k , inl x
% % % to (δ?↔δ             (G₁ ∪ G₂) eσ eδ) (k , inr x) with to (δ?↔δ G₂ eσ eδ) (k , x)
% % % ... | k , x = k , inr x
% % % to (δ?↔δ {c = c}     (G₁ ∗ G₂) eσ eδ) (k , inl (u , v , refl , x , y)) with to (δ?↔δ G₁ eσ eδ) (k , x) | to (substΓσ G₂ eσ) (k , y)
% % % ... | k₁ , x | k₂ , y = max k₁ k₂ , (c ∷ u) , v , refl , weakenˡ {G = G₁} x , weakenʳ {G = G₂} y
% % % to (δ?↔δ {c = c} (G₁ ∗ G₂) eσ eδ) (k , inr (x , y)) with x | to (δ?↔δ G₂ eσ eδ) (k , y)
% % % ... | k₁ , x | k₂ , y = max k₁ k₂ , [] , (c ∷ _) , refl , weakenˡ {G = G₁} x , weakenʳ {G = G₂} y
% % % to (δ?↔δ           (var i)   eσ eδ) (k , x) = zero , to (eδ i) (k , x)
% % % from (δ?↔δ {c = c} (char c') eσ eδ) x with c ≟ c'
% % % from (δ?↔δ {c = c} (char c)  eσ eδ) (k , refl) | yes refl = k , refl
% % % from (δ?↔δ {c = c} (char .c) eσ eδ) (k , refl) | no ¬c≡c = k , ¬c≡c refl
% % % from (δ?↔δ         (A · G)   eσ eδ) (k , x , y) with from (δ?↔δ G eσ eδ) (k , y)
% % % ... | k , y = k , x , y
% % % from (δ?↔δ         (G₁ ∪ G₂) eσ eδ) (k , inl x) with from (δ?↔δ G₁ eσ eδ) (k , x)
% % % ... | k , x = k , inl x
% % % from (δ?↔δ         (G₁ ∪ G₂) eσ eδ) (k , inr x) with from (δ?↔δ G₂ eσ eδ) (k , x)
% % % ... | k , x = k , inr x
% % % from (δ?↔δ {c = c} (G₁ ∗ G₂) eσ eδ) (k , [] , (.c ∷ v) , refl , x , y) with from (δ?↔δ G₂ eσ eδ) (k , y)
% % % ... | k' , y = k' , inr ((k , x) , y)
% % % from (δ?↔δ {c = c} (G₁ ∗ G₂) eσ eδ) (k , (.c ∷ u) , v , refl , x , y) with from (δ?↔δ G₁ eσ eδ) (k , x) | from (substΓσ G₂ eσ) (k , y)
% % % ... | k₁ , x | k₂ , y = max k₁ k₂ , inl (u , v , refl , weakenˡ {G = δ? _ _ _ _ G₁ c} x , weakenʳ {G = subst G₂ _} y)
% % % from (δ?↔δ         (var i)   eσ eδ) (k , x) = from (eδ i) x
% % % 
% % % δ?↔δ (μ G) eσ eδ = δ?↔δμ G eσ eδ
% % % \end{code}
% % % With the exception of these two postulates, we have proven the correctness of our syntactic derivative function.
% % % % \begin{code}[hide]
% % % % substGvar≡G : (G : Gram n) → subst G var ≡ G
% % % % substGvar≡G ∅ = refl
% % % % substGvar≡G ε = refl
% % % % substGvar≡G (char x) = refl
% % % % substGvar≡G (x · G) = cong (x ·_) (substGvar≡G G)
% % % % substGvar≡G (G ∪ G₁) = cong₂ _∪_ (substGvar≡G G) (substGvar≡G G₁)
% % % % substGvar≡G (G ∗ G₁) = cong₂ _∗_ (substGvar≡G G) (substGvar≡G G₁)
% % % % substGvar≡G (μ G) = cong μ (trans (cong (subst G) (funext (λ { zero → refl ; (suc i) → refl }))) (substGvar≡G G))
% % % % substGvar≡G (var _) = refl
% % % % 
% % % % substG⊥≡G : {σ : Fin zero → Gram zero} (G : Gram zero) → subst G σ ≡ G
% % % % substG⊥≡G G = trans (cong (subst G) (funext (λ ()))) (substGvar≡G G)
% % % % 
% % % % ≡→↔ : {x y : Set} → x ≡ y → x ↔ y
% % % % ≡→↔ refl = record { to = λ z → z ; from = λ z → z }
% % % % \end{code}
% % % 
% % % \subsubsection{Parsing}
% % % 
% % % Tying it all together, we show how to parse a string following a grammar. We only care about grammars without variables, so all the environments are empty (\as{λ}~\as{(}\as{)}).
% % % \begin{code}
% % % parse : (G : Gram zero) → (w : String) → Dec (⟦ G ⟧ (λ ()) w)
% % % parse G [] = ν? G (λ ())
% % % parse G (c ∷ cs) = map? (δ?↔δ G (λ ()) (λ ())) (parse (δ? (λ ()) (λ ()) (λ ()) (λ ()) G c) cs)
% % % \end{code}
% % % This is a correct parser for context-free grammars.
% % % 
\begin{code}[hide]%
\>[0]\AgdaKeyword{module}\AgdaSpace{}%
\AgdaModule{4-discussion}\AgdaSpace{}%
\AgdaKeyword{where}\<%
\end{code}

\section{Discussion}

Finally, we want to discuss three aspects of our work: expressiveness, performance, and simplicity.

\jr{TODO: μ-regular expressions have been studied before, cite}
\paragraph{Expressiveness} We conjecture that our grammars which include variables and fixed points can describe any context-free language. We have shown the example of balanced the bracket language which is known to be context-free. Furthermore, Grenrus shows that any context-free grammar can be converted to his grammars \cite{fix-ing-regular-expressions}, which are similar to our grammars. The main problem is showing that mutually recursive nonterminals can be expressed using our simple fixed points, which requires Beki\'c's bisection lemma~\cite{Bekic1984}. Formalizing this in our framework is future work.

Going beyond context-free languages, many practical programming languages cannot be adequately described as context-free languages. For example, features such as associativity, precedence, and indentation sensitivity cannot be expressed directly using context-free grammars. Recent work by Afroozeh and Izmaylova~\cite{one-parser-to-rule-them-all} shows that all these advanced features can be supported if we extend our grammars with data-dependencies. Our framework can form a foundation for such extensions and we consider formalizing it as future work.

\paragraph{Performance}
\jr{cite Jeremy Yallop's work}
For a parser to practically useful, it must at least have linear asymptotic complexity for practical grammars. Might et al. \cite{parsing-with-derivatives} show that naively parsing using derivatives does not achieve that bound, but optimizations might make it possible. In particular, they argue that we could achieve $O(n|G|)$ time complexity (where $|G|$ is the grammar size) if the grammar size stays approximately constant after every derivative. By compacting the grammar, they conjecture it is possible to achieve this bound for any unambiguous grammar. We want to investigate if similar optimizations could be applied to our parser and if we can prove that we achieve this bound.

\paragraph{Simplicity}
One of the main contributions of Elliot's type theoretic formalization of languages~\cite{conal-languages} is its simplicity of implementation and proof. To be able to extend his approach to context-free languages we have had to introduce some complications.
\jr{TODO: finish this paragraph}
% Most notably, we use fuel to define the semantics of our grammars. We have explored other approaches such as using guarded type theory, but we did not manage to significantly simplify our formalization. Furtheremore, we expect that many proofs remain simple despite our fuel-based approach.
\medbreak

In conclusion, we have formalized (acyclic) context-free grammars using a type theoretic approach to provide fertile ground for further formalizations of disambiguation strategies and parsers that are both correct and performant.

%
% TODO:
%
% \begin{credits}
% \subsubsection{\ackname} A bold run-in heading in small font size at the end of the paper is
% used for general acknowledgments, for example: This study was funded
% by X (grant number Y).
% 
% \subsubsection{\discintname}
% It is now necessary to declare any competing interests or to specifically
% state that the authors have no competing interests. Please place the
% statement with a bold run-in heading in small font size beneath the
% (optional) acknowledgments\footnote{If EquinOCS, our proceedings submission
% system, is used, then the disclaimer can be provided directly in the system.},
% for example: The authors have no competing interests to declare that are
% relevant to the content of this article. Or: Author A has received research
% grants from Company W. Author B has received a speaker honorarium from
% Company X and owns stock in Company Y. Author C is a member of committee Z.
% \end{credits}
%
% ---- Bibliography ----
%
% BibTeX users should specify bibliography style 'splncs04'.
% References will then be sorted and formatted in the correct style.

\bibliographystyle{splncs04}
\bibliography{src/references.bib}

\end{document}
